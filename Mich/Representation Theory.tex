\documentclass{article}
\usepackage{graphicx} % Required for inserting images
\usepackage[utf8]{inputenc}
\usepackage{amsmath,amsfonts,amssymb,amsthm}
\usepackage{enumerate,bbm}
\usepackage{leftindex}
\usepackage{tikz,tikz-cd,graphicx,color,mathrsfs,color,hyperref,boldline}
\usepackage{caption,float}
\usepackage[a4paper,margin=1in,footskip=0.25in]{geometry}

\usepackage{listings}
\usepackage{xcolor}

\usepackage{tabularx,capt-of}

\usepackage{blindtext}
%Image-related packages
\usepackage{graphicx}
\usepackage{subcaption}
\usepackage[export]{adjustbox}
\usepackage{lipsum}

%hyperref setup
\hypersetup{
    colorlinks=true,
    linkcolor=blue,
    filecolor=magenta,      
    urlcolor=cyan,
    pdftitle={Overleaf Example},
    pdfpagemode=FullScreen,
    }

%New colors defined below
\definecolor{codegreen}{rgb}{0,0.6,0}
\definecolor{codegray}{rgb}{0.5,0.5,0.5}
\definecolor{codepurple}{rgb}{0.58,0,0.82}
\definecolor{backcolour}{rgb}{0.95,0.95,0.92}

%Code listing style named "mystyle"
\lstdefinestyle{mystyle}{
  backgroundcolor=\color{backcolour}, commentstyle=\color{codegreen},
  keywordstyle=\color{magenta},
  numberstyle=\tiny\color{codegray},
  stringstyle=\color{codepurple},
  basicstyle=\ttfamily\footnotesize,
  breakatwhitespace=false,         
  breaklines=true,                 
  captionpos=b,                    
  keepspaces=true,                 
  numbers=left,                    
  numbersep=5pt,                  
  showspaces=false,                
  showstringspaces=false,
  showtabs=false,                  
  tabsize=2
}

%"mystyle" code listing set
\lstset{style=mystyle}

\theoremstyle{definition}
\newtheorem{defn}{Definition}[section]
\newtheorem{example}[defn]{Example}
\theoremstyle{remark}
\newtheorem{rem}{Remark}
\newtheorem{remS}[section]{defn}
\theoremstyle{plain}
\newtheorem{lem}[defn]{Lemma}
\newtheorem{thm}[defn]{Theorem}
\newtheorem{prop}[defn]{Proposition}
\newtheorem{fact}[defn]{Fact}
\newtheorem{crly}[defn]{Corollary}
\newtheorem{conj}[defn]{Conjecture}

%\newtheorem*{programming*}{Programming Task}

%\newtheorem{innercustomgeneric}{\customgenericname}
%\providecommand{\customgenericname}{}
%\newcommand{\newcustomtheorem}[2]{%
%  \newenvironment{#1}[1]
%  {%
%   \renewcommand\customgenericname{#2}%
%   \renewcommand\theinnercustomgeneric{##1}%
%   \innercustomgeneric
%  }
%  {\endinnercustomgeneric}
%}

%\newcustomtheorem{question}{Question}
%\newcustomtheorem{programming}{Programming Task}

\newcommand{\NN}{\mathbb{N}}
\newcommand{\ZZ}{\mathbb{Z}}
\newcommand{\QQ}{\mathbb{Q}}
\newcommand{\RR}{\mathbb{R}}
\newcommand{\CC}{\mathbb{C}}
\newcommand{\PP}{\mathbb{P}}
\newcommand{\FF}{\mathbb{F}}
\newcommand{\Hom}{\operatorname{Hom}}
\newcommand{\im}{\operatorname{im}}
\newcommand{\id}{\operatorname{id}}
\newcommand{\Ind}{\operatorname{Ind}}
\newcommand{\Res}{\operatorname{Res}}

\newcommand{\calD}{\mathcal{D}}

\newcommand{\sol}{\textit{Solution: }}

\title{Representation theory}
\author{Kevin}
\date{October 2024}

\begin{document}
\maketitle
\section{Some Linear Algebra}
\textit{It was supposed to be a revision of IB linear algebra, but we have a solid understanding of linear algebra, so we will not include redundant stuff in this section.}

We define the tensor product of two vector spaces $V$ and $W$. 
\begin{defn}
    Given two $K$-vector spaces $V,W$, the tensor product $V\otimes_K W$ is the vector space with basis vectors $v\otimes w$, $v\in V, w\in W$ modulo the following relations:
    \begin{itemize}
        \item $(v_1+v_2)\otimes w=v_1\otimes w+ v_2\otimes w$.
        \item $v\otimes (w_1+w_2)=v\otimes w_1+v\otimes w_2$.
        \item For all $k\in K$, $(kv)\otimes w=v\otimes (kw)=k(v\otimes w)$.
    \end{itemize}
\end{defn}
When the underlying field is clear, we can drop the subscript and write $V\otimes W$.
\begin{prop}[Universal Property]
    For any bilinear map $V\times W\to U$, there exists a unique linear map $V\otimes W\to U$ which makes the following diagram commutes.
    \begin{center}
        % https://tikzcd.yichuanshen.de/#N4Igdg9gJgpgziAXAbVABwnAlgFyxMJZABgBpiBdUkANwEMAbAVxiRADUAdTvAW3gAEAdRABfUuky58hFAEZyVWoxZsAqmIkgM2PASJk5S+s1aIO3CH0EjRSmFADm8IqABmAJwi8kZEDggkBWVTNjdNdy8fRAAmagCg6gYsMDMQKDo4AAsHEGoTVXNuPAZYAXDxSO9feMDY-JU0gEc8kAY6ACMYBgAFKT1ZEA8sRyycMQpRIA
\begin{tikzcd}
V\times W \arrow[r, "f"] \arrow[d, "q"']  & U \\
V\otimes W \arrow[ru, "\tilde f", dashed] &  
\end{tikzcd}
    \end{center} where $q$ is the quotient map.
\end{prop}
\begin{proof}
    
\end{proof}
\begin{prop}
    Let $\{v_1,...,v_m\}$, $\{w_1,...,w_n\}$ be basis of $V$ and $W$ respectively. Then $\{v_i\otimes w_j\}_{i,j}$ is a basis of $V\otimes W$.
\end{prop}
\begin{proof}
   We claim that the vectors $v_i\otimes w_j$ span the tensor product. Given $v\otimes w$, we can express $v$ in $\{v_i\}$ and $w$ in $\{w_j\}$ then use the relations. 
   
   We now show that $\{v_i\otimes w_j\}$ is a linearly independent set. Consider the linear functional $\epsilon_i:V\to K$ and $\phi_j:W\to K$ which are dual to $v_i$, $w_j$ respectively. We define $L_{p,q}:V\times W\to K, (v,w)\mapsto \epsilon_p(v)\phi_q(w)$. Note that this map is bilinear, so it descends to a well-defined map $\tilde L_{p,q}:V\otimes W\to K,\ v\otimes w\mapsto \epsilon(v)\phi(w)$.  One can check that $\tilde L_{p,q}$ is actually a linear map and $\tilde L_{p,q}(v_i\otimes w_j)=\delta_{ip}\delta_{jq}$. This implies linear independence.
\end{proof}
\begin{prop}
    There are natural isomorphisms $U\otimes(V\otimes W)\cong (U\otimes V)\otimes W$, and $(U\oplus V)\otimes W\cong (U\otimes W)\oplus (V\otimes W)$.
\end{prop}
\begin{proof}
    Consider the maps $u\otimes (v\otimes w)\mapsto (u\otimes v)\otimes w$ and $(u,v)\otimes w\mapsto (u\otimes w,v\otimes w)$. These clearly preserve the tensor relations and induce linear maps.
\end{proof}
\begin{prop}
    There are natural isomorphisms $W\otimes V^\ast\cong \Hom(V,W)$ and $\Hom(U\otimes V,W)\cong\Hom(U,\Hom(V,W))$.
\end{prop}

Consider linear operators $\alpha\in\operatorname{End}(V)$ and $\beta\in \operatorname{End}(W)$. We can define an operator $\alpha\otimes \beta\in\operatorname{End}(V\otimes W)$ by $(\alpha\otimes \beta)(v\otimes w)=(\alpha v)\otimes (\beta w)$ and extending by linearity to all vectors in $V\otimes W$. Note that the extension is possible as the definition clearly preserves the tensor relations. If we assume that $A_{ij}$ is the matrix of $\alpha$ w.r.t. $\{v_1,...,v_m\}$ and $B_{kl}$ is the matrix of $\beta$ w.r.t. $\{w_1,...,w_n\}$, then $\alpha\otimes \beta$ has matrix $(A\otimes B)_{(i,k)(j,l)}=A_{ij}B_{kl}$. This can be checked by applying $\alpha\otimes\beta$ to each basis vectors. We spell them out here.
\begin{align*}
    (\alpha\otimes\beta)(v_j\otimes w_l)&=(\alpha v_j)\otimes (\beta w_l)\\
    &=\sum_{i,k}(A_{ij}v_i)\otimes(B_{kl}w_k)\\
    &=\sum_{i,k}A_{ij}B_{kl}(v_i\otimes w_k)
\end{align*}
\begin{prop}
    \begin{enumerate}
        \item The eigenvalues of $\alpha\otimes\beta$ are $\lambda_i\mu_k$ where $\lambda_i$ are e-values of $\alpha$ and $\mu_k$ are evalues of $\mu_k$.
        \item $\operatorname{Tr}(A\otimes B)=\operatorname{Tr}(A)\operatorname{Tr}(B)$
        \item $\det(A\otimes B)=\det(A)^n\det(B)^m$.
    \end{enumerate}
\end{prop}
\begin{proof}
    Non-trivial.
\end{proof}
We can construct $n$-fold tensor product $V^{\otimes n}$, and $S_n$ acts on $V^{\otimes n}$ by permuting factors, i.e., $\sigma\cdot (u_1\otimes\cdots\otimes u_n)=u_{\sigma(1)}\otimes \cdots\otimes u_{\sigma(n)}$.
\begin{defn}[Symmetric/exterior power of $V$]
Let $V$ be a $K$-vector space.
    \begin{itemize}
        \item The $n$th symmetric power of $V$ is
        \[\{S^nV=\{a\in V^{\otimes n}:\forall \sigma\in S_n,\ \sigma\cdot a=a\}\]
        \item The $n$th exterior power of $V$ is
        \[\Lambda^nV=\{a\in V^{\otimes n}:\forall\sigma\in S_n\ \sigma\cdot a=\operatorname{sgn}(\sigma)a\}\]
    \end{itemize}
\end{defn}
\begin{prop}
    If $\{v_1,...,v_m\}$ is a basis for $V$, then 
    \begin{itemize}
        \item $\{\frac1n\sum_{\sigma\in S_n}v_{k_{\sigma(1)}}\otimes\cdots\otimes v_{k_{\sigma(n)}}:1\le k_1\le\cdots\le k_n\le m\}$ is a basis for $S^nV$
        \item $\{\frac1n\sum_{\sigma\in S_n}\operatorname{sgn}(\sigma)v_{k_{\sigma(1)}}\otimes\cdots\otimes v_{k_{\sigma(n)}}:1\le k_1<\cdots<k_n\le m\}$ is a basis for $\Lambda^nV$.
    \end{itemize}
\end{prop}
\begin{proof}
    Sheet 3.
\end{proof}
\begin{crly}
    Let $V$ be a $d$-dimensional vector space, then $\dim S^nV=\binom{d+n-1}{n}$, $\dim\Lambda^nV=\binom{d}{n}$.
\end{crly}
\begin{prop}
    $V\otimes V\cong S^2V\oplus\Lambda^2V$.
\end{prop}
\begin{proof}
    The union of basis is a basis for $V\otimes V$, and count the number.
\end{proof}

\section{Basic Stuff}
\begin{defn}
    $G$ is a group, $V$ a vector space over a field $k$, a representation of $G$ on $V$ is a group homomorphism $\rho:G\to GL(V)$
\end{defn}
\begin{rem}
    Synonym 1: $G$ acts on $V$ by linear transformations
    Synonym 2: $V$ is a $kG$-module...
\end{rem}
\begin{defn}
    The degree of $\rho$ is $\dim V$.
\end{defn}
\begin{example}
\begin{enumerate}
    \item $G=\ZZ_2=\{\pm 1\}$, $V=\RR^2$, $1\mapsto I$, $-1\mapsto \begin{pmatrix}
        -1&\\
        &1
    \end{pmatrix}$
    \item (Trivial representation) $G$ any group, $V=k$, $\rho(g)=1\in k^\ast$.
    \item $G=\ZZ$, $V$ vector space. We have 
    \[\{\text{rep of }\ZZ\}\leftrightarrow\{\text{invertible linear transformation}\}\]
    given by $\rho\mapsto \rho(1)$. (This follows from definition)
    \item $G=\ZZ_N$, then similarly, we have
    \[\{\text{rep of }\ZZ_N\}\leftrightarrow\{\text{invertible linear maps }f\text{ s.t. }f^N=I\}\]
\end{enumerate}
\end{example}
\begin{rem}
    If $G$ has group presentation $\langle g_1,...,g_k\mid r_1,...,r_l\rangle$, then
    \begin{align*}
        \{\text{reps of }G\text{ on }V\}\leftrightarrow\{(f_1,...,f_k)\in GL(V)\mid \forall i,\rho(r_i)=I\}
    \end{align*}
\end{rem}
\begin{example}
    $G=S_3$, $V=\RR^2$. Given $g\in G$ make it act on an equilateral triangle by permuting vertices and extending linearly. This gives a representation by definition.
\end{example}

Problem: too many reps. We want to
\begin{itemize}
    \item know when two reps are the ``same'';
    \item break up reps into ``smaller'' reps.
\end{itemize}
\begin{defn}
    Let $\rho:G\to\operatorname{Aut}(V)$, $\rho':G\to V'$ are isomorphic if there is an isomorphism $\varphi:V\to V'$ as vector spaces s.t. $\varphi\circ\rho(g)=\rho'(g)\circ\varphi$ for all $g\in G$.
\end{defn}
Note that in the definition above, $\varphi$ is independent of $g$.

\textbf{Terminology:} $\varphi$ intertwines $\rho,\rho'$. This is the same as saying $\varphi^{-1}$ intertwines $\rho',\rho$.
\begin{example}
    $\rho:\ZZ\to\operatorname{Aut}(V)$ and $\rho':\ZZ\to\operatorname{Aut}(V')$ are isomorphic iff there exists a basis of $V$ and a basis of $V'$ s.t. 
    \[\begin{bmatrix}\text{matrix of }\rho(1)\text{ basis of }V\end{bmatrix}=\begin{bmatrix}
        \text{matrix of }\rho'(1)\text{ wrt to basis of }V'
    \end{bmatrix}\]
    As a corollary, isomorphism classes of reps of $\ZZ$ on $V$ is in bijection correspondence with rational canonical forms.
\end{example}
\begin{rem}
    If two representations $\rho,\rho'$ are isomorphic, then $\dim\rho=\dim\rho'$, but the converse is false, e.g., $\rho:\ZZ\to GL_1(\RR),1\mapsto 1$ and $\rho:\ZZ\to GL_1(\RR), 1\mapsto -1$. Another example is if $G=\ZZ_2$, then the matrices $\operatorname{diag}(-1,1),\operatorname{diag}(1,-1)$ and "off-diag"(1,1) define isomorphic reps which are different from $\operatorname{diag}(-1,-1)$ and $I$.
\end{rem}
    
 \begin{defn}
     A rep $V$ of $G$ is a direct sum of $W,W'$ if 
     \begin{itemize}
         \item they are invariant subspaces
         \item and $V=W\oplus W'$ as a vector space.
     \end{itemize}
 \end{defn}

\subsection{Subreps}
Let $\rho:G\to\operatorname{Aut}(V)$ be a rep. Suppose $W\le V$ is preserved by $G$, i.e., $gW\subseteq W$ for all $g\in G$ (implies $gW=W$ as $g$ is invertible). Then $G$ acts on $W$ and we get $\tilde{\rho}: G\to\operatorname{Aut}(W)$ and we say that $\tilde{\rho}$ is a subrepresentation of $\rho$.
\begin{defn}
    A $G$-invariant subspace $W\le V$ is called a subrepresentation. If $W=0$ or $W=V$ then we say the subrep is trivial. Otherwise we say the subrep is non-trivial
\end{defn}
\begin{defn}
    If there are no non-trivial subreps of $V$ then we say that $V$ is irreducible (simple).
\end{defn}
 \begin{example}
     \begin{itemize}
         \item Any 1-dim rep is irred.
         \item $G=\ZZ_2$, $1\mapsto\operatorname{diag}(1,-1)$ is not irreducible because it has exactly two non-trivial subreps.
         \item $S^3$ acts on an equilateral triangle by permuting vertices, one can extend linearly to $\CC^2$. This rep is irreducible by noting the fact that an invariant subspace is a complex line which corresponds to common eigenvectors.
         \item $G=\ZZ$, $V$ vec. space over $\CC$, $\rho(1)=A$. Can check that the invariant subspaces are precisely defined by Jordan blocks of $A$ when written in JNF. Suppose wrt to some basis $e_1,...,e_n$ one has $A$ in JNF, then $\langle e_1\rangle,..,\langle e_1,...,e_n\langle$ are all invariant subspaces, so the only irreducible rep of $\ZZ$ on vector space over $\CC$ are the $1$-dim reps.
     \end{itemize}
 \end{example}
 \begin{prop}
     Let $\rho:G\to\operatorname{Aut}(V)$ be a rep and $W\le V$ a subspace. Then $W$ is a subrep if and only if for every basis $v_1,...,v_a$ of $W$ and its extension $v_1,...,v_d$ which is a basis of $V$, the matrix of $\rho(g)$ is of the form
     \begin{align*}
     \begin{pmatrix}
         A & B\\ 0&C
     \end{pmatrix}
     \end{align*} for all $g\in G$
 \end{prop}
 \begin{proof}
 Trivial.
 \end{proof}

 \begin{prop}
     $V$ is a direct sum of $W,W'$ if and only if for every basis $v_1,...,v_a$ of $W$ and basis $v_{a+1},...,v_{d}$ of W', the matrix of $\rho(g)$ is of the form
     \begin{align*}
         \begin{pmatrix}
             A&0\\0&B
         \end{pmatrix}
     \end{align*} for all $g\in G$
 \end{prop}
 \begin{proof}
     Trivial.
 \end{proof}
 One can define external direct sums $\rho\oplus\rho'$ as well by requiring the action to be componentwise.
 \begin{defn}
     Given any rep $V$ of $G$ (abuse of notation), write it as a direct sum $V=W_1\oplus...\oplus W_r$, where $W_i$ is $G$-invariant and does not itself break up into direct sum. Such a $W$ is called indecomposable
 \end{defn}
 An example is given by JNF.
 \begin{rem}
     Indecomposable $\not\implies$ Irreducible.
 \end{rem}
 \begin{defn}
     Let $\rho:G\to\operatorname{Aut}(V)$, $\rho':G\to\operatorname{Aut}(V')$ be reps of $G$. A morphism (or $G$-equivariant map) from $\rho$ to $\rho'$ is a linear map $\varphi:V\to V'$ s.t. $\varphi\rho(g)=\rho'(g)\varphi$.

     We write $\operatorname{Hom}_G(V,V')$ for the set of these $G$-equivariant maps. This is a vector space.
     \begin{itemize}
         \item If $\varphi$ is an iso, then $V,V'$ are isomorphic reps;
         \item If $\varphi$ is injective, then $V$ is a subrep of $V'$
         \item If $\varphi$ is surjective, then $V'$ is a quotient rep of $V$.
     \end{itemize}
 \end{defn}
 \begin{example}
     Suppose $G$ acts on a set $X$, then we can lienarize the action and get a representation on $k[X]$ which is the vector space with basis $\{e_x\}_{x\in X}$. Note that this rep is never irreducible.

     Exercise:
     \begin{itemize}
         \item Show that if $X=\{x_1,...,x_d\}$, then $W=\langle e_{x_1}+...+e_{x_d}\rangle$ and $W'=\{\sum a_ix_i\mid \sum_ia_i=0\}$ are both subreps of $k[X]$ and if $\operatorname{char}(k)=0$, $k[X]=W\oplus W'$.
         \item Can you find an example where $W'$ is not irred.
         \item Is $W\oplus W'=k[X]$ true if $\operatorname{char}(k)=p$ prime?
     \end{itemize}
 \end{example}

\textbf{Clarification:} If $\varphi\in\operatorname{Hom}_G(V,V')$ is surjective, then $V/\ker\varphi\cong V'$, so we say that $V'$ is a quotient rep of $V$.

\subsection{Permutation Reps}

If $G$ acts on a set $X$ and $X=X_1\sqcup X_2$ and $GX_i=X_i$, then $k[X]=k[X_1]\oplus k[X_2]$ (direct sum as rep). More generally, $X=\bigsqcup O_i$ where $O_i$ are orbits. If $y\in O_i$, then $G/\operatorname{Stab}(y)$ is in bijective correspondence with $O_i$. Then, $k[X]=\bigoplus_i k[O_i]$. These are called permutation reps

\begin{example}
    $S_3$ acts on $\{1,2,3\}$ then $k[X]=k^3$ and the action is by permuting standard basis vectors.
\end{example}
\begin{defn}
    $\rho:G\to GL(V)$ rep, then $\rho$ is faithful if $\ker\rho$ is trivial.
\end{defn}
\begin{crly}
    If $G$ is simple, then any non-trivial rep is faithful. Every finite group can be embedded as the subgroup of some $GL(V)$ so admits a faithful rep.
\end{crly}
\begin{proof}
    1st part is trivial. The 2nd part uses Cayley's theorem and the fact that $S_n$ embeds in $GL_n(K)$ as the subgroup of permutation matrices.
\end{proof}
\begin{rem}
    Can compose $\rho$ rep with a group hom to get new reps. When the group hom is an inclusion, we say that the resulting rep is the restriction to the subgroup.
\end{rem}

\subsection{Dual, Hom, and Tensor Products}
Let $(\rho, V),\ (\sigma, W)$ be reps of $G$.
\begin{defn}
    We can make $\Hom_k(V,W)$ a representation of $G$ by letting $g$ act on $\Hom_k(V,W)$ by $g\cdot f=\sigma(g)\circ f\circ\rho(g^{-1})$.
\end{defn}
In particular, we can construct the dual $V^\ast$ by letting $g\cdot f=f\circ\rho(g^{-1})$.
\begin{defn}[Tensor product]
    We can define their tensor product $(\rho\otimes\sigma,V\otimes W)$ by $(\rho\otimes \sigma)(g)(v\otimes w)=\rho(g)(v)\otimes\sigma(g)(w)$ and extend by linearity.
\end{defn}
This construction clearly preserves the tensor relations so the extension by linearity is well-defined. Note that this also satisfies the axioms of representations.

More generally, if $(\rho,V)$ is a rep of $G$ and $(\sigma, W)$ is a rep of $H$ (both over the field $K$), then we can define $\rho\otimes\sigma : G\times H\to \operatorname{GL}(V\otimes W),\ (g,h)\mapsto \rho(g)\otimes \sigma(h)$. This is a representation of $G\times H$. The tensor product operation defined above can be realized as the special case given by precomposing with the diagonal embedding
\[G\overset{\Delta}{\longrightarrow} G\times G\overset{\rho\otimes \sigma}{\longrightarrow}\operatorname{GL}(V\otimes W)\]
\begin{prop}
    We have some natural isomorphisms of representations.
    \begin{enumerate}
        \item $\Hom_k(V,W)\cong W\otimes V^\ast$
        \item $\Hom_k(U\otimes V,W)\cong\Hom_k(U,\Hom_k(V,W))$
    \end{enumerate}
    where $U,V,W$ are representations of a group $G$.
\end{prop}
\begin{proof}
    The underlying vector spaces are naturally isomorphic by Proposition 1.4, so it suffices to check that the isomorphisms proposed in the proof are $G$-equivariant.
\end{proof}


\section{Category of representations}
\begin{prop}
    There is a natural bijection between $K[G]$-modules and representations of $G$ on vector spaces over $K$.
\end{prop}
\begin{proof}
    Given a rep $(\rho,V)$ of $G$, we can make it a $K[G]$-module by letting the group algebra act on $V$ in the obvious way, i.e., given an element $\sum_{g\in G}\lambda_gg$, where $\lambda_g\in K$ for all $g$,
    \[\left(\sum_{g\in G}\lambda_gg\right)\cdot v=\sum_g\lambda_g\rho(g)(v)\]
    Moreover, if $\phi:V\to W$ intertwines two representations $(\rho, V),\ (\sigma,W)$, i.e., $\forall g\in G$, $\phi\circ\rho(g)=\sigma(g)\circ\phi$, then it is easy to check that
    \[\left(\sum_g\lambda_gg\right)\cdot \phi(v)=\phi\left(\left(\sum_g\lambda_gg\right)\cdot v\right)\]
    i,e., $\phi$ is a module homomorphism. We can check that this gives a functor from the category of representations of $G$ on vector spaces over $K$ to the category of $K[G]$-modules.
    
    Conversely, given a $K[G]$-module $M$, we can define an action of $G$ on $M$ regarded as a $K$-vector space. [Just let $K\langle e\rangle$ act.] Embed $G$ in $K[G]$ in the obvious way, then the action of $K[G]$ on $M$ yields a representation. Each $K[G]$-module homomorphism induces a morphism between representations under this construction. We can check that this defines a functor from the category of $K[G]$-modules to the category of $K$-representations of $G$.

    It is clear that these two processes are inverses of each other.
\end{proof}
So studying representations is the same as studying $K[G]$-modules.



\section{Complete Reducibility}
\begin{thm}
Let $G$ be a be a finite group. $V$ a vector space over field $k$ s.t. $\operatorname{char}(k)=0$. Suppose $W\le V$ is a $G$-invariant subspace. Then $\exists$ a $G$-invariant complement $W'$ s.t. $V=W\oplus W'$ as reps of $G$.
\end{thm}
We will prove this theorem later. We now discuss some consequences of this theorem.
\begin{crly}
    In the situation of the preceding theorem, $V\cong W_1\oplus...\oplus W_r$ as reps s.t. $W_i$ is irreducible.
\end{crly}
\begin{proof}
    Induction on dimension.
\end{proof}
\begin{example}
    $G=\ZZ_N$, $k$ algebraically closed field with characteristic $0$. We can invoke JNF.
     If $V$ is indecomposable as a rep of $G$, then it's also indecomposable as a rep of $\ZZ$ by precomposing with the projection map, and the matrix of $\rho(1)=A$ is a Jordan block $J_\lambda$. We may compute $J_\lambda^i$ to conclude that $d=1$ and $\lambda^N=1$. So the only Jordan blocks that occur in a finite dim rep of $\ZZ_N$ are $1\times 1$ blocks with e-values $N$th roots of unity, i.e., diagonal. Exercise: write down all invariant subspaces of $V$ is $\rho(1)$ is as above.

     If $k=\FF_p$, $G=\ZZ_p$, then $G$ acts on $\FF_p^2$ by $1\mapsto \begin{pmatrix}
         1&a\\0&1
     \end{pmatrix}$ Show that there is no complement to the invariant line $\langle e_1\rangle$.
\end{example}
\textbf{Exercise:}  prove the theorem if $G$ is abelian and $V$ a v.s. over $\CC$. [Pick a $g\in G$ and find an e-value $\lambda$. Now any $h\in G$ preserves the $\lambda$-eigenspace by commutativity. Induction.]

Recall relevant knowledge on Hermitian inner product. [linear in the 1st argument and conjugate-linear in the 2nd argument.]

\begin{lem}
If $W\le V$ is a $G$-invariant subsapce and $\langle-,-\rangle$ a $G$-invariant inner product on $V$, then $W^\perp$ is $G$-invariant.
\end{lem}
\begin{proof}
    If $x\in W^\perp$ and $g\in G$, then $\langle gx,w\rangle=\langle g^{-1}gx,g^{-1}w\rangle=\langle x,g^{-1}w\rangle=0$.
\end{proof}
\begin{lem}[Weyl's unitary trick]
    $G$ finite group acting on a vector space $V$ over $\CC$. Then there exists a $G$-invariant Hermitian inner product on $V$.
\end{lem}
\begin{proof}
    Choose any Hermitian inner product and denote it $\langle-,-\rangle$. Define a new inner product 
    \[(v,w)=|G|^{-1}\sum_{g\in G}\langle gv,gw\rangle\]
    Most properties hold trivially. We check $G$-invariance and non-degeneracy. Let $h\in G$, then
    \[(hv,hw)=|G|^{-1}\sum_{g\in G}\langle ghv,ghw\rangle=|G|^{-1}\sum_{g'\in G}\langle g'v,g'w\rangle=(v,w)\]
    where we use the change of index $g'=gh$. For non-degeneracy, note that $(v,v)=0\Leftrightarrow\langle gv,gv\rangle=0$ for all $g\in G$ $\Leftrightarrow v=0$.
\end{proof}
\begin{defn}
    A rep $V$ of $G$ is completely reducible if it's isomorphic to a direct sum of irreducible reps.
\end{defn}
\begin{crly}
    Representations of finite group $G$ on vector space $V$ over $\CC$ are completely reducible.
\end{crly}
\begin{proof}
    Any complex rep can be given a unitary structure, i.e., a $G$-invariant Hermitian inner product on $V$, so any subrep has an orthogonal complement. Proceed by induction on $\dim V$.
\end{proof}
\begin{crly}
    If $G\le GL_n(\CC)$ is finite, then $G$ is conjugate to a subgroup of $U(n)$.
\end{crly}
\begin{proof}
    Recall that if $(-,-)$ is any Hermitian inner product, then can choose a basis $v_1,...,v_n$ of $\CC^n$ s.t. $(v_i,v_j)=\delta_{ij}$. Let $X$ be the change of basis matrix from $\{v_1,...,v_n\}$ to the standard basis, then $\langle x,y\rangle=(Xx,Xy)$. Apply this to a $G$-invariant inner product, then 
    \[\langle X^{-1}gXx,X^{-1}gXy\rangle = (gXx,gXy)=(Xx,Yy)=\langle x,y\rangle\]
    so $\{X^{-1}gX:g\in G\}\le U(n)$.
\end{proof}
\begin{crly}
    If $A\in GL_n(\CC)$ and $A^n=I$, then $A$ is diagonalizable.
\end{crly}
\begin{proof}
    $A$ defines a rep of $\ZZ_n$. We have three variants of the same proof.
    \begin{itemize}
        \item $A$ is conjugate to a unitary matrix, so diagonalizable by the preceding corollary.
        \item $v\in\CC^n$ e-vector for $A$. Complete reducibility implies that $\CC^n=\CC v\oplus W$ as reps. Done by induction on dimensions.
        \item The only irreducible reps are one dimensional (e-vectors exist in $\CC$), so $\CC^n$ splits as the direct sum of $A$-invariant lines, so $A$ is diagonal in some basis.
    \end{itemize}
\end{proof}
\begin{thm}[Complete reducibility, Maschke's theorem]
    Let $G$ be a finite group, $V$ a vector space over field $k$ such that $\operatorname{char}(k)\nmid |G|$. If $W\le V$ is $G$-invariant, then there exists a $G$-invariant complement.
\end{thm}
\begin{proof}
    Pick any $W'$ which is a complement of $W$. Let $q:V\to W$ be the projection onto $W$ with $\ker q=W$. We then have $gq(w)=gw=qgw$ for all $w\in W$. Now define $\Bar{q}(v)=|G|^{-1}\sum_{g\in G}gqg^{-1}v$
    then $\Bar{q}$ has several properties.
    \begin{itemize}
        \item $\Bar{q}(V)=W$ as $q(V)=W$.
        \item $gq(v)\in W$ for all $v\in V$ as $W$ is $G$-stable
        \item If $w\in W$, then $\Bar{q}(w)=|G|^{-1}\sum_{g\in G}gqg^{-1}w=|G|^{-1}\sum_{g\in G}gg^{-1}w=w$, so $\Bar{q}$ projects onto $W$.
        \item $\ker\bar{q}$ is $G$-stable because for $h\in G$, $h\bar{q}(v)=|G|^{-1}\sum_{g'\in G}g'qg^{-1}hv=\bar{q}hv$ where $g'=gh$.
        \item Finally, can prove that $\ker\bar q\oplus\operatorname{im}\bar q=V$ using properties of projection (idempotency).
    \end{itemize}
    So we have found a $G$-invariant subspace.
\end{proof}
Illuminating exercise: Do this process for $G=\ZZ_2$.
What fails when $\operatorname{char}(k)\mid G$?
\section{Schur's Lemma}
\begin{thm}[Schur's lemma]
    Let $V, W$ be irreducible reps of a group $G$, where $V, W$ are vector spaces over a field $K$. Then
    \begin{enumerate}
        \item Any $G$-equivariant map $\phi:V\to W$ is either $0$ or an iso.
        \item If in addition $K$ is algebraically closed, then $\dim \Hom_G(V,W)=0,1$. 
    \end{enumerate}
\end{thm}
\begin{proof}
    We prove $1)$. Note that if $\phi$ is non-zero, then $\ker\phi\neq V$ and $\ker\phi$ is an invariant subspace of $V$. Similarly $\im\phi\neq \{0\}$ is also an invariant subspace. Since both $V,W$ are irred reps, we must have $\ker\phi=0$ and $\im\phi=W$, i,e., $\phi$ is an iso.

    Now we prove $2)$. Suppose $\phi_1,\phi_2\in\Hom_G(V,W)$ are non-zero. By $1)$, they are isomorphisms, i.e., invertible, so consider $\phi_1^{-1}\circ\phi_2\in\operatorname{End}_G(V)$. Since $K$ is algebraically closed, we can find an eigenvalue $\lambda\in K$ of $\phi_1^{-1}\circ\phi_2$. Then consider $\ker(\phi_1^{-1}\circ\phi_2-\lambda\id_V)$, which is a non-trivial invariant subspace of $V$. By irreducibility of $V$ we must have $\phi_1^{-1}\circ\phi_2=\lambda\id_V$, i.e., $\phi_2=\lambda\phi_1$, i.e., $\Hom_G(V,W)=K\langle\phi_1\rangle\cong K$.
\end{proof}
\begin{crly}
    Let $K$ is algebraically closed. Suppose $V=\bigoplus_i V_i$ is a decomposition of a rep $V$ into irred components. Then, for each irreducible rep $W$ of $G$,
    \[|\{i:V_i\cong W\}|=\dim\Hom_G(V,W)=\dim\Hom_G(W,V)\]
\end{crly}
\begin{proof}
    This is immediate from Schur's lemma and properties of $\Hom_G(-,-)$.  
\end{proof}
\begin{crly}
    If $G$ admits a faithful irred rep over an algebraically closed field $K$, then $Z(G)$ is cyclic.
\end{crly}
\begin{proof}
    Let $(\rho,V)$ be such a rep and $z\in Z(G)$. By definition, $zg=gz$ for all $g\in G$, so $\rho(z)\rho(g)=\rho(g)\rho(z)$ for all $g\in G$, i.e., $\rho(z)$ is $G$-equivariant, i.e., $\rho(z)\in\operatorname{End_G(V)}$, and $\rho(z)\neq 0$. By Schur's lemma, $\rho(z)=\lambda\id_V$ for some $\lambda\in K^\times$, so $Z(G)$ embeds as a subgroup of $K^\times$, so $Z(G)$ must be cyclic.
\end{proof}
\subsection{Isotypical Decomposition}
\begin{defn}
    Let $V$ be a completely reducible rep of $G$, and let $W$ be any irred rep, then the $W$-isotypical component of $V$ is the smallest subrep of $V$ containing all subreps of $V$ isomorphic to $W$.

    We say that $V$ has a unique isotypical decomposition if it is the direct sum of $W$-isotypical components as $W$ runs through all irred reps of $G$.
\end{defn}
\begin{thm}
    If $G$ is a finite group, then every complex rep of $G$ has a unique isotypical decomposition
\end{thm}
\begin{proof}
    ES2. The proof uses character theory, we will give the proof in the next section.
\end{proof}


\section{Characters}
\begin{defn}
    $\rho:G\to\operatorname{Aut}(V)$, $V$ a vec. space over $k$. The character of $\rho$ is a function $\chi:G\to k$, $g\mapsto\operatorname{tr}(\rho(g))$.
\end{defn}
\begin{rem}
    It is clear that $\chi$ satisfies the following properties.
    \begin{itemize}
        \item $\chi(g)$ doesn't depend on the choice of basis of $V$.
        \item If $\rho,\rho'$ are isomorphic reps via a linear iso represented by matrix $X$, then $\operatorname{tr}(\rho'(g))=X\rho(g) X^{-1}$ for all $g\in G$.
    \end{itemize}
\end{rem}
\begin{lem}
    $\rho$ rep of $G$ on $V$ (over $\CC$).
    \begin{itemize}
        \item $\chi(1)=\dim V$
        \item $\chi(g)=\chi(hgh^{-1})$ for all $h,g\in G$
        \item $\chi(g^{-1})=\chi(g)^\ast$ if $G$ is finite.
        \item If $\chi'$ is the character of $\rho'$ then $\chi_{\rho\oplus\rho'}=\chi+\chi'$ defined as $g\mapsto\chi(g)+\chi'(g)$.
    \end{itemize}
\end{lem}
\begin{proof}
    Trivial.
\end{proof}
\begin{defn}
    The set of class functions is $\mathcal{C}_G=\{f:G\to\CC:\forall h,g\in G, f(ghg^{-1})=f(h)\}$.
\end{defn}
Note that $\mathcal{C}_G$ is a vector space over $\CC$ with basis the functions \[\delta_{\mathcal{O}}:G\to\CC,g\mapsto \begin{cases}1&g\in\mathcal{O}\\ 0&g\not\in\mathcal{O}\end{cases}\]
where $\mathcal{O}$ is a conjugacy class.

Write $G$ as a disjoint union of conj. classes, then $\chi_V\in\mathcal{C}_G$ if $V$ is a complex rep of $G$.

Define a Hermitian inner product on $\mathcal{C}_G$ by
\[\langle f,f'\rangle_G=|G|^{-1}\sum_{g\in G}f(g)\overline{f'(g)}=\sum_{i=1}^r\dfrac{1}{|C_G(x_i)|}f(x_i)\overline{f'(x_i)}\]
where $x_i$ are in distinct orbits.

\begin{lem}
Let $V,W$ be (unitary) complex reps of a finite group $G$, then
    $\chi_{\Hom_k(V,W)}(g)=\overline{\chi_V(g)}\chi_W(g)$
\end{lem}
\begin{proof}
    Given $g\in G$, choose eigenbasis for $\{v_1,...,v_n\}$ of $V$ and $\{w_1,...,w_m\}$ of $W$. This is always possible since $G$ is finite. We get a basis $\{\alpha_{ij}\}$ of $\Hom_k(V,W)$, where $\alpha_{ij}(v_k)=\delta_{jk}w_i$ (extend $k$-lienarly). We have
    \begin{align*}
        (g\cdot\alpha_{ij})(v_k)&=\rho(g)\circ\alpha_{ij}\circ\rho(g)^{-1}(v_k)\\
        &=\rho(g)(\alpha_{ij}(\lambda_k^{-1}v_k))\\
        &=\rho(g)(\lambda_k^{-1}\delta_{jk}w_i)\\
        &=\mu_i\lambda_k^{-1}\delta_{jk}w_i\\
        &=\mu_i\lambda_k^{-1}\alpha_{ij}(v_k)
    \end{align*}
    So $\chi_{\Hom_k(V,W)}(g)=\sum_{i,j}\mu_i\lambda_j^{-1}=\overline{\chi_V(g)}\chi_W(g)$.
\end{proof}
\begin{lem}
    Let $V,W$ be reps of a finite group $G$. Recall that $V^G=\{v\in V:\forall g\in G,\ g\cdot v=v\}\le V$. Then
    \begin{itemize}
        \item $\dim V^G=|G|^{-1}\sum_{g\in G}\chi(g)=\langle1_G,\chi_V\rangle_G$.
        \item $\dim\Hom_G(V,W)=\langle\chi_V,\chi_W\rangle_G$.
    \end{itemize}
\end{lem}
\begin{proof}
    To prove the first part, we note that $\pi:U\to U^G$, $u\mapsto |G|^{-1}\sum_{g\in G} g\cdot u$ is a projection, so $\dim U^G=\operatorname{tr}(\pi)=|G|^{-1}\sum_{g\in G}\chi(g)=\langle 1_G,\chi_V\rangle_G$.

    To prove the second part we simply note that $\Hom_G(V,W)=\Hom_k(V,W)^G$.
\end{proof}
\subsection{Character Table}
In this section, we assume that the group $G$ is finite and that every representation is over $\CC$.
\begin{thm}[Orthogonality]
    If $V,V'$ are irreducible complex of a finite group $G$ with character $\chi,\chi'$, then
    \[\langle\chi,\chi'\rangle=\begin{cases}
        1 & V\cong V'\\
        0 & \text{o/w}
    \end{cases}\]
\end{thm}
\begin{proof}
    This is a combination of the preceding lemma and Schur's lemma. For $V,V'$ irreducible, $\langle \chi,\chi'\rangle_G=\dim\Hom_G(V,V')$ by the preceding lemma, and $\dim\Hom_G(V,V')=0,1$ by Schur's lemma. The dimension is $1$ precisely when there exists an $G$-equivariant iso $V\cong V'$.
\end{proof} 
\begin{thm}[The character table is a square.]
    There are as many irred. complex reps as conj. classes in $G$ (finite), i.e., $\{\chi_V: V\text{ irred.}\}$ is an orthonormal basis of $\mathcal{C}_G$.
\end{thm}
\begin{proof}
    It suffices to show that the set of irreducible characters span $\mathcal{C}_G$ as they clearly form an orthonormal set. Let $\{\chi_1,...,\chi_k\}$ be the complete list of irreducible characters. We can assume that this list is finite as orthogonality means that the cardinality is $\le$ the dimension of $\mathcal{C}_G$. Let $I=\langle\chi_1,...,\chi_k\rangle$. We will prove that $I^\perp=\{0\}$.

    We begin with a general construction. Let $f\in\mathcal{C}_G$. For every representation $(\rho,V)$, we define
    \[\varphi_{f,V}=|G|^{-1}\sum_{g\in G}\overline{f(g)}\rho(g)\in\Hom_k(V,V)\]
    Moreover, we see that $\forall h\in G$,
    \[\rho(h)\varphi_{f,V}\rho(h)^{-1}=|G|^{-1}\sum_{g\in G}\overline{f(g)}\rho(hgh^{-1})=\varphi_{f,V}\]
    Note that conjugating by an element is an automorphism of $G$, so we may change the summing index and use that fact that $f$ is a class function. Hence, $\varphi_{f,V}\in\Hom_G(V,V)$.

    Now we specialize to $f\in I^\perp$ and $V$ an irrep with character $\chi_i$. By Schur's lemma, $\varphi_{f,V}=\lambda\id_V$ for some $\lambda\in\CC$ and $\lambda\dim V=\operatorname{tr}(\varphi_{f,V})=\langle f,\chi_i\rangle_G=0$, so $\varphi_{f,V}=0$. By complete reducibility, every rep of $G$ breaks up into irred components, so $\varphi_{f,V}=0$ for all reps $V$. In particular, if take $V=\CC G$ to be the regular representation, then 
    \[\varphi_{f,\CC G}(e)=|G|^{-1}\sum_{g\in G}\overline{f(g)}g=0\]
    By linear independence, $\overline{f(g)}=0$ for all $g\in G$, so $f=0$.

    Hence, $\{\chi_1,...,\chi_k\}$ form an orthonormal basis for $\mathcal{C}_G$, so $\dim \mathcal{C}_G=k=$ the number of conjugacy classes of $G$.
\end{proof}
\begin{crly}
    If $\rho',\rho'$ are two complex reps of $G$ with the same character, then $\rho\cong\rho'$
\end{crly}

We introduce some notation. If $\rho$ is a rep of $G$, then write $n\rho$ for the $n$-fold direct sum of $\rho$, whose character is then denoted by $n\chi_\rho$. So if $\rho$ is isomorphic to a direct sum of reps, i.e., $\rho=n_1\rho_1\oplus...\oplus n_r\rho_r$, $\rho_i$ irred (complete reducibility) so that $\chi=n_1\chi+...+n_r\chi_r$. But $\langle\chi_i,\chi_j\rangle=\delta_{ij}$, so $\langle\chi_i,\chi_j\rangle=n_i$, so have an expansion in terms of the orthonormal basis $\{\chi_i\}$.

In particular, in any decomposition of a rep $V$ into direct summand of irred. reps, the multiplicity with which the irred. reps occur is $\langle\chi,\chi_i\rangle$, which is independent of the decomposition.
\subsubsection{Consequences}
\begin{prop}[Column orthogonality]
    Suppose $\chi_1,...,\chi_r$ is a complete list of irreducible of characters of $G$, then for each $g,h\in G$, we have
    \[\sum_{i=1}^r\overline{\chi_i(g)}\chi_i(h)=\begin{cases}
        C_G(g) & \text{ if }g,\ h\text{ are conjugate}\\
        0 & \text{o/w}
    \end{cases}\]
    In particular, $|G|=\sum_i(\dim V_i)^2$.
\end{prop}
\begin{proof}
    Let $D$ be the diagonal matrix such that $D_{ii}=1/|C_G(g_i)|$, where each $g_i$ represents a unique conjugacy class in $G$. Then by orthogonality of characters, we have
    \begin{align*}
        \delta_{ij}&=\langle\chi_i,\chi_j\rangle_G\\
        &=\sum_k\dfrac{1}{D_{kk}}\overline{\chi_i(g_k)}\chi_j(g_k)\\
        &=(\bar XD^{-1}X^T)_{ij}
    \end{align*}
    where $X_{pq}=\chi_p(g_q)$. Since $X$ is a square, the matrix
\end{proof}
We introduce the permutation representation of $G$.

Suppose $G$ acts on a finite set $X$, then we can define a rep by letting $G$ act on the vector space $\CC X$ formally spanned by the basis $\{e_x:x\in X\}$. The action is given by linearly extending the action $g\cdot e_x=e_{g\cdot x}$.
\begin{lem}
    If $\chi$ is the character of $\CC X$ then $\chi(g)=|X^g|=|\{x\in X:gx=x\}|$.
\end{lem}
\begin{proof}
    Since $X$ is finite, list all the elements $x_1,...,x_d$. Then w.r.t. the basis $\{e_{x_1},...,e_{x_d}\}$, the matrix of $g$ is given by a permutation matrix, i.e., the $(i,j)$-entry is $1$ if $gx_i=x_j$ and $0$ otherwise.
\end{proof}

\begin{prop}
    Let $V_1,...,V_r$ be the complete list of irreducible representations of a finite group $G$, then as a representation, we have $\CC G\cong\bigoplus_{i=1}^r(\dim V_i)V_i$, where $\CC G$ is the regular representation.
\end{prop}
\begin{proof}
    Note that $\chi_{\CC G}(g)=0$ if $g\neq e$, so by computing $\langle\chi_{\CC G},\chi_i\rangle_G=|G|^{-1}\overline{\chi_{\CC G}(e)}\chi_i(e)=\chi_i(e)=\dim V_i$.
\end{proof}
\begin{prop}[Burnside's lemma]
    Let $G$ be a finite group acting on a finite set $X$, then $$\langle 1_G,\chi_{\CC X}\rangle_G=\#\text{ of $G$-orbit on $X$}$$
\end{prop}
\begin{proof}
    If the action is not transitive, then we may write $X=\bigsqcup_iO_i$ as a disjoint union of orbits, then $\CC X$ breaks into subreps $\bigoplus_i \CC O_i$, then the dot product can be computed by linearity. Hence, we may wlog assume that $G$ acts transitively on $X$, then
    \begin{align*}
        |G|\langle1,\chi_{\CC X}\rangle_G&=\sum_{g\in G}\chi_{\CC X}(g)=\sum_{g\in G}|X^g|\\
        &=|\{(g,x)\in G\times X:gx=x\}|\\
        &=\sum_{x\in X}|\operatorname{Stab}_G(x)|\\
        &=|X|\dfrac{|G|}{|X|}\\
        &=|G|
    \end{align*}
\end{proof}

\begin{crly}
    If $G$ is a finite group acting on $X$ with permutation character $\chi$, then $\langle\chi,\chi\rangle_G$ is the number of $G$-orbit on the product action, where $G$ acts on $X\times X$ by $g\cdot (x,y)=(gx,gy)$.
\end{crly}
\begin{proof}
    Note that $g$ fixes $(x,y)$ iff $g$ fixes both $x$ and $y$, so $\chi_{X\times X}=\chi^2$, the result then follows from properties of inner product and Burnside's lemma.
\end{proof}
\begin{rem}
    Note that the diagonal $\Delta=\{(x,x)\in X\times X\}$ is $G$-stable, so is the off-diagonal $\{(x,y)\in X\times X:x\neq y\}$.
\end{rem}
\begin{defn}
    We say that $G$ acts on $X$ 2-transitively if its action on $X\times X$ has only two orbits. 
\end{defn}

\subsection{Character Ring}
\begin{defn}
    The character ring of a group $G$ is defined as
\[R(G)=\{\chi_1-\chi_2:\text{ $\chi_1$, $\chi_2$ are characters of $G$}\}\subseteq\mathcal{C}_G\]
\end{defn}
Note that $R(G)$ is an additive subgroup of $\mathcal{C}_G$ with addition induced by taking direct sum of two reps. The trivial rep acts as the multiplicative identity of the multiplicative structure given by taking tensor product of two reps.

Let $G,\ H$ be finite groups. The tensor product operation also gives a $\ZZ$-bilinear map
$R(G)\times R(H)\to R(G\times H), (\chi_V,\chi_W)\mapsto \chi_{V\otimes W}$.
\begin{prop}
    Let $(\rho_i,V_i)$, $(\psi_j,W_j)$ be complete lists of complex irreps of $G$ and $H$ respectively, then $\rho_i\otimes\psi_j$ are irreps of $G\times H$ as $i,j$ runs through all possible indices, and every complex irrep of $G\times H$ arrises this way.
\end{prop}
\begin{proof}
    By direct computation of characters, we see that $\{\rho_i\otimes\psi_j:i,j\}$ form an orthonormal set, and we have
    \[\sum_{i,j}\dim(V_i\otimes W_j)^2=\sum_i(\dim V_i)^2\sum_j(\dim W_j)^2=|G||H|=|G\times H|\]
    so the list is complete.
\end{proof}
We can also investigate the characters of symmetric/exterior powers. We can actually assemble the fomula into generating functions. (see ES3)

\section{Induction and Restriction}

\begin{defn}[Induced Rep]
Let $G$ be a finite group and $H\le G$ and $(\rho, W)$ a rep of $H$.
Define $\mathcal{F}(G,W)=\{f:G\to W\}$. This can be made a rep of $G\times H$ via the action $(k,h)\cdot f=(g\mapsto \rho(h)f(k^{-1}gh))$. This can be given the structure of a rep of $G$ by the standard embedding $g\mapsto (g,e_H)$ and as a rep of $H$ by $h\mapsto(e_G,h)$. These two actions commute. Moreover, the $H$-invariant subspace $\mathcal{F}(G,W)^H$ is also $G$-invariant.
    
    The induced representation is $\operatorname{Ind}_H^GW=\mathcal{F}(G,W)^H$, where $G$ acts as described above.
\end{defn}
In terms of module,
\begin{defn}
    Consider a $KH$-module $W$, where $H$ is a subgroup of $G$. We can define a $KG$-module by $K[G]\otimes_{K[H]}W$. [This is essentially extending coefficient]
\end{defn}
We have $\mathcal{F}(G,W)\cong K[G]\otimes_K W$ as $K$-vector spaces. [Note that $W$ is finite dimensional, so $W^\ast\cong W$ by sending basis to dual basis.] In this process, we pick a basis $\{w_1,...,w_n\}$ for $W$ so that $\{g\otimes w_i:g\in G, i=1,...,n\}$ is a basis for $K[G]\otimes_K W$. Each basis vector $g\otimes w_i$ is identified with the map which sends $g$ to $w_i$ and other elements to $0$. Under this identification the action of $G\times H$ becomes $(k,h)\cdot (g\otimes w_i)=kgh^{-1}\otimes\rho(h)w_i$. Given a general function $\sum_{g\in G} g\otimes w_g\in K[G]\otimes_KW$, the $H$-invariant condition is given by
\begin{align*}
    \sum_{g\in G}gh^{-1}\otimes \rho(h)w_g=\sum_{g\in G}g\otimes w_g
\end{align*}
Note that by renaming the index, the first sum can be written as $\sum_{g\in G}g\otimes \rho(h)w_{gh}$, so the invariance condition is
\[\rho(h)w_{gh}=w_g\] for all  $h\in H$.

Note that $K[G]$ can be regarded as a $K[H]$-module. Consider a subset $R=\{r_1,...,r_m\}\subseteq G$ of coset representatives. We claim that $R$ gives a $K[H]$-basis of $K[G]$. Clearly any $g$ belongs to a coset, so $R$ is a $K[H]$-generating set of $K[G]$. We need independence. Suppose there exists $\mu_r\in K[H]$ s.t.
\[0=\sum_{r\in R}\mu_rr\]
Then we can expand the sum to get
\[0=\sum_{r\in R}\sum_{h\in H}\lambda_{r,h}hr\]
where $\lambda_{r,h}\in K$. This forces $\lambda_{r,h}=0$ for all $r,h$ by $K$-linear independence.

The invariant condition can be transferred to give an isomorphism $K[G]\otimes_{K[H]}W\cong\mathcal{F}(G,W)^H$ of $K[G]$-modules. Define the following maps
\begin{align*}
    \phi:&K[G]\otimes_{K[H]}W\to \mathcal{F}(G,W)^H,\\ &\left(\sum_{r\in R}r\otimes w_r\right)\mapsto (f:g=rh\mapsto \rho(h^{-1})w_r)
\end{align*}
where $r\in R$ and $h\in H$ are the unique elements such that $g=rh$.
We can check that the image is indeed $H$-invariant. Let $\tilde h\in H$ and $g\in G$, $(\tilde h\cdot f)(g)=\rho(\tilde h)f(rh\tilde h)=\rho(\tilde h)\rho(\tilde h^{-1})\rho(h^{-1})w_r=f(g)$. 
This map is clearly injective. To see that $\phi$ is surjective, we observe that any $H$-invariant function $f:G\to W$ satisfies the relation $f(gh)=\rho(h^{-1})f(g)$ for all $g\in G$, $h\in H$, i.e., $f$ only depends on the action $\rho$ and the coset. If the action is given, then $f$ is uniquely determined by its values on coset representatives. We also need to check that $\phi$ preserves the $K[G]$-module structure. It is clearly additive. To see that it is $K[G]$-linear, note that for all $k\in G$, we have
\begin{align*}
    \phi\left(k\cdot \sum_{r\in R}r\otimes w_r\right)=(g\mapsto f(k^{-1}g))=k\cdot f
\end{align*}
where $f=\phi\left(\sum_rr\otimes w_r\right)$. So $\phi$ is a $k[G]$-module isomorphism.

Let $W_r\le \operatorname{Ind}_H^GW$ be the subspace of functions supported on $rH$ for each $r\in R$.
\begin{prop}
    Each vector space $W_r$ is isomorphic to $W$ and $\operatorname{Ind}_H^GW\cong\oplus_{r\in R}W_r$ as a vector space. In particular, $\dim \operatorname{Ind}_H^GW=|G:H|\dim W$
\end{prop}
\begin{proof}
    For each $r\in R$, define 
    \[w\in W\mapsto \sum_{h\in H}rh\otimes\rho(h^{-1})w\]
    Clearly RHS is identified with $H$-invariant functions supported on $rH$ bijectively. Also, every $H$-invariant function can be uniquely written as a $K$-linear combination of $H$-invariant functions supported on $rH$, $r\in R$, so $\operatorname{Ind}_H^GW\cong \bigoplus_{r\in R}W_r$ as vector space.
\end{proof}

\begin{prop}[Basic properties]
    \begin{enumerate}
        \item $\operatorname{Ind}(W_1\oplus W_2)\cong \operatorname{Ind}W_1\oplus\operatorname{Ind}W_2$
        \item $\dim\operatorname{Ind}W=|G:H|\dim W$
        \item $\operatorname{Ind}_{\{e\}}^G1$ is the left regular rep of $G$ over $K$.
        \item (Transitivity) If $P\le H\le G$, and $W$ is a rep of $P$, then $\operatorname{Ind}_H^G\operatorname{Ind}_P^HW=\operatorname{Ind}_P^GW$.
    \end{enumerate}
\end{prop}
\begin{proof}
    $1)$: Straightforward. $\operatorname{Ind}(W_1\oplus W_2)\cong K[G]\otimes_{K[H]}(W_1\oplus W_2)\cong (K[G]\otimes_{K[H]}W_1)\oplus (K[G]\otimes_{K[H]}W_2)$.

    $2):$ Preceding proposition.

    $3)$: We define a $K[G]$-module isomorphism $K[G]\otimes_{K\{e\}}K\cong K[G]$. Clearly, LHS is just $K[G]\otimes_K K$, and we have a vector space isomorphism $K[G]\otimes_KK\cong K[G]$ given by $K$-linearly extending $g\otimes 1_K\mapsto g$. It is easy to check that this map intertwines the induced action and the left regular action. [Just check every basis vector.]

    $4)$: See ES3 for the complex case. More generally, over a field $K$, an elements in $\operatorname{Ind}_H^G\operatorname{Ind}_P^HW$ is an element of $\mathcal{F}(G,\mathcal{F}(H,W)^P)^H$, we can identify this with an element $\phi\in\mathcal F(G\times H,W)$ s.t.
    \begin{itemize}
        \item $\phi(gh',h)=\phi(g,h'h)$ for all $h'\in H$. (Invariance under the action of $H$)
        \item $\phi(g,hp)=\rho(p^{-1})\phi(g,h)$ for all $p\in P$. (Invariance under the action of $P$)
    \end{itemize}
    The first relation says that $\phi$ is completely determined by its restriction to $G\times \{e\}$, which is a function $\psi:G\to W$ satisfying $\psi(gp)=\rho(p^{-1})\psi(g)$, i.e., $P$-invariant, i.e., an element in $\operatorname{Ind}_P^GW$. One can show that this is indeed a linear isomorphism, and hence an isomorphism of representations.
\end{proof}
\begin{defn}[Induced class function]
    Let $\varphi\in\mathcal{C}_H$ we define $\operatorname{Ind}_H^G\varphi\in\mathcal{C}_G$ by
    \[(\operatorname{Ind}_H^G\varphi)(g)=\sum_{r\in R}\varphi(r^{-1}gr)=|H|^{-1}\sum_{x\in G}\overset{\circ}{\varphi}(x^{-1}gx)\]
    where $R$ is a set of left coset representatives and $\overset{\circ}{\varphi}$ is the extension of $\varphi$ by zero outside of $H$.
\end{defn}
\begin{thm}
    Let $(\rho,W)$ be a rep of $H\le G$ with character $\chi_W$. We then have
    $\operatorname{Ind}_H^G(\chi_W)=\chi_{\operatorname{Ind}_H^GW}$.
\end{thm}

\begin{proof}
    Recall the vector space decomposition $\Ind_H^GW\cong\bigoplus_{r\in R}W_r$. It is clear from definition that the action of $G$ on $\Ind_H^GW$ permutes $W_r$ according to the action of $G$ on the left cosets $G/H$. [$W_r$ consists of functions supported on $rH$, then the left action of $G$ maps $W_r$ to $W_{gr}$, functions supported on $grH$] We need the trace of $g$. Note that the only contribution comes from cosets $rH$ s.t. $grH=rH$, i.e., $r^{-1}gr\in H$. We need to compute the trace of $g$ on $W_r$. Recall the vector space isomorphism $W\cong W_r$
    \[w\mapsto \sum_{h\in H}rh\otimes\rho(h^{-1})w\]
    We compute
    \begin{align*}
        g\cdot\left(\sum_{h\in H}rh\otimes\rho(h^{-1})w\right)&=\sum_{h\in H}grh\otimes \rho(h^{-1})w\\
        &=\sum_{h\in H}rh'h\otimes\rho(h^{-1})w\\
        &=\sum_{\tilde h\in H}r\tilde h\otimes\rho(\tilde h^{-1})\rho(h')w\\
        &=\sum_{h\in H}rh\otimes\rho(h^{-1})\rho(h')w
    \end{align*}
    The second equality follows from the fact that $r^{-1}gr=h'\in H$. So $g$ acts on $W_r$ the same way as $r^{-1}gr$ acts on $W$ under the isomorphism constructed before, so $\operatorname{tr}(g)=\chi_{W}(r^{-1}gr)$. Sum over all $r$.
\end{proof}

\begin{thm}[Frobenius Reciprocity]
    If $V$ is a rep of $G$ and $W$ is a rep of $H\le G$, then
    \[\Hom_G(V,\operatorname{Ind}_H^GW)\cong\Hom_H(\operatorname{Res}^G_HV,W)\]
    In the language of characters (if the reps are over $\CC$),
    \[\langle\chi_V,\Ind_H^G\chi_{W}\rangle_G=\langle\chi_{V}|_H,\chi_W\rangle_H\]
    i.e., $\operatorname{Ind}_H^G$ and $\operatorname{Res}^G_H$ are adjoints.
\end{thm}

\begin{proof}
  We will show that there is a natural isomorphism $\Hom_G(V,\Ind W)\cong\Hom_H(V,W)$.
  Define
  \begin{align*}
      &\Theta:\Hom_G(V,\mathcal{F}(G,W)^H)\to\Hom_H(V,W)\\
      &\Psi:\Hom_H(V,W)\to\Hom_G(V,\mathcal{F}(G,W)^H)
  \end{align*}
  by $\Theta(f)(v)=f(v)(e)$ and $\Psi(\beta)(v)(g)=\beta(g^{-1}v)$. These maps are clearly linear.
  By direct computation, $(\Psi\Theta f)(v)(g)=\Psi(f(-)(e))(v)(g)=f(g^{-1}v)(e)=f(v)(g)$ by $G$-equivariance, and $(\Theta\Psi\beta)(v)=\Theta(\beta((\bullet)^{-1}-))(v)=\beta(e^{-1}v)=\beta(v)$. We just verified that these are indeed isomorphisms of vector spaces.

  For reps over $\CC$, we can count the dimension and get $\dim\Hom_G(V,\Ind W)=\langle\chi_V,\Ind_H^G\chi_W\rangle_G=\langle\chi_V|_H,\chi_W\rangle_H=\dim\Hom_H(V,W)$.
\end{proof}
The second part of the proof (characters) can also be proved by directly checking the adjoint relation from the formula of induced characters.

\subsection{Mackey's Theory}
We want to study representations of the form $\Res^G_K\Ind_H^GW$ where $G$ is a finite group, $H,K\le G$ are subgroups, and $W$ is a rep of $H$.

There is an action of $K$ on $G/H$ given by $k\cdot gH=kgH$.
\begin{defn}
    The union of orbit of the action above, i.e., $\bigcup_{k\in K}kgH$ is called a double coset, denoted $KgH$. We also say $K\backslash G/H=\{KgH:g\in G\}$.
\end{defn}
An alternative description is the $K\times H$-orbit on $G$, for the left action of $K$ and the right action of $H$.

We introduce some notation: let $(\rho, W)$ be a rep of $H$.
\begin{itemize}
    \item $\leftindex^g{H}=gHg^{-1}\le G$
    \item $(\leftindex^gH,\leftindex^gW)$ is the representation of $\leftindex^gH$ on $W$ given by $\leftindex^g\rho(ghg^{-1})=\rho(h)$.
\end{itemize}
\begin{thm}[Mackey's restriction formula]
    If $G$ is a finite group with subgroups $H,K$, and $(\rho, W)$ is a rep of $H$, then
    \[\Res^G_K\Ind_H^G W\cong \bigoplus_{KgH\in K\backslash G/H}\Ind^K_{K\cap\leftindex^gH}\Res^{\leftindex^gH}_{K\cap\leftindex^gH}\leftindex^gW\]
    as reps.
\end{thm}
\begin{proof}
    For each double coset $KgH$, we define a space
    \[V_{KgH}=\{f\in\Ind_H^GW:\forall x\not\in KgH,\ f(x)=0\}\]
    i.e., the space of functions $G\to W$ with support contained in $KgH$. This is a $K$-invariant subspace of $\Ind_H^GW$. [The action of $K$ preserves this coset $KgH$.] Thus, there is a decomposition of rep of $K$ given by
    \[\Res^G_K\Ind_H^G W\cong\bigoplus_{KgH\in K\backslash G/H}V_{KgH}\]
    We define $$\Theta: V_{KgH}\to\mathcal{F}(K,\leftindex^gW)^{K\cap\leftindex^gH}=\Ind_{K\cap\leftindex^gH}^K\Res^{\leftindex^gH}_{K\cap\leftindex^gH}\leftindex^gW,\ f\mapsto(k\mapsto f(kg))$$
    \begin{itemize}
        \item We need to check that the image is indeed $K\cap\leftindex^gH$-invariant. Suppose $ghg^{-1}\in K$, then by direct computation
        \begin{align*}
            \Theta(f)(kghg^{-1})=f(kghg^{-1}g)=f(kgh)=\rho(h^{-1})\Theta(f)
        \end{align*}
        where we used $H$-invariance of $f$ in the last equality.
        \item We need to check that $\Theta$ is $K$-equivariant. Let $k'\in K$. We compute
        \begin{align*}
            (k'\cdot \Theta(f))(k)=f(k'^{-1}kg)=\Theta(k'\cdot f)(k)
        \end{align*}
        \item $\Theta$ is injective. Suppose $\Theta(f)=0$, i.e., $f(kg)=0$ for all $k\in K$. Then $f(kgh)=\rho(h^{-1})f(kg)=0$ for all $k\in K$ and all $h\in H$, so $f=0$ as we assumed that $f$ is supported in $KgH$.
        \item We compute the dimension of these two spaces. By $H$-invariance, each $f\in V_{KgH}$ is uniquely determined by its values on left $H$-coset representatives, i.e., on $\operatorname{Orb}_{K}(gH)$, so 
        \[\dim V_{KgH}=|\operatorname{Orb}_K(gH)|\dim W=\dfrac{|K|}{|\operatorname{Stab}_K(gH)|}\dim W=\dfrac{|K|}{|K\cap\leftindex^gW|}\dim W\]
        [Note that $k\in K$ fixes $gH$ iff $k=ghg^{-1}$ for some $h\in H$.]
        Also,
        \[\dim\Ind_{K\cap\leftindex^gH}^K\Res^{\leftindex^gH}_{K\cap\leftindex^gH}\leftindex^gW=\dfrac{|K|}{|K\cap\leftindex^gW|}\dim W\]
    \end{itemize} 
    Hence, $\Theta$ is an isomorphism of rep.
\end{proof}
\begin{crly}[Character version of Mackey's formula]
Suppose $\chi$ is the character of a rep $W$ of $H$, then 
    \[\Res^G_K\Ind_H^G\chi=\sum_{KgH\in K\backslash G/H}\Ind_{K\cap\leftindex^gH}^K\leftindex^g\chi\]
    where $\leftindex^g\chi(x)=\chi(g^{-1}xg)$ for all $x\in K\cap\leftindex^gH$.
\end{crly}
\begin{thm}[Mackey's irreducibility criterion]
    If $H\le G$ and $W$ is a \textbf{complex} rep of $H$, then $\Ind_H^GW$ is irreducible if and only if
    \begin{itemize}
        \item $W$ is irreducible and
        \item for each $g\in G\setminus H$, the two representations $\Res^{\leftindex^gH}_{H\cap\leftindex^gH}\leftindex^gW$ and $\Res^H_{H\cap\leftindex^gH}W$ have no irreducible factors in common, i.e., 
    \end{itemize}
\end{thm}
\begin{proof}
    Suppose $\chi$ is the character of $W$.
    \begin{align*}
        \langle\Ind_H^G\chi,\Ind_H^G\chi\rangle_G&=\langle\chi,\Res^G_H\Ind_H^G\chi\rangle_H\\
        &=\sum_{HgH\in H\backslash G/H}\langle\chi,\Ind_{H\cap\leftindex^gH}^H\Res^H_{H\cap\leftindex^gH}\chi\rangle_H\\
        &=\sum_{HgH\in H\backslash G/H}\langle\Res^H_{H\cap\leftindex^gH}\chi,\Res^H_{H\cap\leftindex^gH}\chi\rangle_{H\cap\leftindex^gH}
    \end{align*}
    by Frobenius reciprocity and Mackey's formula. $\Ind_H^GW$ is irreducible precisely when the sum above equals $1$. Observe that the term corresponding to $HeH$ is $\langle\chi,\chi\rangle_{H}\ge 1$, so irreducibility is equivalent to 
    \begin{itemize}
        \item $\langle\chi,\chi\rangle_H=1$, i.e., $W$ is irreducible and
        \item all other terms in the sum vanishes.
    \end{itemize}
\end{proof}
The result holds over any algebraically closed field (where character theory exists).
\begin{crly}
    If $H\trianglelefteq G$ and $W$ is an irrep of $H$, then $\Ind_H^GW$ is irreducible if and only if $\leftindex^g\chi_W\neq\chi_W$ for all $g\in G\setminus H$.
\end{crly}
\begin{proof}
    Since $H$ is normal, $\leftindex^gH=H$ for all $g\in G$, and $\leftindex^g W$ is irreducible as $W$ is. [We cam compute the character.] The second condition in Mackey's criterion is equivalent to $\leftindex^gW\not\cong W$, i.e., $\leftindex^g\chi_W\neq\chi_W$.
\end{proof}

\subsection{Frobenius Groups}
\begin{defn}[Frobenius group]
    A finite group $G$ is a Frobenius group if it has a transitive action on a set $X$ with $|X|>1$ such that $|X^g|\le 1$ for all $g\neq e$ and $\operatorname{Stab}_G(x)\neq\{e\}$ for some (hence all) $x\in X$.
\end{defn}
\begin{lem}
    A group $G$ is Frobenius if and only if $G$ has a proper subgroup $H$ s.t. $H\cap\leftindex^gH=\{e\}$ for all $g\in G\setminus H$.
\end{lem}
\begin{proof}
    Suppose $G$ is Frobenius, so $G$ acts on a set $X$ in a way described in the definition. Let $H=\operatorname{Stab}_G(x)$ for some $x\in X$, then $\leftindex^gH=\operatorname{Stab}_G(gx)$. Since $|X^g|\le 1$, we must have $|\leftindex^gH\cap H|\le 1$ for all $g\neq e$, so $\leftindex^gH\cap H=\{e\}$.

    Conversely, if such a proper subgroup $H$ exists, then consider the left coset action. It is transitive, and $\operatorname{Stab}_G(gH)=\leftindex^gH\neq \{e\}$ as $H$ is non-trivial. If there exists $g,r,r'\in G$ s.t. $grH=rH$ and $gr'H=r'H$, then $g\in\leftindex^{r'}H\cap\leftindex^rH$, so $g=e$. 
\end{proof}
\begin{thm}[Frobenius 1901]
    Let $G$ be a finite group acting transitively on a set $X$. If each $g\in G\setminus\{e\}$ fixes at most one element of $X$, i.e., $|X^g|\le 1$ for all $g\neq e$, then
    \[K=\{e\}\cup\{g\in G:X^g=\varnothing\}\]
    is a normal subgroup of $G$ with $|K|=|X|$
\end{thm}
It follows Frobenius groups cannot be simple. The normal subgroup $K$ is called the Frobenius kernel, and the group $H$ (mentioned in the lemma) is the Frobenius complement.
\begin{proof}
    
\end{proof}

\section{Arithmetic Properties of Characters}
\section{Topological Groups}
\end{document}