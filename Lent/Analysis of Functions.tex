\documentclass{article}
\usepackage{graphicx} % Required for inserting images
\usepackage[utf8]{inputenc}
\usepackage{amsmath,amsfonts,amssymb,amsthm}
\usepackage{enumerate,bbm}
\usepackage{leftindex}
\usepackage{tikz,tikz-cd,graphicx,color,mathrsfs,color,hyperref,boldline}
\usepackage{caption,float}
\usepackage[a4paper,margin=1in,footskip=0.25in]{geometry}
\newcommand{\e}{\varepsilon}
\usepackage{listings}
\usepackage{xcolor}
\usepackage{stmaryrd}
%\usepackage{tipa}
\usepackage{tabularx,capt-of}

\usepackage{blindtext}
%Image-related packages
\usepackage{graphicx}
\usepackage{subcaption}
\usepackage[export]{adjustbox}
\usepackage{lipsum}

%hyperref setup
\hypersetup{
    colorlinks=true,
    linkcolor=blue,
    filecolor=magenta,      
    urlcolor=cyan,
    pdftitle={Overleaf Example},
    pdfpagemode=FullScreen,
    }

%New colors defined below
\definecolor{codegreen}{rgb}{0,0.6,0}
\definecolor{codegray}{rgb}{0.5,0.5,0.5}
\definecolor{codepurple}{rgb}{0.58,0,0.82}
\definecolor{backcolour}{rgb}{0.95,0.95,0.92}

%Code listing style named "mystyle"
\lstdefinestyle{mystyle}{
  backgroundcolor=\color{backcolour}, commentstyle=\color{codegreen},
  keywordstyle=\color{magenta},
  numberstyle=\tiny\color{codegray},
  stringstyle=\color{codepurple},
  basicstyle=\ttfamily\footnotesize,
  breakatwhitespace=false,         
  breaklines=true,                 
  captionpos=b,                    
  keepspaces=true,                 
  numbers=left,                    
  numbersep=5pt,                  
  showspaces=false,                
  showstringspaces=false,
  showtabs=false,                  
  tabsize=2
}

%"mystyle" code listing set
\lstset{style=mystyle}

\theoremstyle{definition}
\newtheorem{defn}{Definition}[section]
\newtheorem{example}[defn]{Example}
\theoremstyle{remark}
\newtheorem{rem}{Remark}
\newtheorem{remS}[section]{defn}
\theoremstyle{plain}
\newtheorem{lem}[defn]{Lemma}
\newtheorem{thm}[defn]{Theorem}
\newtheorem{prop}[defn]{Proposition}
\newtheorem{fact}[defn]{Fact}
\newtheorem{crly}[defn]{Corollary}
\newtheorem{conj}[defn]{Conjecture}

%\newtheorem*{programming*}{Programming Task}

%\newtheorem{innercustomgeneric}{\customgenericname}
%\providecommand{\customgenericname}{}
%\newcommand{\newcustomtheorem}[2]{%
%  \newenvironment{#1}[1]
%  {%
%   \renewcommand\customgenericname{#2}%
%   \renewcommand\theinnercustomgeneric{##1}%
%   \innercustomgeneric
%  }
%  {\endinnercustomgeneric}
%}

%\newcustomtheorem{question}{Question}
%\newcustomtheorem{programming}{Programming Task}

\newcommand{\NN}{\mathbb{N}}
\newcommand{\ZZ}{\mathbb{Z}}
\newcommand{\QQ}{\mathbb{Q}}
\newcommand{\RR}{\mathbb{R}}
\newcommand{\CC}{\mathbb{C}}
\newcommand{\PP}{\mathbb{P}}
\newcommand{\FF}{\mathbb{F}}
\newcommand{\Hom}{\operatorname{Hom}}
\newcommand{\im}{\operatorname{im}}
\newcommand{\id}{\operatorname{id}}
\newcommand{\Ind}{\operatorname{Ind}}
\newcommand{\Res}{\operatorname{Res}}

\newcommand{\calD}{\mathcal{D}}

\newcommand{\sol}{\textit{Solution: }}

\title{Analysis of Functions}
\author{Kevin}
\date{January 2025}

\begin{document}
\maketitle
\section{Review of Basic Concepts}
q: (Owen's signature)\\
\subsection{Probmeas}
\begin{enumerate}
    \item The Lebesgue measure is inner regular, i.e., for all $A\in B(\RR^n)$, $\mu(A)=\sup\{\mu(K):K\subseteq A,\ K\text{ compact}\}$.
    \item Recall that $\mu$ extends to the $\mu$-completion of $\mathcal B$, which equals $M_\mu=\{B\cup A: B\in \mathcal B,A\in\mathcal N,\ \mu(A)=0\}$.
    \item For measurable functions $f:E\to F$, if $(F,\mathcal F)=(\RR,\mathcal B)$ (or $(\CC,\mathcal B)$), then we say that $f$ is Borel. This extends to maps taking values $\pm\infty$ if $f^{-1}({\pm\infty})\in\mathcal E$. If $f$ takes values in $[0,\infty]$, then we say $f\ge 0$ (non-negative).
    \item Recall MCT and DCT.
\end{enumerate}
\subsection{$L^p$-spaces and Approximation}
For $f:(E,\mathcal E,\mu)\to \RR$ (or $\CC$), define
\begin{align*}
    &\|f\|_{L^p}=\left(\int_E |f|^pd\mu\right)^{1/p},\ 1\le p<\infty\\
    &\|f\|_{L^\infty}=\operatorname{ess}\sup|f|=\inf\{\lambda\ge0:|f|\le \lambda\ \text{a.e.}\}
\end{align*}
We use $\|\cdot\|_\infty$ to denote the usual sup-norm. Define $L^p(E,\mu)=\{f:E\to\RR:\text{meas. }\|f\|_{L^p}<\infty\}$

Recall Riesz-Fischer Theorem. Also recall the spaces $C^k(\RR^n)$, the set of all functions on $\RR^n$ with continuous partial derivatives up to order $k$. We note that $C^\infty(\RR^n)=\bigcap_{k\ge 0}C^k(\RR^n)$. Note that this includes unbounded smooth functions. Use subscript $c$ to denote the linear subspaces consisting of compactly supported functions.

\begin{rem}
    $C_c^\infty(\RR^n)$ is non-empty, e.g.,
    \begin{align*}
        \psi(x)=\begin{cases}
            e^{\frac{1}{|x|^2-1}} & |x|<1\\
            0 & \text{o/w}
        \end{cases}
    \end{align*}
\end{rem}
\begin{thm}
    $C^\infty_c(\RR^n)$ is dense in $L^p(\RR^n,dx)$ for $1\le p<\infty$.
\end{thm}

\[\langle\text{-}:\tag{Owen's signature}\]

We admit the following lemma from PM.
\begin{lem}
    $C_c(\RR^n)$ is dense in $L^p$, $1\le p<\infty$.
\end{lem}
Recall convolution and basic properties including commutativity, associativity, and $\int_{\RR^n}f\ast gdx=\int_{\RR^n}f\int_{\RR^n}g$ (translation invariance and Fubini).

Recall multi-index notation $\alpha\in\ZZ_+^n$ is written as $\alpha=(\alpha_1,...,\alpha_n)$ with order $|\alpha|=\alpha_1+...+\alpha_n$ and we set $\alpha!=\alpha_1!\cdots\alpha_n!$, and for $x\in\RR^n$, we write $x^\alpha=x_1^{\alpha_1}\cdots x_n^{\alpha_n}$, so the partial differential operator becomes
\[D^\alpha=\dfrac{\partial^{|\alpha|}}{\partial x^\alpha}=\dfrac{\partial^{|\alpha|}}{\partial x_1^{\alpha_1}...\partial x_n^{\alpha_n}}\]
In particular $D_i=D^{(0,...,1,0,...,0)}$
\begin{thm}
    Let $f\in L^1_{\text{loc}}$ (i.e., $f1_K\in L^1$ for any $K\subseteq\RR^n$ compact), and $g\in C^k_c(\RR^n)$. Then $f\ast g\in C^k(\RR^n)$ and for all $0\le|\alpha|\le k$, we have
    \[D^\alpha(f\ast g)=f\ast(D^\alpha g)\]
\end{thm}
\begin{proof}
    Recall the translation operator $\tau_zh=h(\bullet-z)$, $z\in\RR^n$. Then for all $u\in\RR^n$,
    \[\tau_z(f\ast g)(x)=\int_{\RR^n}g(x-u-y)f(y)dy\]
    Since $g\in C_c(\RR^n)$ we have $|g(x-u-y)|\le\|g\|_\infty1_K$ for all $|u|\le 1$, where $K=K_{x,g}$ is a compact set, so tht $\|g\|_\infty1_K|f|$ gives an integrable upper bounde for the integrand. Since $g(x-u-y)\to g(x-y)$ as $u\to 0$, we have pointwise convergence. Apply DCT, we see that $f\ast g$ is cts.

    Now for $k=1$, we define difference operators $\forall e_i$ (standard basis vector) by $\Delta^i_hg(z)=\frac{g(z+he_i)-g(z)}{h}$ which converges to $D_ig(z)$. We can write
    \begin{align*}
        \Delta_h^i(f\ast g)(x)=\int_{\RR^n}\Delta_h^ig(x-y)f(y)dy
    \end{align*}
    Apply mean value inequality, get $|\Delta_h^ig(x-y)|\le\|D_ig\|_\infty1_K$. Apply DCT, $\Delta_h^i(f\ast g)\to f\ast(D_ig)$, which is continuous, so $f\ast g\in C^1$. Induction...
\end{proof}
\begin{prop}[Continuity of translation in $L^p$]
    Let $1\le p<\infty$. Then $\|\tau_zf-f\|_{L^p}\to 0$ as $z\to 0$ for all $f\in L^p$.
\end{prop}
\begin{proof}
    Hold for cts functions with compact support. Then apply $\varepsilon/3$-argument.
\end{proof}
\begin{thm}[Minkowski's inequality for integrals]
    Let $F:\RR^n\times\RR^n\to\RR$ be a measurable non-negative or $dx\otimes dx$-integrable function. Then
    \[\left\|\int_{\RR^n}F(x,\cdot)dx\right\|_{L^p}\le\int_{\RR^n}\|F(x,\cdot)\|_{L^p}dx\]
\end{thm}
\begin{proof}
    Example sheet.
\end{proof}
\begin{thm}[Mollification/Approximate identity]
    Let $\varphi\in C_c^\infty(\RR^n)$ be non-negative s.t. $\int_{\RR^n}\varphi(x)dx=1$. Define $\varphi_\varepsilon^{-n}\varphi(\cdot/\varepsilon)$, $\varepsilon>0$. Then for $1\le p<\infty$ and any $f\in L^p$,
    \[\|f-\varphi_\varepsilon\ast f\|_{L^p}\overset{\varepsilon\to0}{\longrightarrow} 0\]
\end{thm}
\[\circ-;\tag{Owen's signature}\]
\begin{proof}
    For $f\in L^p$, $x\in\RR^n$,
    \begin{align*}
        |\varphi_\varepsilon\ast f(x)-f(x)|&=\left|\int_{\RR^n}f(x-y)\varepsilon^{-n}\varphi(y/\varepsilon)dy-f(x)\right|\\
        &=\left|\int_{\RR^n}(f(x-\varepsilon u)\varphi(u)-f(x))du\right|\\
        &\le \int_{\RR^n}|f(x-\varepsilon)-f(x)|\varphi(u)du
    \end{align*}
    Apply Minkowski's inequality for integrals,
    \[\|\varphi_\varepsilon\ast f-f\|_{L^p}\le\int_{\RR^n}\|\tau_{\varepsilon u}f=f\|_{L^p}\varphi(u)du\]
    This converges to $0$ as $\varepsilon\to 0$ by DCT.
\end{proof}
In particular, since $C_c(\RR^n)$ is dense in $L^p$ and
$\{\varphi_\varepsilon\ast f:f\in C_c(\RR^n)\}\subseteq C_c^\infty(\RR^n)$, we have also proved that $C_c^\infty(\RR^n)$ is dense in $L^p$.

\subsection{Lebesgue's Differentiation Theorem}
\begin{defn}[Hardy-Littlewood maximal function]
    For $f\in L^1$, $x\in\RR^n$, let $$Mf(x)=\sup_{r>0}\dfrac{1}{|B_r(x)|}\int_{B_r(x)}f(y)dy$$
\end{defn}
\begin{lem}
    For $f\in L^1$, $Mf$ maps $\RR^n$ to $\RR$ and is Borel-measurable, and for all $\lambda>0$,
    \[|\{x:Mf(x)>\lambda\}|\le\dfrac{3^n}{\lambda}\|f\|_{L^1}\]
\end{lem}
\begin{proof}
    Define $A_\lambda=\{x:Mf(x)>\lambda\}$. %If $x\in A_\lambda$, then $\exists r_x>0$ s.t.
    %\[\dfrac{1}{|B_{r_x}(x)|}\int_{B_{r_x}(x)}|f(y)|dy>\lambda\]
    If $x_m\in A_\lambda^c$ s.t. $x_m\to x\in\RR^n$. Then
    \[\dfrac{1}{|B_{r_x}(x_m)|}\int_{\RR^n}1_{B_{r_x}(x_m)}|f(y)|dy\le\lambda\]
    by definition of $A_\lambda^c$. Apply DCT, get a contradiction, so $A^c_\lambda$ is closed, so $A_\lambda$ is open. This gives measurability.

    To prove the inequality, we use the inner regularity of \(\mu\) and take an arbitrary compact subset $K\subseteq A_\lambda$. $K$ has an open cover $\{B_{r_x}(x):x\in A_\lambda\}$. Pass to a finite subcover $B_1,...,B_N$ of such balls. By Wiener's covering lemma (ES1), reduce to a subcollection osf disjoint balls $B_1,...,B_k$ s.t. $$|K|\le 3^n\sum_{i=1}^k|B_i|=\dfrac{3^n}{\lambda}\sum_{i=1}^k\lambda|B_i|\le \dfrac{3^n}{\lambda}\sum_{i=1}^k\int_{B_i}|f(y)|dy\le \dfrac{3^n}{\lambda}\|f\|_{L^1}$$
    By inner regularity, $|A_\lambda|\le\sup\{|K|:K\subseteq A_\lambda\text{ cpt}\}\le\frac{3^n}{\lambda}\|f\|_{L^1}$.
\end{proof}
\begin{thm}
    Let $f\in L^1(\RR^n)$, $B_r(x)$ ball centered at $x$ with radius $r$. Then
    \[\lim_{r\to 0}\dfrac{1}{|B_r(x)|}\int_{B_r(x)}|f(y)-f(x)|dy=0,\ a.e.\tag{$\dagger$}\]
\end{thm}
\begin{rem}
    The set of points $A=\{x\in\RR^n:(\dagger)\}$ are called Lebesgue points of $f$.
\end{rem}
\[\langle(\rangle-:\tag{Owen's Signature}\]
\begin{proof}
    Consider
    \[\bar A_\lambda=\left\{x:\lim_{r\to 0}|B_r(x)^{-1}|\int_{B_r(x)}|f(y)-f(x)|dy>2\lambda\right\}\]
    Let $\varepsilon>0$. Pick $g\in C_c(\RR^n)$ s.t. $\|f-g\|_{L^1}<\varepsilon$. Then
    \[\dfrac{1}{|B_r(x)|}\int_{B_r(x)}|f(y)-f(x)|dy\le \dfrac{1}{|B_r(x)|}\int_{B_r(x)}|f(y)-g(y)|dy+\dfrac{1}{|B_r(x)|}\int_{B_r(x)}|g(y)-g(x)|dy+|f(x)-g(x)|\]
    $g$ is unif. cts, so the second term is small. If $x\in\bar A_\lambda$, either the first term or the third term is $>\lambda$. The third term is bounded using Markov's inequality
    \[\{x:|f(x)-g(x)|>\lambda\}\le\dfrac{\|f-g\|_{L^1}}{\lambda}<\varepsilon/\lambda\]
    For the first term, use HL-maximal inequality,
    \[|\{x:\text{first term}>\lambda\}|\le|\{x:M(f-g)(x)>\lambda\}|\le\dfrac{3^n}{\lambda}\|f-g\|_{L^1}\le 3^n\varepsilon/\lambda\]
    Therefore $|\bar A_\lambda|\le C\varepsilon$.
    So $|A^c|\le|\bigcup_n\bar A_{1/n}|\le\sum_n|\bar A_{1/n}|=0$.
\end{proof}
\begin{rem}
    In particular, for $f\in L^1(\RR), \lim_{h\to 0}\int_x^{x+h}f(y)dy=f(x)$ a.e.
\end{rem}
\begin{thm}[Egorov]
    Let $E\in B(\RR^n)$, $|E|<\infty$. Suppose $f_j:E\to\RR$ measurable s.t. $f_j\to f$ a.e. on $E$. Then
    \[\forall\varepsilon>0,\ \exists A_\varepsilon\ \text{s.t.}\ |E\setminus A_\varepsilon|<\varepsilon \text{ and } f_j\overset{\text{unif}}{\to} f\text{ on }A_\varepsilon\]
\end{thm}
\begin{proof}
    By discarding a null set, we may assume that $f_j\to f$ pointwise on $E$. Define
    \[E_k^m=\{x:\forall j>k,\ |f_j(x)-f(x)|<1/m\}\]
    $E_k^m$ is increasing as $k\to\infty$, and $\bigcup_k E_k^m=E$ by pointwise convergence.
    Pick a subsequence $k_m$ s.t. $|E\setminus E_{k_m}^m|\le\varepsilon 2^{-m}$. Define $A_\varepsilon=\bigcap_m E_{k_m}^m$. For all $x\in A_\varepsilon$, $|f_j(x)-f(x)|<1/m$ whenever $j>k_m$, so the convergence is uniform on $A_\varepsilon$, and
    \[|E\setminus A_\varepsilon|\le\sum_m|E\setminus E_{k_m}^m|\le\varepsilon\]
\end{proof}
\begin{thm}[Lusin]
    Let $|E|<\infty$, $f:E\to\RR$ (or $\CC$) Borel-measurable. Then
    \[\forall\varepsilon>0,\ \exists F_\varepsilon\text{ s.t. }|E\setminus F_\varepsilon|<\varepsilon\text{ and } f:F_\varepsilon\to\RR \text{ cts}\]
\end{thm}
\begin{rem}
    Note that $f$ is not necessarily continuous $F_\varepsilon$ when regarded as a map defined on $E$.
\end{rem}
\begin{proof}
    First prove it for simple functions $f=\sum_{i=1}^ma_i1_{A_i}$ (wlog assume $A_i$ disjoint), where $\bigcup A_i=E$. Use inner regularity to find compact sets $K_k\subseteq A_k$ s.t. $|A_k\setminus K_k|<\varepsilon/m$. $f$ is cts on $\bigcup_k K_k$ and $|E\setminus\bigcup K_k|\le\varepsilon$. For general $f$, approximate $f$ ptwise by simple functions on $E$. Pick $A_\varepsilon$ s.t. $|E\setminus A_\varepsilon|<\varepsilon/2$ s.t. $f_m\to f$ unif. by Egorov. Take $C_m$ compact s.t. $|E\setminus C_m|<\varepsilon 2^{-m-1}$. Then Take $F_\varepsilon=A_\varepsilon\cap\bigcap_mC_m$. Can check that $|E\setminus F_\varepsilon|\le\varepsilon$.
\end{proof}
Recall Riesz representation theorem in Hilbert spaces (bounded linear functionals can be written as taking inner product with a certain element).

Consider two measures $\mu,\nu$ on a measurable space $(E,\mathcal E)$. 
\begin{defn}
    We say that $\nu$ is absolutely continuous w.r.t $\mu$ if $\mu(A)=0\implies\nu(A)=0$ for any $A\in\mathcal E$. We write $\nu\ll\mu$. If $\nu\ll\mu$ and $\mu\ll\nu$ both hold, then we say that $\mu,\nu$ are mutually absolutely continuous.

    If there exists $B\in\mathcal E$ s.t. $0=\mu(B)=\nu(B^c)$, then we say that $\mu,\nu$ are mutually singular, and we write $\mu\perp\nu$.
\end{defn}

\begin{thm}[Radon-Nikodym]
    Let $\mu,\nu$ be finite measures on $(E,\mathcal E)$ s.t. $\nu\ll\mu$. Then $\exists w\in L^1(\mu)$, $w\ge 0$, s.t. for all $A\in\mathcal E$, $\nu(A)=\int_A d\nu=\int_Awd\mu$.
\end{thm}
\begin{rem}\
    \begin{enumerate}
        \item[1)] $w$ is unique.
        \item[2)] Will show that $\int_E hd\nu=\int_E hwd\mu$ for all $h\ge 0$ measurable. In particular, $w=d\nu/d\mu$ (Leibniz notation) is called the Radon-Nikodym derivative (or density) of $\nu$ w.r.t. $\mu$.
        \item[3)] The result extends to $\mu,\nu$ $\sigma$-finite.
    \end{enumerate}
\end{rem}
\begin{proof}[Proof (von Neumann)]
    Define $\alpha=\mu+2\nu$ and $\beta=2\mu+\nu$. On $L^2(\alpha)$, consider the map $\Lambda(f)=\int_Efd\beta$. This is bounded since \[|\Lambda(f)|\le\int_E|f|d\beta\le2\int_E|f|d\alpha\le 2\sqrt{\alpha(E)}\|f\|_{L^2(\alpha)}\]
    By Riesz, there exists $g\in L^2(\alpha)$ s.t. $\Lambda(f)=\int_Egfd\alpha$, i.e., $\int f(2d\mu+d\nu)=\int gf(d\mu+2d\nu)$ for all $f\in L^2(\alpha)$. Rearrange,
    \[\int_Ef(2-g)d\mu=\int_Ef(2g-1)d\nu\]
    Consider $A_j=\{x:g(x)\le\frac{1}{2}-\frac{1}{j}\}$, $j\in\NN$. Thus, by taking $f=1_{A_j}$,
    \[\dfrac32\mu(A_j)\le\int_E f(2g-1)d\mu\le-\dfrac2j\nu(A_j) \]
    So, $g\ge 1/2$ a.e. (w.r.t. both $\mu,\nu$). Similarly, by considering $\{x:g(x)\ge 2+1/j\}$, can prove that $g\le 2$ $\mu$-, $\nu$- a.e. We extend to simple functions and then to non-negative measurable functions by MCT.

    Consider $f=1_{\{x:g(x)=1/2\}}$, then get $\frac{3}{2}\mu(\{x:g(x)=1/2\})=0$, so $\nu(\{x:g(x)=1/2\})=0$.

    Let $h\ge 0$ measurable and define $f=\frac{h}{2g-1}$ and $w=\frac{2-g}{2g-1}$ (define it to be $0$ if $2g-1=0$). Now
    \[\int_E hd\nu=\int_Ef(2g-1)d\nu=\int_Ef(2-g)d\mu=\int_E hwd\mu\]
    Done by taking $h=1_A$. Note that $w\in L^1(\mu)$ since $\int_E wd\mu=\int_E 1_E wd\mu=\nu(E)<\infty$.
\end{proof}

\begin{rem}\
    \begin{enumerate}[1)]
        \item If $\PP$ is a prob measure on $B$ s.t. $\PP\ll dx$, where $dx$ is the Lebesgue measure, then $\frac{d\PP}{dx}=p(x)$ is the Lebesuge prob density of $\PP$. Moreover, there exists a unique decomposition $\PP=\PP_{\ll}+\PP_\perp$. (ESheet)
    \end{enumerate}
\end{rem}

\section{Dual Spaces}

\begin{defn}
    Let $X$ be a topological vector space. The top dual space is \[X'=\{\Lambda:X\to\RR \text{ linear and cts}\}\]
    If $(X,\|\cdot \|)$ is a normed space, then cts$\Leftrightarrow$bounded, and $X'$ has the operator norm. $X''=(X')'$ is called the bidual space.
\end{defn}
We have a point evaluation map $\Lambda\mapsto \Lambda(x)$ for each $x\in X$. [Note that $|\Lambda(x)|\le \|\Lambda\|_{X'}\|x\|_X\le C\|\lambda\|_{X'}$.] Can identify $x\mapsto (\Lambda\mapsto \Lambda(x))$ and regard $X$ as a subspace of $X''$.
\begin{defn}
    If $X''=X$, then say $X$ is reflexive.
\end{defn}
For $1<p<\infty$, consider the H\"older conjugate $q$ of $p$. Each $g\in L^q$ defines a linear functional on $L^p$ by
$\Lambda_g(f)=\int fgdx$. This is bounded since $|\Lambda_g(f)|\le\|g\|_{L^q}\|f\|_{L^p}$. In fact $\|\Lambda_g\|=\|g\|_{L^q}$. Get an embedding $L^q\subseteq (L^p)'$.
\begin{thm}
    For $1\le p<\infty$, $(L^p)'=L^q$. For $1<p<\infty$, $L^p$ is reflexive.
\end{thm}
\begin{rem}
    The result is false for $p=\infty$, so $L^1$ is not reflexive.
\end{rem}
\begin{lem}
    Under the hypothesis of the theorem, let $U\in (L^p)'$ positive, then $\exists g\in L^q$ s.t. $U(f)=\Lambda_g$ and $\|U\|=\|g\|_{L^q}$.
\end{lem}
\begin{proof}[Proof of lemma]
    On $\RR^n$ consider the finite measure $\mu$ with density $e^{-|x|^2}$. Further define, for each $A\in\mathcal B$, the set function
    \[\nu(A)= U(e^{-|x|^2/p}1_A)\ge 0\]
    To show that $\nu$ is countably additive, consider $A_m\in\mathcal B$ s.t. $A_m\downarrow\bigcap _mA_m=\varnothing$ and note $\nu(A_m)= U(e^{-|x|^2/p}1_{A_m})\le\|U\|\|\|e^{-|x|^2/p}1_{A_m}\|_{L^p}\to 0$ by DCT. Hence $\nu$ is a finite measure. Note that $\nu\ll \mu$. [If $\mu(A)=0$, then $\nu(A)\le\|U\|\|\|e^{-|x|^2/p}\|_{L^p}=\|U\|\mu(A)^{1/p}=0$.] By Radon-Nikodym, $\exists\mathcal G\in L^1(\mu)$ non-negative s.t. $\nu(A)=\int_A\mathcal G d\mu$. Consider a simple function $F=\sum_k a_k1_{A_k}$. Compute $$U(e^{-|x|^2/p}F)=\int\sum_k a_k1_{A_k}\mathcal G e^{-|x|^2}dx=\int e^{-|x|^2/p}\sum_{k}a_k1_{A_k}\mathcal G e^{-|x|^2/q}dx$$
    Set $g=\mathcal G e^{-|x|^2/q}$. Note that $\{e^{-|x|^2/p}F:F\text{ simple}\}$ is dense in $L^p\cap \{\ge 0\}$. Since $fg\in L^1$ [Note $\int |fg|=\int|f|g=U(|f|)\le \|U\|\|f\|_{L^p}<\infty$]. Decomposing $fg$ into $f_+g-f_-g$ and taking limits, we see that $U(f)=\int fg$ for all $f\in L^p$.

    (cf. ES1), have $\|g\|_{L^q}=\sup\{\int |fg|:\|f\|_{L^p}\le 1\}= U(|f|)\le\|U\|<\infty$ and $\|U\|=\sup_{\|f\|_{L^p}\le 1}|\int fg|\le\|g\|_{L^q}$ by H\"older, so $\|U\|=\|g\|_{L^q}$.
\end{proof}
\begin{proof}[Proof of thm]
    Note (ES) that $\Lambda\in (L^p)'$ can be uniquely decomposed as $\Lambda_+-\Lambda_-$, where $\Lambda_\pm$ are positive linear functionals ($\Lambda_\pm(h)\ge 0$ for all $h\ge 0$ a.e.). Apply the preceding lemma.
\end{proof}

We can characterize duals of subspace of $L^\infty$, e.g., any finite measure defines a linear functional in $C_c(\RR^n)'$ by $f\mapsto \mu(f)$.

\begin{defn}
    A measure is regular on $\RR^n$ if $\forall\varepsilon>0,\forall A\in\mathcal B,\exists C$ closed, $D$ open s.t. $C\subseteq A\subseteq D$ s.t. $\mu(D\setminus C)<\varepsilon$
\end{defn}
\begin{thm}[Riesz]
    Let $\Lambda\in (C_c(\RR^n))'$ be positive. Then $\exists$ a $\sigma$-algebra $M\supseteq\mathcal B$ and a regular measure $\mu$ on $M$ s.t. $\Lambda(f)=\int_{\RR^n}fd\mu$.
\end{thm}
Proof omitted.
\[(B\tag{Owen's Signature}\]

\section{Weak and Weak\(*\) Topology}
\begin{defn}
    A semi-norm $p$ on a vector space $X$ is a functional $p:X\to[0,\infty)$ s.t.\begin{enumerate}[1)]
        \item $\forall x,y\in X, \ p(x+y)\le p(x)+p(y)$;
        \item $\forall x\in X,\forall \lambda\in \RR \text{ (or $\CC$)},\ p(\lambda x)=|\lambda|p(x)$
    \end{enumerate}
\end{defn}
The collection $\mathcal P$ of seminorms introduces a `locally convex' topology $\tau_{\mathcal P}$ generated by
$$V_x(p,n)=\{y\in X: p(y-x)<1/n\}$$
for $x\in X$, $p\in \mathcal P$, $n\in\NN$.
\begin{defn}
    The family $\mathcal P$ is said to separate points if for any $0\neq x\in X$, there exists $p\in \mathcal P$ s.t. $p(x)\neq 0$.
\end{defn}
Therefore (ES) a sequence $x_n$ converges in $\tau_\mathcal P$ iff for all $p\in \mathcal P$, $p(x_n-x)\to 0$ as $n\to\infty$. This topology is not generally metrizable unless $\mathcal P$ is countable. In that case a metric is given by
\[d_{\mathcal P}(x,y)=\sum_{i=1}^\infty\dfrac{p_i(x-y)}{2^{-i}(1+p_i(x-y))}\]
\begin{defn}
    We say that $(X,\tau_{\mathcal P})$ is a locally convex topological vector space (LCTVS). If it's complete, then we call it a Frechet space.
\end{defn}
Consider the semi-norms given by $p_\Lambda(x)=|\Lambda(x)|$.
\begin{defn}
    The topology $\tau_{\mathcal P}$ induced by $\mathcal P=\{p_\Lambda:\Lambda\in X'\}$ is called the weak topology $\tau_w$. We say that $x_n\to x$ weakly in $X$ or $x_n\rightharpoonup x$ if $\Lambda(x_n)\to \Lambda(x)$ for all $\Lambda\in X'$.
\end{defn}
\begin{defn}
    On the dual space $X'$, we can consider the weak-$\ast$ topology $\tau_{w^\ast}$ induced by $\mathcal P=\{p_x(\Lambda)=|\Lambda(x)|:x\in X\}$. Note that $\Lambda_n\to\Lambda$ weak-$\ast$, or $\Lambda_n\rightharpoonup^\ast\Lambda$ if $\Lambda_n(x)\to\Lambda(x)$ for all $x\in X$.
\end{defn}
\begin{example}\
    \begin{enumerate}[1)]
        \item Consider $L^p(\RR,dx)$. $f_n\to f$ weakly in $L^p$ iff \[\forall g\in L^q,\ \int_{\RR^n}f_ngdx\to\int_{\RR^n}fgdx\tag{$\dagger$}\]
        Since $L^p=(L^q)'$, $\Lambda_{f_n}\to\Lambda_{f}$ weak-$\ast$ (or $f_n\to f$ weak-$\ast$) iff ($\dagger$) holds. For $1<p<\infty$, weak convergence and weak-$\ast$ convergence coincide. (This is true in any reflexive space.)
        \item Consider a probmeas on a metric space $D$ (with Borel $\sigma$-algebra). Let $C_b(D)$ denote the Banach space of bounded cts functions on $D$. Then $\mu(f)$ defines an element $C_b(D)'$. A sequence of probmeas $\mu_n$ converges to $\mu$ in $\tau_{w^\ast}$ if $\mu_n(f)\to \mu(f)$ for all $f\in C_b(D)$. (i.e. weak convergence of laws)
    \end{enumerate}
\end{example}
%When is someone gonna tell him that 'and' has an 'n' in it?
Recall Arzela-Ascoli. A sufficient condition for equicontinuity is given by H\"older continuity, defined as
\[\|f\|_{C^{0,\gamma}}=\|f\|_\infty+\sup_{x\neq y}\dfrac{|f(x)-f(y)|}{|x-y|^\gamma}\]
where $0<\gamma<1$ and 
\[\|f\|_{C^{m,\gamma}}=\sum_{0\le|\alpha|\le m}\|D^\alpha f\|_\infty+\max_{|\alpha|=m}\sup_{x\neq y}\dfrac{|D^\alpha f(x)-D^\alpha f(y)|}{|x-y|^\gamma}\]
So $\{f:\|f\|_{C^{0,\gamma}}\le 1\}$ is compact in $C([0,1])$ by Arzela-Ascoli.

\begin{thm}[Banach-Alaoglu]
    Let $X$ be a normed space. The unit ball $B_1=\{\Lambda\in X':\|\Lambda\|_{X'}\le 1\}$ of $X'$ is compact in weak-$\ast$ topology
\end{thm}
\begin{rem}
    In $(C_b(D))'$ any sequence of probmeas has a weak-$\ast$ convergent subsequence.
\end{rem}
We will prove Banach Alaoglu for $X$ separable.
\[\ddot{T}\tag{Owen's Signature}\]
\begin{lem}
    For a countable dense subset $D=\{x_1,...,x_n\}$ of $X$, consider seminorms $\tilde{\mathcal{P}}=\{p_{x_k}(\Lambda)=|\Lambda(x_k)|:k\in\NN\}$ with induced topology $\tau_{\tilde{\mathcal P}}$. Then $\tau_{w^*}=\tau_{\tilde{\mathcal P}}$ coincide as topologies on $B_1'=\{\Lambda\in X':\|\Lambda\|_{X'}\le 1\}$ and are metrized by
    \[d_{\tilde{\mathcal P}(\Lambda,\Lambda')}=\sum_{k=1}^\infty \dfrac{|\Lambda(x_k)-\Lambda'(x_k)|}{2^k(1+|\Lambda(x_k)+\Lambda'(x_k)|)}\]
\end{lem}
\begin{proof}
    The open sets for $\tau_{\tilde{\mathcal P}}$ are generated by $V(x_k,m)=\{\Lambda:|\Lambda(x_k)|<1/m\}$. To prove that the two topologies are equivalent, it suffices to show that $V(x,n)$ contain some $V(x_k,m)$ for all $x\in X,n\in\NN$. Suppose $x\in X\setminus D$ and pick $x_k\in D$ s.t. $\|x-x_k\|<\varepsilon$. For $\Lambda\in V(x_k,m)$ we have $|\Lambda(x)|\le|\Lambda(x_k-x)|+|\Lambda(x_k)|\le\|\Lambda\|_{X'}\varepsilon+1/m<n$ when $\varepsilon$ is sufficiently small and $m$ sufficiently large, so $V(x_k,m)\subseteq V(x,n)$.

    If $\Lambda_j(x_k)\overset{j\to\infty}{\longrightarrow}\Lambda(x_k)$ for all $k$ then $d_{\tilde{\mathcal P}}(\Lambda_j,\Lambda)\to 0$ by DCT applied to counting measure on $\NN$.
\end{proof}
\begin{thm}
    Let $\Lambda_j\in B_1'$. Then $\exists\Lambda\in B_1'$ s.t. $\Lambda_{j_k}\to\Lambda$ weak-$\ast$.
\end{thm}
\begin{proof}
    Let $D$ be a countable dense subset of $X$. Since $|\Lambda_j(x_k)|\le\|x_k\|<\infty$. Diagonalization argument. Find convergent subsequences $\Lambda_{i,j}$ Find a $\Lambda$ as the limit of $\Lambda_{j,j}$. Need to show linearity and continuity. Note that $\Lambda$ is unif cts on $D$. [If $x,y\in D$ with $\|x-y\|<\varepsilon/2$, then for all $j$ sufficiently large, $|\Lambda_{j,j}(x)-\Lambda(x)|,|\Lambda(y)-\Lambda_{j,j}(y)|<\varepsilon/4$] Apply triangle inequality to $|\Lambda(x)-\Lambda(y)|$. Note that $\Lambda_{j,j}$ is uniformly Lipschitz.
    By uniform continuity, we can extend $\Lambda$ to a unif cts function on $X$.

    To show linearity, let $x,y\in X$, $z=x+ay$ for $a\in\RR$ (or $\CC$) and pick $x',y',z'\in D$ s.t. $\|x-x'\|+|a|\|y-y;\|+\|z-z'\|<\delta$.

    Apply a big triangle inequality.
    \begin{align*}
        |\Lambda(z)-\Lambda(x)-a\Lambda(y)|&\le|\Lambda(z)-\Lambda(z')|+|\Lambda(x)-\Lambda(x')|+|a||\Lambda(y)-\Lambda(y')|\\
        &\le +|\Lambda(z')-\Lambda_{j,j}(z')|+\ ..................................|
    \end{align*}
    each term is small either by unif continuity or unif convergence or linearity of $\Lambda$ on $D$.

    Need to show that $\|\Lambda\|\le 1$ ($|\Lambda(x)|\le|\Lambda(x-x')|+|\Lambda(x')|$) and that the convergence holds on $X$.
\end{proof}
\section{The Hahn-Banach theorem and its consequences}
\begin{defn}
    A functional $p:X\to\RR$ on a real vector space is called sub-linear if \begin{enumerate}[(i)]
        \item $p(x+y)\le p(x)+p(y)$ for all $x,y\in X$
        \item $p(tx)=tp(x)$ for all $t\ge 0$, $x\in X$.
    \end{enumerate}
\end{defn}

\begin{lem}[Bounded extension]
    Let $X$ be a real vector space and $p:X\to\RR$ sub-linear Let $M\subsetneq X$ be a vector subspace, and for $x\in X\setminus M$ define $\tilde{M}=\operatorname{span}(M,x)=\{M+cx:c\in\RR\}$. If $l:M\to \RR$ is a linear form s.t. $l(x)\le p(x)$ for all $x\in M$, then there exists $\tilde l:\tilde M\to\RR$ linear s.t. $\tilde l_{M}=l$ and $\tilde l(x)\le p(x)$ for all $x\in\tilde M$.
\end{lem}
\begin{proof}
    Let $y_1,y_2\in M$. $l(y_1)+l(y_2)=l(y_1+y_2)\leq p(y_1+y_2)\le p(y_1-x)+p(y_2+x)$ for all $x\in X\setminus M$. Rearrange, $l(y_1)-p(y_1-x)\le l(y_2)-p(y_2+x)$. Take sup/inf, 
    \[\sup\{l(y)-p(y-x):y\in M\}\le a\le \inf\{p(y+x)-l(y):y\in M\}\tag{$\ast$}\] for some $a\in\RR$.
    If $z\in\tilde M$, then it has a unique decomposition $z=y+\lambda x$ for some $\lambda\in\RR$. Define $\tilde l(z)=\tilde l(y+\lambda x)=l(y)+\lambda a$. To see $\tilde l\le p$ on $\tilde M$, for $\lambda>0$, write $\tilde l (y+\lambda x)=\lambda (l(y/\lambda)+a)\overset{(\ast)}{\le}\lambda(l(\frac y\lambda)+p(\frac{y}{\lambda}+x)-l(\frac y\lambda))=p(y+\lambda x)$. For $\lambda<0$, let $\mu=-\lambda$ and $\tilde l(y+\lambda x)=\mu(l(\frac y\mu-a))\le \mu(l(\frac y\mu)-l(\frac y\mu)+p(\frac y\mu-x))$.
\end{proof}
To extend $l$ to all of $X$ (for $X$ separable), we can apply the extension lemma inductively to $M_n=\operatorname{span}(M;x_1,...,x_n)$, where $(x_n)_{n\in\NN}$ is a countable dense subset of $X$.

In general, consider $S=\{(N,\tilde l):M\subseteq N\subseteq X\text{ vec.sp. },\  \tilde l|_M=l,\ \tilde l\le p\text{ on }N\}$. Apply Zorn's lemma.

\begin{thm}[Hahn-Banach]
    Let $X$ be a real vector space and $p:X\to\RR$ a sublinear functional. For $M\subseteq X$ vec. subspace, let $l:M\to\RR$ be a linear functional s.t. $l(x)\le p(x)$ for all $x\in M$. Then there exists an extension $\tilde l:X\to\RR$ (linear) s.t. $\tilde l(x)\le p(x)$ for all $x\in X$.
\end{thm}
\begin{rem}
    Extensions need not be unique. If $X$ is non-separable, the result depends on Axiom of Choice.
\end{rem}

\begin{crly}[Norming Functional]
    Let $X$ be a normed linear space. For all $x\in X$, there exists a linear functional $\Lambda=\Lambda_x\in X'$ s.t. $\|\Lambda\|=1$ and $|\Lambda(x)|=\|x\|_X$. In particular, if $\Lambda(x-y)=0$ for all $\Lambda\in X'$, then $x=y$.
\end{crly}
\[\subset]-:\langle\tag{Owen's (infinitely handsome) signature}\]
\begin{proof}
    For $x\in X$ define the vector subspace $M=\{cx:c\in\RR\}$ and consider the linear functional $l(cx)=c\|x\|_X$, so $|l(y)|\le p(y)=\|y\|_X$. By Hahn-Banach, there exists $\Lambda=\Lambda_x:X\to\RR$ s.t. $|\Lambda_x(y)|\le\|y\|_X$, so $\Lambda\in X'$ and $\|\Lambda\|\le 1$. Note that $\|\Lambda\|_{X'}\ge \sup_{y\in M\cap B_X}|l(y)|\ge \|x\|_X$
\end{proof}
\begin{crly}
    The canonical injection of $i:X\hookrightarrow X''$ given by $x\mapsto(\Lambda\to\Lambda(x))$ is an isometric embedding.
\end{crly}
\begin{proof}
    Consider $\|i(x)\|_{X''}=\sup_{\|\Lambda\|\le 1}|\Lambda(x)|\le\|x\|_X$. By taking a norming functional, $\|i(x)\|_{X''}\ge\|\Lambda_x(x)\|=\|x\|$.
\end{proof}
If $X$ is reflexive, then $X$ is isometrically isomorphic to $(X')'$ which is complete. (i.e., reflexive normed lienar space is Banach.)
If $X$ is not reflexive, then $X''$ provides (up to iso) the completion of $X$ for $\|\cdot\|_X$.

In particular, if $X$ is reflexive ,then the weak topology coincides with the weak-$\ast$ topology on $(X')'$, so Banach-Alaoglu the unit ball $B_X$ is compact in $\tau_w$.
\begin{thm}[Hyperplane Separation]
    Let $A,B$ be non-empty disjoint convex sets in a Banach space $X$ over $\RR$.
    \begin{enumerate}[(i)]
        \item If $A$ is open, then $\exists\Lambda\in X'$ and $\gamma\in\RR$ s.t. $\Lambda(a)<\gamma\le \Lambda(b)$ for all $a\in A$ and $b\in B$.
        \item If $A$ is compact and $B$ is closed, then $\exists\Lambda\in X'$ and $\gamma_1,\gamma_2\in\RR$ s.t. $\Lambda(a)<\gamma_1<\gamma_2<\Lambda(b)$ for all $a\in A$, $b\in B$.
    \end{enumerate}
\end{thm}
\begin{proof}
    (i): Pick $a_0\in A$, $b_0\in B$ and let $x_0=b_0-a_0$. Define $C=A-B+x_0$. $C$ is convex, $0\in C$, $x_0\notin C$. $C$ is open. Consider the Minkowski functional defined as 
    \[p_C(x)=\inf\{t>0:x/t\in C\}\]
    Can show (ES) that $p_C$ is 
    \begin{itemize}
        \item sublinear on $X$
        \item there exists $k>0$ s.t. $p_C(x)\le k\|x\|$
        \item $p_C(x)<1$ for $x\in C$ and $p_C(x)\ge 1$ for $x\notin C$.
    \end{itemize}
     Take $$M=\{tx_0:t\in\RR\}$$ and consider the linear functional $l:M\to\RR,\ tx_0=t$. Then $l$ is dominated by $p_C$ since 
     \[l(tx_0)=t\le tp_C(x_0)=p(tx_0)=p_C(tx_0)\]
     for $t>0$ and $l(tx_0)=t\le 0\le p_C(tx_0)$. By Hahn-Banach, there exists $\Lambda:X\to\RR$ s.t. $-k\|x\|\le -p_C(x)\le \Lambda(x)\le p_C(x)\le k\|x\|$ for all $x\in X$, so $\Lambda\in X'$. Pick $a\in A$, $b\in B$. Note that \[\Lambda(a)-\Lambda(b)+\Lambda(x_0)=\Lambda(a-b+x_0)\le p_C(a-b+x_0)<1\]
     So $\Lambda(a)<\sup\Lambda(A)\le \Lambda(b)$.

     (2): $\Lambda(A)$ is compact in $\RR$ and $d=\|A-B\|_X$. Then consider $\tilde A_d=A+B_{d/2}$, where $B_{d/2}=\{y:\|y\|<d/2$ still disjoint from $B$. Apply (1).
\end{proof}
\[???\tag{Owen's missing Signature}\]
\section{Generalized Functions and Distributions}
Consider a topological vec. space $X\subseteq\bigcap_{q\ge 1}L^q(\Omega,dx)$, where $\Omega$ is an open subset of $\RR^n$. Suppose $X$ contains $C_c^\infty(\Omega)$. Let $f\in L^q$, then obtain a linear functional on $X$ given by $\Lambda_f(g)=\int_\Omega fgdx$, $g\in X$. If the embedding $X\hookrightarrow L^q$ is cts, then $\Lambda_f\in X'$. Note that $g=\phi_\epsilon$ is contained in $X$, mollification implies $\Lambda_f=0\implies f=0$ a.e.. So we can identify $\Lambda_f$ with $f$  and study the weak-$\ast$ topology of $X'$ on $L^p$.


Define seminorms on $C^\infty(\Omega)$, $p_N(\phi)=\max_{0\le |\alpha|\le n}\sup_{x\in K_N}|D^\alpha\phi(x)|$, where $K_i\subseteq K_{i+1}$ and $\bigcup _iK_i=\Omega$. We define the Frechet space $\mathcal E(\Omega)=(C^\infty(\Omega),\tau_{\mathcal P})$, where $\mathcal P=\{p_N:N\in\NN\}$. [Note that $\mathcal E(\Omega)$ may contain non-integrable functions.] 

\begin{thm}
    There exists a topology $\tau$ on $C^\infty_c(\Omega)$ s.t.
    \begin{enumerate}[(1)]
        \item vector space operations are cts
        \item a sequence $\phi_j\overset{j\to\infty}{\to} 0$ iff $\exists K\subseteq\Omega$ compact s.t. $\operatorname{supp}(\phi_j)\subseteq K$ for all $j$ and $D^\alpha\phi_j\to 0$ unif. on $K$ for all $0\le |\alpha|<\infty$.
        \item If $T:C^\infty_c(\Omega)\to\RR$ (or $\CC$) is linear, then it's cts iff $T(\phi_j)\to 0$ for all $\phi_j\to 0$ in $\tau$.
    \end{enumerate}
\end{thm}
proof omitted.

\begin{defn}
    We define $\calD=\calD(\Omega)=(C^\infty_c(\Omega),\tau)$, the space of test functions.
\end{defn}
For each $\phi\in C^\infty_c(\Omega)$, define $e^{-j}\phi(\cdot/j)$, then $e^{-j}\phi(\cdot/j)\to 0$ in $D$, but $j^{-2025}\phi(\cdot/j)$ does not converge to $0$ in $D$.

\begin{defn}
    Call $\phi\in C^\infty(\RR^n)$ rapidly decreasing if $\sup_{x\in\RR^n}(1+|x|)^N|D^\alpha\phi(x)|<\infty$ for all $0\le|\alpha|<\infty$ and all $N\in\NN$.
\end{defn}
[Note that $e^{-|x|^2}$ is rapidly decreasing but $(1+|x|)^{-2025}$ is not.]

Define seminorms $\tilde{\mathcal P}=\{\tilde p_N:N\in\NN\}$ with 
\[\tilde p_N=\max_{0\le|\alpha|\le N}\sup_{x\in \RR^n}(1+|x|)^N|D^{\alpha}\phi(x)|\]
Define Frechet space $\mathcal{S}(\RR^n)=(\{\phi\text{ rapidly decreasing}\},\tau_{\tilde{\mathcal P}})$.
This is metrizable since $\tilde{\mathcal P}$ is countable. This is called the Schwartz class.

Clearly $\calD(\Omega)\subsetneq \mathcal E(\Omega)$, $D(\RR^n)\subsetneq\mathcal S(\RR^n)\subsetneq\mathcal E(\RR^n)$ with continuous embedding (ES). We can now define $\calD'(\Omega)=\{T:\calD(\Omega)\to\RR\text{ (or $\CC$) linear and cts}\}$, the space of Schwartz distributions.
We also define $\mathcal E(\Omega)=\{T:\mathcal E(\Omega)\to \RR\text{ (or $\CC$) linear and cts}\}$ the space of compactly supported Schwartz distributions. Finally for $\Omega=\RR^n$ we define
$\mathcal S(\RR^n)=\{T:\mathcal S(\RR^n)\to\RR\text{ (or $\CC$) linear and cts}\}$ the space of tempered distributions. These spaces are equipped with their weak-$\ast$ topologies of pointwise convergence on $\cal D,\mathcal E,\mathcal S$ resp. Have cts embeddings
$\mathcal E'\subset\mathcal D'$, and $\mathcal E(\RR^n)\subset \mathcal S(\RR^n)\subset \calD'(\RR^n)$.

\begin{example}
    Consider $\delta_x(\phi)=\phi(x)$, then if $\phi_j\to0$ in $\cal D,\mathcal E,\mathcal S$ , then $\delta_x(\phi_j)=\phi_j(x)\to 0$ as $j\to\infty$, so $\delta_x\in\mathcal E',\mathcal D',\mathcal S'$.
\end{example}
\[O\frown O\tag{Owen's Signature}\]

Let $f\in L^1_{\text{loc}}(\Omega)$. Then $T_f(\phi)=\int_\Omega f\phi dx$, $\phi\in\calD(\Omega)$. Have $T_f\in\calD'(\Omega)$ since for $\phi_j\to 0$ in $\calD(\Omega)$ we have $T_f\phi_j\to0$ by DCT with dominating function $\sup_{j\in\NN}\|\phi_j\|_\infty1_K|f|\in L^1$ ($K$ compact). Also, $T_f=0$ in $\calD'(\Omega)$ still implies $f=0$ a.e. by applying the mollification theorem (ES) to $f1_{B(x)}$ where $B(x)$ is a ball in $\Omega$ containing $x\in\Omega$, so $L^1_{\text{loc}}\subseteq\calD'$

If $\varphi_\e=\e^{-n}\varphi(\cdot/\e)$, $\varphi\ge 0$, smooth, compactly supported, normalized, then for $g\in\calD(\RR^n)$, $T_{\varphi_\e}(g)=g\ast\varphi_\e(0)\to g(0)=\delta_0(g)$ as $\e\to0$. So $T_{\varphi_\e}\overset{\e\to0}{\to}\delta_0$ in $\calD'(\RR^n)$.
\subsection{Generalized (Distributional) Derivatives in $\calD'(\Omega)$)}
Let $f\in C^1(\Omega)$, then $D_if\in L^1_{\text{loc}}$. Consider $T_{D_if}$. Let $\varphi\in\calD(\Omega)$
\[\int (D_if)\varphi dx\overset{ibp}{=}-\int fD_i\varphi dx=-T_f(D_i\varphi)\]
So we define (for all multi-index $\alpha$) the generalized derivative of any $T\in\calD'(\Omega)$ as
\[(D^\alpha T)(\varphi)=(-1)^{|\alpha|}T(D^\alpha\varphi)\]
so $D^\alpha T\in\calD'(\Omega)$. If $T=T_f$ and $D^\alpha T=T_g$ for some $f,g\in L^1_{\text{loc}}$, then say $g=D^\alpha_wf$ the weak partial derivative of $f$.

\begin{example}
    Let $f(x)=x1_{\{x>0\}}$. Consider $T_f$.
    \[DT_f(\varphi)=-T_f(\varphi')=-\int_0^\infty x\varphi'(x)dx\overset{ibp}{=}\int_0^\infty\varphi(x)dx=\int_\RR H\varphi dx\]
    where $H$ is the Heaviside function, so $H$ is the weak derivative of $f$.

    Consider the second derivative.
    \[D^2T_f(\varphi)=DT_H(\varphi)=-T_H(\varphi')=-\int_0^\infty\varphi'=\varphi(0)=\delta_0(\varphi)\]
    $\delta_0$ cannot be represented by locally integrable functions.

    Have $D^3T_f(\varphi)=-\delta_0(\varphi')=-\varphi'(0)$ which is a Schwartz distribution but not a measure.
\end{example}
\subsection{Multiplication of Distributions with Smooth Functions}
If $f\in L^1_{\text{loc}}$ and $a\in C^\infty(\Omega)$, the $T_{af}(\varphi)=\int_\Omega af\varphi=T_f(a\varphi)$. Note that $a\varphi\in\calD$ if $\varphi\in\calD$, so we define
\[(aT)(\varphi)=T(a\varphi)\]
for $T\in\calD'(\Omega)$ and $a\in C^\infty(\Omega)$.

\subsection{Compactly Supported Distributions}
\begin{prop}
    A linear map $T:\mathcal E(\Omega)\to\RR$ (or $\CC$) is cts iff there exists $K\subseteq\Omega$ compact, $N\in\NN$, and $C>0$ s.t. for all $\varphi\in\mathcal E(\Omega)$
    \[|T(\varphi)|\le C\max_{0\le|\alpha|\le N}\sup_{x\in K}|D^\alpha\varphi(x)|\tag{$\dagger$}\]
\end{prop}\
\begin{proof}
    Suppose $(\dagger)$ holds and $\phi_j\to 0$ in $\mathcal E(\Omega)$. By defn of $\tau_{\mathcal P}$, for $j$ large enough, $K\subseteq K_j$ and RHS of $(\dagger)$ with $\phi=\phi_j$ converges to $0$, so $T(\phi_j)\to 0$, so $T\in\mathcal E'(\Omega)$.

    Conversely, assume $T$ is cts but $(\dagger)$ fails. If $K_j\subseteq K_{j+1}$ is any exhaustion of compact sets of $\Omega$, we obtain a sequence $\varphi_j\in\mathcal E'(\Omega)$ s.t. \[|T(\varphi_j)|\ge j\max_{0\le |\alpha|\le j}\sup_{x\in K_j}|D^\alpha(\varphi_j)|\]
    Define $\psi_j=\frac{\varphi_j}{|T(\varphi_j)|}$. Have
    \[|D^\beta\psi_j(x)|\le\frac1j\frac{|D^\beta\varphi_j(x)|}{\max_{0\le|\alpha|\le j}\sup_{x\in K_j}|D^\alpha\varphi_j(x)|}\overset{e.v.}{\le}\frac1j\to 0\]
    So $\psi_j\to 0$ in $\mathcal E(\Omega)$ but $T(\psi_j)=1$ for all $j$. Contradiction.
\end{proof}
\begin{defn}
    Say $T\in \calD'(\Omega)$ has support in a closed set $K\subseteq \Omega$ if $T(\varphi)=0$ whenever $\varphi\in C^\infty_c(\Omega\setminus K)\subseteq C^\infty_c(\Omega)$.
\end{defn}
The last proposition implies that $T\in\mathcal E'(\Omega)$ is supported in some compact subset of $\Omega$.
If $f\in L^1_{loc}$ s.t. $f=0$ outside of a compact set, then $T_f$ is compactly supported. If $T\in\calD'(\Omega)$ is compactly supported, then so is $D^\alpha T$ for any $\alpha$.
\begin{prop}
    Any $T\in\mathcal E'(\Omega)$ restricts to $T\in\calD'(\Omega)$ of compact support. Any $T\in\calD'(\Omega)$ that is compactly supported extends to $\tilde T\in\mathcal E'(\Omega)$.
\end{prop}
\begin{proof}
    The first claim follows from $(\dagger)$ in the preceding proposition. Conversely, if $K$ is compact and supports $T_1$. Take $\xi\in C^\infty_c(\Omega)$ s.t. $\xi=1$ on $K$ and define $\tilde T(\varphi)=T(\xi\varphi)$, $\varphi\in\mathcal E(\Omega)$, which define an element of $\mathcal E'(\Omega)$.
\end{proof}
\subsection{Convolutions of Distributions}
Notation: Recall the shift operator $\tau_xg=g(\cdot-x)$. Let $\overset{\vee}{g}=g(-\cdot)$, and $\overset{\vee}{\tau_xg}=g(x-\cdot)$.

In this notation $f\ast g(x)=T_f(\overset{\vee}{\tau_xg})$. If $g\in C^\infty_c(\RR^n)$, $\overset{\vee}{\tau_xg}\in C_c^\infty(\RR^n)$, and we can define for $T\in\calD'(\RR^n)$, $\varphi\in \calD(\RR^n)$, the convolution
\[x\mapsto T\ast\varphi(x)=T[\overset{\vee}{\tau_x\varphi}]\]
\begin{thm}
    Let $T\in\calD'(\RR^n)$, $\varphi\in\calD(\RR^n)$, $\alpha$ any multi-index. Then $T\ast\varphi\in C^\infty(\RR^n)$ and $D^\alpha(T\ast\varphi)=(D^\alpha T)\ast\varphi= T\ast(D^\alpha\varphi)$.
\end{thm}
\begin{proof}
    Take $e_i\in\RR^n$ (basis vector), and let $h\to 0$. Write \begin{align*}\frac1h[T\ast\varphi(x+he_i)-T\ast\varphi(x)]&=T\left[\frac{\varphi(x+he_i-\cdot)-\varphi(x-\cdot)}{h}\right]\\
    &\overset{ES2}{\longrightarrow} D_i(\varphi(x-\cdot))\end{align*}
     in $\calD(\RR^n)$ and since $T$ is cts in this topology, $\text{RHS}\overset{h\to 0}{\to} T[D_i\varphi(x-\cdot)]=T(\overset{\vee}{\tau_xD_i\varphi})=T\ast D_i\varphi$. In particular $T\ast \varphi$ is cts for any $\varphi\in\calD(\RR^n)$ and is $T\ast D_i\varphi$, and by iterating we deduce that $T\ast\varphi\in C^\infty(\RR^n)$. In particular, $D^\alpha (T\ast\varphi)=T\ast(D^\alpha\varphi)$.

     Need to prove the first equality. Note 
     \[D^\alpha(\overset{\vee}{\tau_x\varphi})=D^\alpha\varphi(x-\cdot)=(-1)^{|\alpha|}(D^\alpha\varphi)(x-\cdot)=(-1)^{|\alpha|}\overset{\vee}{\tau_xD^\alpha\varphi}\]
     Thus $(D^\alpha)\ast\varphi(x)\overset{}{=}D^\alpha T(\overset{\vee}{\tau_x\varphi})=(-1)^{|\alpha|}T(D^\alpha(\overset{\vee}{\tau_x\varphi}))=\tau(\overset{\vee}{\tau_xD^\alpha\varphi})=T\ast(D^\alpha\varphi)$.
\end{proof}
Notice if $T\in\mathcal E'(\Omega)$, supported in $K$ cpt and $K_x$ the shifted support of $\overset{\vee}{\tau_x\varphi}=\varphi(x-\cdot)$, $\varphi\in C_c^\infty(\RR^n)$. Thus for $|x|$ large enough, $K\cap K_x=\varnothing$ and $T_\ast\varphi\in\calD(\RR^n)$.
\begin{defn}
    Let $T)1\in\calD'(\RR^n)$, $T_2\in\mathcal E'(\RR^n)$. THen define their convolution by the action
    \[(T_1\ast T_2)\ast\varphi(x)=T_1\ast(T_2\ast\varphi)(x)\] for $\varphi\in \calD(\RR^n)$.
\end{defn}
\begin{rem}
    Note that $T_1\ast T_2$ is assigned on $\calD(\RR^n)$ as we can consider $x=0$, $\varphi=\phi(-\cdot)$ so that $T_1\ast T_2(\varphi)(0)=T_1\ast T_2(\phi)$ for $\phi\in\calD(\RR^n)$. 
\end{rem}
\begin{rem}
    Note that $\delta_0$ has cpt support and $\delta_0\ast\varphi=\delta_0[\varphi(x-\cdot)]=\varphi(x)$. Therefore for any $T_1\in\calD'(\RR^n)$, $(T_1\ast\delta_0)\ast\varphi=T_1\ast(\delta_0\ast\varphi)=T_1\ast\varphi$, so $\delta_0\ast[\cdot]$ acts as a right identity on all of $\calD'(\RR^n)$.
\end{rem}
\begin{thm}
    Let $T_1\in\calD'(\RR^n), T_2\in\mathcal E'(\RR^n)$, $\alpha$ any multi-index. Then \[D^\alpha(T_1\ast T_2)=(D^\alpha T_1)\ast T_2=T_1\ast(D^\alpha T_2)\]
\end{thm}
\begin{proof}
    Using the previous theorem and the definitions, for any $\varphi\in\calD(\RR^n)$, we have $D^\alpha(T_1\ast T_2)\overset{\text{thm}}{=}(T_1\ast T_2)\ast D^\alpha\varphi\overset{\text{def}}{=} T_1\ast (T_2\ast D^\alpha\varphi)\overset{\text{thm}}{=} T_1\ast(D^\alpha T_2\ast \varphi)\overset{\text{def}}{=}(T_1\ast D^\alpha T_2)\ast\varphi$.
\end{proof}
\subsection{Fundamental Solutions of Linear PDEs}
Consider a partial differential operator $L=\sum_{|\alpha|\le k}a_\alpha D^\alpha$, $a_\alpha\in C^\infty(\RR^n)$, $k\in\NN$. Consider $Lu=u_0$ for $u,u_0\in\calD'(\RR^n)$. A weak solution $u$ is one s.t. $(Lu)(\varphi)=u_0(\varphi)$ for all $\varphi\in\calD$.
An element $\mathcal G\in\calD'(\RR^n)$ is called a fundamental solution for $L$ if $L\mathcal G=\delta_0$ in $\calD'(\RR^n)$. If $\mathcal G= T_g$ for some $g\in L^1_{\text{loc}}(\RR^n)$, then we call $g$ the Green kernel of $\mathcal G$.
\begin{thm}
    Suppose $L$ has constant coefficients $a_\alpha\in\RR$ (or $\CC$), and $\mathcal G\in\calD'(\RR^n)$ is its fundamental solution. Then, if $u_0\in\mathcal E'(\RR^n)$, a solution $Lu=u_0$ is given by
    \[u=\mathcal G\ast u_0\]
\end{thm}
\begin{rem}
    If $u_0\in\mathcal D$, then $\mathcal G\ast u_0\in C^\infty(\RR^n)$ and the equation $Lu=u_0$ holds pointwise on $\RR^n$.
\end{rem}
\begin{proof}
    By linearity and the previous theorem
    \[Lu=\sum_{|\alpha|\le k}a_\alpha D^\alpha(\mathcal G\ast u_0)=\sum_{|\alpha|\le k}a_\alpha(D^\alpha\mathcal G\ast u_0)=LG\ast u_0=\delta_0\ast u_0=u_0\]
    Note that the last equality follows from the fact that we can swap $\delta_0$ and $u_0$ when $u_0$ is compactly supported.
\end{proof}
\subsection{Fourier Transforms of Distributions}
Recall Fourier transform \[\hat f(u)=\int_{\RR^n}e^{-ix\cdot u}f(x)dx\]
for $f\in L^1$. Since $\hat f\in L^1_{\text{loc}}(\RR^n)$, we can consider the operator
\[T_{\hat f}(\varphi)=\int_{\RR^n}\hat f(u)\varphi(u)du=\int_{\RR^n}\int_{\RR^n}f(x)e^{-ix\cdot u}\varphi(u)dudx=T_f(\hat\varphi)\] for $\varphi\in\calD$. Note that we used Fubini.
Since $\hat\varphi$ is not necessarily in $\calD$, this defn doesn't extend to $\calD'$, and we choose to work with $\mathcal S(\RR^n)$ and $\mathcal S'(\RR^n)$ instead.

Recall Riemann Lebesgue lemma from PM.
\begin{lem}
    Let $f\in L^1(\RR^n)$, then $\hat f\in C_0(\RR^n)$.
\end{lem}
\begin{proof}
    For any $u_j\to u$ in $\RR^n$, we have $e^{-ix\cdot u_j}f(x)\to e^{-ix\cdot u}f(x)$, and this gives a dominating function. By DCT, $\hat f(u_j)\to \hat f(u)$, so $\hat f$ is cts.
    Also have $\|\hat f\|_\infty\le \int_{\RR^n}|f(x)|dx=\|f\|_{L^1}$. For any $f\in L^1$, take a sequence $f_k\in C^\infty_c(\RR^n)$ s.t. $f_n\to f$ in $L^1$, so that $\|\hat f_k-\hat f\|_\infty\le \|f_k-f\|_{L^1}\to 0$ so $\hat f_k\to \hat f$ unif. on $\RR^n$, and $\hat f_k\in C_0(\RR^n)$. Now have $|u_j|\hat f_l(u)|=|\hat{D_jf_k}(u)|\le \|D_jf\|_{L^1}<\infty$. By completeness of $C_0(\RR^n)$, $\hat f\in C_0(\RR^n)$
\end{proof}
\begin{rem}
    Note that Fourier transform does not map $L^1$ onto $C_0(\RR^n)$.
\end{rem}
\[\Lbag;\tag{Owen's signature}\]
\begin{lem}
    Let $f\in L^1(\RR^n)$.\begin{enumerate}[(i)]
        \item If $f_\lambda=\lambda^{-n}f(\cdot/\lambda)$, $\lambda>0$, then $\hat f_\lambda=\hat{f}(\lambda u)$, $u\in\RR^n$
        \item $\mathcal F[\tau_x f](u)=e^{-ix\cdot u}\hat f(u)$, $\mathcal F[e^{i\langle y,\cdot\rangle}f]=\tau_y \hat f$.
        \item If $g\in L^1$, then $f\ast g\in L^1$ and $\mathcal F[f\ast g]=\hat f\cdot \hat g$
    \end{enumerate}
\end{lem}
\begin{proof}
    Fubini and substitution.
\end{proof}
\begin{thm}\
    \begin{enumerate}[(i)]
        \item Let $f\in C^1(\RR^n)$, $f,D_jf\in L^1$ for $j=1,...,n$. Then $\mathcal F[D_jf](u)=iu_j\hat f(u)$ for $u\in\RR^n$.
        \item If $f\in L^1(\RR^n)$ and $\int_{\RR^n}(1+|x|)|f(x)|dx<\infty$, then for any $j=1,...,n$, $u\in\RR^n$, have $D_j\hat f(u)=-i\mathcal F[x_j\hat f(x)]$. In particular, $\hat f\in C^1(\RR^n)$
    \end{enumerate}
\end{thm}
\begin{proof}
    (i) For any $\e>0$, we can pick $f_\e\in C^\infty_c(\RR^n)$ s.t.  $\|f_\e-f\|_{L^1}+\|D_jf_\e-D_jf\|_{L^1}<\e$. [First approximate by $\bar f=f\xi$, with $\xi\in C_c^\infty(\RR^n)$ s.t. $\xi=1$ on $D(0,M)$. Approximate $\bar f$ by $\phi_\e\ast \bar f\to \bar f$ in $L^1$. Also have $D_j(\phi_\e\ast f)=\phi_\e\ast(D_j\bar f)\to D_j\bar f$ as $\e\to 0$.] For such $f_\e$ we see
    \[\mathcal F[D_jf_\e](u)=\int_{\RR^n}e^{ix\cdot u}D_jf_\e(x)dx\overset{ibp}{=}-\int_{\RR^n}iu_jx^{-ix\cdot u}f_\e(x)dx\]
    So \begin{align*}
        |\mathcal F[D_jf](u)-iu_j\hat f(u)|&\le |\mathcal F[D_j\hat f](u)-\mathcal F[D_jf_\e](u)|+|iu_j(\hat f_\e(u)-f(u))|\\
        &\le \|D_jf-D_jf_\e\|+|u_j|\|f_\e-f\|_{L^1}\\ &\le (1+|u_j|)\e
    \end{align*}

    (ii) \begin{align*}
        \frac1h(\hat f(u+he_j)-\hat f(u))&=\int_{\RR^n}\frac1h(e^{-ix\cdot(u+he_j)}-e^{-ix\cdot u})f(x)dx\\
        &=\int_{\RR^n}e^{-ix\cdot u}(\frac{e^{-ix\cdot(he_j)}-1}{h})f(x)dx\overset{\text{DCT}}{\rightarrow}-i\int_{\RR^n}e^{-ix\cdot u}x_jf(x)dx
    \end{align*}
    Dominating function $|x_j|$, which is $|f(x)|dx$-integrable by assumption.
\end{proof}
Recall from PM.
\begin{thm}[Fourier Inversion]
    Let $f\in L^1$ and $\hat f\in L^1$ Then $f(x)=\frac{1}{(2\pi)^n}\int_{\RR^n}e^{ix\cdot u}\hat f(u)dx=\mathcal F^{-1}[\hat f](x)$ a.e.
\end{thm}
Note that for the unique cts representative of $f$, the formula holds everywhere.

Note that $\mathcal F^{-1}[\mathcal F\varphi]=\frac{1}{(2\pi)^n}\mathcal F[\mathcal F\varphi](-\cdot)$, so $\mathcal F^{-1}$ is a Fourier transform, and $\mathcal F^2\varphi=(2\pi)^n\overset{\vee}{\varphi}$.

\begin{thm}
    $\mathcal F$ is a linear automorphism of $\mathcal S(\RR^n)$.
\end{thm}
\begin{proof}
    Can check $\mathcal S(\RR^n)\subseteq L^1(\RR^n)$. If $f\in L^1$, then $\int|f|\le (\sup_{x\in\RR^n}(1+|x|)^{n+1}|f(x))\int_{\RR^n}\frac{dx}{(1+|x|)^{n+1}}<\infty$.
    For multi-indices $\alpha,\beta$,
    \[|u^\alpha||D^\beta\hat f(u)|=|\mathcal F[D^\alpha(x^\beta f)]|(u)\overset{\text{RL}}{\le}\|D^\alpha(x^\beta f)\|_{L^1}\le p_N(f)\] where $p_N(f)$ is an expression of the form in the previous ineq. If $\phi_j\to 0$ in $\mathcal S$, have $p_N(\hat \phi_j)\to 0$ and $\mathcal F:\mathcal S\to \mathcal S$ is cts. Moreover, if $\mathcal F[\phi]=0$ for $\phi\in\mathcal S\subseteq L^1$, then by the Fourier inversion formula, $\phi=\mathcal F^{-1}(0)=0$, so $F$ is injective. For any $\phi\in \mathcal S(\RR^n)$, have $\mathcal F^{-1}\mathcal F\phi=\frac{1}{(2\pi)^n}\mathcal F^2\overset{\vee}{\phi}$, which is the fourier transform of some function, so also surjective.
\end{proof}

\begin{defn}
    For $T\in \mathcal S'(\RR^n)$ we define its distributional Fourier transform $\hat T(\phi)=T(\hat\phi)$ for all $\phi\in\mathcal S(\RR^n)$
\end{defn}
\begin{rem}
    Clearly by the previous theorem, $\hat T\in \mathcal S'(\RR^n)$. If $f\in L^1$, then $T_{\hat f}(\phi)=\int_{\RR^n}\hat f\phi dx\overset{\text{Fubini}}{=}\int_{\RR^n}f\hat\phi= T_f(\hat\phi)$.
\end{rem}
\[(:\Lbag\tag{Owen's Signature with Quiff}\]
\begin{defn}
    Call $\phi$ slowly increasing if $\sup_{x\in\RR^n}(1+|x|)^{-N}|\phi(x)|<\infty$ for some $N$.
\end{defn}
Then $T_\phi\in\mathcal S'(\RR^n)$. Even if $\hat T_\phi$ is given by $T_g$ for some $g\in L^1_{\text{loc}}$, can't conclude $\hat\phi$ is pointwise defined.

If $T_j, T\in\mathcal S'(\RR^n)$ and $T_j\to T$ weak-* in $\mathcal S'(\RR^n)$, then $\hat T_j(\phi)=T_j(\hat\phi)\to T(\hat\phi)=\hat T(\phi)$, so $\mathcal F:\mathcal S'(\RR^n)\to \mathcal S'(\RR^n)$ is sequentially continuous.
One shows further that $\hat T=0\implies T=0$, so $\mathcal F$ is inj. Define $\mathcal F^{-1}T$ via $\mathcal F^{-1}T(\phi)=T(\mathcal F^{-1}\phi)$ for all $\phi$ in $\mathcal S(\RR^n)$. $\mathcal F^{-1}=\frac{1}{(2\pi)^n}\overset{\vee}{\mathcal F}$. Can check $\mathcal F^{-1}[\mathcal FT](\phi)=T(\phi)$.
\begin{thm}
    $\mathcal F$ (Fourier transform) defines a linear automorphism of $\mathcal S'(\RR^n)$.
\end{thm}
\begin{rem}
    Recall Plancherel from PM. $\mathcal F$ extends to the completion by unif continuity. Get an isometry $\frac{1}{(2\pi)^{n/2}}\bar{\mathcal F}$ of $L^2=\overline{L^1\cap L^2}^{L^2}$. If we define $\bar{\mathcal F}T(\phi)=T(\bar{\mathcal F}\phi)=T(\hat\phi)$ for all $\phi\in\mathcal S\subseteq L^1\cap L^2$, then see that $\bar{\mathcal F}=\mathcal F$ on $\mathcal S^1$.

    For any finite measure $\mu$ on $\RR^n$, have $\hat\mu(u)=\int_{\RR^n}e^{-ix\cdot u}d\mu(x)$. Then
    \[\hat T_\mu(\phi)=T_\mu(\hat\phi)=\int_{\RR^n}\int_{\RR^n}e^{-ix\cdot u}\phi(u)dud\mu(x)=T_{\hat\mu}(\phi)\] $\mathcal FT_\mu=T_{\hat\mu}$ in $\mathcal S'(\RR^n)$.

    For $T\in\mathcal E'(\RR^n)$, can define $E(u)=T(e^{-i\langle\cdot,u\rangle})$. Can show that $\hat T= T_E$ in $\mathcal S'(\RR^n)$ with $E$ slowly increasing.
\end{rem}

Note that the product of a slowly increasing func with a rapidly decreasing func is again rapidly decreasing, i.e., in $\mathcal S$. For $T\in \mathcal S'$, define $aT$ for $a\in C^\infty$ slowly increasing by $(aT)(\phi)=T(a\phi)$ for $\phi\in\mathcal S$. For any $T\in\mathcal D'(\RR^n)$, define $(\tau_kT)(\varphi)=T(\tau_{-k}\phi)$ for $k\in\RR^n$.

\begin{lem}
    Let $T\in\mathcal S'(\RR^n)$ and $\alpha$ any multi-index \begin{enumerate}[(i)]
        \item $\mathcal F(\tau_y T)=e^{-i\langle y,\cdot\rangle}\hat T$ and $\mathcal F[e^{-i\langle y,\cdot\rangle}T]=\tau_y T$
        \item $\mathcal F[D^\alpha T]=i^{|\alpha|}u^\alpha\hat T$ and $D^\alpha\hat T=(-i)^{|\alpha|}\mathcal F[x^\alpha T]$
    \end{enumerate}
\end{lem}
\begin{proof}
    Compute.
\end{proof}
\begin{rem}
    $\mathcal F[D^\alpha\delta_0]=i^{|\alpha|}u^\alpha$. So the FT of partial derivatives of Dirac measure span the space of polys.
\end{rem}
\[\text{Owen is ill today}\]
%{\fontspec{Symbola}\symbol{"1F912"}}
\subsection{Periodic Distribution}
\begin{defn}
    $T\in\calD'(\RR^n)$ is periodic if $\tau_k T=T$ for all $k\in\ZZ^n$
\end{defn}\
\begin{defn}
    For $T\in\mathcal E'(\RR^n)$, define the periodization $T_{\text{per}}=\sum_{k\in\ZZ^n}\tau_k T$.
\end{defn}
The fundamental cell of the lattice is $\mathcal Q=[-1/2,1/2)^n$. The indicator $1_{\mathcal Q}$ is not smooth.
\begin{lem}
    There exists $\psi\in C_c^\infty(\RR^n)$ s.t. \begin{enumerate}[(i)]
        \item $\psi\ge 0$
        \item $\operatorname{supp}\psi\subseteq\operatorname{Int}(Q)$ where $Q=[-1,1]^n$.
        \item $\sum_{k\in\ZZ^n}\psi(x-k)=1$ for all $x\in\RR^n$
    \end{enumerate}
    If $\psi'$ is another such function and $T$ is a periodic distribution, then $T(\psi)=T(\psi')$.
\end{lem}
Call this $\psi$ a periodic partition of unity (ppu).
\begin{proof}
    Find $\psi_0\in C_0^\infty$ supported in $\operatorname{Int}(Q)$ s.t. $\psi_0=1$ on $\mathcal Q$. Define $S(x)=\sum_{k\in\ZZ^n}\psi_0(x-k)$. Normalize $\psi(x)=\psi_0(x)/S(x)$.

    If $T$ is a periodic distribution, then
    \[T(\psi)=T(\sum_{g\in \ZZ^n}\tau_g\psi'\psi)=\sum_g\tau_g T(\psi\tau_g\psi')=\sum_gT(\psi'\tau_g\psi)=T(\psi')\]
\end{proof}
Can take $\psi_{0,j}\to 1_{\mathcal Q}$ ptwise and $\sup_j\|\psi_{0,j}\|_\infty<\infty$. Obtain a uniformly bounded sequence of ppu $\psi_j\to 1_{\mathcal Q}$.

\begin{defn}
    For $T\in\calD'(\RR^n)$ periodic, define the mean of $T$ as $M(T)= T(\psi)$ where $\psi$ is any ppu.
\end{defn}
\begin{thm}
    Let $\mathcal E'(\RR^n)$. $T_{\text{per}}$ converges in $\mathcal S'(\RR^n)$. If $T\in\calD'(\RR^n)$ is periodic, then there exists $V\in\mathcal E'(\RR^n)$ s.t. $T=\sum_{g\in\ZZ^n}\tau_g V$ in $\calD'(\RR^n)$.
\end{thm}
\begin{proof}
    For $T\in\mathcal E'$, have a cpt set $K\subseteq B_R$ (ball of radius $R$) and $N,C>0$ s.t. for all $\phi\in\mathcal E$
    \[|T(\phi)|\le C\sup_{x\in K,|\alpha|\le N}|D^\alpha\phi(x)|\]
    Have $1+|g|\le 1+|g+x|+|x|\le 1+|g+x|+R\le (1+R)(1+|g+x|)$, so 
    \[1\le \frac{(1+R)^M(1+|g+x|)^M}{(1+|g|)^M}\] for any $M\in\NN$.
    For all $\phi\in\mathcal S\subseteq\mathcal E$, 
    \[|T\phi|\le C\frac{(1+R)^M}{(1+|g|)^M}\sup_{x\in K,|\alpha|\le N}(1+|g+x|)^M|D^\alpha\phi(x)|\]
    Applies to $\tau_g\phi$, get a similar inequality.

    Since $\sum_{g\in\ZZ^n}(1+|g|)^{-n-1}<\infty$ we deduce
    \[|\sum_g\tau_g T\phi|\le C'\sup_{y\in\RR^n,|\alpha|\le N}(1+|y|)^{n+1}|D^\alpha\phi(y)|\]
    so $\sum_g\tau_g T\in\mathcal S'$ by ES.

    For the converse, let $T$ be periodic and $\phi\in\calD$. If $\psi$ is any ppu, have
    \[T\phi=T(\phi\sum_g\tau_g\psi)=\sum_g T(\psi\tau_{-g}\phi)=\sum_g(\psi T)(\tau_{-g}\phi)=\sum_g\tau_g(\psi T)(\phi)\]
\end{proof}
\[\overset{\rightharpoonup}{\overset{\circ<\circ}{\smile}}\tag{Owen's Signature}\]
\begin{thm}[Convergence of Fourier series in $\mathcal S'$]
  Let $U\in\mathcal D'(\RR^n)$ be periodic. Then $U=\sum_{g\in\ZZ^n}u_g T_{e_{2\pi g}}$ in $\mathcal S'(\RR^n)$, where $e_{2\pi h}=e^{i\langle 2\pi h,\cdot\rangle}$ and with Fourier coefficients $u_g=M(e_{2\pi h})$  
\end{thm}
\begin{lem}
    If $T\in\mathcal S'$ s.t. $(e_{-k}-1)T=0$ for all $k\in\ZZ^n$, then $T=\sum_{g\in\ZZ^n}c_g\delta_{2\pi g}$ in $\mathcal S'$. Have $|c_g|\le C(1+|g|)^N$ for some $N,C>0$
\end{lem}
\begin{proof}
    Let $\Lambda^\ast=\{2\pi g:g\in\ZZ^n\}$. Take $\varphi\in\mathcal D(\RR^n)$ s.t. $\operatorname{supp}\varphi\cap\Lambda^\ast=\varnothing$, so $(e_{-k}-1)^{-1}\varphi\in \calD$ and $T(\varphi)=(e_{-k}-1)T((e_{-k}-1)^{-1}\varphi)=0$ for all $k$, so $T$ is also supported in $\Lambda^\ast$. Now take ppu $\psi$ and consider $\tilde\psi=\psi(\cdot/(2\pi))$. $\operatorname{supp}\tilde\psi\subset\{x\in\RR^n:\forall i,\ -2\pi<x_i<2\pi \}$ and $\sum_{g\in\ZZ^n}\tau_{2\pi g}\tilde\psi =1$ on $\RR^n$. Now define $T_g=(\tau_{2\pi g}\tilde\psi )T$ which is supported in $\{2\pi g\}$ and have $\sum_{g\in\ZZ^n}T_g=\sum_g(\tau_{2\pi g}\tilde\psi) T=T$ (in $\calD'$) and $(e_{-k}-1)T_g=(\tau_{2\pi g}\tilde\psi)(e_{-k}-1)T$ for all $k\in\ZZ^n$.

    Choose $k=g_j$ ($j$-th standard basis vector), have $(e_{-k}-1)T_g=(e^{-ix_j}-1)T_g=(e^{-(x_j-2\pi g)}-1)T_g\overset{\text{Taylor}}{=}(x_j-2\pi g)K(x_j)T_g$, where $K$ is the Taylor poly which doesn't vanish near $2\pi g$, so $(x_j-2\pi  g)T_g=0$.

    Take $\phi\in\mathcal S(\RR^n)$ and apply Taylor expansion to get $\phi(x)=\phi(2\pi g)+\sum_{j=1}^n(x_j-2\pi g)\phi_j(x)$ for some $\phi_j\in\mathcal S$, so $T_g\phi=T_g(\phi(2\pi g))+\sum_{j=1}^n(x_j-2\pi g)T\phi_j=\delta_{2\pi g}(\phi)T_g(1)$. Let $c_g=T_g(1)$.

    $|c_g|=|T_g(\sum_{g'}\tau_{2\pi g'}\tilde\psi)|=|T_g(\tau_{2\pi g}\tilde\psi)|$. Since $T_g\in\mathcal E'\subseteq\mathcal S'$ and we have a characterization of $\mathcal S'$ in ES3, have
    $|c_g|\le C_0\sup_{x\in\RR^n,|\alpha|\le N}(1+|x|)^N|D^\alpha\tilde\psi(x-2\pi g)|$ for some $N\in\NN$, $c>0$, and $\le C_1(1+|x|)^N\sup_{y\in\RR^n,|\alpha\le N|}(1+|y|)^N|D^\alpha\tilde\psi(y)|\le C(1+|g|)^N$. Therefore $T=\sum_{g}c_g\delta_{2\pi g}$ converges in $\mathcal S'$.
\end{proof}
\begin{proof}[Proof of Thm]
    Apply the lemma to $U$. $\hat U=(2\pi)^n\sum_{g\in\ZZ^n}u_g\delta_{2\pi g}$, $u_g=c_g/(2\pi)^n$. Take inverse FT, see $U=\sum_{g\in\ZZ^n}u_g T_{e_{2\pi g}}$. Note that $T\mapsto M[T]$ is cts on $\mathcal S'$, so $M(e_{-2\pi k}U)=\sum_{g\in\ZZ^n}u_gM(e_{-2\pi k}T_{e_{2\pi g}})=\int_{\mathcal Q}e^{i2\pi\langle g-k,x\rangle}dx=1$ if $g=k$  and $0$ otherwise.
\end{proof}
Apply this to $U=\sum_k\delta_k=\sum_{k}\tau_k\delta_0$ with ppu $\psi$ s.t. $\psi(0)=1$. Compute Fourier coeffs.
$M(e_{-2\pi g}U)=\sum_k\delta_k(e_{-2\pi g} \psi)=1$ for all $g$, so $\sum_{k\in\ZZ^n}\delta_k=\sum_{k\in\ZZ^n}T_{e_{2\pi k}}$ in $\mathcal S'$.
Testing this identity on $\phi(x-\cdot)$ for $\phi\in\mathcal S$, $x\in\RR^n$. Get $\sum_k\phi(x-k)=\sum_{k}T_{e_{2\pi k}}\phi(x-\cdot)=\sum_{k}e^{i2\pi k\cdot x}\hat\phi(2\pi k)$ (Poisson summation formula when $x=0$).
%\[\text{\textcorner\%}\]
\[\text{[Owen Broke \LaTeX{} Today]}\]
\section{Sobolev Spaces and Elliptic PDEs}
Let $C^k(\Omega)$ denote th normed space $\{f:\Omega\to\RR:D^\alpha f\text{ exists for all }0\le |\alpha|\le k,\ \|f\|_{C^k}<\infty\}$, where $\|f\|_{C^k}=\sum_{0\le|\alpha|\le k}\|D^\alpha f\|_\infty$. Similarly define the H\"older spaces for $0<\eta<1$ as $C^{k,\eta}(\Omega)=\{f\in C^{k}(\Omega):\|f\|_{C^{k,\eta}}<\infty\}$, where $\|f\|_{C^{k,\eta}}=\|f\|_k+\sum_{|\alpha|=k}\sup_{x\neq y}\frac{|D^\alpha f(x)-D^\alpha f(y)|}{|x-y|^\eta}$.
$C^k$ and $C^{k,\eta}$ are Banach spaces.

We can replace $\|\cdot\|_\infty$ and $D^\alpha$ by $L^p$-norm and the weak derivative $D^\alpha_w$.

\begin{defn}[Sobolev space]
    Let $k\in\ZZ_{\ge0}$, $1\le p\le\infty$. Then $f\in W^{k,p}(\Omega)$ ($\Omega$ open) if $D_w^\alpha f\in L^p$ for all $0\le |\alpha|\le k$. Then norm on $W^{k,p}(\Omega)$ is given by
    \[\|f\|_{W^{k,p}}=\left(\sum_{0\le|\alpha|\le k}\|D_w^\alpha f\|_{L^p}^p\right)^{1/p}\] if $p<\infty$
    and \[\|f\|_{W^{k,\infty}}=\max_{0\le|\alpha|\le k}\|D_w^\alpha f\|_{L^\infty}\]
\end{defn}
When $\Omega=\RR^n,\ p=2$, have
\begin{defn}
    Let $s\in\RR$. Then $H^s(\RR^n)$ consists of $f\in\mathcal S'(\RR^n)$, ($f=T_f$), s.t. $\hat f\in L^2_{\text{loc}}(\RR^n)$ and $\|f\|_{H^s}^2=\int_{\RR^n}|\hat f(u)|^2(1+|u|^2)^sdu$
\end{defn}
Note that $H^s(\RR^n)$ is a Hilbert space for the inner product $(f,g)_{H^s}=\int_{\RR^n}\hat f(u)\overline{\hat g(u)}(1+|u|^2)^sdu$, so $H^s=L^2(\mu_s)$ for some measure $\mu_s$ on $\RR^n$.

By Plancherel, for $s\ge 0$, $H^s(\RR^n)$ consists of elements of $L^2(\RR^n,dx)$
\begin{prop}
    For $s\ge 0$, $H^s(\RR^n)= W^{s,2}(\RR^n)$ with equivalent norms
\end{prop}
\begin{proof}
    
\end{proof}
\begin{thm}[Sobolev embedding]
    Let $s>n/2+k$ for $k\in\NN$ and $f\in H^s$. Then $\exists f^\ast\in C^k(\RR^n)$ s.t. $f^\ast=f$ a.e. and $\|f^\ast\|_{C^k}\le C_{s,n,k}\|f\|_{H^s}$. In particular, there is an embedding $H^s(\RR^n)\hookrightarrow C^k(\RR^n)$.
\end{thm}
\begin{rem}\
\begin{enumerate}
    \item[$\{\varnothing\}$:] Have $H^s\subseteq C^{k,\eta}$ if $s>\frac n2+k+\eta$.
    \item[$\{\varnothing,\{\varnothing\}\}$:]$\bigcap_{s>0}H^s\subseteq C^\infty(\RR^n)$
\end{enumerate}
\end{rem}
\begin{proof}
    Take $f\in \mathcal S(\RR^n)$ and note
    \begin{align*}
        |D^\alpha f(x)|&=|\mathcal F^{-1}[u^\alpha \hat f]|\\
        &\overset{\text{R.L.}}{\le} \frac{1}{(2\pi)^n}\int_{\RR^n}|u|^{|\alpha|}|\hat f(u)|\frac{(1+|u|^2)^{s/2}}{(1+|u|^2)^{s/2}}du\\
        &\overset{\text{C.S.}}{\le}\frac{1}{(2\pi)^n}\left(\int_{\RR^n}\frac{|u|^{2|\alpha|}}{(1+|u|^2)^s}du\right)\left(\int_{\RR^n}|\hat f(u)|^2(1+|u|^2)^sdu\right)\\
        &\le C_{s,n,k}\|f\|_{H^s}
    \end{align*}
    For $f\in H^s$ take $f_n\in\mathcal S$ s.t. $f_n\to f$ in $H^s$ and a.e. (pass to a subseq if necessary). This is Cauchy in $H^s$ and by the same inequality in $C^k$, we have $f_n\to f^\ast$ in $C^k$ by completeness. By uniqueness of limit, we have $f^\ast=f$ a.e. so $f^\ast =f $.
\end{proof}
Consider \[-\nabla^2v+v=f\tag{$\dagger$}\]
where $f\in H^s(\RR^n)$. Have Fourier transform $\hat\nabla^2=-|u|^2$.
\begin{thm}
    There exists a unique solution $v$ in $H^{s+2}(\RR^n)$ to $(\dagger)$ and $\|v\|_{H^{s+2}}\le \|f\|_{H^s}$ (elliptic regularity estimate)
\end{thm}
\begin{proof}
    Take FT get $(1+|u|^2)\hat v=\hat f$ in $\mathcal S'(\RR^n)$. For $f\in L^1_{\text{loc}}$ this has unique soln $\hat v(u)=\frac{\hat f(u)}{1+|u|^2}$, $u\in\RR^n$, so $v=\mathcal F^{-1}\hat v$.
    \[\|v\|^2_{H^{s+2}}=\int_{\RR^n}(1+|u|^2)^{s+r}\frac{\hat f(u)}{(1+|u|^2)^{2}}=\|f\|^2_{H^s}\]
\end{proof}
To study eqns restricted to open sets $\Omega\subseteq\RR^n$ with boundary $\partial\Omega$, need to define the restriction of $f\in H^s$ to $\partial \Omega$. If $f\in H^s$ for $s>n/2$, then Sobolev embedding implies that $f\in C^\e$ for some $\e>0$ and the Sobolev trace $f|_{\partial\Omega}$ exists by uniform continuity. For general $s>1/2$, have
\begin{thm}[Trace thm]
    There exists a bounded linear operator $T:H^s(\RR^n)\to H^{s-1/2}(\RR^{n-1})$, $s>1/2$, s.t. for all $f\in \mathcal S(\RR^n)$, $Tf=f|_{\RR^{n-1}\times \{0\}}$
\end{thm}
\begin{proof}
    ES
\end{proof}
Call $T= T_\Sigma$ for $\Sigma=\RR^{n-1}\times\{0\}$ the boundary trace of $f\in H^s$. By change of coords, this operator extends to $T_{\partial\Omega}$ for sufficiently regular $\Omega$. In particular, we have $T_{\partial\Omega}:H^1(\RR^n)\hookrightarrow H^{1/2}(\partial\Omega)\hookrightarrow L^2(\partial\Omega)$ is bounded linear.
\subsection{$H_0^1(\Omega)$}
Any $f\in C_c^\infty(\Omega)$ (sufficiently regular $\Omega$) extends by zero to an element of $H^1(\RR^n)$ (hence in $H^s(\RR^n)$ for all $s$). Have Hilbert norm
\[\|f\|_{H^1(\RR^n)}=\int_{\RR^n}(1+|u|^2)||\hat f(u)|^2du=(2\pi)^n\int_\Omega(|f(x)|^2+|Df(x)|^2|)dx\] where $Df$ is the gradient vector. Define $H_0^1(\Omega)=\overline{C_c^\infty(\Omega)}^{\|\cdot\|_{H^1}}$ in $H^1(\RR^n)$. This is not $W^{1,2}$ because
\begin{prop}
    Let $f\in H_0^1(\RR^n)$. Then $f(x)=0$ for almost every $x\in\Omega^c$ and if $\partial \Omega$ is sufficiently regular, then $T_{\partial\Omega} f=0$.
\end{prop}
\begin{proof}
    Take $\varphi\in C_c^\infty((\Omega^c)^\circ)$ and take $f_n\in C_c^\infty(\Omega)$ s.t. $f_n\to f$ in $H^1(\RR^n)$. Have $\Lambda_\varphi(h)=\int_{\RR^n}\varphi h$, then $\Lambda_\varphi\in (L^2)'\subseteq(H^1)'$, so $0=\int_{\RR^n}\varphi f_n=\Lambda_\varphi(f_n)\to\Lambda_\varphi f=\int\varphi f=0$, so $\operatorname{supp}(f)\subseteq\Omega$. Similarly, $0=T_{\partial \Omega}f_n\to T_{\partial \Omega}f=0$, so $f=0$ on $\partial\Omega$.
\end{proof}
Consider the BVP 
\[\begin{cases}
    -\nabla^2v+v=f & \text{on }\Omega\\
    v=0 & \text{on }\partial\Omega
\end{cases}\]
Interpret this as
\[\int_\Omega(-\nabla^2v+v)\varphi\overset{ibp}{=}\int_{\partial\Omega}\nabla v\cdot\nabla\varphi+\int_\Omega v\varphi=\int_\Omega f\varphi\] for $f\in L^2$, $v\in H^1$.

Since $C_c^\infty$ is dense in $H_0^1$ and $L^2$, this equation is the same as solving 
\[\langle v,\varphi\rangle_{H^1}=\langle f,\varphi\rangle_{L^2}\tag{$\dagger'$}\]
 for all $\varphi\in H_0^1(\Omega)$.
 \begin{thm}
     For eveyr $f\in L^2(\Omega)$, there exists a unique $v\in H_0^1(\Omega)$ s.t. $(\dagger')$ holds and $\|v\|_{H^1}=\|f\|_{L^2}$. Therefore the solution map $S:f\mapsto v=v_f$ is a bounded linear form $L^2(\Omega)\to H_0^1(\Omega)$ and self-adjoint for $L^2(\Omega)$.
 \end{thm}
 \begin{proof}
     Define $\Lambda_f(\phi)=\int_\Omega f\phi$ so that $\Lambda_f\in(H_0^1)'$ since 
     \[|\Lambda_f(\phi)|\overset{\text{C.S.}}{\le}\|f\|_{L^2}\|\phi\|_{L^2}\le\|f\|_{L^2}\|\phi\|_{H^1}\]
     Hence by Riesz representation thm on $H_0^1$, there exists a uniuque $v\in H_0^1$ s.t. $\langle v,\phi\rangle_{H^1}=\langle f,\phi\rangle_{L^2}$ for all $\phi\in H_0^1$.

     Next take $f_1,f_2\in L^2(\Omega)$, $\alpha\in\RR$, and take $v_1=S(f_1)$ and $v_2=S(f_2)$ and define $v=v_1+\alpha v_2$. Then,
     \[\langle v,\phi\rangle_{H^1}=\langle v_1+\alpha v_2,\phi\rangle_{H^1}=\langle v,\phi\rangle_{H^1}+\alpha\langle v_2,\phi\rangle_{H^1}==\langle f_1,\phi\rangle_{L^2}+\alpha\langle f_2,\phi\rangle_{L^2}=\langle f_1+\alpha f_2,\phi\rangle_{L^2}\]
     So $S(f_1)+\alpha S(f_2)=S(f_1+\alpha f_2)$.
     Also have 
     \[\|S(f)\|_{H^1}=\|v_f\|_{H^1}\overset{\text{Riesz}}{=}\|\Lambda_f\|\le \|f\|_{L^2}\]

    To see it's self-adjoint,
     \[\langle S(f),g\rangle_{L^2}=\langle g, S(f)\rangle_{L^2}=\langle S(f),S(g)\rangle_{H^1}=\langle S(g),S(f)\rangle_{H^1}=\langle f,S(g)\rangle_{L^2}\]
 \end{proof}
 To study regularity of $v$, we introduce
 \[H_{\text{loc}}^s(\Omega)=\{f\in L^2_{\text{loc}}(\Omega):\forall \xi\in C_c^\infty(\Omega),\ f\xi\in H^s(\RR^n)\}\]
 \begin{prop}
     If $f\in H_{\text{loc}}^s$ for $s>k+n/2$, then $f\in C^k(U)$ for any $U$ open s.t. $\bar U\subseteq \Omega$.
 \end{prop}
 \begin{proof}
     Given $U$, pick $\xi\in C_c^\infty$ s.t. $\xi=1$ on $\bar U$ and note that $f\xi\in H^s(\RR^n)\subseteq C^k(\RR^n)$ (Sobolev embedding), so $f=f\xi $on $U$, the result then follows.
 \end{proof}
 \begin{crly}
     $\bigcap_{s>0} H^s_{\text{loc}}(\Omega)\subseteq C^\infty(\Omega)$.
 \end{crly}
 Note that $f\in C^\infty(\Omega)$ may be unbounded at $\partial\Omega$.
 \begin{thm}[Interior regularity]
     Let $f\in L^2(\Omega)$ and suppose $v\in H_0^1$ solves $(\dagger')$. Then $v\in H^2_{\text{loc}}(\Omega)$. If additionally $f\in L^2(\Omega)\cap H^k_{\text{loc}}(\Omega)$, then $v\in H^{k+2}_{\text{loc}}(\Omega)$.
 \end{thm}
 \begin{proof}
 Let $K\subseteq\Omega$ be any compact set, and take $\chi\in C_c^\infty(\Omega)$ s.t. $\chi=1$ on $K$. Take $\varphi\in S(\RR^n)$ and set $\phi=\chi\varphi\in H_0^1$. Then $(\dagger')$ implies
 \[\int_\Omega(Dv\cdot D(\varphi\chi)+v\varphi\chi)dx=\int_\Omega f\varphi\chi dx\] for all $\varphi\in\mathcal S(\RR^n)$.
 Using chain rule and IBP, we rearrange the equation above to get
 \[\int_{\Omega}(D(v\chi)\cdot D\varphi+v\chi\varphi)dx=\int_\Omega g\varphi dx\]
 where $g=-(Dv)\cdot(D\chi)-v D\chi+f\chi\in L^2(\RR^n)$. IBP again, can see that $v\chi$ solves $-\nabla^2(v\chi)+v\chi=g$ in $\mathcal S'(\RR^n)$. Hence by elliptic regularity estimate, $\|v\chi\|_{H^2}\le \|g\|_{L^2}<\infty$.

 To prove $v\in H^2_{\text{loc}}$, take $\xi\in C^\infty_c(\Omega)$ and $K=\operatorname{supp}(\xi)$ s.t. $v\xi=v\chi\xi$. Then $\|v\xi\|_{H^2}=\|v\chi\xi\|_{H^2}\le C_n\|v\chi\|_{H^2}\|\xi\|_{L^2}<\infty$ ($\|fg\|_{L^2}\le \|f\|_{L^2}\|g\|_{L^\infty}$ + chain rule) We recognize that $g\in H^1_{\text{loc}}(\Omega)$ whenever $f\in H^1_{\text{loc}}(\Omega)$ so repeating the preceding argument given $v\in H^3_{\text{loc}}$. Can prove the rest of the theorem using the inequality $\|fg\|_{H^s}\le C_{n,s}\|f\|_{H^s}\|g\|_{C^s}$
 \end{proof}
 \begin{crly}
     If $f\in C^\infty(\Omega)\cap L^2(\Omega)$, then $v\in H_0^1(\Omega)\cap C^\infty(\Omega)$ solves $-\nabla^2 v+v=f$ on $\Omega$ (pointwise)
 \end{crly}
\begin{thm}[Rellich-Kondrashov]
    Let $\Omega\subseteq\RR^n$ be open and bounded. Let $u_j\in H_0^1(\Omega)$ s.t. $\|u_j\|_{H^1}\le K$ for all $j=1,2,...$, and some $K>0$. Then $\exists u\in H_0^1(\Omega)$ s.t. $u_{j_k}\to u$ in $L^2(\Omega)$ along a subsequence.
\end{thm}
\begin{proof}
    By Banach-Alaoglu in $H_0^1(\Omega)$, we obtain $u_{j_k}\rightharpoonup u$ in $H_0^1$ and then in $L^2$ (weakly), and $\|u\|_{H^1}\le K$. Also $u_{j_k}$, $u$ vanish a.e. on $\Omega^c$ so \begin{align*}\|u_{j_k}-u\|_{L^2(\Omega)}&=\|u_{j_k}-u\|_{L^2(\RR^n)}=\frac{1}{(2\pi)^n}\|\hat u_{j_k}-\hat u\|_{L^2(\RR^n)}\\ &=\frac{1}{(2\pi^n)}\left(\int_{|z|>R}|\hat u_{j_k}(z)-\hat u(z)|^2dz+\int_{|z|<R}|\hat u_{j_k}(z)-\hat u(z)|^2dz\right)\\ \end{align*}
    
    Given $\e>0$, have
    \[\int_{|z|>R}|\hat u_{j_k}(z)-\hat u(z)|^2dz\le\int_{|z|>R}\frac{1+|z|^2}{1+|z|^2}(|\hat u_{j_k}(z)|^2+|\hat u(z)|^2)dz\le \frac{2}{1+R^2}(\|u_{j_k}\|_{H^1}^2+\|u\|_{H^1}^2)<\e\] for $R$ sufficiently large.
    
    For $z\in \RR^n$ fixed,
    \begin{align*}
        \hat u_{j_k}(z)=\int_{\RR^n}e^{-ix\cdot z}u_{j_k(x)}dx=\int_{\Omega}e^{-ix\cdot z}u_{j_k}(x)dx=\langle e^{-i\langle\cdot,z\rangle},u_{j_k}\rangle\to \langle e^{-i\langle\cdot,z\rangle},u\rangle_{L^2(\Omega)}=\hat u(z)
    \end{align*}
    by weak convergence. Also \[|\hat u_{j_k}(z)|+|\hat u(z)|\le \|u_{j_k}\|_{L^1(\Omega)}+\|u\|_{L^1(\Omega)}\overset{\text{C.S.}}{\le}C_\omega(\|u_{j_k}\|_{L^2(\Omega)}+\|u\|_{L^2(\Omega)})\le 2C_\Omega K\] which is $dz$-integrable on $\{z:|z|\le R\}$, so $\int_{|z|<R}|\hat u_{j_k}(z)-\hat u(z)|^2dz\to 0$ by DCT.
\end{proof}
\begin{crly}
    The solution operator $S$ from $(\dagger')$ is a compact linear self-adjoint operator on $L^2(\Omega)$
\end{crly}
\begin{proof}
    $S$ maps $L^2$ into $H_0^1$ (bounded linear) and use Rellich-Kondrashov.
\end{proof}
By spectral theorem, there exists ONB $\{w_k:k\in\NN\}$ of $L^2(\Omega)$ and real e-values $\mu_k\downarrow 0$ as $k\to\infty$ s.t. \[Sw_k=\mu_kw_k\] in $L^2(\Omega)$. Thus $w_k\in H_0^1$. For all $\varphi\in H_0^1$
\begin{align*}
    \langle w_k,\varphi\rangle_{L^2}\overset{(\dagger')}{=}\langle Sw_k,\varphi\rangle_{H^1}=\mu_k\langle w_k,\varphi\rangle_{H^1}
\end{align*} Tes $\varphi=w_k$, see that $1=\langle w_k,w_k\rangle_{L^2}=\mu_k\|w_k\|_{H^1}$, so $\mu_k>0$ for all $k$. In $\calD'(\Omega)$,
\begin{align*}
    (-\nabla^2+I)w_k=(-\nabla^2+I)\frac{\mu_k}{\mu_k}w_k=\frac{1}{\mu_k}(-\nabla^2+I)Sw_k=\frac{w_k}{\mu_k}
\end{align*}
Therefore, \[-\nabla^2w_k=\left(\frac{1}{\mu_k}-1\right)w_k=\lambda_kw_k\]
where $\lambda_k=\frac{1}{\mu_k}-1\uparrow\infty$ are the e-values of $-\nabla^2$. The weak form is
\[\langle Dw_k,D\varphi\rangle_{L^2}=\lambda_k\langle w_k,\varphi\rangle_{L^2}\] for all $k$ and all $\varphi\in\calD(\Omega)$ (in fact $H_0^1(\Omega)$)
Note that $w_k=f$ in $(\dagger')$ and in $H_0^1$, so iterating the interior regularity thm, $w_k\in C^\infty(\Omega)$. Thus $-\nabla^2w_k=\lambda_kw_k$ is true on $\Omega$ ptwise.
\begin{thm}[Poincare inequality]
    For all $u\in H_0^1(\Omega)$, \[\frac{\langle Du,Du\rangle_{L^2}}{\langle u,u\rangle_{L^2}}\ge\lambda_1>0\]
\end{thm}
\begin{proof}
    ES
\end{proof}
We can now solve the Dirichlet problem for the Laplace equation
\[\begin{cases}
    -\nabla^2 v=f & \text{on }\Omega\\
    v=0 & \text{on }\partial\Omega
\end{cases}\]
or the weak form: find $v\in H_0^1$ $\langle Dv,D\varphi\rangle_{L^2}=\langle f,\varphi\rangle_{L^2}$ for all $\varphi\in\calD(\Omega)$. Denote this by $(\ast)$
\begin{thm}
    There exists a unique solution $v\in H_0^1(\Omega)$ to $(\ast)$, for any $f\in L^2(\Omega)$. 
    \[v=\sum_{k=1}^\infty\frac{1}{\lambda_k}\langle w_k,f\rangle_{L^2}w_k\]
\end{thm}
\begin{proof}
    Take partial sums $v_J=\sum_{n=1}^J\lambda_k^{-1}\langle w_k,f\rangle_{L^2}w_k$, $J\in\NN$, then (for $J'<J$) 
    \begin{align*}
        \|v_J-v_{J'}\|_{H^1}^2&=\langle v_J-v_{J'},v_J-v_{J'}\rangle_{L^2}+\langle D(v_J-v_{J'}),D(v_J-{v_{J'}})\rangle\\
        &=\sum_{k=J'+1}^J\lambda_k^{-1}\langle w_k,f\rangle^2+\sum_{k,k'=J'+1}^J\lambda_k^{-1}\lambda_{k'}^{-1}\langle f,w_k\rangle\langle f,w_{k'}\rangle\langle Dw_k,Dw_{k'}\rangle_{L^2}\\
        &=\sum_{k=J'+1}^J\lambda_k^{-1}\langle w_k,f\rangle^2+\sum_{k,k'=J'+1}^J\lambda_k^{-1}\lambda_{k'}^{-1}\langle f,w_k\rangle\langle f,w_{k'}\rangle\lambda_k^{-1}\langle w_k,w_{k'}\rangle_{L^2}\\
        &\le \sum_{k=J'+1}^\infty(\lambda_k^{-2}+\lambda_k^{-1}\langle w_k,f\rangle_{L^2}^2\\
        &\le C(\lambda_1)\sum_{k=J'+1}^\infty\langle f,w_k\rangle^2\overset{J'\to\infty}{\longrightarrow}0
    \end{align*}
    So $v_J$ is Cauchy in $H_0^1$, so $v\in H_0^1$
    Can check
    \begin{align*}
        \langle Dv,D\varphi\rangle_{L^2}&\overset{ibp}{=}\langle v,-\nabla^2\varphi\rangle_{L^2}=\sum_{k=1}^\infty\lambda_k^{-1}\langle v,w_k\rangle_{L^2}\langle w_{k},-\nabla^2\varphi\rangle_{L^2}\\
        &=\sum_{k=1}^\infty\langle v,w_k\rangle_{L^2}\langle w_k,\varphi\rangle_{L^2}\\
        &=\langle v,\varphi\rangle_{L^2}
    \end{align*}
    We used that $\langle w_k,-\nabla^2\varphi\rangle_{L^2}=\langle Dw_k,D\varphi\rangle_{L^2}=\lambda_k\langle w_k,\varphi\rangle_{L^2}$.
\end{proof}

\end{document}
