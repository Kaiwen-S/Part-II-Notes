\documentclass{article}
\usepackage{graphicx} % Required for inserting images
\usepackage[utf8]{inputenc}
\usepackage{amsmath,amsfonts,amssymb,amsthm}
\usepackage{enumerate,bbm}
\usepackage{leftindex}
\usepackage{tikz,tikz-cd,graphicx,color,mathrsfs,color,hyperref,boldline}
\usepackage{caption,float}
\usepackage[a4paper,margin=1in,footskip=0.25in]{geometry}
\newcommand{\e}{\varepsilon}
\usepackage{listings}
\usepackage{xcolor}

\usepackage{tabularx,capt-of}

\usepackage{blindtext}
%Image-related packages
\usepackage{graphicx}
\usepackage{subcaption}
\usepackage[export]{adjustbox}
\usepackage{lipsum}

%hyperref setup
\hypersetup{
    colorlinks=true,
    linkcolor=blue,
    filecolor=magenta,      
    urlcolor=cyan,
    pdftitle={Overleaf Example},
    pdfpagemode=FullScreen,
    }

%New colors defined below
\definecolor{codegreen}{rgb}{0,0.6,0}
\definecolor{codegray}{rgb}{0.5,0.5,0.5}
\definecolor{codepurple}{rgb}{0.58,0,0.82}
\definecolor{backcolour}{rgb}{0.95,0.95,0.92}

%Code listing style named "mystyle"
\lstdefinestyle{mystyle}{
  backgroundcolor=\color{backcolour}, commentstyle=\color{codegreen},
  keywordstyle=\color{magenta},
  numberstyle=\tiny\color{codegray},
  stringstyle=\color{codepurple},
  basicstyle=\ttfamily\footnotesize,
  breakatwhitespace=false,         
  breaklines=true,                 
  captionpos=b,                    
  keepspaces=true,                 
  numbers=left,                    
  numbersep=5pt,                  
  showspaces=false,                
  showstringspaces=false,
  showtabs=false,                  
  tabsize=2
}

%"mystyle" code listing set
\lstset{style=mystyle}

\theoremstyle{definition}
\newtheorem{defn}{Definition}[section]
\newtheorem{example}[defn]{Example}
\theoremstyle{remark}
\newtheorem{rem}{Remark}
\newtheorem{remS}[section]{defn}
\theoremstyle{plain}
\newtheorem{lem}[defn]{Lemma}
\newtheorem{thm}[defn]{Theorem}
\newtheorem{prop}[defn]{Proposition}
\newtheorem{fact}[defn]{Fact}
\newtheorem{crly}[defn]{Corollary}
\newtheorem{conj}[defn]{Conjecture}

%\newtheorem*{programming*}{Programming Task}

%\newtheorem{innercustomgeneric}{\customgenericname}
%\providecommand{\customgenericname}{}
%\newcommand{\newcustomtheorem}[2]{%
%  \newenvironment{#1}[1]
%  {%
%   \renewcommand\customgenericname{#2}%
%   \renewcommand\theinnercustomgeneric{##1}%
%   \innercustomgeneric
%  }
%  {\endinnercustomgeneric}
%}

%\newcustomtheorem{question}{Question}
%\newcustomtheorem{programming}{Programming Task}

\newcommand{\NN}{\mathbb{N}}
\newcommand{\ZZ}{\mathbb{Z}}
\newcommand{\QQ}{\mathbb{Q}}
\newcommand{\RR}{\mathbb{R}}
\newcommand{\CC}{\mathbb{C}}
\newcommand{\PP}{\mathbb{P}}
\newcommand{\FF}{\mathbb{F}}
\newcommand{\Hom}{\operatorname{Hom}}
\newcommand{\im}{\operatorname{im}}
\newcommand{\id}{\operatorname{id}}
\newcommand{\Ind}{\operatorname{Ind}}
\newcommand{\Res}{\operatorname{Res}}

\newcommand{\calD}{\mathcal{D}}

\newcommand{\sol}{\textit{Solution: }}

\title{Analysis of Functions}
\author{Kevin}
\date{January 2025}

\begin{document}
\maketitle
\section{Review of Basic Concepts}
q: (Owen's signature)\\
\subsection{Probmeas}
\begin{enumerate}
    \item The Lebesgue measure is inner regular, i.e., for all $A\in B(\RR^n)$, $\mu(A)=\sup\{\mu(K):K\subseteq A,\ K\text{ compact}\}$.
    \item Recall that $\mu$ extends to the $\mu$-completion of $\mathcal B$, which equals $M_\mu=\{B\cup A: B\in \mathcal B,A\in\mathcal N,\ \mu(A)=0\}$.
    \item For measurable functions $f:E\to F$, if $(F,\mathcal F)=(\RR,\mathcal B)$ (or $(\CC,\mathcal B)$), then we say that $f$ is Borel. This extends to maps taking values $\pm\infty$ if $f^{-1}({\pm\infty})\in\mathcal E$. If $f$ takes values in $[0,\infty]$, then we say $f\ge 0$ (non-negative).
    \item Recall MCT and DCT.
\end{enumerate}
\subsection{$L^p$-spaces and Approximation}
For $f:(E,\mathcal E,\mu)\to \RR$ (or $\CC$), define
\begin{align*}
    &\|f\|_{L^p}=\left(\int_E |f|^pd\mu\right)^{1/p},\ 1\le p<\infty\\
    &\|f\|_{L^\infty}=\operatorname{ess}\sup|f|=\inf\{\lambda\ge0:|f|\le \lambda\ \text{a.e.}\}
\end{align*}
We use $\|\cdot\|_\infty$ to denote the usual sup-norm. Define $L^p(E,\mu)=\{f:E\to\RR:\text{meas. }\|f\|_{L^p}<\infty\}$

Recall Riesz-Fischer Theorem. Also recall the spaces $C^k(\RR^n)$, the set of all functions on $\RR^n$ with continuous partial derivatives up to order $k$. We note that $C^\infty(\RR^n)=\bigcap_{k\ge 0}C^k(\RR^n)$. Note that this includes unbounded smooth functions. Use subscript $c$ to denote the linear subspaces consisting of compactly supported functions.

\begin{rem}
    $C_c^\infty(\RR^n)$ is non-empty, e.g.,
    \begin{align*}
        \psi(x)=\begin{cases}
            e^{\frac{1}{|x|^2-1}} & |x|<1\\
            0 & \text{o/w}
        \end{cases}
    \end{align*}
\end{rem}
\begin{thm}
    $C^\infty_c(\RR^n)$ is dense in $L^p(\RR^n,dx)$ for $1\le p<\infty$.
\end{thm}

\[\langle\text{-}:\tag{Owen's signature}\]

We admit the following lemma from PM.
\begin{lem}
    $C_c(\RR^n)$ is dense in $L^p$, $1\le p<\infty$.
\end{lem}
Recall convolution and basic properties including commutativity, associativity, and $\int_{\RR^n}f\ast gdx=\int_{\RR^n}f\int_{\RR^n}g$ (translation invariance and Fubini).

Recall multi-index notation $\alpha\in\ZZ_+^n$ is written as $\alpha=(\alpha_1,...,\alpha_n)$ with order $|\alpha|=\alpha_1+...+\alpha_n$ and we set $\alpha!=\alpha_1!\cdots\alpha_n!$, and for $x\in\RR^n$, we write $x^\alpha=x_1^{\alpha_1}\cdots x_n^{\alpha_n}$, so the partial differential operator becomes
\[D^\alpha=\dfrac{\partial^{|\alpha|}}{\partial x^\alpha}=\dfrac{\partial^{|\alpha|}}{\partial x_1^{\alpha_1}...\partial x_n^{\alpha_n}}\]
In particular $D_i=D^{(0,...,1,0,...,0)}$
\begin{thm}
    Let $f\in L^1_{\text{loc}}$ (i.e., $f1_K\in L^1$ for any $K\subseteq\RR^n$ compact), and $g\in C^k_c(\RR^n)$. Then $f\ast g\in C^k(\RR^n)$ and for all $0\le|\alpha|\le k$, we have
    \[D^\alpha(f\ast g)=f\ast(D^\alpha g)\]
\end{thm}
\begin{proof}
    Recall the translation operator $\tau_zh=h(\bullet-z)$, $z\in\RR^n$. Then for all $u\in\RR^n$,
    \[\tau_z(f\ast g)(x)=\int_{\RR^n}g(x-u-y)f(y)dy\]
    Since $g\in C_c(\RR^n)$ we have $|g(x-u-y)|\le\|g\|_\infty1_K$ for all $|u|\le 1$, where $K=K_{x,g}$ is a compact set, so tht $\|g\|_\infty1_K|f|$ gives an integrable upper bounde for the integrand. Since $g(x-u-y)\to g(x-y)$ as $u\to 0$, we have pointwise convergence. Apply DCT, we see that $f\ast g$ is cts.

    Now for $k=1$, we define difference operators $\forall e_i$ (standard basis vector) by $\Delta^i_hg(z)=\frac{g(z+he_i)-g(z)}{h}$ which converges to $D_ig(z)$. We can write
    \begin{align*}
        \Delta_h^i(f\ast g)(x)=\int_{\RR^n}\Delta_h^ig(x-y)f(y)dy
    \end{align*}
    Apply mean value inequality, get $|\Delta_h^ig(x-y)|\le\|D_ig\|_\infty1_K$. Apply DCT, $\Delta_h^i(f\ast g)\to f\ast(D_ig)$, which is continuous, so $f\ast g\in C^1$. Induction...
\end{proof}
\begin{prop}[Continuity of translation in $L^p$]
    Let $1\le p<\infty$. Then $\|\tau_zf-f\|_{L^p}\to 0$ as $z\to 0$ for all $f\in L^p$.
\end{prop}
\begin{proof}
    Hold for cts functions with compact support. Then apply $\varepsilon/3$-argument.
\end{proof}
\begin{thm}[Minkowski's inequality for integrals]
    Let $F:\RR^n\times\RR^n\to\RR$ be a measurable non-negative or $dx\otimes dx$-integrable function. Then
    \[\left\|\int_{\RR^n}F(x,\cdot)dx\right\|_{L^p}\le\int_{\RR^n}\|F(x,\cdot)\|_{L^p}dx\]
\end{thm}
\begin{proof}
    Example sheet.
\end{proof}
\begin{thm}[Mollification/Approximate identity]
    Let $\varphi\in C_c^\infty(\RR^n)$ be non-negative s.t. $\int_{\RR^n}\varphi(x)dx=1$. Define $\varphi_\varepsilon^{-n}\varphi(\cdot/\varepsilon)$, $\varepsilon>0$. Then for $1\le p<\infty$ and any $f\in L^p$,
    \[\|f-\varphi_\varepsilon\ast f\|_{L^p}\overset{\varepsilon\to0}{\longrightarrow} 0\]
\end{thm}
\[\circ-;\tag{Owen's signature}\]
\begin{proof}
    For $f\in L^p$, $x\in\RR^n$,
    \begin{align*}
        |\varphi_\varepsilon\ast f(x)-f(x)|&=\left|\int_{\RR^n}f(x-y)\varepsilon^{-n}\varphi(y/\varepsilon)dy-f(x)\right|\\
        &=\left|\int_{\RR^n}(f(x-\varepsilon u)\varphi(u)-f(x))du\right|\\
        &\le \int_{\RR^n}|f(x-\varepsilon)-f(x)|\varphi(u)du
    \end{align*}
    Apply Minkowski's inequality for integrals,
    \[\|\varphi_\varepsilon\ast f-f\|_{L^p}\le\int_{\RR^n}\|\tau_{\varepsilon u}f=f\|_{L^p}\varphi(u)du\]
    This converges to $0$ as $\varepsilon\to 0$ by DCT.
\end{proof}
In particular, since $C_c(\RR^n)$ is dense in $L^p$ and
$\{\varphi_\varepsilon\ast f:f\in C_c(\RR^n)\}\subseteq C_c^\infty(\RR^n)$, we have also proved that $C_c^\infty(\RR^n)$ is dense in $L^p$.

\subsection{Lebesgue's Differentiation Theorem}
\begin{defn}[Hardy-Littlewood maximal function]
    For $f\in L^1$, $x\in\RR^n$, let $$Mf(x)=\sup_{r>0}\dfrac{1}{|B_r(x)|}\int_{B_r(x)}f(y)dy$$
\end{defn}
\begin{lem}
    For $f\in L^1$, $Mf$ maps $\RR^n$ to $\RR$ and is Borel-measurable, and for all $\lambda>0$,
    \[|\{x:Mf(x)>\lambda\}|\le\dfrac{3^n}{\lambda}\|f\|_{L^1}\]
\end{lem}
\begin{proof}
    Define $A_\lambda=\{x:Mf(x)>\lambda\}$. %If $x\in A_\lambda$, then $\exists r_x>0$ s.t.
    %\[\dfrac{1}{|B_{r_x}(x)|}\int_{B_{r_x}(x)}|f(y)|dy>\lambda\]
    If $x_m\in A_\lambda^c$ s.t. $x_m\to x\in\RR^n$. Then
    \[\dfrac{1}{|B_{r_x}(x_m)|}\int_{\RR^n}1_{B_{r_x}(x_m)}|f(y)|dy\le\lambda\]
    by definition of $A_\lambda^c$. Apply DCT, get a contradiction, so $A^c_\lambda$ is closed, so $A_\lambda$ is open. This gives measurability.

    To prove the inequality, we use the inner regularity of \(\mu\) and take an arbitrary compact subset $K\subseteq A_\lambda$. $K$ has an open cover $\{B_{r_x}(x):x\in A_\lambda\}$. Pass to a finite subcover $B_1,...,B_N$ of such balls. By Wiener's covering lemma (ES1), reduce to a subcollection osf disjoint balls $B_1,...,B_k$ s.t. $$|K|\le 3^n\sum_{i=1}^k|B_i|=\dfrac{3^n}{\lambda}\sum_{i=1}^k\lambda|B_i|\le \dfrac{3^n}{\lambda}\sum_{i=1}^k\int_{B_i}|f(y)|dy\le \dfrac{3^n}{\lambda}\|f\|_{L^1}$$
    By inner regularity, $|A_\lambda|\le\sup\{|K|:K\subseteq A_\lambda\text{ cpt}\}\le\frac{3^n}{\lambda}\|f\|_{L^1}$.
\end{proof}
\begin{thm}
    Let $f\in L^1(\RR^n)$, $B_r(x)$ ball centered at $x$ with radius $r$. Then
    \[\lim_{r\to 0}\dfrac{1}{|B_r(x)|}\int_{B_r(x)}|f(y)-f(x)|dy=0,\ a.e.\tag{$\dagger$}\]
\end{thm}
\begin{rem}
    The set of points $A=\{x\in\RR^n:(\dagger)\}$ are called Lebesgue points of $f$.
\end{rem}
\[\langle(\rangle-:\tag{Owen's Signature}\]
\begin{proof}
    Consider
    \[\bar A_\lambda=\left\{x:\lim_{r\to 0}|B_r(x)^{-1}|\int_{B_r(x)}|f(y)-f(x)|dy>2\lambda\right\}\]
    Let $\varepsilon>0$. Pick $g\in C_c(\RR^n)$ s.t. $\|f-g\|_{L^1}<\varepsilon$. Then
    \[\dfrac{1}{|B_r(x)|}\int_{B_r(x)}|f(y)-f(x)|dy\le \dfrac{1}{|B_r(x)|}\int_{B_r(x)}|f(y)-g(y)|dy+\dfrac{1}{|B_r(x)|}\int_{B_r(x)}|g(y)-g(x)|dy+|f(x)-g(x)|\]
    $g$ is unif. cts, so the second term is small. If $x\in\bar A_\lambda$, either the first term or the third term is $>\lambda$. The third term is bounded using Markov's inequality
    \[\{x:|f(x)-g(x)|>\lambda\}\le\dfrac{\|f-g\|_{L^1}}{\lambda}<\varepsilon/\lambda\]
    For the first term, use HL-maximal inequality,
    \[|\{x:\text{first term}>\lambda\}|\le|\{x:M(f-g)(x)>\lambda\}|\le\dfrac{3^n}{\lambda}\|f-g\|_{L^1}\le 3^n\varepsilon/\lambda\]
    Therefore $|\bar A_\lambda|\le C\varepsilon$.
    So $|A^c|\le|\bigcup_n\bar A_{1/n}|\le\sum_n|\bar A_{1/n}|=0$.
\end{proof}
\begin{rem}
    In particular, for $f\in L^1(\RR), \lim_{h\to 0}\int_x^{x+h}f(y)dy=f(x)$ a.e.
\end{rem}
\begin{thm}[Egorov]
    Let $E\in B(\RR^n)$, $|E|<\infty$. Suppose $f_j:E\to\RR$ measurable s.t. $f_j\to f$ a.e. on $E$. Then
    \[\forall\varepsilon>0,\ \exists A_\varepsilon\ \text{s.t.}\ |E\setminus A_\varepsilon|<\varepsilon \text{ and } f_j\overset{\text{unif}}{\to} f\text{ on }A_\varepsilon\]
\end{thm}
\begin{proof}
    By discarding a null set, we may assume that $f_j\to f$ pointwise on $E$. Define
    \[E_k^m=\{x:\forall j>k,\ |f_j(x)-f(x)|<1/m\}\]
    $E_k^m$ is increasing as $k\to\infty$, and $\bigcup_k E_k^m=E$ by pointwise convergence.
    Pick a subsequence $k_m$ s.t. $|E\setminus E_{k_m}^m|\le\varepsilon 2^{-m}$. Define $A_\varepsilon=\bigcap_m E_{k_m}^m$. For all $x\in A_\varepsilon$, $|f_j(x)-f(x)|<1/m$ whenever $j>k_m$, so the convergence is uniform on $A_\varepsilon$, and
    \[|E\setminus A_\varepsilon|\le\sum_m|E\setminus E_{k_m}^m|\le\varepsilon\]
\end{proof}
\begin{thm}[Lusin]
    Let $|E|<\infty$, $f:E\to\RR$ (or $\CC$) Borel-measurable. Then
    \[\forall\varepsilon>0,\ \exists F_\varepsilon\text{ s.t. }|E\setminus F_\varepsilon|<\varepsilon\text{ and } f:F_\varepsilon\to\RR \text{ cts}\]
\end{thm}
\begin{rem}
    Note that $f$ is not necessarily continuous $F_\varepsilon$ when regarded as a map defined on $E$.
\end{rem}
\begin{proof}
    First prove it for simple functions $f=\sum_{i=1}^ma_i1_{A_i}$ (wlog assume $A_i$ disjoint), where $\bigcup A_i=E$. Use inner regularity to find compact sets $K_k\subseteq A_k$ s.t. $|A_k\setminus K_k|<\varepsilon/m$. $f$ is cts on $\bigcup_k K_k$ and $|E\setminus\bigcup K_k|\le\varepsilon$. For general $f$, approximate $f$ ptwise by simple functions on $E$. Pick $A_\varepsilon$ s.t. $|E\setminus A_\varepsilon|<\varepsilon/2$ s.t. $f_m\to f$ unif. by Egorov. Take $C_m$ compact s.t. $|E\setminus C_m|<\varepsilon 2^{-m-1}$. Then Take $F_\varepsilon=A_\varepsilon\cap\bigcap_mC_m$. Can check that $|E\setminus F_\varepsilon|\le\varepsilon$.
\end{proof}
Recall Riesz representation theorem in Hilbert spaces (bounded linear functionals can be written as taking inner product with a certain element).

Consider two measures $\mu,\nu$ on a measurable space $(E,\mathcal E)$. 
\begin{defn}
    We say that $\nu$ is absolutely continuous w.r.t $\mu$ if $\mu(A)=0\implies\nu(A)=0$ for any $A\in\mathcal E$. We write $\nu\ll\mu$. If $\nu\ll\mu$ and $\mu\ll\nu$ both hold, then we say that $\mu,\nu$ are mutually absolutely continuous.

    If there exists $B\in\mathcal E$ s.t. $0=\mu(B)=\nu(B^c)$, then we say that $\mu,\nu$ are mutually singular, and we write $\mu\perp\nu$.
\end{defn}

\begin{thm}[Radon-Nikodym]
    Let $\mu,\nu$ be finite measures on $(E,\mathcal E)$ s.t. $\nu\ll\mu$. Then $\exists w\in L^1(\mu)$, $w\ge 0$, s.t. for all $A\in\mathcal E$, $\nu(A)=\int_A d\nu=\int_Awd\mu$.
\end{thm}
\begin{rem}\
    \begin{enumerate}
        \item[1)] $w$ is unique.
        \item[2)] Will show that $\int_E hd\nu=\int_E hwd\mu$ for all $h\ge 0$ measurable. In particular, $w=d\nu/d\mu$ (Leibniz notation) is called the Radon-Nikodym derivative (or density) of $\nu$ w.r.t. $\mu$.
        \item[3)] The result extends to $\mu,\nu$ $\sigma$-finite.
    \end{enumerate}
\end{rem}
\begin{proof}[Proof (von Neumann)]
    Define $\alpha=\mu+2\nu$ and $\beta=2\mu+\nu$. On $L^2(\alpha)$, consider the map $\Lambda(f)=\int_Efd\beta$. This is bounded since \[|\Lambda(f)|\le\int_E|f|d\beta\le2\int_E|f|d\alpha\le 2\sqrt{\alpha(E)}\|f\|_{L^2(\alpha)}\]
    By Riesz, there exists $g\in L^2(\alpha)$ s.t. $\Lambda(f)=\int_Egfd\alpha$, i.e., $\int f(2d\mu+d\nu)=\int gf(d\mu+2d\nu)$ for all $f\in L^2(\alpha)$. Rearrange,
    \[\int_Ef(2-g)d\mu=\int_Ef(2g-1)d\nu\]
    Consider $A_j=\{x:g(x)\le\frac{1}{2}-\frac{1}{j}\}$, $j\in\NN$. Thus, by taking $f=1_{A_j}$,
    \[\dfrac32\mu(A_j)\le\int_E f(2g-1)d\mu\le-\dfrac2j\nu(A_j) \]
    So, $g\ge 1/2$ a.e. (w.r.t. both $\mu,\nu$). Similarly, by considering $\{x:g(x)\ge 2+1/j\}$, can prove that $g\le 2$ $\mu$-, $\nu$- a.e. We extend to simple functions and then to non-negative measurable functions by MCT.

    Consider $f=1_{\{x:g(x)=1/2\}}$, then get $\frac{3}{2}\mu(\{x:g(x)=1/2\})=0$, so $\nu(\{x:g(x)=1/2\})=0$.

    Let $h\ge 0$ measurable and define $f=\frac{h}{2g-1}$ and $w=\frac{2-g}{2g-1}$ (define it to be $0$ if $2g-1=0$). Now
    \[\int_E hd\nu=\int_Ef(2g-1)d\nu=\int_Ef(2-g)d\mu=\int_E hwd\mu\]
    Done by taking $h=1_A$. Note that $w\in L^1(\mu)$ since $\int_E wd\mu=\int_E 1_E wd\mu=\nu(E)<\infty$.
\end{proof}

\begin{rem}\
    \begin{enumerate}[1)]
        \item If $\PP$ is a prob measure on $B$ s.t. $\PP\ll dx$, where $dx$ is the Lebesgue measure, then $\frac{d\PP}{dx}=p(x)$ is the Lebesuge prob density of $\PP$. Moreover, there exists a unique decomposition $\PP=\PP_{\ll}+\PP_\perp$. (ESheet)
    \end{enumerate}
\end{rem}

\section{Dual Spaces}

\begin{defn}
    Let $X$ be a topological vector space. The top dual space is \[X'=\{\Lambda:X\to\RR \text{ linear and cts}\}\]
    If $(X,\|\cdot \|)$ is a normed space, then cts$\Leftrightarrow$bounded, and $X'$ has the operator norm. $X''=(X')'$ is called the bidual space.
\end{defn}
We have a point evaluation map $\Lambda\mapsto \Lambda(x)$ for each $x\in X$. [Note that $|\Lambda(x)|\le \|\Lambda\|_{X'}\|x\|_X\le C\|\lambda\|_{X'}$.] Can identify $x\mapsto (\Lambda\mapsto \Lambda(x))$ and regard $X$ as a subspace of $X''$.
\begin{defn}
    If $X''=X$, then say $X$ is reflexive.
\end{defn}
For $1<p<\infty$, consider the H\"older conjugate $q$ of $p$. Each $g\in L^q$ defines a linear functional on $L^p$ by
$\Lambda_g(f)=\int fgdx$. This is bounded since $|\Lambda_g(f)|\le\|g\|_{L^q}\|f\|_{L^p}$. In fact $\|\Lambda_g\|=\|g\|_{L^q}$. Get an embedding $L^q\subseteq (L^p)'$.
\begin{thm}
    For $1\le p<\infty$, $(L^p)'=L^q$. For $1<p<\infty$, $L^p$ is reflexive.
\end{thm}
\begin{rem}
    The result is false for $p=\infty$, so $L^1$ is not reflexive.
\end{rem}
\begin{lem}
    Under the hypothesis of the theorem, let $U\in (L^p)'$ positive, then $\exists g\in L^q$ s.t. $U(f)=\Lambda_g$ and $\|U\|=\|g\|_{L^q}$.
\end{lem}
\begin{proof}[Proof of lemma]
    On $\RR^n$ consider the finite measure $\mu$ with density $e^{-|x|^2}$. Further define, for each $A\in\mathcal B$, the set function
    \[\nu(A)= U(e^{-|x|^2/p}1_A)\ge 0\]
    To show that $\nu$ is countably additive, consider $A_m\in\mathcal B$ s.t. $A_m\downarrow\bigcap _mA_m=\varnothing$ and note $\nu(A_m)= U(e^{-|x|^2/p}1_{A_m})\le\|U\|\|\|e^{-|x|^2/p}1_{A_m}\|_{L^p}\to 0$ by DCT. Hence $\nu$ is a finite measure. Note that $\nu\ll \mu$. [If $\mu(A)=0$, then $\nu(A)\le\|U\|\|\|e^{-|x|^2/p}\|_{L^p}=\|U\|\mu(A)^{1/p}=0$.] By Radon-Nikodym, $\exists\mathcal G\in L^1(\mu)$ non-negative s.t. $\nu(A)=\int_A\mathcal G d\mu$. Consider a simple function $F=\sum_k a_k1_{A_k}$. Compute $$U(e^{-|x|^2/p}F)=\int\sum_k a_k1_{A_k}\mathcal G e^{-|x|^2}dx=\int e^{-|x|^2/p}\sum_{k}a_k1_{A_k}\mathcal G e^{-|x|^2/q}dx$$
    Set $g=\mathcal G e^{-|x|^2/q}$. Note that $\{e^{-|x|^2/p}F:F\text{ simple}\}$ is dense in $L^p\cap \{\ge 0\}$. Since $fg\in L^1$ [Note $\int |fg|=\int|f|g=U(|f|)\le \|U\|\|f\|_{L^p}<\infty$]. Decomposing $fg$ into $f_+g-f_-g$ and taking limits, we see that $U(f)=\int fg$ for all $f\in L^p$.

    (cf. ES1), have $\|g\|_{L^q}=\sup\{\int |fg|:\|f\|_{L^p}\le 1\}= U(|f|)\le\|U\|<\infty$ and $\|U\|=\sup_{\|f\|_{L^p}\le 1}|\int fg|\le\|g\|_{L^q}$ by H\"older, so $\|U\|=\|g\|_{L^q}$.
\end{proof}
\begin{proof}[Proof of thm]
    Note (ES) that $\Lambda\in (L^p)'$ can be uniquely decomposed as $\Lambda_+-\Lambda_-$, where $\Lambda_\pm$ are positive linear functionals ($\Lambda_\pm(h)\ge 0$ for all $h\ge 0$ a.e.). Apply the preceding lemma.
\end{proof}

We can characterize duals of subspace of $L^\infty$, e.g., any finite measure defines a linear functional in $C_c(\RR^n)'$ by $f\mapsto \mu(f)$.

\begin{defn}
    A measure is regular on $\RR^n$ if $\forall\varepsilon>0,\forall A\in\mathcal B,\exists C$ closed, $D$ open s.t. $C\subseteq A\subseteq D$ s.t. $\mu(D\setminus C)<\varepsilon$
\end{defn}
\begin{thm}[Riesz]
    Let $\Lambda\in (C_c(\RR^n))'$ be positive. Then $\exists$ a $\sigma$-algebra $M\supseteq\mathcal B$ and a regular measure $\mu$ on $M$ s.t. $\Lambda(f)=\int_{\RR^n}fd\mu$.
\end{thm}
Proof omitted.
\[(B\tag{Owen's Signature}\]

\section{Weak and Weak\(*\) Topology}
\begin{defn}
    A semi-norm $p$ on a vector space $X$ is a functional $p:X\to[0,\infty)$ s.t.\begin{enumerate}[1)]
        \item $\forall x,y\in X, \ p(x+y)\le p(x)+p(y)$;
        \item $\forall x\in X,\forall \lambda\in \RR \text{ (or $\CC$)},\ p(\lambda x)=|\lambda|p(x)$
    \end{enumerate}
\end{defn}
The collection $\mathcal P$ of seminorms introduces a `locally convex' topology $\tau_{\mathcal P}$ generated by
$$V_x(p,n)=\{y\in X: p(y-x)<1/n\}$$
for $x\in X$, $p\in \mathcal P$, $n\in\NN$.
\begin{defn}
    The family $\mathcal P$ is said to separate points if for any $0\neq x\in X$, there exists $p\in \mathcal P$ s.t. $p(x)\neq 0$.
\end{defn}
Therefore (ES) a sequence $x_n$ converges in $\tau_\mathcal P$ iff for all $p\in \mathcal P$, $p(x_n-x)\to 0$ as $n\to\infty$. This topology is not generally metrizable unless $\mathcal P$ is countable. In that case a metric is given by
\[d_{\mathcal P}(x,y)=\sum_{i=1}^\infty\dfrac{p_i(x-y)}{2^{-i}(1+p_i(x-y))}\]
\begin{defn}
    We say that $(X,\tau_{\mathcal P})$ is a locally convex topological vector space (LCTVS). If it's complete, then we call it a Frechet space.
\end{defn}
Consider the semi-norms given by $p_\Lambda(x)=|\Lambda(x)|$.
\begin{defn}
    The topology $\tau_{\mathcal P}$ induced by $\mathcal P=\{p_\Lambda:\Lambda\in X'\}$ is called the weak topology $\tau_w$. We say that $x_n\to x$ weakly in $X$ or $x_n\rightharpoonup x$ if $\Lambda(x_n)\to \Lambda(x)$ for all $\Lambda\in X'$.
\end{defn}
\begin{defn}
    On the dual space $X'$, we can consider the weak-$\ast$ topology $\tau_{w^\ast}$ induced by $\mathcal P=\{p_x(\Lambda)=|\Lambda(x)|:x\in X\}$. Note that $\Lambda_n\to\Lambda$ weak-$\ast$, or $\Lambda_n\rightharpoonup^\ast\Lambda$ if $\Lambda_n(x)\to\Lambda(x)$ for all $x\in X$.
\end{defn}
\begin{example}\
    \begin{enumerate}[1)]
        \item Consider $L^p(\RR,dx)$. $f_n\to f$ weakly in $L^p$ iff \[\forall g\in L^q,\ \int_{\RR^n}f_ngdx\to\int_{\RR^n}fgdx\tag{$\dagger$}\]
        Since $L^p=(L^q)'$, $\Lambda_{f_n}\to\Lambda_{f}$ weak-$\ast$ (or $f_n\to f$ weak-$\ast$) iff ($\dagger$) holds. For $1<p<\infty$, weak convergence and weak-$\ast$ convergence coincide. (This is true in any reflexive space.)
        \item Consider a probmeas on a metric space $D$ (with Borel $\sigma$-algebra). Let $C_b(D)$ denote the Banach space of bounded cts functions on $D$. Then $\mu(f)$ defines an element $C_b(D)'$. A sequence of probmeas $\mu_n$ converges to $\mu$ in $\tau_{w^\ast}$ if $\mu_n(f)\to \mu(f)$ for all $f\in C_b(D)$. (i.e. weak convergence of laws)
    \end{enumerate}
\end{example}
%When is someone gonna tell him that 'and' has an 'n' in it?
Recall Arzela-Ascoli. A sufficient condition for equicontinuity is given by H\"older continuity, defined as
\[\|f\|_{C^{0,\gamma}}=\|f\|_\infty+\sup_{x\neq y}\dfrac{|f(x)-f(y)|}{|x-y|^\gamma}\]
where $0<\gamma<1$ and 
\[\|f\|_{C^{m,\gamma}}=\sum_{0\le|\alpha|\le m}\|D^\alpha f\|_\infty+\max_{|\alpha|=m}\sup_{x\neq y}\dfrac{|D^\alpha f(x)-D^\alpha f(y)|}{|x-y|^\gamma}\]
So $\{f:\|f\|_{C^{0,\gamma}}\le 1\}$ is compact in $C([0,1])$ by Arzela-Ascoli.

\begin{thm}[Banach-Alaoglu]
    Let $X$ be a normed space. The unit ball $B_1=\{\Lambda\in X':\|\Lambda\|_{X'}\le 1\}$ of $X'$ is compact in weak-$\ast$ topology
\end{thm}
\begin{rem}
    In $(C_b(D))'$ any sequence of probmeas has a weak-$\ast$ convergent subsequence.
\end{rem}
We will prove Banach Alaoglu for $X$ separable.
\[\ddot{T}\tag{Owen's Signature}\]
\begin{lem}
    For a countable dense subset $D=\{x_1,...,x_n\}$ of $X$, consider seminorms $\tilde{\mathcal{P}}=\{p_{x_k}(\Lambda)=|\Lambda(x_k)|:k\in\NN\}$ with induced topology $\tau_{\tilde{\mathcal P}}$. Then $\tau_{w^*}=\tau_{\tilde{\mathcal P}}$ coincide as topologies on $B_1'=\{\Lambda\in X':\|\Lambda\|_{X'}\le 1\}$ and are metrized by
    \[d_{\tilde{\mathcal P}(\Lambda,\Lambda')}=\sum_{k=1}^\infty \dfrac{|\Lambda(x_k)-\Lambda'(x_k)|}{2^k(1+|\Lambda(x_k)+\Lambda'(x_k)|)}\]
\end{lem}
\begin{proof}
    The open sets for $\tau_{\tilde{\mathcal P}}$ are generated by $V(x_k,m)=\{\Lambda:|\Lambda(x_k)|<1/m\}$. To prove that the two topologies are equivalent, it suffices to show that $V(x,n)$ contain some $V(x_k,m)$ for all $x\in X,n\in\NN$. Suppose $x\in X\setminus D$ and pick $x_k\in D$ s.t. $\|x-x_k\|<\varepsilon$. For $\Lambda\in V(x_k,m)$ we have $|\Lambda(x)|\le|\Lambda(x_k-x)|+|\Lambda(x_k)|\le\|\Lambda\|_{X'}\varepsilon+1/m<n$ when $\varepsilon$ is sufficiently small and $m$ sufficiently large, so $V(x_k,m)\subseteq V(x,n)$.

    If $\Lambda_j(x_k)\overset{j\to\infty}{\longrightarrow}\Lambda(x_k)$ for all $k$ then $d_{\tilde{\mathcal P}}(\Lambda_j,\Lambda)\to 0$ by DCT applied to counting measure on $\NN$.
\end{proof}
\begin{thm}
    Let $\Lambda_j\in B_1'$. Then $\exists\Lambda\in B_1'$ s.t. $\Lambda_{j_k}\to\Lambda$ weak-$\ast$.
\end{thm}
\begin{proof}
    Let $D$ be a countable dense subset of $X$. Since $|\Lambda_j(x_k)|\le\|x_k\|<\infty$. Diagonalization argument. Find convergent subsequences $\Lambda_{i,j}$ Find a $\Lambda$ as the limit of $\Lambda_{j,j}$. Need to show linearity and continuity. Note that $\Lambda$ is unif cts on $D$. [If $x,y\in D$ with $\|x-y\|<\varepsilon/2$, then for all $j$ sufficiently large, $|\Lambda_{j,j}(x)-\Lambda(x)|,|\Lambda(y)-\Lambda_{j,j}(y)|<\varepsilon/4$] Apply triangle inequality to $|\Lambda(x)-\Lambda(y)|$. Note that $\Lambda_{j,j}$ is uniformly Lipschitz.
    By uniform continuity, we can extend $\Lambda$ to a unif cts function on $X$.

    To show linearity, let $x,y\in X$, $z=x+ay$ for $a\in\RR$ (or $\CC$) and pick $x',y',z'\in D$ s.t. $\|x-x'\|+|a|\|y-y;\|+\|z-z'\|<\delta$.

    Apply a big triangle inequality.
    \begin{align*}
        |\Lambda(z)-\Lambda(x)-a\Lambda(y)|&\le|\Lambda(z)-\Lambda(z')|+|\Lambda(x)-\Lambda(x')|+|a||\Lambda(y)-\Lambda(y')|\\
        &\le +|\Lambda(z')-\Lambda_{j,j}(z')|+\ ..................................|
    \end{align*}
    each term is small either by unif continuity or unif convergence or linearity of $\Lambda$ on $D$.

    Need to show that $\|\Lambda\|\le 1$ ($|\Lambda(x)|\le|\Lambda(x-x')|+|\Lambda(x')|$) and that the convergence holds on $X$.
\end{proof}
\section{The Hahn-Banach theorem and its consequences}
\begin{defn}
    A functional $p:X\to\RR$ on a real vector space is called sub-linear if \begin{enumerate}[(i)]
        \item $p(x+y)\le p(x)+p(y)$ for all $x,y\in X$
        \item $p(tx)=tp(x)$ for all $t\ge 0$, $x\in X$.
    \end{enumerate}
\end{defn}

\begin{lem}[Bounded extension]
    Let $X$ be a real vector space and $p:X\to\RR$ sub-linear Let $M\subsetneq X$ be a vector subspace, and for $x\in X\setminus M$ define $\tilde{M}=\operatorname{span}(M,x)=\{M+cx:c\in\RR\}$. If $l:M\to \RR$ is a linear form s.t. $l(x)\le p(x)$ for all $x\in M$, then there exists $\tilde l:\tilde M\to\RR$ linear s.t. $\tilde l_{M}=l$ and $\tilde l(x)\le p(x)$ for all $x\in\tilde M$.
\end{lem}
\begin{proof}
    Let $y_1,y_2\in M$. $l(y_1)+l(y_2)=l(y_1+y_2)\leq p(y_1+y_2)\le p(y_1-x)+p(y_2+x)$ for all $x\in X\setminus M$. Rearrange, $l(y_1)-p(y_1-x)\le l(y_2)-p(y_2+x)$. Take sup/inf, 
    \[\sup\{l(y)-p(y-x):y\in M\}\le a\le \inf\{p(y+x)-l(y):y\in M\}\tag{$\ast$}\] for some $a\in\RR$.
    If $z\in\tilde M$, then it has a unique decomposition $z=y+\lambda x$ for some $\lambda\in\RR$. Define $\tilde l(z)=\tilde l(y+\lambda x)=l(y)+\lambda a$. To see $\tilde l\le p$ on $\tilde M$, for $\lambda>0$, write $\tilde l (y+\lambda x)=\lambda (l(y/\lambda)+a)\overset{(\ast)}{\le}\lambda(l(\frac y\lambda)+p(\frac{y}{\lambda}+x)-l(\frac y\lambda))=p(y+\lambda x)$. For $\lambda<0$, let $\mu=-\lambda$ and $\tilde l(y+\lambda x)=\mu(l(\frac y\mu-a))\le \mu(l(\frac y\mu)-l(\frac y\mu)+p(\frac y\mu-x))$.
\end{proof}
To extend $l$ to all of $X$ (for $X$ separable), we can apply the extension lemma inductively to $M_n=\operatorname{span}(M;x_1,...,x_n)$, where $(x_n)_{n\in\NN}$ is a countable dense subset of $X$.

In general, consider $S=\{(N,\tilde l):M\subseteq N\subseteq X\text{ vec.sp. },\  \tilde l|_M=l,\ \tilde l\le p\text{ on }N\}$. Apply Zorn's lemma.

\begin{thm}[Hahn-Banach]
    Let $X$ be a real vector space and $p:X\to\RR$ a sublinear functional. For $M\subseteq X$ vec. subspace, let $l:M\to\RR$ be a linear functional s.t. $l(x)\le p(x)$ for all $x\in M$. Then there exists an extension $\tilde l:X\to\RR$ (linear) s.t. $\tilde l(x)\le p(x)$ for all $x\in X$.
\end{thm}
\begin{rem}
    Extensions need not be unique. If $X$ is non-separable, the result depends on Axiom of Choice.
\end{rem}

\begin{crly}[Norming Functional]
    Let $X$ be a normed linear space. For all $x\in X$, there exists a linear functional $\Lambda=\Lambda_x\in X'$ s.t. $\|\Lambda\|=1$ and $|\Lambda(x)|=\|x\|_X$. In particular, if $\Lambda(x-y)=0$ for all $\Lambda\in X'$, then $x=y$.
\end{crly}
\[\subset]-:\langle\tag{Owen's (infinitely handsome) signature}\]
\begin{proof}
    For $x\in X$ define the vector subspace $M=\{cx:c\in\RR\}$ and consider the linear functional $l(cx)=c\|x\|_X$, so $|l(y)|\le p(y)=\|y\|_X$. By Hahn-Banach, there exists $\Lambda=\Lambda_x:X\to\RR$ s.t. $|\Lambda_x(y)|\le\|y\|_X$, so $\Lambda\in X'$ and $\|\Lambda\|\le 1$. Note that $\|\Lambda\|_{X'}\ge \sup_{y\in M\cap B_X}|l(y)|\ge \|x\|_X$
\end{proof}
\begin{crly}
    The canonical injection of $i:X\hookrightarrow X''$ given by $x\mapsto(\Lambda\to\Lambda(x))$ is an isometric embedding.
\end{crly}
\begin{proof}
    Consider $\|i(x)\|_{X''}=\sup_{\|\Lambda\|\le 1}|\Lambda(x)|\le\|x\|_X$. By taking a norming functional, $\|i(x)\|_{X''}\ge\|\Lambda_x(x)\|=\|x\|$.
\end{proof}
If $X$ is reflexive, then $X$ is isometrically isomorphic to $(X')'$ which is complete. (i.e., reflexive normed lienar space is Banach.)
If $X$ is not reflexive, then $X''$ provides (up to iso) the completion of $X$ for $\|\cdot\|_X$.

In particular, if $X$ is reflexive ,then the weak topology coincides with the weak-$\ast$ topology on $(X')'$, so Banach-Alaoglu the unit ball $B_X$ is compact in $\tau_w$.
\begin{thm}[Hyperplane Separation]
    Let $A,B$ be non-empty disjoint convex sets in a Banach space $X$ over $\RR$.
    \begin{enumerate}[(i)]
        \item If $A$ is open, then $\exists\Lambda\in X'$ and $\gamma\in\RR$ s.t. $\Lambda(a)<\gamma\le \Lambda(b)$ for all $a\in A$ and $b\in B$.
        \item If $A$ is compact and $B$ is closed, then $\exists\Lambda\in X'$ and $\gamma_1,\gamma_2\in\RR$ s.t. $\Lambda(a)<\gamma_1<\gamma_2<\Lambda(b)$ for all $a\in A$, $b\in B$.
    \end{enumerate}
\end{thm}
\begin{proof}
    (i): Pick $a_0\in A$, $b_0\in B$ and let $x_0=b_0-a_0$. Define $C=A-B+x_0$. $C$ is convex, $0\in C$, $x_0\notin C$. $C$ is open. Consider the Minkowski functional defined as 
    \[p_C(x)=\inf\{t>0:x/t\in C\}\]
    Can show (ES) that $p_C$ is 
    \begin{itemize}
        \item sublinear on $X$
        \item there exists $k>0$ s.t. $p_C(x)\le k\|x\|$
        \item $p_C(x)<1$ for $x\in C$ and $p_C(x)\ge 1$ for $x\notin C$.
    \end{itemize}
     Take $$M=\{tx_0:t\in\RR\}$$ and consider the linear functional $l:M\to\RR,\ tx_0=t$. Then $l$ is dominated by $p_C$ since 
     \[l(tx_0)=t\le tp_C(x_0)=p(tx_0)=p_C(tx_0)\]
     for $t>0$ and $l(tx_0)=t\le 0\le p_C(tx_0)$. By Hahn-Banach, there exists $\Lambda:X\to\RR$ s.t. $-k\|x\|\le -p_C(x)\le \Lambda(x)\le p_C(x)\le k\|x\|$ for all $x\in X$, so $\Lambda\in X'$. Pick $a\in A$, $b\in B$. Note that \[\Lambda(a)-\Lambda(b)+\Lambda(x_0)=\Lambda(a-b+x_0)\le p_C(a-b+x_0)<1\]
     So $\Lambda(a)<\sup\Lambda(A)\le \Lambda(b)$.

     (2): $\Lambda(A)$ is compact in $\RR$ and $d=\|A-B\|_X$. Then consider $\tilde A_d=A+B_{d/2}$, where $B_{d/2}=\{y:\|y\|<d/2$ still disjoint from $B$. Apply (1).
\end{proof}
\[???\tag{Owen's missing Signature}\]
\section{Generalized Functions and Distributions}
Consider a topological vec. space $X\subseteq\bigcap_{q\ge 1}L^q(\Omega,dx)$, where $\Omega$ is an open subset of $\RR^n$. Suppose $X$ contains $C_c^\infty(\Omega)$. Let $f\in L^q$, then obtain a linear functional on $X$ given by $\Lambda_f(g)=\int_\Omega fgdx$, $g\in X$. If the embedding $X\hookrightarrow L^q$ is cts, then $\Lambda_f\in X'$. Note that $g=\phi_\epsilon$ is contained in $X$, mollification implies $\Lambda_f=0\implies f=0$ a.e.. So we can identify $\Lambda_f$ with $f$  and study the weak-$\ast$ topology of $X'$ on $L^p$.


Define seminorms on $C^\infty(\Omega)$, $p_N(\phi)=\max_{0\le |\alpha|\le n}\sup_{x\in K_N}|D^\alpha\phi(x)|$, where $K_i\subseteq K_{i+1}$ and $\bigcup _iK_i=\Omega$. We define the Frechet space $\mathcal E(\Omega)=(C^\infty(\Omega),\tau_{\mathcal P})$, where $\mathcal P=\{p_N:N\in\NN\}$. [Note that $\mathcal E(\Omega)$ may contain non-integrable functions.] 

\begin{thm}
    There exists a topology $\tau$ on $C^\infty_c(\Omega)$ s.t.
    \begin{enumerate}[(1)]
        \item vector space operations are cts
        \item a sequence $\phi_j\overset{j\to\infty}{\to} 0$ iff $\exists K\subseteq\Omega$ compact s.t. $\operatorname{supp}(\phi_j)\subseteq K$ for all $j$ and $D^\alpha\phi_j\to 0$ unif. on $K$ for all $0\le |\alpha|<\infty$.
        \item If $T:C^\infty_c(\Omega)\to\RR$ (or $\CC$) is linear, then it's cts iff $T(\phi_j)\to 0$ for all $\phi_j\to 0$ in $\tau$.
    \end{enumerate}
\end{thm}
proof omitted.

\begin{defn}
    We define $\calD=\calD(\Omega)=(C^\infty_c(\Omega),\tau)$, the space of test functions.
\end{defn}
For each $\phi\in C^\infty_c(\Omega)$, define $e^{-j}\phi(\cdot/j)$, then $e^{-j}\phi(\cdot/j)\to 0$ in $D$, but $j^{-2025}\phi(\cdot/j)$ does not converge to $0$ in $D$.

\begin{defn}
    Call $\phi\in C^\infty(\RR^n)$ rapidly decreasing if $\sup_{x\in\RR^n}(1+|x|)^N|D^\alpha\phi(x)|<\infty$ for all $0\le|\alpha|<\infty$ and all $N\in\NN$.
\end{defn}
[Note that $e^{-|x|^2}$ is rapidly decreasing but $(1+|x|)^{-2025}$ is not.]

Define seminorms $\tilde{\mathcal P}=\{\tilde p_N:N\in\NN\}$ with 
\[\tilde p_N=\max_{0\le|\alpha|\le N}\sup_{x\in \RR^n}(1+|x|)^N|D^{\alpha}\phi(x)|\]
Define Frechet space $\mathcal{S}(\RR^n)=(\{\phi\text{ rapidly decreasing}\},\tau_{\tilde{\mathcal P}})$.
This is metrizable since $\tilde{\mathcal P}$ is countable. This is called the Schwartz class.

Clearly $\calD(\Omega)\subsetneq \mathcal E(\Omega)$, $D(\RR^n)\subsetneq\mathcal S(\RR^n)\subsetneq\mathcal E(\RR^n)$ with continuous embedding (ES). We can now define $\calD'(\Omega)=\{T:\calD(\Omega)\to\RR\text{ (or $\CC$) linear and cts}\}$, the space of Schwartz distributions.
We also define $\mathcal E(\Omega)=\{T:\mathcal E(\Omega)\to \RR\text{ (or $\CC$) linear and cts}\}$ the space of compactly supported Schwartz distributions. Finally for $\Omega=\RR^n$ we define
$\mathcal S(\RR^n)=\{T:\mathcal S(\RR^n)\to\RR\text{ (or $\CC$) linear and cts}\}$ the space of tempered distributions. These spaces are equipped with their weak-$\ast$ topologies of pointwise convergence on $\cal D,\mathcal E,\mathcal S$ resp. Have cts embeddings
$\mathcal E'\subset\mathcal D'$, and $\mathcal E(\RR^n)\subset \mathcal S(\RR^n)\subset \calD'(\RR^n)$.

\begin{example}
    Consider $\delta_x(\phi)=\phi(x)$, then if $\phi_j\to0$ in $\cal D,\mathcal E,\mathcal S$ , then $\delta_x(\phi_j)=\phi_j(x)\to 0$ as $j\to\infty$, so $\delta_x\in\mathcal E',\mathcal D',\mathcal S'$.
\end{example}
\[O\frown O\tag{Owen's Signature}\]

Let $f\in L^1_{\text{loc}}(\Omega)$. Then $T_f(\phi)=\int_\Omega f\phi dx$, $\phi\in\calD(\Omega)$. Have $T_f\in\calD'(\Omega)$ since for $\phi_j\to 0$ in $\calD(\Omega)$ we have $T_f\phi_j\to0$ by DCT with dominating function $\sup_{j\in\NN}\|\phi_j\|_\infty1_K|f|\in L^1$ ($K$ compact). Also, $T_f=0$ in $\calD'(\Omega)$ still implies $f=0$ a.e. by applying the mollification theorem (ES) to $f1_{B(x)}$ where $B(x)$ is a ball in $\Omega$ containing $x\in\Omega$, so $L^1_{\text{loc}}\subseteq\calD'$

If $\varphi_\e=\e^{-n}\varphi(\cdot/\e)$, $\varphi\ge 0$, smooth, compactly supported, normalized, then for $g\in\calD(\RR^n)$, $T_{\varphi_\e}(g)=g\ast\varphi_\e(0)\to g(0)=\delta_0(g)$ as $\e\to0$. So $T_{\varphi_\e}\overset{\e\to0}{\to}\delta_0$ in $\calD'(\RR^n)$.
\subsection{Generalized (Distributional) Derivatives in $\calD'(\Omega)$)}
Let $f\in C^1(\Omega)$, then $D_if\in L^1_{\text{loc}}$. Consider $T_{D_if}$. Let $\varphi\in\calD(\Omega)$
\[\int (D_if)\varphi dx\overset{ibp}{=}-\int fD_i\varphi dx=-T_f(D_i\varphi)\]
So we define (for all multi-index $\alpha$) the generalized derivative of any $T\in\calD'(\Omega)$ as
\[(D^\alpha T)(\varphi)=(-1)^{|\alpha|}T(D^\alpha\varphi)\]
so $D^\alpha T\in\calD'(\Omega)$. If $T=T_f$ and $D^\alpha T=T_g$ for some $f,g\in L^1_{\text{loc}}$, then say $g=D^\alpha_wf$ the weak partial derivative of $f$.

\begin{example}
    Let $f(x)=x1_{\{x>0\}}$. Consider $T_f$.
    \[DT_f(\varphi)=-T_f(\varphi')=-\int_0^\infty x\varphi'(x)dx\overset{ibp}{=}\int_0^\infty\varphi(x)dx=\int_\RR H\varphi dx\]
    where $H$ is the Heaviside function, so $H$ is the weak derivative of $f$.

    Consider the second derivative.
    \[D^2T_f(\varphi)=DT_H(\varphi)=-T_H(\varphi')=-\int_0^\infty\varphi'=\varphi(0)=\delta_0(\varphi)\]
    $\delta_0$ cannot be represented by locally integrable functions.

    Have $D^3T_f(\varphi)=-\delta_0(\varphi')=-\varphi'(0)$ which is a Schwartz distribution but not a measure.
\end{example}
\subsection{Multiplication of Distributions with Smooth Functions}
If $f\in L^1_{\text{loc}}$ and $a\in C^\infty(\Omega)$, the $T_{af}(\varphi)=\int_\Omega af\varphi=T_f(a\varphi)$. Note that $a\varphi\in\calD$ if $\varphi\in\calD$, so we define
\[(aT)(\varphi)=T(a\varphi)\]
for $T\in\calD'(\Omega)$ and $a\in C^\infty(\Omega)$.

\subsection{Compactly Supported Distributions}
\begin{prop}
    A linear map $T:\mathcal E(\Omega)\to\RR$ (or $\CC$) is cts iff there exists $K\subseteq\Omega$ compact, $N\in\NN$, and $C>0$ s.t. for all $\varphi\in\mathcal E(\Omega)$
    \[|T(\varphi)|\le C\max_{0\le|\alpha|\le N}\sup_{x\in K}|D^\alpha\varphi(x)|\tag{$\dagger$}\]
\end{prop}\
\begin{proof}
    Suppose $(\dagger)$ holds and $\phi_j\to 0$ in $\mathcal E(\Omega)$. By defn of $\tau_{\mathcal P}$, for $j$ large enough, $K\subseteq K_j$ and RHS of $(\dagger)$ with $\phi=\phi_j$ converges to $0$, so $T(\phi_j)\to 0$, so $T\in\mathcal E'(\Omega)$.

    Conversely, assume $T$ is cts but $(\dagger)$ fails. If $K_j\subseteq K_{j+1}$ is any exhaustion of compact sets of $\Omega$, we obtain a sequence $\varphi_j\in\mathcal E'(\Omega)$ s.t. \[|T(\varphi_j)|\ge j\max_{0\le |\alpha|\le j}\sup_{x\in K_j}|D^\alpha(\varphi_j)|\]
    Define $\psi_j=\frac{\varphi_j}{|T(\varphi_j)|}$. Have
    \[|D^\beta\psi_j(x)|\le\frac1j\frac{|D^\beta\varphi_j(x)|}{\max_{0\le|\alpha|\le j}\sup_{x\in K_j}|D^\alpha\varphi_j(x)|}\overset{e.v.}{\le}\frac1j\to 0\]
    So $\psi_j\to 0$ in $\mathcal E(\Omega)$ but $T(\psi_j)=1$ for all $j$. Contradiction.
\end{proof}
\begin{defn}
    Say $T\in \calD'(\Omega)$ has support in a closed set $K\subseteq \Omega$ if $T(\varphi)=0$ whenever $\varphi\in C^\infty_c(\Omega\setminus K)\subseteq C^\infty_c(\Omega)$.
\end{defn}
The last proposition implies that $T\in\mathcal E'(\Omega)$ is supported in some compact subset of $\Omega$.
\end{document}
