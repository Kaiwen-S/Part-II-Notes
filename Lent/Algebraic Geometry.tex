\documentclass{article}
\usepackage{graphicx} % Required for inserting images
\usepackage[utf8]{inputenc}
\usepackage{amsmath,amsfonts,amssymb,amsthm}
\usepackage{enumerate,bbm}
\usepackage{leftindex}
\usepackage{tikz,tikz-cd,graphicx,color,mathrsfs,color,hyperref,boldline}
\usepackage{caption,float}
\usepackage[a4paper,margin=1in,footskip=0.25in]{geometry}

\usepackage{listings}
\usepackage{xcolor}

\usepackage{tabularx,capt-of}

\usepackage{blindtext}
%Image-related packages
\usepackage{graphicx}
\usepackage{subcaption}
\usepackage[export]{adjustbox}
\usepackage{lipsum}

%hyperref setup
\hypersetup{
    colorlinks=true,
    linkcolor=blue,
    filecolor=magenta,      
    urlcolor=cyan,
    pdftitle={Overleaf Example},
    pdfpagemode=FullScreen,
    }

%New colors defined below
\definecolor{codegreen}{rgb}{0,0.6,0}
\definecolor{codegray}{rgb}{0.5,0.5,0.5}
\definecolor{codepurple}{rgb}{0.58,0,0.82}
\definecolor{backcolour}{rgb}{0.95,0.95,0.92}

%Code listing style named "mystyle"
\lstdefinestyle{mystyle}{
  backgroundcolor=\color{backcolour}, commentstyle=\color{codegreen},
  keywordstyle=\color{magenta},
  numberstyle=\tiny\color{codegray},
  stringstyle=\color{codepurple},
  basicstyle=\ttfamily\footnotesize,
  breakatwhitespace=false,         
  breaklines=true,                 
  captionpos=b,                    
  keepspaces=true,                 
  numbers=left,                    
  numbersep=5pt,                  
  showspaces=false,                
  showstringspaces=false,
  showtabs=false,                  
  tabsize=2
}

%"mystyle" code listing set
\lstset{style=mystyle}

\theoremstyle{definition}
\newtheorem{defn}{Definition}[section]
\newtheorem{example}[defn]{Example}
\theoremstyle{remark}
\newtheorem{rem}{Remark}
\newtheorem{remS}[section]{defn}
\theoremstyle{plain}
\newtheorem{lem}[defn]{Lemma}
\newtheorem{thm}[defn]{Theorem}
\newtheorem{prop}[defn]{Proposition}
\newtheorem{fact}[defn]{Fact}
\newtheorem{crly}[defn]{Corollary}
\newtheorem{conj}[defn]{Conjecture}

%\newtheorem*{programming*}{Programming Task}

%\newtheorem{innercustomgeneric}{\customgenericname}
%\providecommand{\customgenericname}{}
%\newcommand{\newcustomtheorem}[2]{%
%  \newenvironment{#1}[1]
%  {%
%   \renewcommand\customgenericname{#2}%
%   \renewcommand\theinnercustomgeneric{##1}%
%   \innercustomgeneric
%  }
%  {\endinnercustomgeneric}
%}

%\newcustomtheorem{question}{Question}
%\newcustomtheorem{programming}{Programming Task}

\newcommand{\NN}{\mathbb{N}}
\newcommand{\ZZ}{\mathbb{Z}}
\newcommand{\QQ}{\mathbb{Q}}
\newcommand{\RR}{\mathbb{R}}
\newcommand{\CC}{\mathbb{C}}
\newcommand{\PP}{\mathbb{P}}
\newcommand{\FF}{\mathbb{F}}
\newcommand{\Hom}{\operatorname{Hom}}
\newcommand{\im}{\operatorname{im}}
\newcommand{\id}{\operatorname{id}}
\newcommand{\Ind}{\operatorname{Ind}}
\newcommand{\Res}{\operatorname{Res}}

\newcommand{\calD}{\mathcal{D}}

\newcommand{\sol}{\textit{Solution: }}

\title{Algebraic Geometry}
\author{Kevin}
\date{January 2025}

\begin{document}
\maketitle
\section{Motivating Examples and Introduction}
\[\backslash:\tag{Owen's signature}\]
Consider $f(x,y)=x^2+y^2-1$ over $\RR$. Roots of $f$: unit circle.
\begin{itemize}
    \item $1$-dim manifold
    \item smooth
    \item irreducible
\end{itemize}
Algebraically, consider the quotient ring $R[x,y]/I$, where $I=(x^2+y^2-1)$.
\begin{itemize}
    \item transcendence degree $1$ over $\RR$.
    \item localizations are independent of choices
    \item $I$ is a prime ideal
\end{itemize}
\begin{example}
    $x^n+y^n=z^n$ over $\ZZ$. (no non-trivial solutions when $n\ge 3$) Assume $n=2$. Can identify (not a bijective correspondence) the solution sets with $\{(x,y)\in\QQ^2:x^2+y^2=1\}$. Consider the line $L_t:y=tx+1$. $L_t$ meets the circle at $(-2t/(1+t^2),(1-t^2)/(1+t^2))$. In $\QQ^2$, if $t\in \QQ$, then 
    \[t\leftrightarrow\left(\dfrac{-2t}{1+t^2},\dfrac{1-t^2}{1+t^2}\right)\]
    is a fully algebraic identification of the solution set with the base field.
\end{example}
\begin{rem}
    Care about intersections.
\end{rem}
\begin{example}
    FTA. The set of zeros of a polynomial over $\CC$ is $\{(z,y)=(z,p(z)):z\in\CC\}\cap\{(z,y):y=0\}\subseteq\CC^2$.
\end{example}
\begin{defn}
    Let $L/k$ be a field extension. We say $x\in L$ is algebraic over $k$ if there exists a non-zero $p_x\in k[z]$ s.. $p_x(x)=0$. Otherwise, $x$ is transcendental over $k$. Say $L/k$ is algebraic if all elemnts of $L$ are algebraic over $k$
\end{defn}
Recall that every field $k$ has a (unique up to iso) maximal algebraic extension $\bar k$, its algebraic closure.

\textbf{In this course, we work over an algebraically closed field of characteristic $0$.}

\subsection{The Projective Plane}
FTA predicts that every two lines in a plane intersect at a point, unless they are parallel.
\begin{defn}
    The projective plane: $\PP^2_k=\PP^2=\{(x,y,z)\in k^3\setminus\{(0,0,0)\}\}/\{(x,y,z)\sim\lambda(x,y,z),\lambda\neq 0\}$.
\end{defn}
Denote $[(x,y,z)]=(x:y:z)$. We have an inclusion $k^2\hookrightarrow\PP_k^2, (x,y)\mapsto (x:y:1)$. The points at infinity are $\{(x:y:0)\}\subseteq\PP_k^2$.

A line $ax+by-c=0$ in $k^2$ dosn't have well-defined solution set in $\PP^2$, but its homogenization $ax+by+cz=0$ does.

\begin{defn}
    A projective plane curve is
    \[\{(x:y:z)\in\PP^2\colon F(x,y,z)=0\}\]
    for some non-zero homogeneous poly $F$.
\end{defn}
If $f\in k[x,y]$, then $z^{\deg f}f(x/z,y/z)=F(x,y,z)$ is homogeneous of the same degree.
\begin{defn}
    If $C=\{f(x,y)=0\}\subseteq k^2$ and $F$ is the homogenization of $f$, we say that $\{(x:y:z)\in\PP^2:F(x,y,z)=0\}$ is the projective closure of $C$ in $\PP^2$.
\end{defn}
\end{document}