\documentclass{article}
\usepackage{graphicx} % Required for inserting images
\usepackage[utf8]{inputenc}
\usepackage{amsmath,amsfonts,amssymb,amsthm}
\usepackage{enumerate,bbm}
\usepackage{tikz,graphicx,color,mathrsfs,color,hyperref,boldline,tikz-cd}
\usepackage{caption,float}
\usepackage[a4paper,margin=1in,footskip=0.25in]{geometry}

\usepackage{listings}
\usepackage{xcolor}

\usepackage{tabularx,capt-of}

\usepackage{blindtext}
%Image-related packages
\usepackage{graphicx}
\usepackage{subcaption}
\usepackage[export]{adjustbox}
\usepackage{lipsum}

%hyperref setup
\hypersetup{
    colorlinks=true,
    linkcolor=blue,
    filecolor=magenta,      
    urlcolor=cyan,
    pdftitle={Overleaf Example},
    pdfpagemode=FullScreen,
    }

%New colors defined below
\definecolor{codegreen}{rgb}{0,0.6,0}
\definecolor{codegray}{rgb}{0.5,0.5,0.5}
\definecolor{codepurple}{rgb}{0.58,0,0.82}
\definecolor{backcolour}{rgb}{0.95,0.95,0.92}

%Code listing style named "mystyle"
\lstdefinestyle{mystyle}{
  backgroundcolor=\color{backcolour}, commentstyle=\color{codegreen},
  keywordstyle=\color{magenta},
  numberstyle=\tiny\color{codegray},
  stringstyle=\color{codepurple},
  basicstyle=\ttfamily\footnotesize,
  breakatwhitespace=false,         
  breaklines=true,                 
  captionpos=b,                    
  keepspaces=true,                 
  numbers=left,                    
  numbersep=5pt,                  
  showspaces=false,                
  showstringspaces=false,
  showtabs=false,                  
  tabsize=2
}

%"mystyle" code listing set
\lstset{style=mystyle}

\theoremstyle{definition}
\newtheorem{defn}{Definition}[section]
\newtheorem{example}[defn]{Example}
\theoremstyle{remark}
\newtheorem{rem}{Remark}
\newtheorem{remS}[section]{defn}
\newtheorem{lem}[defn]{Lemma}
\theoremstyle{plain}
\newtheorem{thm}[defn]{Theorem}
\newtheorem{prop}[defn]{Proposition}
\newtheorem{fact}[defn]{Fact}
\newtheorem{crly}[defn]{Corollary}
\newtheorem{conj}[defn]{Conjecture}

%\newtheorem*{programming*}{Programming Task}
%\newtheorem{innercustomgeneric}{\customgenericname}
%\providecommand{\customgenericname}{}
%\newcommand{\newcustomtheorem}[2]{%
%  \newenvironment{#1}[1]
%  {%
%   \renewcommand\customgenericname{#2}%
%   \renewcommand\theinnercustomgeneric{##1}%
%   \innercustomgeneric
%  }
%  {\endinnercustomgeneric}
%}

%\newcustomtheorem{question}{Question}
%\newcustomtheorem{programming}{Programming Task}

\newcommand{\NN}{\mathbb{N}}
\newcommand{\ZZ}{\mathbb{Z}}
\newcommand{\QQ}{\mathbb{Q}}
\newcommand{\RR}{\mathbb{R}}
\newcommand{\CC}{\mathbb{C}}
\newcommand{\PP}{\mathbb{P}}
\newcommand{\FF}{\mathbb{F}}

\newcommand{\calD}{\mathcal{D}}

\newcommand{\sol}{\textit{Solution: }}
\newcommand{\Res}{\operatorname{Res}}

\title{III Algebraic Topology}
\author{Kevin}
\date{October 2024}

\begin{document}

\maketitle

\section{Homotopy}
\begin{defn}
    A \textbf{homotopy} between continuous maps $f,g:X\to Y$ is a continuous map $H:X\times I\to Y$ s.t. $H(x,0)=f(x)$, $H(x,1)=g(x)$ for all $x\in X$. Notation: $f\simeq g$
\end{defn}
Note that ``$\sim$'' defines an equivalence relation on $\operatorname{Maps}(X,Y)$.
\begin{defn}
    A map $f:X\to Y$ is a homotopy equivalence if $\exists$ a map $g:Y\to X$ s.t $fg\simeq \operatorname{id}_Y$ and $gf\simeq \operatorname{id}_X$. Notation: $X\sim Y$.
\end{defn}
\begin{example}
    Homeomorphisms; the inclusion $\{0\}\hookrightarrow\RR^n$; the inclusion $S^{n-1}\hookrightarrow\RR^n$. For the latter two, we have obvious retractions and the homotopy is given by straight line homotopy (interpolating linearly).
\end{example}
\begin{defn}
    A space $X$ is contractible if $X\simeq \{\ast\}$.
\end{defn}

\section{Singular (co)homology}
\subsection{Chain complexes}
\begin{defn}
    A chain complex is a sequence of abelian groups and homomorphisms (called differentials or boundary maps), denoted $(C_\bullet,d)$ s.t. $d_n\circ d_{n-1}=0$. Elements in $C_n$ are called $n$-chains.
\end{defn}
\begin{defn}
    The homology of $(C_\bullet,d)$ is 
    \[H_n(C)=\ker d_n/\operatorname{im}(d_{n-1})\]
\end{defn}
\begin{defn}
    A chain map $f:(C_\bullet,d^C)\to(D_\bullet,d^D)$ is a family of homomorphisms $f_n:C_n\to D_n$ s.t. $f_{n-1}\circ d_n^C=d_{n}^D\circ f_n$.
\end{defn}
\begin{lem}
    A chain map induces a well-defined homomorphism on homology $H_n(C)\overset{f_\ast}{\to}H_n(D), [x]\mapsto[f_n(x)]$.
\end{lem}
\begin{proof}
If $[x]\in H_n(C)$, then $f_n(x)$ is a cycle by direct computation. If $[x]=[y]$, then $x-y=d_{n+1}^C(z)\implies f(x)-f(y)=d_{n+1}(f(z))$, so $[f_n(x)]=[f_n(y)]$.
\end{proof}
\subsection{Singular chain complex \& Singular homology}
Define $\Delta^n\colon=\{(t_0,...,t_n)\in\RR^{n+1}\colon\forall i\ge0, t_i\ge 0, \sum_it_i=1\}$ to be the standard $n$-simplex. We have standard maps $\delta^i:\Delta^{n-1}\to \Delta^n$ given by
\[(t_0,...,t_{n-1})\mapsto (t_0,...,t_{i-1},0,t_i,...,t_{n-1})\]
which embeds the $\Delta^{n-1}$ as the $i$th face of $\Delta^n$ which is opposite of the $i$th vertex.
\begin{defn}
    Let $X$ be a topological space.
    \begin{itemize}
        \item A singular $n$-simplex in $X$ is a map $\sigma:\Delta^n\to X$.
        \item The singular chain complex of $X$ is given by $C_n(X):=\text{ free abelian grp over the }n\text{- simplices of }X$ with boundary maps $d_n(\sigma)=\sum_{i=0}^n(-1)^i(\sigma\circ\delta^i)$.
    \end{itemize}
\end{defn}
\begin{lem}
    $(C_\bullet(X),d)$ is a chain complex
\end{lem}
We first observe that the following diagram commutes.
\begin{center}
    % https://tikzcd.yichuanshen.de/#N4Igdg9gJgpgziAXAbVABwnAlgFyxMJZABgBpiBdUkANwEMAbAVxiRAB12ARGBnOgHrAwAWgBMAXxATS6TLnyEUARnJVajFm048+g4SOVSZc7HgJExa6vWatEHbr34DCE9TCgBzeEVAAzACcIAFskMhAcCCRVDTttdlg9AH0sEGoGOgAjXgAFeXMlEAYYfxx0kBywKCQAZmITECDQ8OoomJtNe0ck-mTgACtDKQzsvILFNkCsLwALcuoqmsQResbmsMRY9sQrOK0HTl66VIrMnIZ8s0mHabmFyphqpFWG2Sbgze3o3c74w8SzhOAzOY0uEwsDhKZQqSzqDQoEiAA
\begin{tikzcd}
\Delta^{n-2} \arrow[r, "\delta_i", bend left] \arrow[r, "\delta_{j-1}"', bend right] & \Delta^{n-1} \arrow[r, "\delta_i"', bend right] \arrow[r, "\delta_j", bend left] & \Delta^n
\end{tikzcd}
\end{center}
\begin{proof}
    Let $\sigma\in C_n(X)$.
    \begin{align*}
        d_{n-1}(d_n(\sigma))&=\sum_{i=0}^{n-1}(-1)^i\sum_{j=0}^n(-1)^j(\sigma\circ\delta_j\delta_i)\\
        &=\sum_{i<j}(-1)^{i+j}(\sigma\circ\delta_j\circ\delta_i)+\sum_{i\ge j}(-1)^{i+j}(\sigma\circ\delta_j\circ\delta_i)\\
        &=\sum_{l\le k}(-1)^{l+k+1}(\sigma\circ\delta_l\circ\delta_k)+\sum_{i\ge j}(-1)^{i+j}(\sigma\circ\delta_j\circ\delta_i)\\
        &=0
    \end{align*}
    where we used the substitution $k=j-1$ and $l=i$ in the second to last line.
\end{proof}
\begin{defn}[Singular homology]
    $H_i(X):=H_i(C(X))$.
\end{defn}
We have functoriality. If $f:X\to Y$ is a map, then it induces a chain map $f_\#:C(X)\to C(Y),\sigma\mapsto f\circ\sigma$, so one gets $f_\ast:H_\ast(X)\to H_\ast(Y)$.

\subsection{Computations}
\begin{example}
    $H_n(pt)=\ZZ$ if $n=0$ and $0$ otherwise. Observe that for all $n$, there is a unique singular $n$-simplex, denoted $\sigma_n$ which is the constant simplex, and 
    \[d_n(\sigma_n)=\sum_{i=0}^n(-1)^i(\sigma_n\circ\delta_i)=\sum_{i=0}^n(-1)^i\sigma_{n-1}=\begin{cases}
        \sigma_{n-1}& n\text{ even}\\
        0 & n\text{ odd}
    \end{cases}\]
\end{example}
\begin{example}
    If $X=\bigsqcup_{\alpha\in I}X_\alpha$, then $C_n(X)=\bigoplus_{\alpha\in I}C_n(X_\alpha)$, which implies that $H_n(X)\cong\bigoplus_{\alpha\in I}H_n(X_\alpha)$.
\end{example}
\begin{example}
    If $X\neq\varnothing$ and is path-connected, then $H_0(X)\cong \ZZ$.
    \begin{proof}
        Observe that $C_0(X)=\ZZ\langle X\rangle$. Define an augmentation map $\epsilon:C_0(X)\to\ZZ$ by $\sum h_\sigma\sigma\mapsto\sum h_\sigma$. This is clearly surjective. We want to show that $\ker(\epsilon)=\operatorname{im}(d_1)$. Clearly $\ker\epsilon\supseteq\operatorname{im}d_1$ as $(\epsilon\circ d_1)(\sigma)=\epsilon(\sigma(1)-\sigma(0))=1-1=0$. Conversely, if $\epsilon(\sum h_\sigma\sigma)=0$, then $\sum h_\sigma=0$. For each $\sigma$ in the summation, choose $\tau_\sigma:\Delta^1\to X$ s.t. $\tau_\sigma(0)=x_0$ and $\tau_\sigma(1)=\sigma$ (a point). This is possible by path-connectedness. By direct computation, we get
        \begin{align*}
            d_1\sum h_\sigma\sigma=\sum h_\sigma(\sigma-x_0)=\sum h_\sigma\sigma-\sum h_\sigma x=\sum h_\sigma\sigma
        \end{align*}
        so $\sum h_\sigma\sigma$ is a boundary, and we are done.
    \end{proof}
\end{example}
\begin{crly}
    If $X$ is any space, then $H_0(X)=\ZZ\langle\text{path components of }X\rangle$.
\end{crly}
\subsection{Singular cohomology}
A cochain complex is of the form $0\to C^0\overset{d^0}{\to} C^1\overset{d^1}{\to}C^2\to....$ s.t. $d^nd^{n-1}=0$. Define $H^n(C^\bullet,d)=\ker d^n/\operatorname{im}d^{n-1}$. 
\begin{defn}
    The singular cochain complex of a topological space $X$ is $C^n(X)=\operatorname{Hom}(C_n(X),\ZZ)$ with $d^n(\varphi)=\varphi\circ d_{n+1}$. The singular cohomology of $X$ is defined to be the cohomology of the singular cochain complex.
\end{defn}
Note that a map $f:X\to Y$ induces $f^\#:C^\bullet(Y)\to C^\bullet(X)$, $\varphi\mapsto\varphi(f\circ - )$, so induces $f^\ast: H^\ast(Y)\to H^\ast(X)$.

\section{Four Major Tools for Computation}
\begin{thm}[Homotopy invariance] If $f\simeq g:X\to Y$, then $f_\ast=g_\ast$ and $f^\ast=g^\ast$.
\end{thm}
\begin{thm}[Mayer-Vietoris Sequence]
    $X=A\cup B$ with $A,B$ open. There is a natural homomorphism $\partial_{MV}:H_n(X)\to H_{n-1}(A\cap B)$ s.t.
    \begin{center}
        % https://tikzcd.yichuanshen.de/#N4Igdg9gJgpgziAXAbVABwnAlgFyxMJZABgBpiBdUkANwEMAbAVxiRADpOQBfU9TXPkIoAjOSq1GLNgAkA+sDABqEdwAUADQCUPPiAzY8BIgCZx1es1aIQ8sGoCCAHScBjOmgAEAIR29+hkJEAMzmklaycvYOWi4QaMxwnnZqvroBgsYoACxhltI2Kdrp+gJGwsgArHlS1hxc3BIwUADm8ESgAGYAThAAtkhkIDgQSGLhBWBMDAzUDHQARjAMAAplQTYMMJ04JT39Y9QjSGYTdS5odN14jAoAsgBq3CBzi8trgVkgWzt7vQOIU7HRChM5sNRYOQxOQuOhwHAuPB9eAQuS+GFOOG7V5LVbrL4-Xb+ED7AGg4G5ME2NQAKyhWgxWIAtLS0QzYfCXt83njPsJvtsiXpSUhKcDqlSpjMce98fzCTwKNwgA
\begin{tikzcd}
... \arrow[r] & H_{n+1}(X) \arrow[r, "\partial_{MV}"] & H_n(A\cap B) \arrow[r, "(i_A)_\ast\times(i_B)_\ast"] & H_n(A)\oplus H_n(B) \arrow[r, "(j_A)_\ast-(j_B)_\ast"] & H_n(X) \arrow[r] & ...
\end{tikzcd}
    \end{center}
    is exact.
\end{thm}
\begin{rem}
    \begin{itemize}
        \item Naturality means the the connecting homomorphism commutes with maps fo the form $f:(X;A,B)\to (Y;U,V)$, where $X=A\cup B$ and $Y=U\cup V$, $f(A)\subseteq U, f(B)\subseteq V$.
        \item If $\alpha\in H_n(X)$ is s.t. $\alpha=[a+b]$ where $a\in C_n(A), b\in C_n(B)$, then $\partial_{MV}\alpha=[d_na]=-[d_nb]$.
    \end{itemize}
\end{rem}
Have an SES $0\to C_\bullet(A)\overset{i}{\hookrightarrow} C_\bullet(X)\overset{q}{\to}C_\bullet(X,A)\to 0$.
\begin{thm}[Homology exact sequence]
There is a natural hom $\partial:H_n(X,A)\to H_{n-1}(A)$ s.t. the sequence
\begin{center}
    % https://tikzcd.yichuanshen.de/#N4Igdg9gJgpgziAXAbVABwnAlgFyxMJZABgBpiBdUkANwEMAbAVxiRADpOQBfU9TXPkIoAjOSq1GLNgAkA+sDABqEdwAUADVIBBAJQ8+IDNjwEiAJnHV6zVohDywavQf4mhRAMxXJt2XKcNfV43QTMUABYfG2l7R00dYMNjMOFkAFZoqTsOLm4JGCgAc3giUAAzACcIAFskMhAcCCQxX1iwJgYGagY6ACMYBgAFAVNhEAYYcpxXECraluompEs2nIAddbQ6SrxGEB7+wZH3cImpmZC56rrEVeXEbzW2LDlNujgZw4Hh0Y97SbTWbzW5PB5RZ72ACOb3WHy+EyOv1O40Bl0MIKQEIemUhHS632OfzOaJ4FG4QA
\begin{tikzcd}
... \arrow[r] & {H_{n+1}(X,A)} \arrow[r, "\partial"] & H_n(A) \arrow[r, "i_\ast"] & H_n(X) \arrow[r, "q_\ast"] & {H_n(X,A)} \arrow[r] & ...
\end{tikzcd}
\end{center}
is exact. Moreover, if $[x]\in C_n(X,A)$ is a relative cycle, then $d_b[x]=[d_nx]=0\in C_{n-1}(X,A)$, so $d_nx\in C_{n-1}(A)$, so $\partial[[x]]=[d_nx]\in H_{n-1}(A)$.
\end{thm}
\begin{rem}
    In this case, $\partial$ is natural w.r.t. maps of pairs.
\end{rem}
\begin{thm}[Excision Theorem]
Let $(X,A)$ be a pair and $Z\subseteq A$ s.t. $\overline{Z}\subseteq A^\circ$, then the inclusion $(X\setminus Z,A\setminus Z)\hookrightarrow (X,A)$ induces  isomorphisms \[H_n(X\setminus Z,A\setminus Z)\overset{\sim}{\to}H_n(X,A)\]
and
\[H^n(X\setminus Z,A\setminus Z)\overset{\sim}{\leftarrow}H^n(X,A)\]
\end{thm}
The cohomological version is analogous.

\subsection{Degree}
\begin{defn}
    Degree of a map $S^n\to S^n$ is the integer multiplication in the induced map on top homology.
\end{defn}
\begin{thm}[Brower's fixed point theorem]
    Self maps $D^n\to D^n$ has a fixed point.
\end{thm}
\begin{proof}
    This is equivalent to the no retraction theorem.
\end{proof}
\begin{prop}
    A reflection $r:S^n\to S^n$ has degree $-1$.
\end{prop}
\begin{proof}
    Decompose $S^n$ as the union of two hemispheres and use naturality of MV sequence to perform induction.
\end{proof}
\begin{crly}[Hairy ball theorem]
    $S^n$ has a nowhere vanishing vector field iff $n$ odd.
\end{crly}
\begin{proof}
    If $n=2k+1$ odd, then $S^{2k+1}\subseteq \CC^{n+1}$, then $v(x)=ix$ gives a valid vector field.

    If $v:S^n\to \RR^{n+1}$ is a non-vanishing vector field, then can define $w(x)=v(x)/|v(x)|$. Have a homotopy $H:S^n\times[0,\pi]\to \RR^{n+1}, (x,t)\mapsto \cos(t)x+\sin(t)w(x)$. So the antipodal map is homotopic to the identity, so $n$ must be odd.
\end{proof}
\end{document}
