\documentclass{article}
\usepackage{graphicx} % Required for inserting images
\usepackage[utf8]{inputenc}
\usepackage{amsmath,amsfonts,amssymb,amsthm}
\usepackage{enumerate,bbm}
\usepackage{leftindex}
\usepackage{tikz,tikz-cd,graphicx,color,mathrsfs,color,hyperref,boldline}
\usepackage{caption,float}
\usepackage[a4paper,margin=1in,footskip=0.25in]{geometry}

\usepackage{listings}
\usepackage{xcolor}

\usepackage{tabularx,capt-of}

\usepackage{blindtext}
%Image-related packages
\usepackage{graphicx}
\usepackage{subcaption}
\usepackage[export]{adjustbox}
\usepackage{lipsum}

%hyperref setup
\hypersetup{
    colorlinks=true,
    linkcolor=blue,
    filecolor=magenta,      
    urlcolor=cyan,
    pdftitle={Overleaf Example},
    pdfpagemode=FullScreen,
    }

%New colors defined below
\definecolor{codegreen}{rgb}{0,0.6,0}
\definecolor{codegray}{rgb}{0.5,0.5,0.5}
\definecolor{codepurple}{rgb}{0.58,0,0.82}
\definecolor{backcolour}{rgb}{0.95,0.95,0.92}

%Code listing style named "mystyle"
\lstdefinestyle{mystyle}{
  backgroundcolor=\color{backcolour}, commentstyle=\color{codegreen},
  keywordstyle=\color{magenta},
  numberstyle=\tiny\color{codegray},
  stringstyle=\color{codepurple},
  basicstyle=\ttfamily\footnotesize,
  breakatwhitespace=false,         
  breaklines=true,                 
  captionpos=b,                    
  keepspaces=true,                 
  numbers=left,                    
  numbersep=5pt,                  
  showspaces=false,                
  showstringspaces=false,
  showtabs=false,                  
  tabsize=2
}

%"mystyle" code listing set
\lstset{style=mystyle}

\theoremstyle{definition}
\newtheorem{defn}{Definition}[section]
\newtheorem{example}[defn]{Example}
\theoremstyle{remark}
\newtheorem{rem}{Remark}
\newtheorem{remS}[section]{defn}
\theoremstyle{plain}
\newtheorem{lem}[defn]{Lemma}
\newtheorem{thm}[defn]{Theorem}
\newtheorem{prop}[defn]{Proposition}
\newtheorem{fact}[defn]{Fact}
\newtheorem{crly}[defn]{Corollary}
\newtheorem{conj}[defn]{Conjecture}

%\newtheorem*{programming*}{Programming Task}

%\newtheorem{innercustomgeneric}{\customgenericname}
%\providecommand{\customgenericname}{}
%\newcommand{\newcustomtheorem}[2]{%
%  \newenvironment{#1}[1]
%  {%
%   \renewcommand\customgenericname{#2}%
%   \renewcommand\theinnercustomgeneric{##1}%
%   \innercustomgeneric
%  }
%  {\endinnercustomgeneric}
%}

%\newcustomtheorem{question}{Question}
%\newcustomtheorem{programming}{Programming Task}

\newcommand{\NN}{\mathbb{N}}
\newcommand{\ZZ}{\mathbb{Z}}
\newcommand{\QQ}{\mathbb{Q}}
\newcommand{\RR}{\mathbb{R}}
\newcommand{\CC}{\mathbb{C}}
\newcommand{\PP}{\mathbb{P}}
\newcommand{\FF}{\mathbb{F}}
\newcommand{\Hom}{\operatorname{Hom}}
\newcommand{\im}{\operatorname{im}}
\newcommand{\id}{\operatorname{id}}
\newcommand{\Ind}{\operatorname{Ind}}
\newcommand{\Res}{\operatorname{Res}}

\newcommand{\calD}{\mathcal{D}}

\newcommand{\sol}{\textit{Solution: }}

\title{Topics in Analysis}
\author{Kevin}
\date{January 2025}

\begin{document}
\maketitle
\section{Review of Metric Spaces}
Notation: open ball of radius $\delta$ will be denoted $B_\delta(x)$.

Revised some results about metric spaces.

A proof of IVT using a compactness argument:
\begin{lem}[Discrete IVT]
    Let $f:\{0,1,...,n\}\to\RR$ with $f(0)<0$ and $f(n)\ge 0$, then $\exists m$ s.t. $f(m-1)<0$, $f(m)\ge 0$.
\end{lem}
\begin{proof}
    Can prove it using WOP.

    Alternatively (an idea which will be useful later), define $g:\{0,...,n\}\to\RR$ by $g(m)=1$ if $f(m)\ge 0$ and $0$ if $f(m)<0$. Then $g(n)-g(0)=\sum_n(g(m)-g(m-1))$, so there exists $g(m)-g(m-1)>0$, and we are done. 
\end{proof}
Now can prove IVT. $F:[a,b]\to\RR$, $F(a)<0,\ F(b)>0$, consider $f_n:\{0,...,n\}\to\RR$ defined by $f_n(m)=F(a+\frac{m}{n}(b-a))$. Apply lemma, get a sequence $x_n$ s.t. $F(x_n)\ge 0$ and $F(x_n-(b-a)/n)<0$. Apply Bolzano-Weierstrass.

\begin{thm}[Brouwer's fixed point theorem in one dimension]
    If $f:[a,b]\to[a,b]$ cts, then $f$ has a fixed point.
\end{thm}
This is a trivial consequence of IVT.

\begin{rem}
    In fact, Brouwer's fixed point theorem in one dimension is equivalent to IVT.
\end{rem}

We will generalize this result.

\begin{thm}
    TFAE,
    \begin{enumerate}
        \item[(a)] Any continuous self maps of $D^2$ has a fixed point. (2D Brouwer's fixed point theorem)
        \item[(b)] There is no retraction $D^2\to \partial D^2$. (No retraction theorem)
        \item[(c)] Let $T$ be a (closed solid) triangle with sides $I_1,I_2,I_3$. Let $A_1,A_2,A_3$ be closed subsets of $T$ s.t. $I_j\subseteq A_j$ and $A_1\cup A_2\cup A_3=T$. Then $A_1\cap A_2\cap A_3\neq\varnothing$.
        \item[(d)] There is no continuous function $f:T\to\partial T$ s.t. $\forall x\in I_j$, $f(x)\in I_j$. (No quasi-retraction theorem)
    \end{enumerate}
\end{thm}
\begin{proof}
    (a)$\Rightarrow$(b): Suppose (b) fails and $f:D\to\partial D$ is a retraction. Let $\sigma\neq\id$ be a non-trivial rotation about $(0,0)$. Then $\sigma\circ f:D\to D$ is continuous and has no fixed points. Contradiction.

    (b)$\Rightarrow$(a): standard proof of Brouwer's fixed point theorem from no retraction theorem. Construct a retraction by mapping $x\in D$ to the intersection of the ray from $f(x)$ to $x$ with $\partial D$. 

    (a)$\Rightarrow$(d): Suppose (d) fails. Take $T$ to be an equilateral triangle with vertices $1,\omega,\omega^2\in\CC$ (cube roots of unity). For each polar angle $\theta$, define $\rho(\theta)=d(0,x)$, where $x$ is the point on $\partial T$ with argument $\theta$. $\rho$ is cts in $\theta$ and $1/2\le\rho(\theta)\le 1$ for all $\theta$. Define $\varphi:D\to T$, $\varphi(z)=\rho(\theta)z$ if $z\in\rho(\theta) e^{i\theta}$ This is a homeomorphism. $\varphi(\partial D)=\partial T$. Now let $g:D\to D$, $g=\varphi^{-1}\circ f\circ\varphi$. Then $\varphi^{-1}(I_j)$ is a $2\pi/3$ radian arc of $\partial D$. If $x\in\varphi^{-1}(I_j)$ and $\varphi(x)\in I$, then $f(\varphi(x))\in I_j$, so $g(x)=\in\varphi^{-1}(I_j)$. Also $x\in D$, have $\varphi(x)\in T\Rightarrow f(\varphi(x))\in\partial T\Rightarrow g(x)\in\partial D$. Let $\pi$ be the antipodal map, then $\pi\circ g$ has no fixed points.

    (c)$\Rightarrow$(d): Suppose $f:T\to\partial T$ is a quasi-retraction. Let $A_j=f^{-1}(I_j)$ for $j=1,2,3$. These are closed subsets of $T$ and $A_1\cup A_2\cup A_3=T$. Since $I_1\cap I_2\cap I_3=\varnothing$, we deduce that $A_1\cap A_2\cap A_3=\varnothing$, which contradicts (c).

    (d)$\Rightarrow$(c): Choose a convenient triangle. Let $T=\{(x,y,z)\in\RR^3: x,y,z\ge 0,\ x+y+z=1\}$. This is the standard 2-simplex spanned by $e_1,e_2,e_3$. Let $I_j=\langle e_k:k\neq j\rangle$. If (c) fails, then we can find $A_j$, $j=1,2,3$ with $I_j\subseteq A_j$, $\bigcup_j A_j=T$, and $\bigcap_jA_j=\varnothing$. Define
    \[f(x)=\dfrac{(d(x,A_1),d(x,A_2),d(x,A_3)}{\sum_jd(x,A_j)}\]
    Note that $A_j$ is closed and bounded for all $j$, so each $d(x,A_j)$ is finite, and $\bigcap_jA_j=\varnothing$ ensures that the denominator is never $0$. $f$ is also cts. If $x\in I_j\subseteq A_j$, then $d(x,A_j)=0$, then $f(x)\in I_j$, so $f(I_j)\subseteq I_j$. Contradiction.

    (d)$\Rightarrow$(b): Suppose (b) fails, i.e., we can find a retraction $f:D\to\partial D$. Recall the homeomorphism $\varphi:D\to T$ from an earlier part of the proof. Let $g=\varphi\circ f\circ\varphi^{-1}$, then $g:T\to \partial T$ is continuous. If $x\in T$, then $f(\varphi^{-1}(x))\in\partial D$, then $g(x)\in\partial T$. If $x\in\partial T$, then $\varphi^{-1}(x)\in\partial D$, then $f(\varphi^{-1}(x))=\varphi^{-1}(x)$, so $g(x)=x$. In particular, $g$ is a quasi-retraction.
\end{proof}



\end{document}