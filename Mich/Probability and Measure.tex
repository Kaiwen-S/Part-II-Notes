\documentclass{article}
\usepackage{graphicx} % Required for inserting images
\usepackage[utf8]{inputenc}
\usepackage{amsmath,amsfonts,amssymb,amsthm}
\usepackage{enumerate,bbm}
\usepackage{tikz,graphicx,color,mathrsfs,color,hyperref,boldline}
\usepackage{caption,float}
\usepackage[a4paper,margin=1in,footskip=0.25in]{geometry}

\usepackage{listings}
\usepackage{xcolor}

\usepackage{tabularx,capt-of}

\usepackage{blindtext}
%Image-related packages
\usepackage{graphicx}
\usepackage{subcaption}
\usepackage[export]{adjustbox}
\usepackage{lipsum}

%hyperref setup
\hypersetup{
    colorlinks=true,
    linkcolor=blue,
    filecolor=magenta,      
    urlcolor=cyan,
    pdftitle={Overleaf Example},
    pdfpagemode=FullScreen,
    }

%New colors defined below
\definecolor{codegreen}{rgb}{0,0.6,0}
\definecolor{codegray}{rgb}{0.5,0.5,0.5}
\definecolor{codepurple}{rgb}{0.58,0,0.82}
\definecolor{backcolour}{rgb}{0.95,0.95,0.92}

%Code listing style named "mystyle"
\lstdefinestyle{mystyle}{
  backgroundcolor=\color{backcolour}, commentstyle=\color{codegreen},
  keywordstyle=\color{magenta},
  numberstyle=\tiny\color{codegray},
  stringstyle=\color{codepurple},
  basicstyle=\ttfamily\footnotesize,
  breakatwhitespace=false,         
  breaklines=true,                 
  captionpos=b,                    
  keepspaces=true,                 
  numbers=left,                    
  numbersep=5pt,                  
  showspaces=false,                
  showstringspaces=false,
  showtabs=false,                  
  tabsize=2
}

%"mystyle" code listing set
\lstset{style=mystyle}

\theoremstyle{definition}
\newtheorem{defn}{Definition}[section]
\newtheorem{example}[defn]{Example}
\theoremstyle{remark}
\newtheorem{rem}{Remark}
\newtheorem{remS}[section]{defn}

\theoremstyle{plain}
\newtheorem{thm}[defn]{Theorem}
\newtheorem{prop}[defn]{Proposition}
\newtheorem{fact}[defn]{Fact}
\newtheorem{crly}[defn]{Corollary}
\newtheorem{conj}[defn]{Conjecture}
\newtheorem{lem}[defn]{Lemma}

%\newtheorem*{programming*}{Programming Task}

%\newtheorem{innercustomgeneric}{\customgenericname}
%\providecommand{\customgenericname}{}
%\newcommand{\newcustomtheorem}[2]{%
%  \newenvironment{#1}[1]
%  {%
%   \renewcommand\customgenericname{#2}%
%   \renewcommand\theinnercustomgeneric{##1}%
%   \innercustomgeneric
%  }
%  {\endinnercustomgeneric}
%}

%\newcustomtheorem{question}{Question}
%\newcustomtheorem{programming}{Programming Task}

\newcommand{\NN}{\mathbb{N}}
\newcommand{\ZZ}{\mathbb{Z}}
\newcommand{\QQ}{\mathbb{Q}}
\newcommand{\RR}{\mathbb{R}}
\newcommand{\CC}{\mathbb{C}}
\newcommand{\PP}{\mathbb{P}}
\newcommand{\FF}{\mathbb{F}}

\newcommand{\calD}{\mathcal{D}}

\newcommand{\sol}{\textit{Solution: }}
\newcommand{\Res}{\operatorname{Res}}

\title{Probability and Measure}
\author{Kevin}
\date{October 2024}

\begin{document}
\maketitle
\section{Integration}
\begin{defn}
    A tribe ($\sigma$-algebra) on a set $E$ is a family $\mathcal{A}\subseteq\mathcal{P}(E)$ s.t.
    \begin{enumerate}
        \item $E\in \mathcal{A}$;
        \item $A\in\mathcal{A}\implies E\setminus A\in\mathcal{A}$;
        \item $(A_n)_{n\in\NN}\in\mathcal{A}\implies\bigcup_{n\in\NN}A_n\in\mathcal{A}$.
    \end{enumerate}
\end{defn}
\begin{rem}
\begin{itemize}
    \item If in 3 we only allow finite union, then get the definition of an algebra
    \item 1 and 2 implies $\varnothing\in\mathcal{A}$
    \item By De Morgan's law, $\mathcal{A}$ is closed under countable intersection
    \item Do not assume $A_n$ distinct, so can take $A_n=\varnothing$ for all but finitely many $n$, so $\mathcal{A}$ is closed under finite set operations.
\end{itemize}   
\end{rem}
\begin{defn}
    Let $E$ be a set, $\mathcal{A}$ a $\sigma$-algebra. We say that
    \begin{itemize}
        \item $(E,\mathcal{A})$ is a measurable space;
        \item $A\in\mathcal{A}$ is a measurable set.
    \end{itemize}
\end{defn}
\begin{prop}
    $\mathcal{A},\mathcal{A}'$ $\sigma$-algebra $\implies$ $\mathcal{A}\cap\mathcal{A}'$ is also a $\sigma$-algebra
\end{prop}
\begin{proof}
Trivial.
\end{proof}
\begin{defn}[$\sigma$-algebra generated by set] $C\subseteq\mathcal{P}(E)$, then the smallest $\sigma$-algebra
 containing $C$ is called the $\sigma$-algebra generated by $C$, denoted $\sigma(C)$. One has
 \[\sigma(C)=\bigcap_{C\subseteq\mathcal{A}\ \sigma-\text{alg}}\mathcal{A}\]
\end{defn}
\begin{example}[Borel $\sigma$-algebra]
Let $E$ be a topological space. $\mathcal{O}\subseteq\mathcal{P}(E)$ is the set of open sets Then, $\sigma(\mathcal{O})$ is called the Borel $\sigma$-algebra of $E$, usually denoted $B(E)$.
\end{example}
\textbf{Exercise:} $B(\RR)$ is generated by $(-\infty,a)$, $a\in\QQ$.
\begin{defn}[Separable metric space]
    A metric space is separable if it contains a countable dense subset.
\end{defn}
\begin{defn}[Product measure]
    Let $(E_1,\mathcal{A}_1)$ and $(E_2,\mathcal{A}_2)$ be measurable spaces. The product measure is $\mathcal{A}_1\otimes\mathcal{A}_2=\sigma(\{A_1\times A_2:A_1\in\mathcal{A}_1, A_2\in\mathcal{A}_2\})$
\end{defn}
\begin{prop}
    If $E_1$ and $E_2$ are separable metric spaces, then $B(E_1)\otimes B(E_2)=B(E_1\times E_2)$.
\end{prop}
\subsection{Positive Measure in a Tribe}
\begin{defn}
    Let $(E,\mathcal{A})$ be a measurable sapce. A positive measure on this space is a map $\mu:\mathcal{A}\to[0,\infty]$ s.t.
    \begin{enumerate}
        \item $\mu(\varnothing)=0$.
        \item ($\sigma$-additivity) For all sequences $(A_n)_{n\in \NN}$ of pairwise disjoint measurable sets, $$\mu(\bigsqcup_n A_n)=\sum_n\mu(A_n):=\lim_{N\to\infty}\sum_{n=1}^N\mu(A_n)$$
    \end{enumerate}
\end{defn}
\begin{prop}
    \begin{itemize}
        \item $A\subseteq B\implies\mu(A)\le\mu(B)$. Moreover, if $\mu(A)<\infty$ then $\mu(B\setminus A)=\mu(B)-\mu(A)$.
        \item $A,B$ measurable sets, then $\mu(A)+\mu(B)=\mu(A\cup B)+\mu(A\cap B)$.
        \item $(A_n)$ increasing family, then $\mu(\bigcup_nA_n)=\lim_n\mu(A_n)$
        \item $(B_n)$ decreasing family and $\mu(B_0)<\infty$, then $\mu(\cap_nB_n)=\lim_n\mu(B_n)$.
    \end{itemize}
\end{prop}
\begin{proof}
    To prove (i), note that $B=A\sqcup (B\setminus A)$ and use positivity. (ii): use $A=(A\setminus B)\sqcup (A\cap B)$ and $B=(B\setminus A)\sqcup(A\cap B)$. (iii): Define $C_0=A_0$ and $C_n=A_n\setminus A_{n-1}$, then $A_n=\bigsqcup_{j=0}^n C_j$, so $\mu(A_n)=\mu(\bigcup_{j=0}^nA_j)=\sum_{j=0}^n\mu(C_j)$. So
    \[\mu(\bigcup_n A_n)=\mu(\bigsqcup_j C_j)=\lim_n\sum_{j=0}^n\mu(C_j)=\lim_n\mu(A_n)\]
    For (iv), Define $A_n=B_0\setminus B_n$ and apply (iii).
\end{proof}
\begin{example}
\begin{itemize}
    \item Counting measure on $\NN$, $\mathcal{A}=\mathcal{P}(\NN)$ and $\mu(A)=|A|$.
    \item Lebesgue measure in $\RR$. $\lambda$ is the unique meausre defined in $B(\RR)$ s.t. $\lambda((a,b))=b-a$ for $a<b$.
\end{itemize}
\end{example}
We introduce some terms.
\begin{itemize}
    \item $\mu$ is called finite if $\mu(E)<\infty$
    \item $\mu(E)$ is called the total measure
    \item If $\mu(E)=1$, then it's a probability measure.
    \item $\mu$ is called $\sigma$-finite if there exists an increasing sequence $E_n$ in $\mathcal{A}$ s.t. $E=\bigcup_n E_n$ and $\mu(E_n)<\infty$ for all $n$.
    \item An atom is a point $x\in E$ s.t. $\mu(\{x\})\neq 0$. A measure without atoms is called diffuse.
\end{itemize}
\subsection{Monotone Class and Uniqueness of Measures}
If $\mathcal{A}=\sigma(C)$ and $\mu,\nu$ two measures, then $\mu=\nu$ on $\mathcal{A}$ iff $\mu=\nu$ on $C$.
\begin{defn}[Monotone Class]
Let $\mathcal{M}\subseteq\mathcal{P}(E)$. Then $\mathcal{M}$ is a monotone class if 
\begin{enumerate}
    \item $E\in\mathcal{M}$
    \item $A,B\in\mathcal{M}\implies B\setminus A\in\mathcal{M}$. [This is equivalent to saying $A,B\in\mathcal{M}, A\subseteq B\implies B\setminus A\in\mathcal{M}$.]
    \item If $(A_n)$ is increasing in $\mathcal{M}$, then $\bigcup_nA_n\in\mathcal{M}$.
\end{enumerate}
\end{defn}
\begin{lem}
    A monotone class is a $\sigma$-algebra iff it's closed under finite intersection
\end{lem}
\begin{proof}
    $\Rightarrow$ is trivial. For the other implication, suppose $\mathcal{M}$ is closed under finite intersection. Taking complement, $\mathcal{M}$ is closed under finite union. For any sequence $(B_n)$ in $\mathcal{M}$, define $A_0=B_0$, $A_1=A_0\cup B_1$, etc. We then get an increasing sequence $(A_n)$, and $\bigcup_n B_n=\bigcup_n A_n\in\mathcal{M}$, and we are done.
\end{proof}
\begin{rem}
    If $\mathcal{M},\mathcal{M'}$ are monotone classes $\implies$ $\mathcal{M}\cap\mathcal{M'}$ is also a monotone class, so can define $\mathcal{M}(C)$ to be the smallest monotone class containing $C$ for $C\subseteq\mathcal{P}(E)$.
\end{rem}
\begin{thm}[Monotone class lemma]
    If $C\subseteq\mathcal{P}(E)$ is closed under finite intersection, then $\mathcal{M}(C)=\sigma(C)$.
\end{thm}
\begin{proof}
    Tribe implies monotone, so $\mathcal{M}(C)\subseteq\sigma(C)$. 
    
    Conversely, we want to show that $\sigma(C)\subseteq \mathcal{M}(C)$. If we can show that $\mathcal{M}(C)$ is a sigma algebra then our claimed inclusion holds by definition of $\sigma(C)$ being smallest $\sigma$-algebra containing $C$.

    It suffices to show that $\mathcal{M}(C)$ is closed under finite intersection because of the preceding lemma. To do this, define $\mathcal{M}_A=\{B\in\mathcal{M}(C):A\cap B\in\mathcal{M}(C)\}$ for each $A\in\mathcal{C}$. Our goal is to show that $\mathcal{M}_A=\mathcal{M}(C)$ as then the closure under finite intersection becomes a trivial consequence. Clearly, $\mathcal{M}_A\subseteq \mathcal{M}(C)$ by definition. We now prove the other inclusion.
    
    First, if $A\in C$, then we note that $B\in C\implies A\cap B\in C\implies B\in\mathcal{M}_A$ as $C$ is closed under finite intersection by assumption. We now check that $\mathcal{M}_A$ satisfies the axioms of monotone classes.
    \begin{itemize}
        \item $A\cap E=A\implies E\in\mathcal{A}$;
        \item If $B\subseteq B'$ for $B,B'\in\mathcal{M}_A$, then $A\cap B\subseteq A\cap B'$ are both in $\mathcal{M}_C$, so $A\cap(B'\setminus B)=(A\cap B')\setminus(A\cap B)\in\mathcal{M}_A$
        \item Let $(B_n)$ be an increasing family in $\mathcal{M}_A$. Then $A\cap B_n\in\mathcal{M}(C)$ for all $n$, so $A\cap\bigcup_nB_n=\bigcup_n(A\cap B_n)\in\mathcal{M}(C)$, so $\bigcup_n B_n\in\mathcal{M}_A$.
    \end{itemize}
    This proves that $\mathcal{M}_A$ is a monotone class if $A\in C$. In other words, this says that $\forall A\in C ,\ \forall B\in\mathcal{M}(C),\ A\cap B\in\mathcal{M}(C)$ ($\ast$).

    If $A\in \mathcal{M}(C)$ (not necessarily in $C$), then ($\ast$) says that $C\subseteq\mathcal{M}(C)$, and the same argument as before shows that $\mathcal{M}_A$ is a monotone class, which implies that $\mathcal{M}(C)\subseteq\mathcal{M}_A$.

    Hence, $\mathcal{M}(C)=\mathcal{M}_A$, and we are done.
\end{proof}
\begin{crly}[Uniqueness of measure]
    Let $\mu,\nu$ be two measures on a measurable space $(E,\mathcal{A})$. Suppose there exists $C\subseteq\mathcal{A}$ s.t. 
    \begin{itemize}
        \item $C$ is closed under finite intersections
        \item $\mathcal{A}=\sigma(C)$
        \item For all $A\in C$, $\mu(A)=\nu(A)$
    \end{itemize}
    Suppose $\mu,\nu$ satisfies one of the following:
    \begin{itemize}
        \item $\mu(E)=\nu(E)<\infty$;
        \item $\mu,\nu$ both $\sigma$-finite on $C$, i.e., $\exists (E_n)$ in $C$ s.t. $\bigcup_nE_n=E$ and $\nu(E_n),\mu(E_n)<\infty$.
    \end{itemize}
    Then $\forall A\in\mathcal{A}$, $\mu(A)=\nu(A)$.
\end{crly}
\begin{proof}
    Suppose both measures are finite. Define $\mathcal{G}=\{A\in\mathcal{A}:\mu(A)=\nu(A)\}$. Clearly $\mathcal{G}\subseteq\mathcal{A}$, so we need to show the other inclusion. Assumption on finiteness of total measures implies $C\subseteq\mathcal{G}$. We claim that $\mathcal{G}$ is a monotone class. Indeed, one can check
    \begin{itemize}
        \item $E\in\mathcal{G}$
        \item If $A\subseteq B$ are elements of $\mathcal{G}$, then $\mu(B\setminus A)=\mu(B)-\mu(A)=\nu(B)-\nu(A)=\nu(B\setminus A)$, so $B\setminus A\subseteq \mathcal{G}$.
        \item If $(A_n)$ is an increasing sequence in $\mathcal{G}$, then $\mu(\bigcup A_n)=\lim_n\mu(A_n)=\lim_n\nu(A_n)=\nu(\bigcup A_n)$.
    \end{itemize}
    So $\mathcal{G}\supseteq\mathcal{M}(C)=\mathcal{A}$ The last equality follows from the preceding theorem.

    Now assume $\mu,\nu$ are $\sigma$-finite \textbf{on} $C$. We find an increasing sequence $(E_n)$ in $C$ s.t. $\bigcup_n E_n=E$ and $\nu(E_n)=\mu(E_n)<\infty$. Then, for all $A\in C$, $A\cap E_n\in C$. Define $\mu_n(A\cap E_n)$ and $\nu(A\cap E_n)$.
\end{proof}

\begin{defn}[Negligible set]
 $(E,\mathcal{A},\mu)$ measure space. A set $N\in\mathcal{P}(E)$ is said to be negligible if $\exists A\in \mathcal{A}$ s.t. $N\subseteq A$ and $\mu(A)=0$.
\end{defn}
A property that hoolds for every $x\in E\setminus N$, $N$ negligible, is said to holds almost everywhere (a.e.).
\begin{lem}
    $(N_j)_{j\ge0}$, $N_j$ negligible $\implies $ $\bigcup_j N_j$ negligible.
\end{lem}
\begin{proof}
    For each $j$, choose $A_j\supseteq N_j$ in $\mathcal{A}$. Define $C_0=A_0$, $C_n=A_n\setminus(\bigcup_{j=0}^{n-1})A_j$, then \[\mu(\bigcup_j A_j)\le\sum_j\mu(A_j)=0\]
    Also, \[\bigcup_j N_j\subseteq\bigcup_j A_j\]
    So the claim follows from the inequality and positivity of $\mu$.
\end{proof}
\begin{defn}
    Let $(E,\mathcal{A},\mu)$ be a measure space. If for all $N$ negligible $N\in\mathcal{A}$, then $\mathcal{A}$ is said to be complete (for $\mu$).
\end{defn}

\subsection{Measurable Functions}
\begin{defn}
    $(E,\mathcal{A}),(F,\mathcal{B})$ measurable sapces, $f:E\to F$ is said to be measurable if $\forall B\in\mathcal{B}$, $f^{-1}(B)\in\mathcal{A}$.
\end{defn}
When $E,F$ are topological spaces, we use the Borel $\sigma$-algebra as the default $\sigma$-algebra.
\begin{prop}
    Composition of measurable function is measurable
\end{prop}
\begin{proof}
    Trivial.
\end{proof}
\begin{prop}
    If $\mathcal{B}=\sigma(C)$, then $f:E\to F$ is measurable iff $\forall B\in C$, $f^{-1}(B)\in\mathcal{A}$.
\end{prop}
\begin{proof}
    We define $\mathcal{G}=\{B\in\mathcal{B}:f^{-1}(B)\in\mathcal{A}\}$. We claim that $\mathcal{G}$ is a $\sigma$-algebra containing $\mathcal{B}$.
\end{proof}
\begin{example}
    For a real valued function from some measurable space $(E,\mathcal{A})$, checking measurability is the same as checking $\forall \alpha\in\RR$, $f^{-1}((\alpha,\infty))\in\mathcal{A}$.
\end{example}

\begin{lem}
    $f_1:(E,\mathcal{A})\to(F_1,\mathcal{B}_1)$, $f_2:(E,\mathcal{A})\to(F_2,\mathcal{B}_2)$. Then $f:(E,\mathcal{A})\to(F_1\times F_2,\mathcal{B}_1\otimes\mathcal{B}_2), x\mapsto (f_1(x),f_2(x))$ is measurable if and only if $f_1,f_2$ are measurable.
\end{lem}
\begin{proof}
    The product measure is generated by $C=\{B_1\times B_2:B\in \mathcal{B}_1, B_2\in\mathcal{B}_2\}$. 
    \begin{itemize}
        \item ``$\Leftarrow$'': $f^{-1}(B_1\times B_2)=f_1^{-1}(B_1)\cap f_2^{-1}(B_2)$.
        \item ``$\Rightarrow$'':  $f_1^{-1}(B_1)=f^{-1}(B_1)\cap E=f^{-1}(B_1\times F_2)$. Similarly for $f_2$.
    \end{itemize}
\end{proof}
\begin{crly}
    Product, linear comb, min/max with $0$ of measurable functions are measurable. (Obviously assume additional structures on the range)
\end{crly}
\begin{proof}
    $(x_1,x_2)\mapsto x_1x_2$ is measurable.
\end{proof}

We allow functions with values in $\overline{\RR}=\RR\cup\{\pm\infty\}$. $B(\overline{\RR})$ is generated by $[-\infty,a)$ and $(a,\infty]$, $a\in\RR$.

\begin{prop}
    $f_n(E,\mathcal{A})\to\RR$ measurable. Then $\sup_nf_n,\inf_nf_n,\lim\sup f_n,\lim\inf f_n$ are all measurable.
\end{prop}
\begin{proof}
    ???????????
\end{proof}
\subsection{Lebesgue Integration}
We first defin the Lebesgue integral of a positive function.
\begin{defn}
    $(E,\mathcal{A})$ measurable space. $f:(E,\mathcal{A})\to\RR$ is called a simple function if $f(x)=\sum_{j=1}^na_j1_{A_j}(x)$ where
    \begin{itemize}
        \item $A_j\in\mathcal{A}$;
        \item $E=\bigsqcup_{j=1}^n A_j$.
    \end{itemize}
\end{defn}
\begin{rem}
    If $a_j$ are all distinct, then the expression above is called the canonical representation of $f$.
\end{rem}
\begin{defn}[Integral of simple functions]
    $f:(E,\mathcal{A},\mu)\to\RR_+$. Then \[\int_E fd\mu=\sum_{j=1}^na_j\mu(A_j)\]
\end{defn}
\textbf{Convention:} If $a_j=0$ and $\mu(A_j)=\infty$, then $a_j\mu(A_j)=0$.
\begin{rem}
    It can be shown that this definition does not depend on representations.
\end{rem}
\begin{prop}
    $f,g:(E,\mathcal{A},\mu)\to\RR_+$ simple, then
    \begin{enumerate}
        \item For all $a,b\in\RR_+$, $\int_E(af+bg)d\mu=a\int_E fd\mu+b\int_E gd\mu$.
        \item $f\le g\implies \int_E fd\mu\le \int_E gd\mu$
    \end{enumerate}
\end{prop}
\begin{proof}
    To prove 1, we observe that $af+bg$ is simple. Indeed, if $f=\sum_1^N\beta_i1_{B_i}$ and $g=\sum_1^{N'}\beta_i'1_{B_i'}$, then setting $C_{ij}=B_i\cap B_j'$, get $af+bg=\sum_{i,j}\gamma_{ij}1_{C_{ij}}$, where $\gamma_{ij}=a\beta_i+b\beta_j'$. Apply the definition,
    \begin{align*}
        \int_E(af+bg)d\mu&=a\sum_i\beta_i\sum_j\mu(C_{ij})+b\sum_j\beta_j'\sum_i\mu(C_{ij})\\
        &=a\sum_i\beta_i\mu(B_i)+b\sum_j\beta_j'\mu(B_j')\\
        &=a\int_Efd\mu+b\int_Egd\mu
    \end{align*}
    To prove 2, observe that if $f\le g$, then $g=(g-f)+f$, and use 1.
\end{proof}
\begin{defn}
    We define $\mathcal{E}_+=\{\text{positive simple functions}\}$. 
    
    We say that $f$ is positive measurable on $(E,\mathcal{A})$ if it's measurable and takes value on $\overline{\RR}_+$.
\end{defn}
\begin{defn}[Lebesgue integral of positive measurable functions]
    $f:(E,\mathcal{A},\mu)\to\overline{\RR}_+$ measurable. Define
    \[\int_E fd\mu=\sup_{h\in\mathcal{E}_+,h\le f}\int_E hd\mu\]
\end{defn}
\begin{rem}
    \begin{itemize}
        \item If $f$ is simple, then this is consistent with the original definition.
        \item $f\le g$ pos. meas. implies that $\int_E fd\mu\le\int_E gd\mu$.
        \item If $\mu(\{x\in E:f(x)>0\})=0$, then $\int fd\mu=0$. [If $h\in\mathcal{E}_+$ and $h\le f$, then $\int_E hd\mu=0$.]
    \end{itemize}
\end{rem}
\begin{thm}[Monotone Convergence Theorem]
    Let $f_n:(E,\mathcal{A},\mu)\to\overline{\RR}_+$ be an increasing sequence of measurbale function. Let $f=\sup_n f_n$, then $\int_E fd\mu=\sup_n\int_E f_nd\mu$.
\end{thm}
\begin{proof}
    $f_n$ is an increasing sequence, so $f\ge f_n$ for all $n$, so $\int_E fd\mu\ge \int_E f_nd\mu$ for all n, so $\int fd\mu\ge \sup_n\int_E f_nd\mu$.

    Want to show the other inequality. Since the sequence is monotone $f_n(x)\to f(x)$ for each $x$ (Note that $\overline{\RR}_+$ is the completion of $\RR_+$, so this is possible). 

    Pick $h\in\mathcal{E}_+$ s.t. $h\le f$.
    Let $\epsilon>0$ be given. 
    Define $E_n=\{x\in E:f_n(x)\ge (1-\epsilon)h(x)\}$ This has two properties:
    \begin{itemize}
        \item $E_n$ is an increasing sequence (since $f_n$ is increasing)
        \item $E=\bigcup_n E_n$.
    \end{itemize}
    By definition, $f_n\ge (1-\epsilon)1_{E_n}\cdot h$, so $\int_E fd\mu\ge(1-\epsilon)1_{E_n}hd\mu=(1-\epsilon)\sum_{k=1}^Ka_k\mu(A_k\cap E_n)$ if we assume that $h=\sum_{k=1}^Ka_k1_{A_k}$. By continuity of measure, $\mu(A_k\cap E_n)\to\mu(A_k)$ as $n\to\infty$. So
    \begin{align*}
        \sup_n\int_E fd\mu\ge(1-\epsilon)\lim_{n\to\infty}\sum_{k=1}^Ka_k\mu(A_k\cap E_n)=(1-\epsilon)\sum_{k=1}^Ka_K\mu(A_k)=(1-\epsilon)\int_E hd\mu.
    \end{align*}
    Take supremum over $h\in\mathcal{E}_+$ s.t. $h\le f$ on RHS,
    \[\sup_n\int_Ef_nd\mu\ge(1-\epsilon)\int_Efd\mu\]
    But $\epsilon$ is arbitrary, so
    \[\sup_n\int_Efd\mu\ge\int_Efd\mu\]
\end{proof}
\begin{lem}
    $f$ meas. non-negative. Then $\exists f_n$ increasing sequence of simple functions s.t. $\forall x\in E$, $\lim_n f_n(x)=f(x)$.
\end{lem}
\begin{proof}
    $n\in \NN$. Consider $[n,\infty)$, $[0,n]$.
    Define 
    \[A_\infty^{(n)}=f^{-1}[n,\infty),\ A_i^{(n)}=f^{-1}[i/2^n,(i+1)/2^n)\] for $i=0,1,...,n2^n-1$.
    Define
    \[f_n=\sum_{i=0}^{n2^n-1}\dfrac{i}{2^n}1_{A_i^{(n)}}+n1_{A_\infty^{(n)}}\]
    Have several properties
    \begin{enumerate}
        \item $f_n\le f$;
        \item The partition $\bigsqcup_i[i2^{-n-1},(i+1)2^{-n+1})\cup [n+1,\infty)$ is finer than $\bigsqcup_i[i2^{-n},(i+1)2^{-n})\cup[n,\infty)$, so $f_{n+1}\ge f_n$ for all $n$, so $(f_n)$ is an increasing sequence.
        \item On $A^{(n)}=\bigsqcup_iA_i^{(n)}=f^{-1}[0,n)$, have $f_n(x)\le f(x)\le f_n(x)+2^{-n}$. Also, $A^{(n)}$ is an increasing sequence s.t. $\bigcup_n A^{(n)}=f^{-1}(\RR^+)$, so $\forall x\in f^{-1}(\RR^+)$, $f_n(x)\to f(x)$ as $n\to\infty$. On the remaining set $f^{-1}(\{\infty\})=\cap_nA_\infty^{(n)}$, meaning that $f_n(x)\ge n$, so $\lim_nf_n(x)=\infty=f(x)$.
    \end{enumerate}
    So $f_n$ is an increasing seqeuence converging to $f$ pointwise.
\end{proof}
\begin{prop}[Linearity for measurable positive functions]
    If $f,g$ are measurable non-negative functions then for all $a,b\in\RR^+$, $\int (af+bg)d\mu=a\int fd\mu+b\int gd\mu$.
\end{prop}
\begin{proof}
    Apply the preceding lemma and monotone convergence theorem.
\end{proof}
\begin{prop}
    $f_n$ meas. non-negative, then \[\int\left(\sum_{n=0}^\infty f_n\right) d\mu=\sum_{n=0}^\infty\int f_nd\mu\]
\end{prop}
\begin{proof}
    Define partial sums and apply monotone convergence theorem.
\end{proof}
\begin{prop}
    Let $f,g$ be measurable non-negative functions on a measure space $(E,\mathcal{A},\mu)$.
    \begin{enumerate}
        \item (Markov's inequality) $\forall a>0$, $\mu(\{x\in E:f(x)\ge a\})\le \frac{1}{a}\int fd\mu$.
        \item $\int fd\mu<\infty\implies f<\infty$ a.e.
        \item $\int fd\mu=0\Leftrightarrow f=0$ a.e.
        \item $f=g$ a.e.$\implies \int fd\mu=\int gd\mu$.
    \end{enumerate}
\end{prop}
\begin{proof}
    (i) Observe $A_a=f^{-1}[a,\infty]$ is a measurable since $f$ is measurable. Observe that $f\ge a1_{A_a}$. This implies that $\int fd\mu\ge \int a1_{A_a}d\mu=a\mu(A_a)$.

    (ii) Let $A_a=f^{-1}[a,\infty]$ then $A_\infty=f^{-1}\{\infty\}=\cap_nA_n$ (intersection of a decreasing sequence). We have $\mu(A_\infty)=\mu(\cap_nA_n)=\lim_n\mu(A_n)$. Markov implies that $\mu(A_n)\le \frac{1}{n}\int fd\mu\to 0$ as $n\to\infty$ as the integral is finite, so $\mu(A_\infty)=0$.

    (iii) If $f=0$ a.e., then there if $h\le f$ is any non-negative simple function then $h=0$ a.e. and $\int hd\mu=0$, so by definition $\int fd\mu=0$. Conversely, if $\int fd\mu=0$, then define $B_n=f^{-1}[1/n,\infty]$, then $B_n$ is an increasing sequence and $\bigcup_nB_n=f^{-1}(0,\infty]$. Markove implies that $0=n\int fd\mu\ge \mu(B_n)\to \mu(\bigcup_nB_n)$ as $n\to\infty$. 

    (iv) (Caution) Did not define $\int fd\mu$ if $f$ takes negative values. Define two functions $f\vee g=\max(f,g)$ and $f\wedge g=\min(f,g)$. Both are measurable and $f\vee g\ge f\wedge g$, so $f\vee g-f\wedge g\ge 0$. $f=g$ a.e. $\implies f\vee g-f\wedge g=0$ a.e., so $\int f\vee gd\mu=\int f\wedge gd\mu+\int(f\vee g-f\wedge g)d\mu=\int f\wedge gd\mu$. Also, have $f\wedge g\le f,g\le f\vee g$, so $\int fd\mu=\int gd\mu$.
\end{proof}
\begin{thm}[Fatou's lemma]
    $(E,\mathcal{A},\mu)$ measurable space, $(f_n)_{n\ge0}$ meas. non-negative functions. Then
    \[\int(\lim\inf f_n)d\mu\le\lim\inf\int f_nd\mu\]
\end{thm}
\begin{proof}
    Let $F_n=\inf_{k\ge n}f_n$. This is meas. non-negative and increasing. By monotone convergence theorem,
    \[\lim \int F_nd\mu=\int\lim F_nd\mu=\int\lim\inf f_nd\mu\]
    Observe that $F_n\le f_n$, so $\int F_n\le\int f_nd\mu$, so
    \[\lim\inf\int F_nd\mu\le\lim\inf\int f_nd\mu\]
\end{proof}

\subsection{Integration of Real-Valued Functions}
\begin{defn}
    $f:(E,\mathcal{A},\mu)\to\RR$ meas. We say that $f$ is $\mu$-integrable if $\int|f|d\mu<\infty$ and define
    \[\int fd\mu=\int f^+d\mu-\int f^-d\mu\]
    where $f^+=\max(f,0)$, $f^-=\max(-f,0)$.
\end{defn}
Let $\mathcal{L}^1(E,\mathcal{A},\mu)$ be the set of $\mu$-integrable functions.
\begin{prop}
    \begin{enumerate}
        \item $f\in\mathcal{L}^1(E,\mathcal{A},\mu)\implies |\int fd\mu|\le \int|f|d\mu$.
        \item $\mathcal{L}^1(E,\mathcal{A},\mu)$ is a vector space and $f\mapsto \int fd\mu$ is a linear form.
        \item $f,g\in \mathcal{L}^1(E,\mathcal{A},\mu),f\le g\implies \int fd\mu\le \int gd\mu$
        \item $f,g\in \mathcal{L}^1(E,\mathcal{A},\mu)$. Then $f=g$ a.e. implies that $\int fd\mu=\int gd\mu$.
    \end{enumerate}
\end{prop}
\begin{proof}
    (i) $f=f^+-f^-$ and $|f|=f^++f^-$, then $|\int f^+d\mu-\int f^-d\mu|\le \int f^+d\mu+\infty^-d\mu$.

    (ii) $f^+-f^-+g^+-g^-=f+g=(f+g)^+-(f+g)^-$. Rearrange to get $f^-+g^-+(f+g)^+=f^++g^++(f+g)^-$. Integrate both sides and use linearity for meas. non-negative functions.

    (iii) $f\le g$, so $g=f+(g-f)$.

    (iv) $f=g$ a.e., so $f^{\pm}=g^\pm$ a.e.
\end{proof}
\begin{rem}
    For complex valued function, define the integral by the integral of real/imag parts. Can still check that $|\int fd\mu|\le\int |f|d\mu$.
\end{rem}
\begin{thm}[Lebesgue's Dominated Convergence Theorem]
    $(f_n)_{n\ge 0}$ sequence in $\mathcal{L}^1(E,\mathcal{A},\mu)$ (real or complex valued) Suppose
    \begin{itemize}
        \item $\exists f$ meas. s.t. $f_n(x)\to f(x)$ a.e.
        \item $\exists g$ meas. non-negative and integrable  s.t. $\forall n\ge 0$ $|f_n(x)|\le |g(x)|$ a.e.
    \end{itemize}
    Then
    \[\lim_n\int f_nd\mu=\int fd\mu\]
    and \[\lim_n\int |f_n-f|d\mu=0\]
\end{thm}
\begin{proof}
    First prove the theorem when the assumptions hold everywhere. In this situation we know that $f_n\to f$ pointwise and $|f_n|\le g$ for all $x$. In particular $|f|\le g$, so $f$ is integrable.

    We have $|f_n|\le g\implies |f-f_n|\le 2g\implies 2g-|f-f_n|\ge 0$. By Fatou's lemma, 
    \[\lim\inf\int(2g-|f-f_n|)d\mu\ge\int\lim\inf(2g-|f-f_n|)d\mu=\int 2gd\mu\]
    Since the domination is independent of $n$ we have
    \begin{align*}
        \lim\inf\int(2g-|f-f_n|)d\mu=\int 2g-\lim\sup\int|f-f_n|d\mu\ge\int 2gd\mu
    \end{align*}
    So $\lim\sup|f-f_n|=0$, so $\lim\int |f-f_n|d\mu=0$, and \[\lim_n\left|\int fd\mu-\int f_nd\mu\right|=0\]

    To prove the general case, define $B_n=\{x\in E:|f_n(x)|\le g(x)\}$ and $B_{\lim}=\{x\in E:\lim_nf_n(x)=f(x)\}$. 
    Assumption implies that $B_n^c$ and $B_{\lim}^c$ are negligible, so there exists measurable sets $N_n$ and $N_{\lim}$ s.t. $B_n^c\subset N_n$ and $B_{\lim}\subset N_{\lim}$ and $\mu(N_n)=\mu(N_{\lim})=0$ for all $n$. 
    Then 
    \[\left(B_{\lim}\cap\bigcap_nB_n\right)^c=B_{\lim}^c\cup\bigcup_nB_n^c\subset N_{\lim}\cup\bigcup_nN_n:=N\]
    so must be negligible. Consider the restriction $\Tilde{f}_n=f1_{N^c},\tilde f=f1_{N^c}$. Apply the first part and note that since $N$ is negligible, the integral doesn't change under restriction.
\end{proof}
\subsection{Continuity and Differentiability of the Integral}
\begin{thm}[Continuity]
    $(E,\mathcal{A},\mu)$ measure space, $(U,d)$ metric space. $f:U\times E\to\RR$ s.t.
    \begin{enumerate}
        \item $\forall u\in U$, $x\mapsto f(u,x)$ is meas.
        \item Fix $u_0$, $u\mapsto f(u,x)$ is cts at $u_0$ for almost all $x\in E$.
        \item $\exists g\in \mathcal{L}^1(E,\mathcal{A},\mu)$ s.t. $\forall u\in U$, $|f(u,x)|\le g(x)$ for almost every $x$.
    \end{enumerate}
    Then $u\mapsto \int_f(u,x)d\mu$ is well-defined for $u\in U$ and cts at $u_0$.
\end{thm}
\begin{thm}[Differentiability]
    $(E,\mathcal{A},\mu)$ a measure space, $I\subseteq\RR$, and $f:I\times E\to\RR$ s.t.
    \begin{enumerate}
        \item $u\mapsto f(u,x)$ is diff at $u_0$ for almost every $x$ with derivative $\frac{\partial f}{\partial u}(u_0,x)$.
        \item $\exists g\in\mathcal{L}^1(E,\mathcal{A},\mu)$ s.t. $\forall u\in I$, $|f(u,x)-f(u_0,x)|\le g(x)|u-u_0|$ for almost all $x\in E$.
        \item $\forall u\in I$, $x\mapsto f(u,x)$ is 
    \end{enumerate}Lebesgue integrable.
        Then $F(u)=\int f(u,x)d\mu$ is cts and differentiable at $u_0$ with $F'(u_0)=\int\frac{\partial f}{\partial u}(u_0,x)d\mu$.
\end{thm}
\begin{defn}
    $(E,\mathcal{A},\mu)$ measure space. The completed $\sigma$-algebra $\bar{\mathcal{A}}$ is defined by $\bar{\mathcal{A}}=\sigma(A\cup\mathcal{N})$, where $\mathcal{N}$ is the collection of negligible sets.
\end{defn}
(Exercise: There eixsts a unique measure $\bar\mu$ on $(E,\bar{\mathcal{A}})$ which extends $\mu$.)
\begin{crly}
    If $f:\RR\to\RR$ is measurable wrt $B(\RR)$ and $g:\RR\to \RR$ is s.t. $g=f$ a.e. Then $g$ is meas. w.r.t. $\overline{B(\RR)}$ (?????????)
\end{crly}

\subsection{Connection with Riemann Integral}
\begin{thm}
    $f:I\to\RR$ Riemann integrable with integral $S(f)$. Then $f$ is Lebesgue integrable and $S(f)=\int_Ifd\lambda$.
\end{thm}
\begin{proof}
    This is a consequence of dominated convergence theorem.
    Pick $(h_n)$ stair s.t. $h_n\le f$. Then $\lim_n S(h_n)=S(f)$. Similarly $\tilde{h}_n$ stair s.t. $\tilde h_n\ge f$ with $\lim_n\tilde S(h_n)=S(f)$. Replacing $h_n=\max\{h_1,....,h_n\}$ if necessary, may assume $\tilde h_n$ is increasing. Similarly, may assume $\tilde h_n$ is decreasing. Then $h_\infty=\lim h_n$ and $\tilde h_\infty=\lim \tilde h_n$ are both measurable. Apply dominated convergence theorem ($h_n\le f$),
    \[\int h_\infty d\lambda=\lim\int h_nd\lambda\overset{\text{def}}{=}\lim S(h_n)=S(f)\]
    Also ($-\tilde h_n\le -f$),
    \[\int \tilde h_\infty=\lim\int\tilde h_nd\lambda=\lim S(\tilde h_n)=S(f)\]
    $h_\infty$ and $\tilde h_\infty$ satisfy
    \begin{itemize}
        \item measurable
        \item $h_\infty\le f\le\tilde h_n$
        \item $\int h_\infty d\lambda=\int\tilde h_\infty d\lambda=S(f)$
    \end{itemize}
    So $h_\infty=f=\tilde{h_\infty}$ almost everywhere (use prop 1.35-3 by taking differences). So $f$ is measurable (assuming we are working with the completed borel $\sigma$-algebra) and $\int fd\lambda=S(f)$.
\end{proof}
\begin{example}[Lebesgue non-measurable subset of $\RR$]
Require Zorn's lemma (Axiom of choice). Consider $\RR/\QQ$. For every equiv class $a\in\RR/\QQ$. Pick a representative $x_a\in[0,1]$. Define $F=\{x_a:a\in\RR/\QQ\}\subset[0,1]$.

\textbf{Claim: $F\not\in B(\RR)$.} If $F\in B(\RR)$, then $\forall q\in\QQ$, $q+F$ is measurable. Every $x\in \RR$ belongs to a class $a\in\RR/\QQ$, and every class can be written as $x_a+\QQ$. So $\bigcup_{q\in\QQ}(q+F)=\RR$. Clearly $\lambda(F)\neq 0$, so $\lambda(F)=c>0$. Observe that $q+F$ and $q'+F$ are either disjoint or identical. Then $\bigcup_{q\in\QQ\cap[0,1]}(q+F)\subset[0,2]$ but then $\sum_{q\in\QQ\cap[0,1]}\lambda(q+F)\le 2$, which is absurd. So $F$ is non-measurable.
\end{example}

\subsection{Riemann-Stieltjes Integral}
\begin{thm}
    \begin{enumerate}
        \item $\mu$ finite measure on $(\RR, B(\RR))$. The distribution function of $\mu$ is defined by 
        \[\forall x\in\RR, F_\mu(x)=\mu((-\infty,x])\]
        Then $F_\mu$ is increasing, bounded, right-continuous, and $\lim_{x\to-\infty} F_\mu(x)=0$
        \item If $F:\RR\to\RR$ is increasing, bounded, right-continuous, and $\lim_{x\to-\infty}F(x)=0$, then $\exists!\mu$ on $B(\RR)$ finite s.t. $F=F_\mu$.
    \end{enumerate}
\end{thm}
\begin{proof}
    Lecture 9.
\end{proof}
\end{document}