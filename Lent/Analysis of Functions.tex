\documentclass{article}
\usepackage{graphicx} % Required for inserting images
\usepackage[utf8]{inputenc}
\usepackage{amsmath,amsfonts,amssymb,amsthm}
\usepackage{enumerate,bbm}
\usepackage{leftindex}
\usepackage{tikz,tikz-cd,graphicx,color,mathrsfs,color,hyperref,boldline}
\usepackage{caption,float}
\usepackage[a4paper,margin=1in,footskip=0.25in]{geometry}

\usepackage{listings}
\usepackage{xcolor}

\usepackage{tabularx,capt-of}

\usepackage{blindtext}
%Image-related packages
\usepackage{graphicx}
\usepackage{subcaption}
\usepackage[export]{adjustbox}
\usepackage{lipsum}

%hyperref setup
\hypersetup{
    colorlinks=true,
    linkcolor=blue,
    filecolor=magenta,      
    urlcolor=cyan,
    pdftitle={Overleaf Example},
    pdfpagemode=FullScreen,
    }

%New colors defined below
\definecolor{codegreen}{rgb}{0,0.6,0}
\definecolor{codegray}{rgb}{0.5,0.5,0.5}
\definecolor{codepurple}{rgb}{0.58,0,0.82}
\definecolor{backcolour}{rgb}{0.95,0.95,0.92}

%Code listing style named "mystyle"
\lstdefinestyle{mystyle}{
  backgroundcolor=\color{backcolour}, commentstyle=\color{codegreen},
  keywordstyle=\color{magenta},
  numberstyle=\tiny\color{codegray},
  stringstyle=\color{codepurple},
  basicstyle=\ttfamily\footnotesize,
  breakatwhitespace=false,         
  breaklines=true,                 
  captionpos=b,                    
  keepspaces=true,                 
  numbers=left,                    
  numbersep=5pt,                  
  showspaces=false,                
  showstringspaces=false,
  showtabs=false,                  
  tabsize=2
}

%"mystyle" code listing set
\lstset{style=mystyle}

\theoremstyle{definition}
\newtheorem{defn}{Definition}[section]
\newtheorem{example}[defn]{Example}
\theoremstyle{remark}
\newtheorem{rem}{Remark}
\newtheorem{remS}[section]{defn}
\theoremstyle{plain}
\newtheorem{lem}[defn]{Lemma}
\newtheorem{thm}[defn]{Theorem}
\newtheorem{prop}[defn]{Proposition}
\newtheorem{fact}[defn]{Fact}
\newtheorem{crly}[defn]{Corollary}
\newtheorem{conj}[defn]{Conjecture}

%\newtheorem*{programming*}{Programming Task}

%\newtheorem{innercustomgeneric}{\customgenericname}
%\providecommand{\customgenericname}{}
%\newcommand{\newcustomtheorem}[2]{%
%  \newenvironment{#1}[1]
%  {%
%   \renewcommand\customgenericname{#2}%
%   \renewcommand\theinnercustomgeneric{##1}%
%   \innercustomgeneric
%  }
%  {\endinnercustomgeneric}
%}

%\newcustomtheorem{question}{Question}
%\newcustomtheorem{programming}{Programming Task}

\newcommand{\NN}{\mathbb{N}}
\newcommand{\ZZ}{\mathbb{Z}}
\newcommand{\QQ}{\mathbb{Q}}
\newcommand{\RR}{\mathbb{R}}
\newcommand{\CC}{\mathbb{C}}
\newcommand{\PP}{\mathbb{P}}
\newcommand{\FF}{\mathbb{F}}
\newcommand{\Hom}{\operatorname{Hom}}
\newcommand{\im}{\operatorname{im}}
\newcommand{\id}{\operatorname{id}}
\newcommand{\Ind}{\operatorname{Ind}}
\newcommand{\Res}{\operatorname{Res}}

\newcommand{\calD}{\mathcal{D}}

\newcommand{\sol}{\textit{Solution: }}

\title{Analysis of Functions}
\author{Kevin}
\date{January 2025}

\begin{document}
\maketitle
\section{Review of Basic Concepts}
q: (Owen's signature)\\
\subsection{Probmeas}
\begin{enumerate}
    \item The Lebesgue measure is inner regular, i.e., for all $A\in B(\RR^n)$, $\mu(A)=\sup\{\mu(K):K\subseteq A,\ K\text{ compact}\}$.
    \item Recall that $\mu$ extends to the $\mu$-completion of $\mathcal B$, which equals $M_\mu=\{B\cup A: B\in \mathcal B,A\in\mathcal N,\ \mu(A)=0\}$.
    \item For measurable functions $f:E\to F$, if $(F,\mathcal F)=(\RR,\mathcal B)$ (or $(\CC,\mathcal B)$), then we say that $f$ is Borel. This extends to maps taking values $\pm\infty$ if $f^{-1}({\pm\infty})\in\mathcal E$. If $f$ takes values in $[0,\infty]$, then we say $f\ge 0$ (non-negative).
    \item Recall MCT and DCT.
\end{enumerate}
\subsection{$L^p$-spaces and Approximation}
For $f:(E,\mathcal E,\mu)\to \RR$ (or $\CC$), define
\begin{align*}
    &\|f\|_{L^p}=\left(\int_E |f|^pd\mu\right)^{1/p},\ 1\le p<\infty\\
    &\|f\|_{L^\infty}=\operatorname{ess}\sup|f|=\int\{\lambda>0:|f|\le \lambda\ \text{a.e.}\}
\end{align*}
We use $\|\cdot\|_\infty$ to denote the usual sup-norm. Define $L^p(E,\mu)=\{f:E\to\RR:\text{meas. }\|f\|_{L^p}<\infty\}$

Recall Riesz-Fischer Theorem. Also recall the spaces $C^k(\RR^n)$, the set of all functions on $\RR^n$ with continuous partial derivatives up to order $k$. We note that $C^\infty(\RR^n)=\bigcap_{k\ge 0}C^k(\RR^n)$. Note that this includes unbounded smooth functions. Use subscript $c$ to denote the linear subspaces consisting of compactly supported functions.

\begin{rem}
    $C_c^\infty(\RR^n)$ is non-empty, e.g.,
    \begin{align*}
        \psi(x)=\begin{cases}
            e^{\frac{1}{|x|^2-1}} & |x|<1\\
            0 & \text{o/w}
        \end{cases}
    \end{align*}
\end{rem}
\begin{thm}
    $C^\infty_c(\RR^n)$ is dense in $L^p(\RR^n,dx)$ for $1\le p<\infty$.
\end{thm}

\[\langle\text{-}:\tag{Owen's signature}\]

We admit the following lemma from PM.
\begin{lem}
    $C_c(\RR^n)$ is dense in $L^p$, $1\le p<\infty$.
\end{lem}
Recall convolution and basic properties including commutativity, associativity, and $\int_{\RR^n}f\ast gdx=\int_{\RR^n}f\int_{\RR^n}g$ (translation invariance and Fubini).

Recall multi-index notation $\alpha\in\ZZ_+^n$ is written as $\alpha=(\alpha_1,...,\alpha_n)$ with order $|\alpha|=\alpha_1+...+\alpha_n$ and we set $\alpha!=\alpha_1!\cdots\alpha_n!$, and for $x\in\RR^n$, we write $x^\alpha=x_1^{\alpha_1}\cdots x_n^{\alpha_n}$, so the partial differential operator becomes
\[D^\alpha=\dfrac{\partial^{|\alpha|}}{\partial x^\alpha}=\dfrac{\partial^{|\alpha|}}{\partial x_1^{\alpha_1}...\partial x_n^{\alpha_n}}\]
In particular $D_i=D^{(0,...,1,0,...,0)}$
\begin{thm}
    Let $f\in L^1_{\text{loc}}$ (i.e., $f1_K\in L^1$ for any $K\subseteq\RR^n$ compact), and $g\in C^k_c(\RR^n)$. Then $f\ast g\in C^k(\RR^n)$ and for all $0\le|\alpha|\le k$, we have
    \[D^\alpha(f\ast g)=f\ast(D^\alpha g)\]
\end{thm}
\begin{proof}
    Recall the translation operator $\tau_zh=h(\bullet-z)$, $z\in\RR^n$. Then for all $u\in\RR^n$,
    \[\tau_z(f\ast g)(x)=\int_{\RR^n}g(x-u-y)f(y)dy\]
    Since $g\in C_c(\RR^n)$ we have $|g(x-u-y)|\le\|g\|_\infty1_K$ for all $|u|\le 1$, where $K=K_{x,g}$ is a compact set, so tht $\|g\|_\infty1_K|f|$ gives an integrable upper bounde for the integrand. Since $g(x-u-y)\to g(x-y)$ as $u\to 0$, we have pointwise convergence. Apply DCT, we see that $f\ast g$ is cts.

    Now for $k=1$, we define difference operators $\forall e_i$ (standard basis vector) by $\Delta^i_hg(z)=\frac{g(z+he_i)-g(z)}{h}$ which converges to $D_ig(z)$. We can write
    \begin{align*}
        \Delta_h^i(f\ast g)(x)=\int_{\RR^n}\Delta_h^ig(x-y)f(y)dy
    \end{align*}
    Apply mean value inequality, get $|\Delta_h^ig(x-y)|\le\|D_ig\|_\infty1_K$. Apply DCT, $\Delta_h^i(f\ast g)\to f\ast(D_ig)$, which is continuous, so $f\ast g\in C^1$. Induction...
\end{proof}
\begin{prop}[Continuity of translation in $L^p$]
    Let $1\le p<\infty$. Then $\|\tau_zf-f\|_{L^p}\to 0$ as $z\to 0$ for all $f\in L^p$.
\end{prop}
\begin{proof}
    Hold for cts functions with compact support. Then apply $\epsilon/3$-argument.
\end{proof}
\begin{thm}[Minkowski's inequality for integrals]
    Let $F:\RR^n\times\RR^n\to\RR$ be a measurable non-negative or $dx\otimes dx$-integrable function. Then
    \[\left\|\int_{\RR^n}F(x,\cdot)dx\right\|_{L^p}\le\int_{\RR^n}\|F(x,\cdot)\|_{L^p}dx\]
\end{thm}
\begin{proof}
    Example sheet.
\end{proof}
\begin{thm}[Mollification/Approximate identity]
    Let $\varphi\in C_c^\infty(\RR^n)$ be non-negative s.t. $\int_{\RR^n}\varphi(x)dx=1$. Define $\varphi_\epsilon^{-n}\varphi(\cdot/\epsilon)$, $\epsilon>0$. Then for $1\le p<\infty$ and any $f\in L^p$,
    \[\|f-\varphi_\epsilon\ast f\|_{L^p}\overset{\epsilon\to0}{\longrightarrow} 0\]
\end{thm}
\end{document}
