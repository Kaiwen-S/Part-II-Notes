\documentclass{article}
\usepackage{graphicx} % Required for inserting images
\usepackage[utf8]{inputenc}
\usepackage{amsmath,amsfonts,amssymb,amsthm}
\usepackage{enumerate,bbm}
\usepackage{leftindex}
\usepackage{tikz,tikz-cd,graphicx,color,mathrsfs,color,hyperref,boldline}
\usepackage{caption,float}
\usepackage[a4paper,margin=1in,footskip=0.25in]{geometry}

\usepackage{listings}
\usepackage{xcolor}

\usepackage{tabularx,capt-of}

\usepackage{blindtext}
%Image-related packages
\usepackage{graphicx}
\usepackage{subcaption}
\usepackage[export]{adjustbox}
\usepackage{lipsum}

%hyperref setup
\hypersetup{
    colorlinks=true,
    linkcolor=blue,
    filecolor=magenta,      
    urlcolor=cyan,
    pdftitle={Overleaf Example},
    pdfpagemode=FullScreen,
    }

%New colors defined below
\definecolor{codegreen}{rgb}{0,0.6,0}
\definecolor{codegray}{rgb}{0.5,0.5,0.5}
\definecolor{codepurple}{rgb}{0.58,0,0.82}
\definecolor{backcolour}{rgb}{0.95,0.95,0.92}

%Code listing style named "mystyle"
\lstdefinestyle{mystyle}{
  backgroundcolor=\color{backcolour}, commentstyle=\color{codegreen},
  keywordstyle=\color{magenta},
  numberstyle=\tiny\color{codegray},
  stringstyle=\color{codepurple},
  basicstyle=\ttfamily\footnotesize,
  breakatwhitespace=false,         
  breaklines=true,                 
  captionpos=b,                    
  keepspaces=true,                 
  numbers=left,                    
  numbersep=5pt,                  
  showspaces=false,                
  showstringspaces=false,
  showtabs=false,                  
  tabsize=2
}

%"mystyle" code listing set
\lstset{style=mystyle}

\theoremstyle{definition}
\newtheorem{defn}{Definition}[section]
\newtheorem{example}[defn]{Example}
\theoremstyle{remark}
\newtheorem{rem}{Remark}
\newtheorem{remS}[section]{defn}
\theoremstyle{plain}
\newtheorem{lem}[defn]{Lemma}
\newtheorem{thm}[defn]{Theorem}
\newtheorem{prop}[defn]{Proposition}
\newtheorem{fact}[defn]{Fact}
\newtheorem{crly}[defn]{Corollary}
\newtheorem{conj}[defn]{Conjecture}

%\newtheorem*{programming*}{Programming Task}

%\newtheorem{innercustomgeneric}{\customgenericname}
%\providecommand{\customgenericname}{}
%\newcommand{\newcustomtheorem}[2]{%
%  \newenvironment{#1}[1]
%  {%
%   \renewcommand\customgenericname{#2}%
%   \renewcommand\theinnercustomgeneric{##1}%
%   \innercustomgeneric
%  }
%  {\endinnercustomgeneric}
%}

%\newcustomtheorem{question}{Question}
%\newcustomtheorem{programming}{Programming Task}

\newcommand{\NN}{\mathbb{N}}
\newcommand{\ZZ}{\mathbb{Z}}
\newcommand{\QQ}{\mathbb{Q}}
\newcommand{\RR}{\mathbb{R}}
\newcommand{\CC}{\mathbb{C}}
\newcommand{\PP}{\mathbb{P}}
\newcommand{\FF}{\mathbb{F}}
\newcommand{\Hom}{\operatorname{Hom}}
\newcommand{\im}{\operatorname{im}}
\newcommand{\id}{\operatorname{id}}
\newcommand{\Ind}{\operatorname{Ind}}
\newcommand{\Res}{\operatorname{Res}}

\newcommand{\calD}{\mathcal{D}}

\newcommand{\sol}{\textit{Solution: }}

\title{Number Fields}
\author{Kevin}
\date{January 2025}

\begin{document}
\maketitle
\section{Section 1}
(: (\textit{Owen's signature})
\subsection*{What?}
Field Extension: \(L\vert K\)




\begin{defn}
    A number field is a subfield $K\subseteq \CC$ s.t. $[K:\QQ]<\infty$.
\end{defn}
Let $\alpha$ be an algebraic number. There exists $f_\alpha\in\QQ[X]$ of minimal positive deg s.t. $f_\alpha(\alpha)=0$. 
\begin{itemize}
    \item (monic) min poly: leading coeff 1.
    \item min poly in $\ZZ[X]$: clearing denominators.
\end{itemize}
$[\QQ(\alpha):\QQ]=\deg f_\alpha$. These are all the examples by primitive element theorem.
\begin{example}
    Quadratic fields, cyclotomic fields. 
\end{example}
\subsection*{Why?}
We don't care... (so why are we here?) idk

\[);\models\tag{Owen's Signature}\]

\begin{defn}
    An algebraic number $\alpha$ is an algebraic integer if $f_\alpha\in\ZZ[X]$ (i.e., its monic minimal polynomial has integer coefficient)
\end{defn}
\begin{rem}
    If $\alpha$ is a root of a monic poly $f\in\ZZ[X]$, then $\alpha$ is an algebraic integer.
\end{rem}
\begin{example}
    Let $K=\QQ(\sqrt m)$
    If $m\not\equiv 3\pmod 4$, then $\mathcal O_K=\ZZ[\sqrt m]$. If $m\equiv 1\pmod 4$, then $\mathcal O_K=\ZZ[(1+\sqrt m)/2]$.

    For cyclotomic field $K=\QQ(\theta_n)$, $n\ge 3$, $\mathcal O_K=\ZZ[\theta_m]$. (We will prove this for $n$ prime later.)
\end{example}
\begin{thm}
    The set of algebraic integers $\mathcal{O}$ is a ring.
\end{thm}
Notation: We write $\mathcal{O}_K=K\cap\mathcal O$ for $K$ number field, and we call $\mathcal O_K$ the ring of integers in $K$.
\begin{proof}
    GRM ES4.s
\end{proof}
\begin{table}[H]
    \centering
    \begin{tabular}{c|c|c}
         0&$\longleftarrow^\bullet$ & $\overline\longrightarrow$ \\
         \hline
         $\leftrightarrow$&O &\\
         \hline
        \(\tilde{\mathcal{O}}\) & &\(\overset{\circ}{\chi}\)
    \end{tabular}
    \caption{Noughts and Arrows}
    \label{tab:my_label}
\end{table}

\begin{prop}
    Let $\alpha\in\CC$ be an algebraic number. TFAE,
    \begin{enumerate}[(1)]
        \item[(1)] $\alpha$ is an algebraic integer;
        \item $\ZZ[\alpha]$ is f.g. as a $\ZZ$-module
        \item $\exists$ a f.g. $\ZZ$-module $M$ that is invariant under multiplication by $\alpha$.
        \item
    \end{enumerate}
\end{prop}
\begin{proof}
    (1)$\Rightarrow$(2)$\Rightarrow$(3) is clear. Need to prove (3)$\Rightarrow$(1). Let $M=\ZZ\beta_1\oplus...\oplus \ZZ\beta_k$ be a module satisfying (3). Left multiplication by $\alpha$ is $\ZZ$-linear. Let $A$ be the matrix, then take the char poly of $A$.
\end{proof}

\end{document}