\documentclass{article}
\usepackage{graphicx} % Required for inserting images
\usepackage[utf8]{inputenc}
\usepackage{amsmath,amsfonts,amssymb,amsthm}
\usepackage{enumerate,bbm}
\usepackage{leftindex}
\usepackage{tikz,tikz-cd,graphicx,color,mathrsfs,color,hyperref,boldline}
\usepackage{caption,float}
\usepackage[a4paper,margin=1in,footskip=0.25in]{geometry}

\usepackage{listings}
\usepackage{xcolor}

\usepackage{tabularx,capt-of}

\usepackage{blindtext}
%Image-related packages
\usepackage{graphicx}
\usepackage{subcaption}
\usepackage[export]{adjustbox}
\usepackage{lipsum}

%hyperref setup
\hypersetup{
    colorlinks=true,
    linkcolor=blue,
    filecolor=magenta,      
    urlcolor=cyan,
    pdftitle={Overleaf Example},
    pdfpagemode=FullScreen,
    }

%New colors defined below
\definecolor{codegreen}{rgb}{0,0.6,0}
\definecolor{codegray}{rgb}{0.5,0.5,0.5}
\definecolor{codepurple}{rgb}{0.58,0,0.82}
\definecolor{backcolour}{rgb}{0.95,0.95,0.92}

%Code listing style named "mystyle"
\lstdefinestyle{mystyle}{
  backgroundcolor=\color{backcolour}, commentstyle=\color{codegreen},
  keywordstyle=\color{magenta},
  numberstyle=\tiny\color{codegray},
  stringstyle=\color{codepurple},
  basicstyle=\ttfamily\footnotesize,
  breakatwhitespace=false,         
  breaklines=true,                 
  captionpos=b,                    
  keepspaces=true,                 
  numbers=left,                    
  numbersep=5pt,                  
  showspaces=false,                
  showstringspaces=false,
  showtabs=false,                  
  tabsize=2
}

%"mystyle" code listing set
\lstset{style=mystyle}

\theoremstyle{definition}
\newtheorem{defn}{Definition}[section]
\newtheorem{example}[defn]{Example}
\theoremstyle{remark}
\newtheorem{rem}{Remark}
\newtheorem{remS}[section]{defn}
\theoremstyle{plain}
\newtheorem{lem}[defn]{Lemma}
\newtheorem{thm}[defn]{Theorem}
\newtheorem{prop}[defn]{Proposition}
\newtheorem{fact}[defn]{Fact}
\newtheorem{crly}[defn]{Corollary}
\newtheorem{conj}[defn]{Conjecture}

%\newtheorem*{programming*}{Programming Task}

%\newtheorem{innercustomgeneric}{\customgenericname}
%\providecommand{\customgenericname}{}
%\newcommand{\newcustomtheorem}[2]{%
%  \newenvironment{#1}[1]
%  {%
%   \renewcommand\customgenericname{#2}%
%   \renewcommand\theinnercustomgeneric{##1}%
%   \innercustomgeneric
%  }
%  {\endinnercustomgeneric}
%}

%\newcustomtheorem{question}{Question}
%\newcustomtheorem{programming}{Programming Task}

\newcommand{\NN}{\mathbb{N}}
\newcommand{\ZZ}{\mathbb{Z}}
\newcommand{\QQ}{\mathbb{Q}}
\newcommand{\RR}{\mathbb{R}}
\newcommand{\CC}{\mathbb{C}}
\newcommand{\PP}{\mathbb{P}}
\newcommand{\FF}{\mathbb{F}}
\newcommand{\Hom}{\operatorname{Hom}}
\newcommand{\im}{\operatorname{im}}
\newcommand{\id}{\operatorname{id}}
\newcommand{\Ind}{\operatorname{Ind}}
\newcommand{\Res}{\operatorname{Res}}

\newcommand{\calD}{\mathcal{D}}

\newcommand{\sol}{\textit{Solution: }}

\title{Topics in Analysis}
\author{Kevin}
\date{January 2025}

\begin{document}
\maketitle
\section*{Review of Metric Spaces}
Notation: open ball of radius $\delta$ will be denoted $B_\delta(x)$.

Revised some results about metric spaces.

\section{Brouwer's Fixed Point Theorem}
A proof of IVT using a compactness argument:
\begin{lem}[Discrete IVT]
    Let $f:\{0,1,...,n\}\to\RR$ with $f(0)<0$ and $f(n)\ge 0$, then $\exists m$ s.t. $f(m-1)<0$, $f(m)\ge 0$.
\end{lem}
\begin{proof}
    Can prove it using WOP.

    Alternatively (an idea which will be useful later), define $g:\{0,...,n\}\to\RR$ by $g(m)=1$ if $f(m)\ge 0$ and $0$ if $f(m)<0$. Then $g(n)-g(0)=\sum_n(g(m)-g(m-1))$, so there exists $g(m)-g(m-1)>0$, and we are done. 
\end{proof}
Now can prove IVT. $F:[a,b]\to\RR$, $F(a)<0,\ F(b)>0$, consider $f_n:\{0,...,n\}\to\RR$ defined by $f_n(m)=F(a+\frac{m}{n}(b-a))$. Apply lemma, get a sequence $x_n$ s.t. $F(x_n)\ge 0$ and $F(x_n-(b-a)/n)<0$. Apply Bolzano-Weierstrass.

\begin{thm}[Brouwer's fixed point theorem in one dimension]
    If $f:[a,b]\to[a,b]$ cts, then $f$ has a fixed point.
\end{thm}
This is a trivial consequence of IVT.

\begin{rem}
    In fact, Brouwer's fixed point theorem in one dimension is equivalent to IVT.
\end{rem}

We will generalize this result.

\begin{thm}
    TFAE,
    \begin{enumerate}
        \item[(a)] Any continuous self maps of $D^2$ has a fixed point. (2D Brouwer's fixed point theorem)
        \item[(b)] There is no retraction $D^2\to \partial D^2$. (No retraction theorem)
        \item[(c)] Let $T$ be a (closed solid) triangle with sides $I_1,I_2,I_3$. Let $A_1,A_2,A_3$ be closed subsets of $T$ s.t. $I_j\subseteq A_j$ and $A_1\cup A_2\cup A_3=T$. Then $A_1\cap A_2\cap A_3\neq\varnothing$.
        \item[(d)] There is no continuous function $f:T\to\partial T$ s.t. $\forall x\in I_j$, $f(x)\in I_j$. (No quasi-retraction theorem)
    \end{enumerate}
\end{thm}
\begin{proof}
    (a)$\Rightarrow$(b): Suppose (b) fails and $f:D\to\partial D$ is a retraction. Let $\sigma\neq\id$ be a non-trivial rotation about $(0,0)$. Then $\sigma\circ f:D\to D$ is continuous and has no fixed points. Contradiction.

    (b)$\Rightarrow$(a): standard proof of Brouwer's fixed point theorem from no retraction theorem. Construct a retraction by mapping $x\in D$ to the intersection of the ray from $f(x)$ to $x$ with $\partial D$. 

    (a)$\Rightarrow$(d): Suppose (d) fails. Take $T$ to be an equilateral triangle with vertices $1,\omega,\omega^2\in\CC$ (cube roots of unity). For each polar angle $\theta$, define $\rho(\theta)=d(0,x)$, where $x$ is the point on $\partial T$ with argument $\theta$. $\rho$ is cts in $\theta$ and $1/2\le\rho(\theta)\le 1$ for all $\theta$. Define $\varphi:D\to T$, $\varphi(z)=\rho(\theta)z$ if $z\in\rho(\theta) e^{i\theta}$ This is a homeomorphism. $\varphi(\partial D)=\partial T$. Now let $g:D\to D$, $g=\varphi^{-1}\circ f\circ\varphi$. Then $\varphi^{-1}(I_j)$ is a $2\pi/3$ radian arc of $\partial D$. If $x\in\varphi^{-1}(I_j)$ and $\varphi(x)\in I$, then $f(\varphi(x))\in I_j$, so $g(x)=\in\varphi^{-1}(I_j)$. Also $x\in D$, have $\varphi(x)\in T\Rightarrow f(\varphi(x))\in\partial T\Rightarrow g(x)\in\partial D$. Let $\pi$ be the antipodal map, then $\pi\circ g$ has no fixed points.

    (c)$\Rightarrow$(d): Suppose $f:T\to\partial T$ is a quasi-retraction. Let $A_j=f^{-1}(I_j)$ for $j=1,2,3$. These are closed subsets of $T$ and $A_1\cup A_2\cup A_3=T$. Since $I_1\cap I_2\cap I_3=\varnothing$, we deduce that $A_1\cap A_2\cap A_3=\varnothing$, which contradicts (c).

    (d)$\Rightarrow$(c): Choose a convenient triangle. Let $T=\{(x,y,z)\in\RR^3: x,y,z\ge 0,\ x+y+z=1\}$. This is the standard 2-simplex spanned by $e_1,e_2,e_3$. Let $I_j=\langle e_k:k\neq j\rangle$. If (c) fails, then we can find $A_j$, $j=1,2,3$ with $I_j\subseteq A_j$, $\bigcup_j A_j=T$, and $\bigcap_jA_j=\varnothing$. Define
    \[f(x)=\dfrac{(d(x,A_1),d(x,A_2),d(x,A_3)}{\sum_jd(x,A_j)}\]
    Note that $A_j$ is closed and bounded for all $j$, so each $d(x,A_j)$ is finite, and $\bigcap_jA_j=\varnothing$ ensures that the denominator is never $0$. $f$ is also cts. If $x\in I_j\subseteq A_j$, then $d(x,A_j)=0$, then $f(x)\in I_j$, so $f(I_j)\subseteq I_j$. Contradiction.

    (d)$\Rightarrow$(b): Suppose (b) fails, i.e., we can find a retraction $f:D\to\partial D$. Recall the homeomorphism $\varphi:D\to T$ from an earlier part of the proof. Let $g=\varphi\circ f\circ\varphi^{-1}$, then $g:T\to \partial T$ is continuous. If $x\in T$, then $f(\varphi^{-1}(x))\in\partial D$, then $g(x)\in\partial T$. If $x\in\partial T$, then $\varphi^{-1}(x)\in\partial D$, then $f(\varphi^{-1}(x))=\varphi^{-1}(x)$, so $g(x)=x$. In particular, $g$ is a quasi-retraction.
\end{proof}

We prove (c). First, some technical results.
\begin{defn}
    For $n\ge 2$, we define the $n$th subdivision of a triangle $T$ (or the $n$th triangular grid) by dissecting each edge into $n$ equal subintervals and join them by lines parallel to the sides.
\end{defn}
\begin{thm}[Discrete version of (c)]
    Let $T$ be a triangle with edges $I_j$, $j=1,2,3$, subdivided into a triangular grid. Suppose each vertex of the grid has one of $3$ colors $R_1,R_2,R_3$ s.t. $\forall$ vertex in $I_j$ is either $R_j$ or $R_{j-1}$ (index mod $3$). Then there exists a triangle in the grid s.t. all vertices have distinct colors (multicolored).
\end{thm}
\begin{proof}
    Given an edge between two adjacent vertices of the grid and a direction, we assign a value to that edge: if the two vertices have the same color then $0$, if $R_j\to R_{j-1}$ then $1$, if $R_j\to R_{j+1}$, then $-1$. (Observation: if we have a small triangle in the grid, then summing over its edges counterclockwise gives a non-zero value if and only if all three vertices have distinct colors. In this situation, the value is $\pm 3$.)
    Now consider the the sum over all small triangles in the grid (each counterclockwise), then each internal edge appears exactly twice with opposite orientation, so we only pick up values from the boundary of the large triangle $T$. (The contribution is $1$ from each $I_j$ since we can only have two colors.) The total is $3$, so as we sum over each small triangle,  there must exist at least one small triangle with non-zero contribution, and this happens precisely when it is multicolored.
\end{proof}
We now prove (c).
\begin{proof}
    For each $n\ge 2$, consider the $n$th triangular grid in $T$. For all vertices $x$ of the grid, if $x$ has color $R_j$ then $x\in A_j$. Can do this in a way so that all vertices in $I_j$ are $R_j$ or $R_{j-1}$. Apply the preceding theorem for each $n$. We see that for each $n$, there exists a triangle with vertices $x_n,y_n,z_n$ which multi-colored. Find a subsequence $x_{n_k}\to x\in T$ by sequential compactness. Then $y_{n_k},z_{n_k}\to x$. By closedness of $A_j$, we see that $x\in\bigcap_{j=1}^3A_j$. 
\end{proof}
\begin{rem}
    Brouwer's fixed point theorem holds for any compact convex set in $D\subseteq\RR^n$.
\end{rem}

\section{The Degree of a Closed Curve and Winding Number}
Recall from CA that we can define $\operatorname{arg}_a$ for $z$ in the cut plane.
\begin{lem}
    $\arg_a$ is cts on its domain.
\end{lem}
So if $f:[0,1]\to \CC$ is a cts function to a cut plane, then we have a continuous choice of argument defined on $f$.
\begin{thm}
    Let $f:[0,1]\to\CC^\ast$ be cts. Then there is a cts choice of argument on $f$, i.e., $\exists\theta:[0,1]\to\RR$ cts s.t.$f(t)=|f(t)|e^{i\theta(t)}$. This choice is unique up to adding $2n\pi,\ n\in\ZZ$.
\end{thm}
\begin{proof}
    By uniform continuity, partition $[0,1]$ into a number of segments $[t_i,t_{i+1}]$ s.t. $|f(t_k)-f(t_{k-1})|<\epsilon$, where $\epsilon$ is s.t. $B_\epsilon(0)\subseteq \CC\setminus\im(f)$. 

    Induction: suppose there is a cts choice $\theta$ on $[0,t_k]$. Note that $\im(f|_{[t_k,t_{k+1}]})\subseteq B_\epsilon(f(t_k))$. This lies in a cut plane, so we choose a cts $\operatorname{arg}_a$ s.t. $\theta(t_k)=\arg_a(t_k)$, so the argument function is extendable.

    For uniqueness, if $\phi(t)$ is another choice, then $\theta(t)-\phi(t)$ is a cts function to $2n\ZZ$.
\end{proof}
\begin{defn}
    $f:[0,1]\to\CC^\ast$ be a closed path. The winding number (degree) of $f$ about $0$ is $\frac{1}{2\pi}(\theta(1)-\theta(0))$ where $\theta$ is a choice of argument on $f$.
\end{defn}

\begin{lem}
    $f,h:[0,1]\to\CC^\ast$. two cts closed paths, $w(fh,0)=w(f,0)+w(h,0)$.
\end{lem}
\begin{proof}
    If $\theta,\phi$ are cts arg for $f,h$ respectively, then $\theta+\phi$ is a cts arg of $fh$.
\end{proof}
\begin{crly}[`Dog-walking' lemma]
    Let $f:[0,1]\to\CC^\ast$ be a closed path. $g:[0,1]\to\CC$ another closed path s.t. $|g(t)|<|f(t)|$ for all $t$. Then $w(f,0)=w(f+g,0)$. 
\end{crly}
\begin{proof}
    Write $f+g=f(1+g/f)$. Let $h=1+g/f$, then $\im h$ lies in the right half plane, so $h$ has a cts arg and $w(h,0)=0$. Done by the lemma.
\end{proof}
Define homotopy between two paths.
\begin{lem}
    Let $f_1,f_2:[0,1]\to\CC^\ast$ be closed paths. If $f_1\simeq_F f_2$, where $F:[0,1]^2\to\CC^\ast$, then $w(f_1,0)=w(f_2,0)$.
\end{lem}
\begin{proof}
    Note that $\im F$ is closed in $\CC$, so can find $\epsilon>0$ s.t. $|F(s,t)|>\epsilon$ for all $s,t$. By unif continuity, find $\delta>0$ s.t. $d((s,t),(s',t'))<\delta\Rightarrow|F(s,t)-F(s',t')|<\epsilon$. Choose a dissection $\{0=s_0<s_1<\ldots<s_n=1\}$ s.t. $s_{k+1}-s_k<\delta$. Then for all $k$ and $t$, we have
    \[F(s_k,t)=F(s_{k+1},t)+h_k(t)\]
    where $|h(t)|<\epsilon$. Since $|F(s_{k+1},t)|>\epsilon$, we are in the situation of the previous lemma. Done.
\end{proof}
\begin{crly}[Topological argument principle]
    Let $D$ be the closed unit disk in $\CC$ and $f:D\to \CC$ cts s.t. $f(z)\neq 0$ whenever $|z|=1$. Let $g:[0,1]\to\CC$ with $g(t)=f(e^{i2\pi t})$. If $w(g,0)\neq 0$, then $f$ has a root in $D$.
\end{crly}
\begin{proof}
    Suppose $f$ is never zero in $D$. We define $F(s,t)=f(se^{i2\pi t})$ for $s,t\in [0,1]^2$, i.e., $g$ is null-homotopic, so has winding number zero.
\end{proof}
\begin{crly}
    No retraction $D\to\partial D$.
\end{crly}
\begin{proof}
    If such a retraction exists, then define $g(t)$ as in the previous corollary. Then $w(g,0)=1$, so there exists $z$ s.t. $f(z)=0$. Contradiction.
\end{proof}
\begin{crly}[FTA]
    Every non-constant complex poly has a root in $\CC$.
\end{crly}
\begin{proof}
    Let $p(z)=a_nz^n+\ldots+a_0$, $a_n\neq 0$. Choose $r>1$ s.t. $|a_nr|>|a_{n-1}|+\ldots+|a_0|$. Then for $|z|=r$, have
    \begin{align*}
        |a_nz^n|&> (|a_{n-1}|+\ldots+|a_0|)r^{n-1}\\
        &\ge |a_{n-1}z^{n-1}|+\ldots+|a_0|\\
        &\ge |a_{n-1}z^{n-1}+\ldots+a_0|
    \end{align*}
    Let $f:D\to\CC,\ z\mapsto p(rz)$, so $f(z)=g(z)+h(z)$ where $g(z)=a_n(rz)^n$. We have $|g(z)|>|h(z)|$ when $|z|=1$. Consider the paths $u=f(e^{i2\pi t}), v=g(e^{i2\pi t}), a=h(e^{i2\pi t})$. Have $|a(t)|<|v(t)|$. Apply dog-walking lemma, get $w(u,0)=w(v,0)=n$. [Write $a_n=qe^{i\alpha}$, define $\theta(t)=2\pi nt+\alpha$ is a cts choice of argument]. So there exists $z_0$ s.t. $f(z_0)=0$, so $p(rz_0)=0$.
\end{proof}

\section{Polynomial Approximation}
\begin{prop}[Chebyshev Inequality]
    Let $X$ be a square integrable r.v. with $\mathbb EX=\mu$ and $\operatorname{var}(X)=\sigma^2$, then
    \[\PP(|X-\mu|\ge c\sigma)\le\dfrac{1}{c^2}\]
\end{prop}
\begin{proof}
    $\sigma^2=\mathbb E[(X-\mu)^2]\ge (c\sigma^2)\mathbb E[1_{|X-\mu|\ge c\sigma}]$
\end{proof}
\begin{crly}
    Let $X_1,...,X_n$ be i.i.d Bernoulli variables with parameter $t\in[0,1]$. Let $Y_n(t)=\frac{1}{n}\sum_{i=1}^n X_i$. Then
    \[\forall \delta>0,\ \PP(|Y_n(t)-t|\ge\delta)\le\dfrac{1}{\delta^2n}\]
\end{crly}
\begin{proof}
    Observe that $\sigma^2=\operatorname{var}(Y_n)\le 1/n$. Apply Chebyshev with $c=\delta\sqrt n$.
    \[\PP(|Y_n-t|\ge \delta)\le \PP(|Y_n-t|\ge (\sigma\sqrt n)\delta)\le\dfrac{1}{\delta^2n}\]
    Note that $\sigma<1/\sqrt{n}$, so the first inequality makes sense.
\end{proof}
\begin{thm}[Weierstrass approximation theorem]
    Let $f:[0,1]\to\RR$ be cts. For all $\epsilon>0$, there exists a poly $p$ s.t. $\|p-f\|_\infty<\epsilon$.
\end{thm}
\begin{proof}
    For each $n=1,2,3,...$, let $Y=Y_n(t)$ be the r.v. defined in the previous corollary.

    \textbf{Step 1:} We show that $\mathbb E[f(Y_n(t))]$ is a polynomial in $t$. Compute
    \[\mathbb E[f(Y_n(t))]=\sum_{k=0}^n f(k/n)\PP(Y_n(t)=k/n)=\sum_{k=0}^nf(k/n)\binom{n}{k}t^k(1-t)^{n-k}\]

    \textbf{Step 2:} We show that for all $\epsilon>0$, there exists a sufficiently large $n$ s.t. $|\mathbb E[f(Y_n(t))]-f(t)|<\epsilon$ for all $t\in [0,1]$. We know that $\exists M>0$ s.t. $|f(t)|<M$ for all $t$. Let $\epsilon>0$. Uniform continuity implies that $|f(s)-f(t)|\le \epsilon/2$ whenever $|s-t|\le\delta$. Now $$\PP(|f(Y_n(t))-f(t)|\ge\epsilon)\le \PP(|Y_n-t|\ge\delta)\le\dfrac{1}{\delta^2n}$$
    Choose $n$ s.t. $\frac{1}{\delta^2n}<\frac{\epsilon}{4M}$ so that
    \begin{align*}|\mathbb E(f(Y_n(t))-f(t))|&\le\mathbb E|f(Y_n(t))-f(t)|\\
    &=\mathbb E(|f(Y_n(t))-f(t)|1_{|f(Y_n(t))-f(t)|<\epsilon/2})+\mathbb E(|f(Y_n(t))-f(t)|1_{|f(Y_n(t))-f(t)|\ge\epsilon/2})\\
    &\le\frac{\epsilon}{2}+2M\frac{\epsilon}{4M}=\epsilon
    \end{align*}
    and we are done.
\end{proof}
\begin{crly}
    Let $f:[0,1]\to\RR$ cts. Then there exists a seuquence of polys $(p_k)_{k\in\NN}$ s.t. $p_k\to f$ unif. on $[0,1]$ as $k\to\infty$.
\end{crly}
\begin{thm}[Chebyshev equal ripple criterion]
    Let $f:[0,1]\to\RR$ be cts and $p$ a polynomial of degree $p<n$. Suppose there exists $a\le a_0<a_1<\cdots<a_n\le b$ s.t. $\|f-p\|_\infty$ such that \textbf{exactly one of the following holds}\begin{enumerate}[(1)]
        \item $f(a_k)-p(a_k)=(-1)^k\|f-p\|_\infty$ for all $k=0,...,n$
        \item $f(a_k)-p(a_k)=(-1)^{k+1}\|f-p\|_\infty$ for all $k=0,...,n$.
    \end{enumerate}
    Then, we have $\|f-p\|_\infty\le\|q-f\|_\infty$ for all polynomials $q$ with $\deg q<n$. (i.e. $p$ is hte best poly approximation of $f$ of degree less than $n$)
\end{thm}
\begin{proof}
    Suppose $p$ has property $(1)$ and $q$ is a poly of deg $<n$ with $\|q-f\|_\infty<\|p-f\|_\infty$. Then
    \begin{itemize}
        \item for $k$ even, we have $f(a_k)-p(a_k)=\|f-p\|_\infty$ and $f(a_k)-q(a_k)<r$ and $f(a_k)-q(a_k)<r$, so $q(a_k)>p(a_k)$
        \item for $k$ odd, similar argument shows that $q(a_k)<p(a_k)$.
    \end{itemize}
    By IVT, we can find a root of $p(t)-q(t)$ between $(a_k,a_{k+1})$, so $p(t)-q(t)$ has $n$ roots. But this has degree $<n$, so $p\equiv q$.    
\end{proof}
In fact the equal ripple condition is also necessary.

We try to find a poly of deg $<n$ which best approximates $t^n$.
\begin{lem}
    $\cos(n\theta)$ is a poly of deg $n$ in $\cos\theta$ with leading coeff $2^{n-1}$.
\end{lem}
\begin{proof}
    Expand $(\cos\theta+i\sin\theta)^n$.
\end{proof}
\begin{defn}
    The $n$th Chebyshev polynomial $T_n(x)$ is defined by $T_n(\cos\theta)=\cos(n\theta)$.
\end{defn}
Observe that $|T_n(t)|\le 1$ for $t\in[-1,1]$ with roots $t=\cos\frac{(2l-1)\pi}{2n}$ for $l=1,...,n$. It reaches extremum at $\cos(k\pi/n)$. If $a_k=\cos(k\pi/n)$, then $1=a_0>\cdots>a_n=-1$ and $T_n(a_k)=(-1)^k$.

We can write $T_n(t)=2^{n-1}(t^n-S_n(t))$ for some poly $S_n$, so $S_n(t)=t^n-2^{1-n}T_n(t)$. Since $|t^n-S_n(t)|\le 2^{1-n}$ and $a_k^n-S_n(a_k)=2^{1-n}T_n(a_k)=(-1)^k2^{1-n}$. Therefore by Chebyshev criterion, we get
\begin{prop}
    $S_n(t)$ is the best poly approximation (deg $<n$) of $t^n$ on $[-1,1]$. For any polynomial $Q(t)$ of degree $<n$, we have $\|t^n-Q(t)\|_\infty\ge 2^{1-n}$.
\end{prop}
\begin{crly}\
    \begin{enumerate}[(i)]
        \item If $p:[-1,1]\to\RR$ defined by $\sum_{j=0}^na_jt^j$ with $|a_n|\ge 1$, then $\|p\|_\infty\ge 2^{-n+1}$.
        \item $\forall n\in\NN$, $\exists\epsilon=\epsilon(n)>0$ s.t. if $p(t)=\sum_{j=1}^na_kt^j$ (so deg $\le n$) and $|a_k|\ge 1$ for some $n\ge k\ge 0$, then $\|p\|_\infty\ge\epsilon$.
    \end{enumerate}
\end{crly}
\begin{proof}
    (i): Consider
    \[\|p\|_\infty=|a_n|\sup_{t\in[-1,1]}\left|t^n+\sum_{j=0}^{n-1}\frac{a_n}{|a_n|}t^n\right|\ge 2^{1-n}\]
    (ii) Proceed by induction. If $n=0$, then pick $\epsilon=1$. Suppose the result holds for all poly of deg $\le n$. Consider $p(t)=a_{n+1}t^{n+1}+\sum_{j=0}^na_jt^j=a_{n+1}t^{n+1}+Q(t)$.
    \begin{itemize}
        \item If $|a_{n+1}|<\epsilon(n)/2$, then $\|p\|_\infty\ge\|Q\|_\infty-|a_{n+1}|\ge\epsilon(n)/2$.
        \item If $|a_{n+1}|\ge\epsilon(n)/2$, then $\|p\|\ge \|t^n+\frac{1}{a_{n+1}}Q(t)\|_\infty\ge |a_{n+1}|2^{-n}\ge\epsilon(n)2^{-1-n}$, so we set $\epsilon(n+1)=\epsilon(n)2^{-1-n}$.
    \end{itemize} 
\end{proof}
\begin{thm}
    Let $f:[-1,1]\to\RR$ be a cts function. For all $n\in\NN$, there exists a poly $p$ of deg $\le n$ s.t. $\|p-f\|_\infty\le\|q-f\|_\infty$ for all polys $q$ of deg $\le n$.
\end{thm}
\begin{proof}
    Let $q(t)=\sum_{j=0}^nb_jt^j$.
    \begin{itemize}
        \item Suppose $\exists j$ s.t. $b_j\ge\frac{2\|f\|_\infty+1}{\epsilon(n)}$. Then $\|q\|_\infty\ge\epsilon(n)\frac{2\|f\|_\infty+1}{\epsilon(n)}=2\|f\|_\infty+1$ by (ii) of the previous corollary. In this case, $\|q-f\|_\infty\ge2\|f\|_\infty+1-\|f\|_\infty>\|f\|_\infty$. Certainly not the best poly approx.
        \item Otherwise, let $$E=\{q(t)=\sum_{j=0}^nb_jt^n:\forall j,\ |b_j|<\frac{2\|f\|_\infty+1}{\epsilon(n)}\}$$
        Define $\kappa=\inf_{q\in E}\|q-f\|_\infty$. Then there exists $q_m\in E$ s.t. $\|q_m-f\|_\infty\to\kappa$ as $m\to\infty$. Write $q_m(t)=\sum_{j=0}^nb_j(m)t^j$. Note that the coefficients (regarded as a vector) is a bounded sequence in $\RR^{n+1}$, so admits a convergent subsequence by B-W. Let $p^\ast$ denote the limit. Have $\|q_{m_k}-p^\ast\|\to 0$.
    \end{itemize}
    Can check that $p^\ast$ is the best approximation.
\end{proof}

\section{Gaussian Integration}
Recall inner products and $L^2$ norm on $C[-1,1]$.
\begin{lem}
    There exists a sequence of polynomials $(p_n)_{n\in\NN}\in C[-1,1]$ with $\deg p_n=n$ and $\langle p_m,p_n\rangle=\delta_{nm}$. Moreover, $p_n$ is unique up to scalar multiple.
\end{lem}
Can fix $\|p_n\|_2=1$ and require the leading coefficient is $>0$.
\begin{proof}
    Start with $\{t^i:i\in\NN\}$. Apply Gram-Schmidt.
\end{proof}
\begin{crly}
    If $f$ is a poly of degree $<n$, then $\int_{-1}^1fp_n=0$.
\end{crly}
\begin{lem}
    $p_n$ has distinct roots, all in $[-1,1]$.
\end{lem}
\begin{proof}
    Suppose $x_1,...,x_k$ are all the distinct root of $p_n$ having odd multiplicity. Consider $q(t)=(t-x_1)(t-x_k)$, $k\le n$. Then $q$ changes sign precisely when $p_n$ changes sign, so $qp_n$ doesn't change sign. Since $qp_n$ is non-zero, we have $\langle q,p_n\rangle\neq 0$, so $k=n$ (forced by the preceding corollary).
\end{proof}
\begin{defn}
    $p_n$ discussed above is called the $n$th Legendre polynomial.
\end{defn}
\begin{thm}\
    \begin{enumerate}[(i)]
        \item Let $\alpha_1,...,\alpha_n$ be the roots of $n$th Legendre polynomial $p_n$ and $A_1,...,A_n$ chosen so that $\int_{-1}^1P(t)dt=\sum_{j=1}^nA_jP(\alpha_j)$ whenever $\deg p<n$. Then, $\int_{-1}^1Q(t)dt=\sum_{j=1}^nA_jQ(\alpha_j)$ whenever $Q$ is a poly of degree $<2n$.
        \item If $\beta_j,B_j$ are alternative choices s.t. (i) holds, then $\{\beta_j\}=\{\alpha_j\}$ are the roots of Legendre polynomial.
    \end{enumerate}
\end{thm}
\begin{proof}
    (i): If $\deg Q<2n$, then $Q=qp_n+r$ for some poly $r$ s.t. $\deg r<n$. Note that $\deg q<n$ so that 
    $\int_{-1}^1Q(t)dt=\int_{-1}^1 r(t)dt=\sum_jA_jr(\alpha_j)=\sum_j A_j(q(\alpha_j)p_n(\alpha_j)+r(\alpha_j))=\sum_jA_jQ(\alpha_j)$.

    (ii): Suppose we have another set of choices. So $\int_{-1}^1Q(t)dt=\sum_jB_jQ(\beta_j)$ whenever $\deg Q<2n$. Consider $p(t)=\prod_j(t-\beta_j)$. Consider $\langle p,q\rangle$ for $\deg q<n$. Note that $\deg pq<2n$, so $\int_{-1}^1pq=\sum_jB_jp(\beta_j)q(\beta_j)=0$. By the uniqueness of the construction of $p_n$, $\beta_j$ are roots of $p_n$.
\end{proof}
\begin{lem}
    Let $\alpha_1,...,\alpha_n$ be the roots of the $n$th Legendre poly and $A_1,...,A_n$ numbers defined as in the previous theorem. Then \begin{enumerate}[(i)]
        \item $\sum_iA_i=2$
        \item $A_i\ge 0$ for all $i=1,...,n$
    \end{enumerate}
\end{lem}
\begin{proof}
    (i): Apply the first part of the preceding theorem to $Q(t)=1$.

    (ii): Define $Q_i(t)=\prod_{j\neq i}(t-\alpha_j)^2$
\end{proof}
\begin{thm}
    For $n\ge 1$, let $\alpha_1,...,\alpha_n$ be the roots of the $n$th Legendre poly and $A_1,...,A_n$ as in the preceding theorem.
    Let $f:[-1,1]\to\RR$ be a continuous function and $\epsilon>0$.
    Then for all sufficiently large $n$
    \[\left|\int_{-1}^1f(t)dt-\sum_{i=1}^nA_if(\alpha_i)\right|<\epsilon\]
\end{thm}
\begin{proof}
    By Weierstrass approximation theorem, there exists a polynomial $p$ s.t. $\sup_{t\in[-1,1]}|f(t)-p(t)|<\epsilon/4$. For all $n>\frac12\deg p$, have
    \begin{align*}
        \left|\int_{-1}^1f(t)dt-\sum_{i=1}^nA_if(\alpha_i)\right|&\le\left|\int_{-1}^1(f(t)-p(t))dt\right|+\left|\sum_iA_i(p(\alpha_i)-f(\alpha_i))\right|\\
    \end{align*}
    The first term is $\le \epsilon/2$. The second term is $\le\epsilon/2$ by the preceding lemma.
\end{proof}
\begin{rem}
    Can further show that \begin{enumerate}[(i)]
        \item Legendre poly $p_n=c\frac{d^n}{dt^n}(1-t^2)^n$ (cf. ES2 Q11)
        \item Chebyshev polys can be presented as an orthogonal seuqnece by integrating w.r.t. a weight function. (cf. ES2 Q13)
    \end{enumerate}
\end{rem}

\section{Approximation by Complex Polynomials}
When $S\subseteq\CC$ is any set, a function being holo'c on $S$ means that it is holo'c on some open set containing $S$.
\end{document}