\documentclass{article}
\usepackage{graphicx} % Required for inserting images
\usepackage[utf8]{inputenc}
\usepackage{amsmath,amsfonts,amssymb,amsthm}
\usepackage{enumerate,bbm}
\usepackage{leftindex}
\usepackage{tikz,tikz-cd,graphicx,color,mathrsfs,color,hyperref,boldline}
\usepackage{caption,float}
\usepackage[a4paper,margin=1in,footskip=0.25in]{geometry}
\usepackage{wasysym}
\usepackage{listings}
\usepackage{xcolor}

\usepackage{tabularx,capt-of}

\usepackage{blindtext}
%Image-related packages
\usepackage{graphicx}
\usepackage{subcaption}
\usepackage[export]{adjustbox}
\usepackage{lipsum}

%hyperref setup
\hypersetup{
    colorlinks=true,
    linkcolor=blue,
    filecolor=magenta,      
    urlcolor=cyan,
    pdftitle={Overleaf Example},
    pdfpagemode=FullScreen,
    }

%New colors defined below
\definecolor{codegreen}{rgb}{0,0.6,0}
\definecolor{codegray}{rgb}{0.5,0.5,0.5}
\definecolor{codepurple}{rgb}{0.58,0,0.82}
\definecolor{backcolour}{rgb}{0.95,0.95,0.92}

%Code listing style named "mystyle"
\lstdefinestyle{mystyle}{
  backgroundcolor=\color{backcolour}, commentstyle=\color{codegreen},
  keywordstyle=\color{magenta},
  numberstyle=\tiny\color{codegray},
  stringstyle=\color{codepurple},
  basicstyle=\ttfamily\footnotesize,
  breakatwhitespace=false,         
  breaklines=true,                 
  captionpos=b,                    
  keepspaces=true,                 
  numbers=left,                    
  numbersep=5pt,                  
  showspaces=false,                
  showstringspaces=false,
  showtabs=false,                  
  tabsize=2
}

%"mystyle" code listing set
\lstset{style=mystyle}

\theoremstyle{definition}
\newtheorem{defn}{Definition}[section]
\newtheorem{example}[defn]{Example}
\theoremstyle{remark}
\newtheorem{rem}{Remark}
\newtheorem{remS}[section]{defn}
\theoremstyle{plain}
\newtheorem{lem}[defn]{Lemma}
\newtheorem{thm}[defn]{Theorem}
\newtheorem{prop}[defn]{Proposition}
\newtheorem{fact}[defn]{Fact}
\newtheorem{crly}[defn]{Corollary}
\newtheorem{conj}[defn]{Conjecture}

%\newtheorem*{programming*}{Programming Task}

%\newtheorem{innercustomgeneric}{\customgenericname}
%\providecommand{\customgenericname}{}
%\newcommand{\newcustomtheorem}[2]{%
%  \newenvironment{#1}[1]
%  {%
%   \renewcommand\customgenericname{#2}%
%   \renewcommand\theinnercustomgeneric{##1}%
%   \innercustomgeneric
%  }
%  {\endinnercustomgeneric}
%}

%\newcustomtheorem{question}{Question}
%\newcustomtheorem{programming}{Programming Task}

\newcommand{\NN}{\mathbb{N}}
\newcommand{\ZZ}{\mathbb{Z}}
\newcommand{\QQ}{\mathbb{Q}}
\newcommand{\RR}{\mathbb{R}}
\newcommand{\CC}{\mathbb{C}}
\newcommand{\PP}{\mathbb{P}}
\newcommand{\FF}{\mathbb{F}}
\newcommand{\Hom}{\operatorname{Hom}}
\newcommand{\im}{\operatorname{im}}
\newcommand{\id}{\operatorname{id}}
\newcommand{\Ind}{\operatorname{Ind}}
\newcommand{\Res}{\operatorname{Res}}
\newcommand{\e}{\varepsilon}
\newcommand{\calD}{\mathcal{D}}

\newcommand{\sol}{\textit{Solution: }}

\title{Logic and Set Theory}
\author{Kevin}
\date{January 2025}

\begin{document}
\maketitle
\section{Propositional Logic}
\((\; :\langle\) (by Owen)
\begin{defn}[Propositional logic]
    The language of propositional logic consists of a set $P$ of primitive propositions and the set $L=L(P)$ of propositions (or compound propositions) built as follows
    \begin{enumerate}
        \item[(i)] $P\subseteq L$
        \item[(ii)] $\bot\in L$ ($\bot$ is called `false' or `bottom')
        \item[(iii)] If $p,q\in L$, then $(p\Rightarrow q)\in L$.
    \end{enumerate}
\end{defn}
\begin{example}
    Often $P=\{p_1,p_2,...\}$ (we need a countable infinity supply). Then $(p_1\Rightarrow p_2),\ (\bot\Rightarrow(p_1\Rightarrow p_2)),\ ((p_1\Rightarrow p_2)\Rightarrow(p_1\Rightarrow p_2))\in L$. Also, if $p\in L$, then $((p\Rightarrow \bot)\Rightarrow\bot)\in L$.
\end{example}
\begin{rem}\
    \begin{itemize}
        \item The set $L$ is built from $P$ and $\bot$ inductively, which means that $L=\bigcup_{n\in\NN}L_n$, where $L_1=P\cup\{\bot\}$ and for $n\in \NN$, $L_{n+1}=L_n\cup\{(p\Rightarrow q)\colon p,q\in L_n\}$
        \item Every proposition is a finite string of symbols in the alphabet $P\cup\{\bot,\Rightarrow,(,)\}$. It's easy to check that $L$ is the smallest subset of the set $\Sigma$ of all such strings satsifying (i)-(iii). Note that $L\neq\Sigma$, e.g. $)\Rightarrow p$ is not a proposition.
        \item Every proposition $p$ is built uniquely using (i)-(iii), i.e., either $p\in P$ or $p=\bot$ or $p=(q\Rightarrow r)$ for some unique $q,r\in L$.
        \item We will introduce $\wedge,\ \vee$ later.
    \end{itemize}
\end{rem}
\subsection{Semantic Entailment}
\begin{defn}[Valuation]
    A valuation on $L$ is a function $v:L\to\{0,1\}$ s.t. 
    \begin{enumerate}
        \item[(i)] $v(\bot)=0$
        \item[(ii)] $v(p\Rightarrow q)=\begin{cases}
            0 & v(p)=1 \text{ and } v(q)=0\\
            1 & \text{o/w}
        \end{cases}$
    \end{enumerate}
\end{defn}
\begin{prop}\
    \begin{enumerate}
        \item[(i)] If $v,v'$ are valuations and $v|_P=v'|_P$, then $v=v'$.
        \item[(ii)] For any function $w:P\to\{0,1\}$, there exists a valuation on $L$ s.t. $v|_P=w$.
    \end{enumerate}
\end{prop}
\begin{proof}\
    \begin{enumerate}
        \item[(i)] We have $v(\bot)=v'(\bot)=0$ by definition, so $v|_{L_1}=v'|_{L_1}$. Continue inductively. If $v|_{L_n}=v'|_{L_{n}}$, then for all $p\in L_{n+1}\setminus L_n$, there exists unique $q,r\in L_n$ s.t. $p=(q\Rightarrow r)$. Then $v(q\Rightarrow r)=v'(q\Rightarrow r)$.
        \item[(ii)] Define $v(p)=w(p)$ for all $p\in P$. Extend $v$ to $L$ inductively.
    \end{enumerate}
\end{proof}
\begin{defn}[Tautology]
    Let $t\in L$. We say that $t$ is a tautology if $v(t)=1$ for all valuations $v$ on $L$.
\end{defn}
\begin{example}
    The following are all tautologies
    \begin{enumerate}
        \item[(i)] $(p\Rightarrow (q\Rightarrow p))$, $p,q\in L$. A true statement is implied by anything.
        \item[(ii)] $(\neg\neg p\Rightarrow p)$, i.e., $(((p\Rightarrow\bot)\Rightarrow\bot)\Rightarrow p)$ for any $p\in L$. `The law of excluded middle', can be written as $(\neg p\vee q)$
        \item[(iii)] $(p\Rightarrow(q\Rightarrow r)\Rightarrow ((p\Rightarrow q)\Rightarrow(p\Rightarrow r)))$. If not a tautology then there is a valuation s.t. the whole thing evaluates to $0$, then $v((p\Rightarrow(q\Rightarrow r)))=1$ and $v((p\Rightarrow q)\Rightarrow(p\Rightarrow r))=0$. Then $v(p\Rightarrow q)=1$ and $v(p\Rightarrow r)=0$, so $v(p)=1, v(r)=0, v(q)=1$, then $v(p\Rightarrow (q\Rightarrow r))=0$, contradiction.
    \end{enumerate}
\end{example}
\textbf{Notations:}
\begin{itemize}
    \item (true/top) $\top=(\bot\Rightarrow\bot)$
    \item (not) $\neg p=(p\Rightarrow\bot)$
    \item (or) $p\vee q=(\neg p\Rightarrow q)=((p\Rightarrow\bot)\Rightarrow q)$
    \item (and) $p\wedge q=\neg (p\Rightarrow\neg q)$
\end{itemize}
\begin{defn}[Semantic entailment]
    Let $S\subseteq L$, $t\in L$. Say $S$ (semantically) entails $t$ if for any valuation $v$, $(\forall s\in S, v(s)=1)\implies v(t)=1$. We denote this by $S\models t$.
\end{defn}
Note that $t$ is a tautology iff $\varnothing\vDash t$, or simply written $\models t$.
\begin{example}\
    $\{p,p\Rightarrow q\}\models q$.
\end{example}
\begin{defn}[Model?]
    For $t\in L$, valuation $v$, say $t$ is true in $v$ (or $v$ is a model of $t$) if $v(t)=1$. For $S\subseteq L$, valuation $v$, say $v$ is a model of $S$ if $v(s)=1$ for all $s\in S$.
\end{defn}
Note that $S\models t$ iff $t$ is true in every model of $S$.

\subsection{Syntactic Entailment}
A notion of proof consists of axioms and rules of deduction. In propositional logic we adopt the following propositions as axioms
\begin{enumerate}
    \item[(A1)] $p\Rightarrow(q\Rightarrow p)$, $p,q\in L$
    \item[(A2)] $(p\Rightarrow(q\Rightarrow r))\Rightarrow((p\Rightarrow q)\Rightarrow(p\Rightarrow r))$, $p,q,r\in L$
    \item[(A3)] $\neg\neg p\Rightarrow p$, $p\in L$
\end{enumerate}
Note that all axioms are tautologies. These are called `axiom schemes'. We have one rule of deduction called modus ponens (MP): from $p$ and $p\Rightarrow q$, we can deduce $q$.

Given $S\subseteq L$, $t\in L$, a proof of $t$ from $S$ is a finite sequence $t_1,...,t_n$ in $L$ s.t. $t_n=t$ and for each $i$, 
\begin{enumerate}
    \item[(i)] either $t_i$ is an axiom 
    \item[(ii)] or $t_i\in S$ (premise or hypothesis)
    \item[(iii)] $t_i$ is obtained by modus ponens from earlier lines: $\exists j,k<i$ s.t. $t_k\Rightarrow (t_j\Rightarrow t_i)$
\end{enumerate}
Say $S$ proves $t$ ot $S$ syntactically entails $t$ if there is a proof of $t$ from $S$. Write $S\vdash t$.
If $\varnothing\vdash t$ (or write $\vdash t$), then say $t$ is a theorem.
\begin{example}
    
\end{example}
\begin{prop}[Deduction theorem]
    Let $S\subseteq L$, $p,q\in L$. Then $S\vdash (p\Rightarrow q)$ iff $S\cup \{p\}\vdash q$.
\end{prop}
\begin{rem}
    So $\Rightarrow$ does behave like implications in formal proofs.
\end{rem}
\begin{proof}
    If $S\vdash (p\Rightarrow q)$, then we can write down a proof of $p\Rightarrow q$ from $S$ and add the following lines to obtain a proof of $q$ from $S\cup\{p\}$: Add $p$ (premise), then $q$ (MP).

    Now assume $S\cup\{p\}\vdash q$. Let $t_1,...,t_n$ be a proof of $q$ from $S\cup\{p\}$. We prove that $S\vdash (p\Rightarrow t_i)$ by induction on $i$.
    \begin{enumerate}
        \item[Case 1.] $t_i$ is an axiom or $t_i\in S$. Then $t_i\Rightarrow(p\Rightarrow t_i)$ by (A1) and $t_i$ (premise or axiom), so $p\Rightarrow t_i$ (MP)
        \item[Case 2.] $t_i=p$. Then $S\vdash (p\Rightarrow p)$ since $\vdash (p\Rightarrow p)$.
        \item[Case 3.] $\exists j,k<i$ s.t. $t_k=(t_j\Rightarrow t_i)$. By induction we can write down proofs of $(p\Rightarrow t_j)$ and $p\Rightarrow (t_j\Rightarrow t_i)$ from $S$ and add the following: $(p\Rightarrow (t_j\Rightarrow t_i))\Rightarrow((p\Rightarrow t_j)\Rightarrow(p\Rightarrow t_i))$ by (A2) and $(p\Rightarrow t_j)\Rightarrow(p\Rightarrow t_i)$ by (MP), and $(p\Rightarrow t_i)$ by (MP). This is a proof of $p\Rightarrow t_i$ from $S$.
    \end{enumerate}
\end{proof}
\textbf{Aim:} to prove the completeness theorem: $S\vDash t$ iff $S\vdash t$. This consists of \textit{soundness} ($S\vdash$ implies $S\vDash t$) and \textit{adequacy} ($S\vDash t$ implies $S\vdash t$).
\begin{prop}[Soundness theorem]
    Let $S\subseteq L$, $t\in L$. Then $S\vdash t$ implies $S\vDash t$.
\end{prop}
\begin{proof}
    Let $t_1,...,t_n=t$ be a proof of $t$ from $S$. Let $v$ be a model of $S$. We prove $v(t_i)=1$ by induction on $i$. If $t_i$ is a premise, then $v(t_i)=1$. If $t_i$ is an axiom, then $v(t_i)=1$ since all axioms are tautologies. If $\exists j,k<i$ s.t. $t_k=(t_j\Rightarrow t_i)$, then by induction, $v(t_j)=v(t_j\Rightarrow t_i)=1$, so $v(t_i)=1$.
\end{proof}
\[(\circ\tag{Owen's cyclop signature}\]\[\leftindex^\circ\smile^\circ\]
\begin{defn}[Inconsistency]
    Let $S\subseteq L$, say $S$ is inconsistent if $S\vdash\bot$. Otherwise, $S$ is consistent.
\end{defn}
\begin{thm}[Model existence lemma]
    Let $S\subseteq L$. Then if $S$ is consistent, then $S$ has a model.
\end{thm}
\begin{proof}
    Assume $P$ (primitive props) is countable (general case in chap 3), so $L=\bigcup L_n$ is countable. Enumerate $t_1,t_2,...$. If $T\subseteq L$ is consistent and $t\in L$, then $T\cup\{t\}$ or $T\cup\{\neg t\}$ is consistent. If not then $T\cup\{t\}$ and $T\cup\{\neg t\}$ entails $\bot$. By deduction theorem $T\vdash\neg t$, $T\vdash(\neg t\Rightarrow\bot)$ and hence $T\vdash \bot$ by MP. Contradiction. 

    So enlarge $S$ by including $t$ or $\neg t$ to obtain $S_1$ (consistent). Inductively get $S_1, S_2,...$. Let $\bar S=\bigcup S_n$. Note that $S\subseteq\bar S$  and $\forall t\in T$, either $t\in\bar S$ or $\neg t\in \bar S$. Note that $\bar S$ is consistent.

    Note that $\bar S$ is deductively closed, i.e., $if t\in L$ and $\bar S\vdash t$, then $t\in\bar S$.

    Define a valuation $v:L\to\{0,1\}$ by $v(t)=1$ if $\bar S\vdash t$ and $0$ o/w. Need to check that this is a valuation and then we are done.
\end{proof}
\begin{crly}[Adequacy Theorem]
    Let $S\subseteq L$, $t\in L$. If $S\vDash t$, then $S\vdash t$.
\end{crly}    
\begin{proof}
    $S\vDash t\implies S\cup\{\neg t\}\vDash \bot\overset{\text{Thm}}{\implies} S\cup\{\neg t\}\vdash\bot\implies S\vdash\neg\neg t$. By deduction theorem, add $\neg\neg t\Rightarrow t$ and $t$ (MP), get $S\vdash t$.
\end{proof}
\begin{thm}[Completeness theorem]
    $S\subseteq L$, $t\in L$. Then $S\vDash t$ iff $S\vdash t$.
\end{thm}

\begin{crly}[Compactness Theorem]
    $S\subseteq L$, $t\in L$. If $S\vDash t$, then $\exists$ finite subset $S'\subseteq S$ s.t. $S'\vDash t$.
\end{crly}
\begin{crly}
    $S\subseteq L$. If every finite subset of $S$ has a model, then $S$ has a model
\end{crly}
\begin{crly}[Decidability Theorem]
    $S\subseteq L$ finite, $t\in L$. There is an algorithm that can determine whether $S\vdash t$ in finite time.
\end{crly}

\section{Well-Orderings and Ordinals}
A linare roder (i.e.g. total order) on a set $X$ is a relation $<$ s.t.
\begin{enumerate}
    \item[(i)] irreflexive $\forall x\in X, \neg(x<x)$;
    \item[(ii)] transitive
    \item[(iii)] trichotomous
\end{enumerate}
\[\ell:\tag{Owen's signature}\]
\begin{defn}[Well ordering]
    A well-ordering of a set $X$ is a linear order $<$ on $X$ s.t. every non-empty subset of $X$ has a least element:
    \[(\forall S\subseteq X)(S\neq\varnothing\Rightarrow(\exists x\in S)(\forall y\in S)(x\le y))\]
\end{defn}
\begin{defn}[Order isomorphism]
     Let $X, Y$ be linearly ordered sets. Say $X, Y$ are order-isomorphic (o.i.) if there exists an order-preserving bijection $f:X\to Y$. Such an $f$ is called an order-isomorphism.
\end{defn}
\begin{defn}[Initial segment]
    A subset $I$ of a l.o. set $X$ is an initial segment (i.s.) of $X$ if $\forall x, y\in X$, $(x\in I\wedge y<x)\Rightarrow y\in I$.
\end{defn}
\begin{rem}
    If $I$ is a proper i.s.  of a w.o. set $X$, then $I=I_x=\{y\in X:y<x\}$ for $x$ the least element of $X\setminus I$.
\end{rem}
\begin{lem}
    Let $X, Y$ be w.o. sets, $I$ an i.s. of $Y$ and $f:X\to I$ an o.i. Then for every $x\in X$, we have $f(x)=\min(Y\setminus\{f(y):y<x\})$.
\end{lem}
\begin{proof}
    Fix $x\in X$ and let $A=Y\setminus\{f(y):y<x\}$. Note that $A\neq\varnothing$. Let $a=\min A$. Then $a\le f(x)$, so $a\in I$, so $a=f(z)$ some $z\in X$. Since $f(z)\le f(x)$, so $z\le x$ If $z<x$, then $x\not\in A$. Contradiction. 
\end{proof}
\begin{prop}[Induction]
    Let $X$ be a w.o. set and $S\subseteq X$ s.t. for every $x\in X$ if $y\in S$ for all $y<x$, then $x\in S$. Then $S=X$.
\end{prop}
\begin{proof}
    If not, then pick $x=\min X\setminus S$. If $y<x$, then $y\in S$. Assumption says $x\in S$. Contradiction.
\end{proof}
\begin{prop}
    Let $X, Y$ be o.i. w.o. sets. Then there is a unique o.i. $X\to Y$
\end{prop}
\begin{proof}
    Assume $f,g:X\to Y$ are o.i. Fix $x\in X$ assume $\forall y<x$, $f(y)=g(y)$. By lemma $f(x)=\min \{Y\setminus f(y):y<x\}=g(x)$. Done by the last proposition.
\end{proof}
\[*\smile\circ\tag{Owen's Signature}\]
\begin{proof}[Definition by recursion]
    \begin{proof}[Definition by recursion]
        \begin{proof}[Definition by recursion]
            etc.
            \texttt{return(f)}
        \end{proof}
    \end{proof}
\end{proof}
\texttt{f}
\[O:\tag{Owen's signature}\]

\begin{defn}
    Given w.o. sets $(X,<_X)$, $(Y,<_Y)$, say $Y$ extends $X$ if $X\subseteq Y$ and $<_X$ is the restriction of $<_Y$ to $X$, and $X$ is an i.s. of $Y$. Say $\{X_i:i\in I\}$ is nested if $\forall i,j\in I$, either $X_i$ extends $X_j$ or $X_j$ extends $X_i$.
\end{defn}
\begin{prop}
    Let $\{X_i:i\in I\}$ be a nested set of w.o. sets. Then there exists w.o. set $X$ s.t. $X_i\subseteq X$ for all $i\in I$.
\end{prop}
\begin{proof}
    Let $X=\bigcup_{i\in I}X_i$. For $x,y\in X$, we let $x<y$ iff $\exists i\in I$ s.t. $x,y\in X_i$ and $x<_i y$. Since $X_i$ are nested, it follows that $<$ on $X$ is a well-defined order s.t. each $X_i$ is an i.s. of $X$.
    Let $S\subseteq X, S\neq\varnothing$. Then $\exists i\in I$ s.t. $X_i\cap S\neq\varnothing$. Since $X_i$ is w.o., $X_i\cap S$ has a least element $x$. $X_i$ is an i.s. of $X$ so $x$ is a least element of $S$.
\end{proof}

\begin{defn}
    An ordinal is a w.o. set. Identify ordinals that are o.i. to each other. The order type of a w.o. set $X$ is the unique ordinal to which it's o.i.
\end{defn}
\begin{defn}
    Let $\alpha,\beta$ be ordinals and $X, Y$ be o.i. sets of order type $\alpha,\beta$, resp. We define $\alpha\le\beta$ to mean that $X\le Y$, $\alpha<\beta$ if $X<Y$. Let $\alpha^+$ be the order type of $X^+$. \text{it}{qwertyuiop} by D\"oonlad.
\end{defn}

\begin{prop}
    Let $\alpha$ be an ordinal. Then the ordinals \(<\alpha\) form a w.o. set of order type $\alpha$.
\end{prop}
\begin{proof}
    \textit{kjkjkffkhzzzzc} by D\"oonlad.\\
    \text{it}{sdjvhksdjhvkjhsdfuvb} By O\"en
\end{proof}
\begin{prop}[Burat-Forti paradox]
    The ordinals do not form a set.
\end{prop}
\begin{proof}
    If $X$ is a set consisting of all ordinals. $X$ is w.o. Let $\alpha$ be the order type of $X$, so $\alpha\in X$, and $I_\alpha$ is a proper i.s. of $X$ of order type $\alpha$, so $I_\alpha$ is o.i. to $X$. Contradiction.
\end{proof}
\[[-:\tag{Owen's Signature}\]
\begin{thm}
    There exists uncountable ordinals.
\end{thm}
\begin{proof}
    Let $A=\{(M,R)\in \mathscr P(\NN)\times\mathscr P(\NN^2):R\text{ is a well-ordering of }M\}$.
\end{proof}
\begin{thm}[Hartog's lemma]
    For any set $X$ there is an ordinal $\gamma$ that doesn't inject into $X$.
\end{thm}
\begin{proof}
    ............................By inspection.?
\end{proof}
The least ordinal that doesn't inject into $X$ is called $\gamma(X)$.
\subsection*{Ordinal arithmetic}
\begin{prop}
    
\end{prop}
\[O\smile O\tag{Owen's signature}\]
% Requires: \usepackage{amsmath}
\begin{equation}
   \left( \frac{\dot{a}(t)}{a(t)} \right)^2 = \frac{8\pi G}{3}\rho - \frac{k}{a(t)^2} + \frac{\Lambda}{3}
   \label{eq:friedmann_1}
\end{equation}
\[\overset{\overset{\aquarius}{\odot\odot}}{\frown}\tag{Freddie}\]

\[\overset{\overset{\aquarius}{O_{\hat{}}O}}{\smile}\]

\[\overset{\rotatebox[origin=c]{60}{$\sim$}}{\rotatebox[origin=c]{90}{\aquarius}}\overset{\overset{\overset{\backslash\ /}{\aquarius}}{O\vert O}}{\smile}\overset{\rotatebox[origin=c]{120}{$\sim$}}{\rotatebox[origin=c]{90}{\aquarius}}\]
\end{document}