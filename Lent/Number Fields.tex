\documentclass{article}
\usepackage{graphicx} % Required for inserting images
\usepackage[utf8]{inputenc}
\usepackage{amsmath,amsfonts,amssymb,amsthm}
\usepackage{enumerate,bbm}
\usepackage{leftindex}
\usepackage{tikz,tikz-cd,graphicx,color,mathrsfs,color,hyperref,boldline}
\usepackage{caption,float}
\usepackage[a4paper,margin=1in,footskip=0.25in]{geometry}

\usepackage{listings}
\usepackage{xcolor}

\usepackage{tabularx,capt-of}

\usepackage{blindtext}
%Image-related packages
\usepackage{graphicx}
\usepackage{subcaption}
\usepackage[export]{adjustbox}
\usepackage{lipsum}

%hyperref setup
\hypersetup{
    colorlinks=true,
    linkcolor=blue,
    filecolor=magenta,      
    urlcolor=cyan,
    pdftitle={Overleaf Example},
    pdfpagemode=FullScreen,
    }

%New colors defined below
\definecolor{codegreen}{rgb}{0,0.6,0}
\definecolor{codegray}{rgb}{0.5,0.5,0.5}
\definecolor{codepurple}{rgb}{0.58,0,0.82}
\definecolor{backcolour}{rgb}{0.95,0.95,0.92}

%Code listing style named "mystyle"
\lstdefinestyle{mystyle}{
  backgroundcolor=\color{backcolour}, commentstyle=\color{codegreen},
  keywordstyle=\color{magenta},
  numberstyle=\tiny\color{codegray},
  stringstyle=\color{codepurple},
  basicstyle=\ttfamily\footnotesize,
  breakatwhitespace=false,         
  breaklines=true,                 
  captionpos=b,                    
  keepspaces=true,                 
  numbers=left,                    
  numbersep=5pt,                  
  showspaces=false,                
  showstringspaces=false,
  showtabs=false,                  
  tabsize=2
}

%"mystyle" code listing set
\lstset{style=mystyle}

\theoremstyle{definition}
\newtheorem{defn}{Definition}[section]
\newtheorem{example}[defn]{Example}
\theoremstyle{remark}
\newtheorem{rem}{Remark}
\newtheorem{remS}[section]{defn}
\theoremstyle{plain}
\newtheorem{lem}[defn]{Lemma}
\newtheorem{thm}[defn]{Theorem}
\newtheorem{prop}[defn]{Proposition}
\newtheorem{fact}[defn]{Fact}
\newtheorem{crly}[defn]{Corollary}
\newtheorem{conj}[defn]{Conjecture}

%\newtheorem*{programming*}{Programming Task}

%\newtheorem{innercustomgeneric}{\customgenericname}
%\providecommand{\customgenericname}{}
%\newcommand{\newcustomtheorem}[2]{%
%  \newenvironment{#1}[1]
%  {%
%   \renewcommand\customgenericname{#2}%
%   \renewcommand\theinnercustomgeneric{##1}%
%   \innercustomgeneric
%  }
%  {\endinnercustomgeneric}
%}

%\newcustomtheorem{question}{Question}
%\newcustomtheorem{programming}{Programming Task}

\newcommand{\NN}{\mathbb{N}}
\newcommand{\ZZ}{\mathbb{Z}}
\newcommand{\QQ}{\mathbb{Q}}
\newcommand{\RR}{\mathbb{R}}
\newcommand{\CC}{\mathbb{C}}
\newcommand{\PP}{\mathbb{P}}
\newcommand{\FF}{\mathbb{F}}
\newcommand{\Hom}{\operatorname{Hom}}
\newcommand{\im}{\operatorname{im}}
\newcommand{\id}{\operatorname{id}}
\newcommand{\Ind}{\operatorname{Ind}}
\newcommand{\Res}{\operatorname{Res}}
\newcommand{\tr}{\operatorname{Tr}}
\newcommand{\disc}{\operatorname{disc}}

\newcommand{\calD}{\mathcal{D}}

\newcommand{\sol}{\textit{Solution: }}

\title{Number Fields}
\author{Kevin}
\date{January 2025}

\begin{document}
\maketitle
\section{???}
(: (\textit{Owen's signature})
%\subsection*{What?}
%Field Extension: \(L\vert K\)




\begin{defn}
    A number field is a subfield $K\subseteq \CC$ s.t. $[K:\QQ]<\infty$.
\end{defn}
Let $\alpha$ be an algebraic number. There exists $f_\alpha\in\QQ[X]$ of minimal positive deg s.t. $f_\alpha(\alpha)=0$. 
\begin{itemize}
    \item (monic) min poly: leading coeff 1.
    \item min poly in $\ZZ[X]$: clearing denominators.
\end{itemize}
$[\QQ(\alpha):\QQ]=\deg f_\alpha$. These are all the examples by primitive element theorem.
\begin{example}
    Quadratic fields, cyclotomic fields. 
\end{example}
%\subsection*{Why?}
%We don't care... (so why are we here?) idk

\[);\models\tag{Owen's Signature}\]

\begin{defn}
    An algebraic number $\alpha$ is an algebraic integer if $f_\alpha\in\ZZ[X]$ (i.e., its monic minimal polynomial has integer coefficient)
\end{defn}
\begin{rem}
    If $\alpha$ is a root of a monic poly $f\in\ZZ[X]$, then $\alpha$ is an algebraic integer.
\end{rem}
\begin{example}
    Let $K=\QQ(\sqrt m)$
    If $m\not\equiv 3\pmod 4$, then $\mathcal O_K=\ZZ[\sqrt m]$. If $m\equiv 1\pmod 4$, then $\mathcal O_K=\ZZ[(1+\sqrt m)/2]$.

    For cyclotomic field $K=\QQ(\theta_n)$, $n\ge 3$, $\mathcal O_K=\ZZ[\theta_m]$. (We will prove this for $n$ prime later.)
\end{example}
\begin{thm}
    The set of algebraic integers $\mathcal{O}$ is a ring.
\end{thm}
Notation: We write $\mathcal{O}_K=K\cap\mathcal O$ for $K$ number field, and we call $\mathcal O_K$ the ring of integers in $K$.
\begin{proof}
    GRM ES4.s
\end{proof}
\begin{table}[H]
    \centering
    \begin{tabular}{c|c|c}
         0&$\longleftarrow^\bullet$ & $\overline\longrightarrow$ \\
         \hline
         $\leftrightarrow$&O &\\
         \hline
        \(\tilde{\mathcal{O}}\) & &\(\overset{\circ}{\chi}\)
    \end{tabular}
    \caption{Noughts and Arrows}
    \label{tab:my_label}
\end{table}

\begin{prop}
    Let $\alpha\in\CC$ be an algebraic number. TFAE,
    \begin{enumerate}
        \item[(i)] $\alpha$ is an algebraic integer;
        \item[(ii)] $\ZZ[\alpha]$ is f.g. as a $\ZZ$-module
        \item[(iii)] $\exists$ a f.g. $\ZZ$-module $M$ that is invariant under multiplication by $\alpha$.
    \end{enumerate}
\end{prop}
\begin{proof}
    (1)$\Rightarrow$(2)$\Rightarrow$(3) is clear. Need to prove (3)$\Rightarrow$(1). Let $M=\ZZ\beta_1\oplus...\oplus \ZZ\beta_k$ be a module satisfying (3). Left multiplication by $\alpha$ is $\ZZ$-linear. Let $A$ be the matrix, then take the char poly of $A$.
\end{proof}
\[\{-:\tag{Owen's Signature}\]
\section{Additive Structure of $\mathcal{O}_K$}
Let $K$ be a number field of deg $d$. If $\mathcal O_K=\ZZ\alpha_1\oplus...\oplus\ZZ\alpha_d$ for some $\alpha_i\in \mathcal O_K$, then we say that $\{\alpha_i\}$ is an integral basis. Always have an integral basis (assume $\mathcal O_K$ is f.g.): structure theorem and linear independence $\Rightarrow$ $\mathcal O_K$ is has rank $r\le d$; $\mathcal O_K$ contains $d$ linearly independent elements in $K$ ($1,\alpha,\alpha^2,...,\alpha^{d-1}$, where $K=\QQ(\alpha)$).

Find $\mathcal O_K$:
\begin{enumerate}
    \item Find $d$ L.I. elements $\alpha_i^{(1)}$ of $\mathcal O_K$ and let $M_1=\bigoplus\ZZ\alpha_i^{(1)}$
    \item If $M_1=\mathcal O_K$, then done. Otherwise, can find $\beta\in\mathcal O_K\setminus M_1$. and consider $M_2=M_1+\ZZ\beta$ and express in normal form $M_2=\bigoplus\ZZ\alpha_i^{(2)}$.
    \item Iterate this process.
\end{enumerate}
Need to prove that this process terminates. 

Recall trace and norm from Galois theory. (I have recalled. Pls hurry up.) Recall basic properties of trace and norm. (I have recalled. Pls hurry up.) Abbreviate $N_{K/\QQ}=N_K,\tr_{K/\QQ}=\tr_K$. Recall more properties of trace and norm. (I have recalled. Pls hurry up.)

%Would you rather a faster lectured course on number fields? (Groj) Y/N? *************************************************

\begin{defn}
    Let $\alpha_1,...,\alpha_d\in K$, where $K$ is a number field of deg $d$. The discriminant is defined as $\disc(\alpha_1,...,\alpha_d)=\det(\sigma_i(\alpha_j))^2$, where $\sigma_i\in\Hom(K,\CC)$.
\end{defn}
\begin{example}
    $\disc(1,\alpha,\alpha^2,...,\alpha^{d-1})=(-1)^{\frac{n(n-1)}{2}}N(f'(\alpha))$.
\end{example}
\begin{lem}
    For $\alpha_1,...,\alpha_d\in K$, then
    $\disc(\alpha_1,...,\alpha_d)=\det(\tr_K(\alpha_i\alpha_j))$. If $\alpha_1,...,\alpha_d\in\mathcal O_K$, then $\disc(\alpha_1,...,\alpha_d)\in\ZZ$.
\end{lem}
\begin{proof}
    Compute,
    \begin{align*}
        [\tr(\alpha_i\alpha_k)]_{ik}=[\sigma_i(\alpha_j)]_{ij}[\sigma_{j}(\alpha_k)]_{jk}
    \end{align*}
    Take det.
\end{proof}
\begin{lem}
    $\disc(\alpha_1,...,\alpha_d)=0$ iff $\alpha_1,...,\alpha_d$ are linearly dependent over $\QQ$.
\end{lem}
\begin{proof}
    If linearly dependent, then $[\tr(\alpha_i\alpha_j)]$ is singular. 
    
    Suppose $\alpha_1,...,\alpha_d$ are linearly indpe over $\QQ$. If $\tr(\alpha_i\alpha_j)$ is singular, find $a_1,...,a_d\in\QQ$ not all zero s.t. $a_1\tr(\alpha_1\alpha_j)+\ldots+a_d\tr(\alpha_1\alpha_j)=0$, i.e., $\tr((a_1\alpha_1+\ldots+a_d\alpha_d)\alpha_j)=0$ for all $j$. Let $\gamma=a_1\alpha_1+\ldots+a_d\alpha_d$. This is non-zero by independence. So $\tr(\gamma\beta)\neq 0$ for all $\beta$ (linear combination of $\alpha_i$). In particular, take $\beta=\gamma^{-1}$. Contradiction.
\end{proof}
\begin{crly}
    $\alpha_1,...,\alpha_d$ are l.i. over $\QQ$ iff the set of vectors $\{(\sigma_i(\alpha_j))_{1\le i\le d}:j\in\{1,...,d\}\}$ is l.i. over $\CC$.
\end{crly}
Recall we have $d$ embeddings in to $\CC$. Let $r$ be the number of real embeddings. Suppose $\sigma_1,...,\sigma_r$ are real embeddings. Let $s=\frac{d-r}{2}$. and write $\tau_1,...,\tau_s$, $\bar\tau_1,...,\bar\tau_s$ for the remaining embeddings. Consider $\Sigma:K\to\RR^d$ defined by
\begin{align*}
    \Sigma(\alpha)=\begin{pmatrix}
        \sigma_1(\alpha)\\ \vdots\\ \sigma_r(\alpha)\\ \Re(\tau_1(\alpha))\\ \Im(\tau_1(\alpha))\\\vdots\\ \Im(\tau_n(\alpha_n))
    \end{pmatrix}
\end{align*}
$\Sigma$ is a hom of additive groups.
\begin{prop}
    Let $\alpha_1,..,\alpha_d\in K$. Then $\disc(\alpha_1,...,\alpha_d)=(-4)^s\det(\Sigma(\alpha_1),\cdots,\Sigma(\alpha_d))^2$.
    In particular, $\alpha_1,..,\alpha_d$ are l.i. over $\QQ$ iff $\Sigma(\alpha_1),...,\Sigma(\alpha_n)$ are l.i. over $\RR$
\end{prop}
\begin{proof}
    Write $\disc(\alpha_1,...,\alpha_d)=\det(\sigma_i(\alpha_j))^2$. Row operations.
\end{proof}

\begin{defn}
Let $\Lambda=\ZZ v_1\oplus\cdots\oplus\ZZ v_d$, where $v_1,...,v_d$ are l.i. over $\RR$. This is called a lattice. A fundamental domain for $\Lambda$ is a bounded Borel set of $\RR^d$ that contains exactly one element from each coset $x+\Lambda$ for $x\in\RR^d$.
\end{defn}
The volume of the fundamental parallelepiped $\Sigma(\alpha_1),...,\Sigma(\alpha_d)$ is $|\disc(\alpha_1,...,\alpha_d)|^{1/2}$
\begin{lem}
    For a lattice $\Lambda\subseteq\RR^d$, all fundamental domains have the same volume.
\end{lem}
\begin{proof}
    Let $F_1, F_2$ be  fundamental domains. Then $F_1=\bigsqcup_{v\in\Lambda}F_1\cap(v+F_2)$ and similarly for $F_2$. Note that $(F_1\cap (F_2+v))-v=(F_1-v)\cap F_2$. Compute
    \[\operatorname{vol}(F_1)=\sum_{v\in\Lambda}\operatorname{vol}(F_1\cap (v+F_2))=\sum_{v\in\Lambda}\operatorname{vol}((F_1-v)\cap F_2)=\operatorname{vol}(F_2)\]
\end{proof}
\begin{defn}
    The co-volume of a lattice $\operatorname{covol}(\Lambda)$ is the volume of any fundamental domain.
\end{defn}
\begin{prop}
    Let $K$ be a number field.
    Let $\alpha_1,...,\alpha_d\in K$, $\beta_1,...,\beta_d\in K$ be $\QQ$-indep tuples. Let $A\in\operatorname{GL}_d(\QQ)$ be s.t. $(\beta_i)_i=A(\alpha_i)_i$. Then $\disc(\beta_1,...,\beta_d)=(\det A)^2\disc(\alpha_1,...,\alpha_d)$. If $\beta_1,...,\beta_d\in \ZZ\alpha_1+\cdots+\ZZ\alpha_d$, then $|\disc(\beta_1,...,\beta_d)|\ge|\disc(\alpha_1,...,\alpha_d)|$. Finally, $\disc$ depends only on the module generated by the tuple, i.e., if two tuples generate the same module then they have the same disc.
\end{prop}
\[)=\tag{Owen's Signature}\]
\begin{defn}
    For a $\ZZ$-module $M\subseteq K$ (where $K$ is a number field), we define $\disc(M)$ as the discriminant of any generating tuple.

    For a number field $K$, define $\disc K=\disc\mathcal O_K$.
\end{defn}
\begin{proof}
    Have $[\sigma_j{\beta_i}]_{ij}=A[\sigma_j\alpha_i]_{ij}$. Take det, get the first formula.

    If $\beta_1,...,\beta_d\in\ZZ\alpha_1+\cdots\ZZ\alpha_d$, then $\ZZ$ has integer entries, so $(\det A)^2\geqslant 1$. Get the inequality. If they generate the same module, then by a symmetric argument, can show that $\le$ also holds.
\end{proof}
\begin{prop}
    Let $M_1\subseteq M_2$ be two $\ZZ$-modules of rank $d$ in a number field $K$. Then 
    \[|M_2/M_1|^2=\dfrac{\disc M_1}{\disc M_2}\]
\end{prop}
\begin{thm}[Smith Normal Form]
    Let $M_1\subseteq M_2$ be two $\ZZ$-mod. Then $\exists$ free generators $\alpha_1,...,\alpha_d$ of $M_2$ s.t. $a_1|a_2|\cdots|a_d\in\ZZ_{\ge 0}$ s.t. $a_1\alpha_1,...,a_d\alpha_d$ are free generators for $M_1$.
\end{thm}
\begin{thm}
    A tuple $\alpha_1,...,\alpha_d\in\mathcal O_K$ is an integral basis [$\mathcal O_K=\ZZ\alpha_1\oplus\cdots\oplus\ZZ\alpha_d$] iff $|\disc(\alpha_1,...,\alpha_d)|$ is minimal among all tuples with with non-zero discriminant.
\end{thm}
\begin{proof}
    $(\Leftarrow)$: Suppose $\alpha_1,...,\alpha_d\in\mathcal O_K$ has the least positive $|\disc|$. Let $\beta\in\mathcal O_K$. Have
    \[1\le |(M+\ZZ\beta)/M|^2=\dfrac{\disc(M)}{\disc(M+\ZZ\beta)}\le1\]
    so $M=M+\ZZ\beta$.

    $(\Rightarrow)$: If not minimal then can find $\beta_1,...,\beta_d$ with smaller disc., so $\ZZ\beta_1+\cdots+\ZZ\beta_d\nsubseteq\ZZ\alpha_1+\cdots+\ZZ\alpha_d$.
\end{proof}
\begin{prop}
    Let $\alpha_1,...,\alpha_d\in\mathcal O_K$ be $\QQ$-linearly independent. Then all elements of $\mathcal O_K$ can be written in the form
    \[\dfrac{a_1\alpha_1+\cdots+a_d\alpha_d}{q}\]
    where $a_1,...,a_d\in\ZZ$ and $q\in\ZZ$ with $q^2\mid \frac{\disc(\alpha_1,...,\alpha_d)}{\disc(\mathcal O_K)}$.
\end{prop}
\begin{proof}
    Let $\beta\in\mathcal O_K$ and $q=|(M+\ZZ\beta)/M|$, where $M=\ZZ\alpha_1+\cdots+\ZZ\alpha_d$.
    \[q^2\Bigg|\frac{\disc(\alpha_1,...,\alpha_d)}{\disc(M+\ZZ\beta)}\Bigg|\frac{\disc(\alpha_1,...,\alpha_d)}{\disc(\mathcal O_K)}\]
    Have $q\beta\in M$ (consider the order).
\end{proof}
\section{Unique Factorization of Ideals}
\[(\%\tag{Owen's Signature}\]
Recall definition of an ideal from GRM (I recall please hurry up).

Notation: $\langle\alpha\rangle$ denotes the principal ideal generated by $\alpha$.

Recall the defn of the product of two ideals. (I have recalled please hurry up). (This turns ideals into a semigroup and $\alpha\mapsto\langle\alpha\rangle$ is a homomorphism). Recall how to add two ideals.

\begin{thm}
    Let $K$ be a number field. Then every non-zero ideal of $\mathcal O_K$ is a product of non-zero prime ideals, and this factorization is unique up to reordering.
\end{thm}
We assume that rings are commutative and unital.

Recall definition of integral domain (I have recalled please hurry up)
Recall definition of field (I have recalled please hurry up) [A non-zero field where every non-zero element has a multiplicative inverse is a field.]
Recall definition of field of fractions (I have recalled please hurry up!)
Recall definition of a prime ideal (I have recalled please hurry UP!)
Recall definition of a maximal ideal (I have RECALLED.........)

\begin{lem}
    Let $K$ be a number field. $\mathcal O_K$ is Noetherian.
\end{lem}
\begin{proof}
    Every ideal is a $\ZZ$-submodule of $\mathcal O_K$, which is f.g. as a $\ZZ$-submodule, so it's f.g. as a $\mathcal O_K$-submodule (ideal).
\end{proof}
Note that this implies that every collection of ideals has a maximal element.
\[):zzzzzzzz\tag{Owen's Signature}\]
\begin{lem}
    A non-zero ideal $I\trianglelefteq\mathcal O_K$ prime iff it's maximal.
\end{lem}
\begin{proof}
    $\Rightarrow:$ Suffices to prove that $\mathcal O_K/I$ is a finite integral domain for $I$ prime. Can reduce to principal prime ideals $\langle\beta\rangle\subseteq I$ since $\mathcal O_K/I$ is a quotient of $\mathcal O_K/\langle\beta\rangle$. Let $\alpha_1,...,\alpha_d$ be an integral basis of $\mathcal O_K$, then $\beta\alpha_1,...,\beta\alpha_d$ are free $\ZZ$-module generators of $\langle\beta\rangle$. Same rank, so $\mathcal O_K/\langle\beta\rangle$ is finite.
\end{proof}
\begin{lem}
    Let $M\subset K$ be a f.g. $\mathcal O_K$-module. If $\alpha\in K$ and $\alpha M\subseteq M$, then $\alpha\in\mathcal O_K$.
\end{lem}
i.e., $\mathcal O_K$ is integrally closed.
\begin{proof}
    Let  $\beta_1,...,\beta_d$ be an integral basis of $\mathcal O_K$ and let $M=\alpha\mathcal O_K+\cdots+\alpha_k\mathcal O_K$, then $M=\sum_{i,j}\ZZ\alpha_i\beta_j$, so $M$ is f.g. as a $\ZZ$-mod.
\end{proof}

\begin{defn}[Dedekind domain]
    Noetherian, Integrally closed, Prime ideal iff maximal
\end{defn}
\begin{defn}
    A f.g. $\mathcal O_K$-submodule of $K$ is called a fractional ideal.
\end{defn}
\begin{rem}
    A fractional ideal $I\subseteq K$ is an (integral) ideal iff $I\subseteq \mathcal O_K$.
\end{rem}
\begin{lem}
    If $I\subseteq K$ is a fractional ideal, then $\exists a\in\ZZ$ s.t. $aI\trianglelefteq \mathcal O_K$ is an (integral) ideal. Conversely, if $I\trianglelefteq \mathcal O_K$ is an (integral) ideal, then $\alpha I$ is a fractional ideal for all $\alpha\in K$.
\end{lem}
\begin{proof}
    
\end{proof}
Multiplication can be extended to fractional ideals.
\begin{lem}
    Let $\mathfrak p\trianglelefteq\mathcal O_K$ be a prime idea. Then there exists a fractional ideal $\mathfrak p'$ s.t. $\mathfrak p\mathfrak p'=(1)$.
\end{lem}
\begin{proof}
    Let $\mathfrak p'=\{\alpha\in K:\alpha\mathfrak p\subseteq \mathcal O_K\}$. This is a $\mathcal O_K$-module. $\beta \mathfrak p'\subseteq\mathcal O_K$ for all $\beta\in\mathfrak p$. $\mathfrak p'$ is a f.g. $\mathcal O_K$-mod, i.e., a fractional ideal. Since $\mathfrak p\mathfrak p'\subseteq\mathcal O_K$, so it's an ideal. On the other hand ,$\mathfrak p'\mathfrak p\supseteq \mathfrak p$ and $\mathfrak p$ is prime, hence maximal. Either $\mathfrak p\mathfrak p'=(1)$ or $\mathfrak p\mathfrak p'=\mathfrak p$. If the latter holds, then $\alpha\mathfrak p\subseteq \mathfrak p$ and $\alpha\in\mathcal O_K$. Contradiction (integral closedness)
\end{proof}
\begin{thm}
    In a Dedekind domain, have unique factorization domain of ideals.
\end{thm}
\begin{proof}

\end{proof}
\end{document}