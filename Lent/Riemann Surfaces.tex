\documentclass{article}
\usepackage{graphicx} % Required for inserting images
\usepackage[utf8]{inputenc}
\usepackage{amsmath,amsfonts,amssymb,amsthm}
\usepackage{enumerate,bbm}
\usepackage{leftindex}
\usepackage{tikz,tikz-cd,graphicx,color,mathrsfs,color,hyperref,boldline}
\usepackage{caption,float}
\usepackage[a4paper,margin=1in,footskip=0.25in]{geometry}

\usepackage{listings}
\usepackage{xcolor}

\usepackage{tabularx,capt-of}

\usepackage{blindtext}
%Image-related packages
\usepackage{graphicx}
\usepackage{subcaption}
\usepackage[export]{adjustbox}
\usepackage{lipsum}

%hyperref setup
\hypersetup{
    colorlinks=true,
    linkcolor=blue,
    filecolor=magenta,      
    urlcolor=cyan,
    pdftitle={Overleaf Example},
    pdfpagemode=FullScreen,
    }

%New colors defined below
\definecolor{codegreen}{rgb}{0,0.6,0}
\definecolor{codegray}{rgb}{0.5,0.5,0.5}
\definecolor{codepurple}{rgb}{0.58,0,0.82}
\definecolor{backcolour}{rgb}{0.95,0.95,0.92}

%Code listing style named "mystyle"
\lstdefinestyle{mystyle}{
  backgroundcolor=\color{backcolour}, commentstyle=\color{codegreen},
  keywordstyle=\color{magenta},
  numberstyle=\tiny\color{codegray},
  stringstyle=\color{codepurple},
  basicstyle=\ttfamily\footnotesize,
  breakatwhitespace=false,         
  breaklines=true,                 
  captionpos=b,                    
  keepspaces=true,                 
  numbers=left,                    
  numbersep=5pt,                  
  showspaces=false,                
  showstringspaces=false,
  showtabs=false,                  
  tabsize=2
}

%"mystyle" code listing set
\lstset{style=mystyle}

\theoremstyle{definition}
\newtheorem{defn}{Definition}[section]
\newtheorem{example}[defn]{Example}
\theoremstyle{remark}
\newtheorem{rem}{Remark}
\newtheorem{remS}[section]{defn}
\theoremstyle{plain}
\newtheorem{lem}[defn]{Lemma}
\newtheorem{thm}[defn]{Theorem}
\newtheorem{prop}[defn]{Proposition}
\newtheorem{fact}[defn]{Fact}
\newtheorem{crly}[defn]{Corollary}
\newtheorem{conj}[defn]{Conjecture}

%\newtheorem*{programming*}{Programming Task}

%\newtheorem{innercustomgeneric}{\customgenericname}
%\providecommand{\customgenericname}{}
%\newcommand{\newcustomtheorem}[2]{%
%  \newenvironment{#1}[1]
%  {%
%   \renewcommand\customgenericname{#2}%
%   \renewcommand\theinnercustomgeneric{##1}%
%   \innercustomgeneric
%  }
%  {\endinnercustomgeneric}
%}

%\newcustomtheorem{question}{Question}
%\newcustomtheorem{programming}{Programming Task}

\newcommand{\NN}{\mathbb{N}}
\newcommand{\ZZ}{\mathbb{Z}}
\newcommand{\QQ}{\mathbb{Q}}
\newcommand{\RR}{\mathbb{R}}
\newcommand{\CC}{\mathbb{C}}
\newcommand{\PP}{\mathbb{P}}
\newcommand{\FF}{\mathbb{F}}
\newcommand{\Hom}{\operatorname{Hom}}
\newcommand{\im}{\operatorname{im}}
\newcommand{\id}{\operatorname{id}}
\newcommand{\Ind}{\operatorname{Ind}}
\newcommand{\Res}{\operatorname{Res}}
\newcommand{\e}{\varepsilon}

\newcommand{\calD}{\mathcal{D}}

\newcommand{\sol}{\textit{Solution: }}

\title{Riemann Surfaces}
\author{Kevin}
\date{January 2025}

\begin{document}
\maketitle
\section{The Complex Plane}
): (Owen's signature)
\subsection{Review of Complex Analysis}
\begin{itemize}
    \item A domain $U\subseteq \CC$ is an open (non-empty) and connected (hence path-connected) subset. 
    \item Disk centered at $z_0$ with radius $r$: $D(z_0,r)$.
    \item Punctured disk $D'(z_0,r)$.
\end{itemize}
If $f:U\to\CC$ is complex-diff, then $\det J=|f'|^2$, where $J$ is the Jacobian of $f$ regarded as a function on $\RR^2$.
\begin{defn}
    $f:U\to\CC$, $U$ domain, is holo'c on $U$ if $f$ is $\CC$-diff for all $z_0\in U$.
\end{defn}
\begin{thm}[Taylor]
    If $f$ is holo'c on $U$ and $D(z_0,r)\subseteq U$, then $f(z)=\sum_{n=0}^\infty a_n(z-z_0)^n$ converges on $D(z_0,r)$ with $a_n=f^{(n)}(z_0)/n!$. Then $f$ holo'c on $U$ implies that $f^{(n)}$ holo'c on $U$.
\end{thm}
\begin{thm}[Identity theorem]
    $f,g$ be holo'c on a domain $U$, then either $f\equiv g$ or for all $z_0\in U$ there exists $r>0$ s.t. $f(z)\neq g(z)$ for all $z\in D'(z_0,r)\subseteq U$.
\end{thm}

Recall that if $f:D'(z_0,r)\to\CC$ is holo'c then $z_0$ is an isolated singularity.
In this situation we have the (unique) Laurent expansion at $z_0$, $f(z)=\sum_{n=-\infty}^\infty a_n(z-z_0)^n$ on $D'(z_0,r)$.
\begin{thm}
    Exactly one occurs
    \begin{itemize}
        \item Removable
        \item Pole
        \item Isolated essential.
    \end{itemize}
\end{thm}
\begin{defn}
    $f:U\to\CC$ ($U$ domain) meromorphic if it's holo'c away from poles.
\end{defn}
\begin{thm}[Open mapping theorem]
    $f$ holo'c on a domain $U$. Then $f$ is either constant or an open map.
\end{thm}
\begin{crly}[Inverse function theorem]
    If $f$ is holo'c on a domain, then
    \begin{enumerate}
        \item $f$ injective $\implies$ $f^{-1}$ is holo'c with $(f^{-1})'(z)=1/f'(f^{-1}(z))$.
        \item if $f'(z_0)\neq 0$ for some $z_0\in U$, then there exists an open nbd (nbd will always be open in this course) $N$ of $z_0$ s.t. $f:N\to f(N)$ is biholomorphic (i.e., $f$ and $f^{-1}$ are both holo'c).
    \end{enumerate}
\end{crly}

\subsection{Continuation of Power Series}
\textit{Notation: $\Delta:=D(0,1)$, and $T=\{z\mid |z|=1\}$.}

Suppose we have power series with radius of convergence $R$ (not $0$ or $\infty$) so $f:D(z_0,R)\to\CC$ holo'c. WLOG, assume $R=1$ and power series as domain $\Delta$.
\begin{defn}
    A point $z\in T$ is regular if $\exists D(z,r)$ and $g:D(z,r)\to\CC$ holo'c s.t. $f\equiv g$ on $\Delta\cap D(z,r)$. Otherwise, we say that $z$ is singular.
\end{defn}
So $z$ being regular means that $f$ can be extended holomorphically to $\bar f$ on $\Delta\cup D(z,r)$.
\begin{example}
    \begin{enumerate}
        \item[(i)] $f(z)=\sum_nz^n$ diverges on $T$ but the $f(z)=1/(1-z)$ means that $T\setminus\{1\}$ are regular points
        \item[(ii)] Take $f(z)=\sum_{n\ge 2}\frac{z^n}{n(n-1)}$ is absolutely conv. on $T$. Note that $1$ is singular. If it's regular, then $f''$ extends but $f''$ is the geometric series.
    \end{enumerate}
\end{example}
\begin{rem}
    \begin{enumerate}
        \item[(i)] The set of regular points is open in $T$.
        \item[(ii)] For $w\in\Delta$, let $\rho(w)$ be radius of convergence of the Taylor series at $w$. For $z\in T$, have $z$ regular if and only if $\rho(z/2)>1/2$.
    \end{enumerate}
\end{rem}
\begin{thm}
    If $f(z)=\sum_{n\ge 0} a_nz^n$ has r.o.c. $1$, then there is a singular point on $T$.
\end{thm}
\begin{proof}
    If not, then for all $w\in T$, $\exists r_w>0$ s.t. $f$ can be extended holomorphically to $f_w$ on $\Delta\cup D(w,r_w)$. Note that $D(w,r_w)\cap D(w',r_{w'})$ is convex (if non-empty). Let $F:\Delta\cup\bigcup_{w\in T}D(w,r_w)\to\CC$ be the extension of $f$. $F$ is well-defined. If $z\in D(w,r_w)\cap D(w',r_{w'})$ then $f_w,f_{w'}$ both holo'c on $D(w,r_w)\cap D(w',r_{w'})\cup\Delta$. By identity theorem, they agree. Now, we note that the domain of $F$ includes $D(o,R)$ for some $R>1$. (If not, take $z_n$ with $1|z_n|<1+1/n$) and $z_n\not\in\operatorname{dom}(F)$, then by sequential compactness, get a contradiction.) Contradiction with r.o.c being $1$.
\end{proof}
\begin{crly}
    If $a_n\ge0$ for all $n$, then $1$ is singular.
\end{crly}
\begin{proof}
    For $z\in T$ and each $k$, we have $f^{(k)}(z/2)=\sum_{n=k}^\infty n(n-1)...(n-k+1)a_n(z/2)^{n-k}$. So $|f^{(k)}(z/2)|\le |f^{(k)}(1/2)|$, so $\rho(z/2)\ge \rho(1/2)$. If $\rho(1/2)>1/2$, then all $z\in T$ is regular. Contradiction.
\end{proof}

\subsection{Complex Logarithm}
Notation: $L_\theta=\{re^{i\theta}:r\ge 0\}$.
\begin{defn}
    For any $\alpha\in\RR$, define $\log_\alpha:\CC\setminus L_{\alpha+\pi}\to\CC$ as $\log_\alpha(re^{i\theta})=\ln r+i\theta$, where $\theta\in(\alpha-\pi,\alpha+\pi)$.
\end{defn}

\subsection{Analytic Continuation (Plane)}
Let $D$ be a fixed domain of $\CC$.
\begin{defn}
    A function element of $D$ is a pair $(f,U)$, where $U$ is a subdomain $(\subseteq D)$ and $f$ is a holo'c function defined on $U$.
\end{defn}
\begin{defn}
    Let $(f,U)$ and $(g,V)$ be function elements of $D$. Say $(g,V)$ is a direct analytic continuation of $(f,U)$ if $U\cap V\neq\varnothing$ and $f\equiv g$ on $U\cap V$.
\end{defn}
\begin{rem}
    The relation of being direct analytic continuation is relfexive and symmetric but not transitive. For instance, $f=\log_{\pi/2}$ on its domain, $g=\log_0$ on the right half plane, $h=\log_{-\pi/2}$ on its domain.
\end{rem}
\begin{defn}
    $(g,V)$ is an analytic continuation of $(f,U)$ along some path $\gamma$ if there exists function elements $(f, U)=(f_1,U_1), (f_2,U_2),...,(f_n,U_n)=(g,V)$ and a dissection $\{t_0=0\le t_1\le...\le t_n=1\}$ s.t. $(f_{i+1},U_{i+1})$ is a direct analytic continuation of $(f_i, U_i)$ and $\gamma([t_{i-1},t_i])\subseteq U_i$. Write $(f,U)\approx (g,V)$.

    An equivalence class $\mathfrak F$ is called a complete analytic function of $D$.
\end{defn}
Note that if $(g,V)$ and $(h,V)$ are both analytic continuations of $(f,U)$ along the \textbf{same} path $\gamma$, then $g\equiv h$ on $V$. (See ES1 or later).

\begin{rem}
    Can do the same for meromorphic continuation.
\end{rem}
\[\subset\vert:\tag{Owen's Signature}\]
\subsection{INFORMAL Examples of Riemann Surfaces}
\begin{enumerate}[1)]
    \item Take disjoint copies $\CC_k^\ast$ of $\CC^\ast$ indexed by $k\in\ZZ$,so $(z,k)\in\CC_k^\ast$. On $\CC_k^\ast$ define $f_k(z)=\ln r+i\theta$ for $z=re^{i\theta}$ and $(2k-1)\pi<\theta\le(2k+1)\pi$. Glue different copies along the cut. Get a space $S$. Then can define $F:S\to\CC$ given by $(z,k)\mapsto f_k(z)$. $F$ is cts, bijective and left invertible. Can do the same for $z^{1/n}$ except get $n$ copies of $\CC^\ast$.
    \item[3)] $\sqrt{z^2-1}$. Cut $[-1,1]$. Take two copies of ``cut complex plane'', $P_0,P_1$. Glue.
    \item What's the difference between a piano, a fish and a pot of glue? (?????????????????????????)
    \item[Ans] You can tune a piano but you can't tuna fish!!!
\end{enumerate}
\subsection{Abstract Riemann Surfaces}
\begin{defn}
    A Riemann Surface $R$ is a (non-empty) connected Hausdorff top. space with a collection of charts (homeo to open subsets of $\CC$) such that the transition functions are holo'c.
\end{defn}
Note the transition function $\tau_{ij}=\varphi_i\varphi_j^{-1}$ is biholo'c. In particular, any Riemann surface is orientable. Second countability is automatic from the definition. (Rado 1925)

Two atlases are said to be compatible (Same Riemann surface) if their union is an atlas.

\[\overset{..}{\tilde{}}\tag{Owen's Signature}\]

\begin{prop}
    The Riemann sphere $\CC_\infty$ is the top. space $\CC\cup\{\infty\}$ (open sets are open subsets of $\CC$ if $\infty\not\in U$ or $U^c$ is compact in $\CC$ if $\infty\in U$.) Charts $\varphi_0:\CC\to\CC$ and $\varphi_1(\CC_\infty\setminus\{0\})$
\end{prop}

Given $P(z,w)\in\CC[z,w]$ Consider $S=\{(z,w)\in\CC^2:P(z,w)=0,(\partial P/\partial z,\partial P/\partial w)\neq (0,0)\}$. Then each conn. component is a Riemann surface. Suppose $(z_0,w_0)\in S$ and $\partial P/\partial z\neq 0$ at $z_0$. Get $f_{w_0}=P(-,w_0)$. By shrinking domain, assume $f_{w_0}D(z_0,r)\to\im(f_{w_0})$ Let $\gamma=\partial D(z_0,r)$ oriented counterclockwise.



\[\supset;\tag{Owen's Signature}\]
\section{Maps on Riemann Surfaces}
\subsection{Analytic Maps}
\begin{defn}
    For Riemann Surfaces $R,S$, a cts function $f:R\to S$ is analytic if for all charts $(\varphi, U)$ on $R$ and $(\psi, W)$ on $S$, have (if $U\cap f^{-1}(W)\neq\varnothing$) $\psi f\varphi^{-1}:\varphi(U\cap f^{-1}(W))\to \CC$ is holomorphic.
\end{defn}
Note that if $R=S$ as top. spaces and $f=\id$, then two atlases are compatible iff $\id$ is analytic.

Consider $\bar\CC$ ($\CC$ with chart given by conjugation), then any biholomorphic map $f:\CC\to\CC$ is not analytic as a map $f:\bar\CC\to\CC$, but the map $f(z)=\bar z$ is analytic.

Let $U$ be a domain in $\CC$. A cts function $f:U\to\CC_\infty$ is analytic iff $f$ is holo'c on $U\setminus f^{-1}(\{\infty\})$ and $g=1/f(z)$ is holo'c on $U\setminus f^{-1}(\{0\})$. If $g$ is never $0$, then $f$ is holo'c. Otherwise $|f(z)|\to \infty$ near the zeros of $g$, i.e., $f$ has poles, i.e., $f$ is meromophic on $U$.

A map $f:\CC_\infty\to\CC$ is analytic iff $f:\CC\to\CC$ and $f(1/z)$ are both holomorphic. By compactness, $f$ is bounded, so $f$ is constant by Liouville.
\begin{defn}
    Let $R$ be a Riemann surface. An analytic function $f$ on $R$ is $f:R\to\CC$ (analytic), and a meromorphic function $g$ on $R$ is $g:R\to\CC_\infty$ (analytic)
\end{defn}
\begin{prop}
    $f:\CC_\infty\to\CC_\infty$ is analytic and nonconstant iff
    \[f(z)=\dfrac{c(z-\alpha_1)\cdots(z-\alpha_m)}{(z-\beta_1)\cdots(z-\beta_n)}\]
    where $c\neq 0$ and $\alpha_i\neq\beta_j$ with $\infty\mapsto\begin{cases}
        \infty & m>n\\
        c & m=n\\
        0 & m<n
    \end{cases}$
\end{prop}
\begin{proof}
    Analytic iff $f(z)$ and $f(1/z)$ are both meromorphic on $\CC$.
    
    ($\Leftarrow$): Clear
    ($\Rightarrow$): If $f(\infty)\neq\infty$. Replace by $1/f$. By continuity, there exists $K>0$ s.t. no poles in $|z|>K$, so we have a finite number of poles $\{\zeta_1,...,\zeta_i\}$. Have Laurent series around each pole. Let $Q_j(z)$ be the principal part of the Laurent series around $\zeta_j$. Let $\bar f=f-Q_1-\cdots-Q_i$, then $f$ has removable singularities at each $\zeta_j$ and holo'c elsewhere (including $\infty$). So $\bar f:\CC_\infty\to\CC$ is analytic, so constant.

    
\end{proof}

\begin{thm}[Identity theorem for Riemann surfaces]
    If $f,g:R\to S$ are analytic maps between Riemann surfaces, then $f\equiv g$ on $R$ or $\forall w\in R$, $\exists \mathcal N$ open nbd of $w$ s.t. $f\neq g$ on $\mathcal N\setminus\{w\}$.
\end{thm}
\begin{proof}
    Let $E=\{w\in R:\exists\mathcal N\ni w,\ f\equiv g\text{ on }\mathcal N\}$, $F=\{w\in R:\exists\mathcal N\ni w,\ f\neq g\text{ on }\mathcal N\setminus\{w\}\}$. Need to show $E\cup F= R$.

    If $f(w)\neq g(w)$. Find $U,V$ open and separating $f(w),g(w)$, then $\mathcal N=f^{-1}(U)\cap f^{-1}(V)$ shows $w\in F$.

    If $f(w)=g(w)$. Take charts $(\varphi, U)$ in $R$ around $w$ and $(\psi, W)$ in $S$ around $f(w)$. $\psi f\varphi^{-1}$ and $\psi g\varphi^{-1}$ are holo'c on $\varphi(U\cap f^{-1}(U)\cap g^{-1}(V))$ and agree at $\varphi(w)$. By identity theorem in CA, there is a disk $D(\varphi(w),r)$ s.t. $\psi f\varphi^{-1},\psi g\varphi^{-1}\varphi^{-1}$ either agree or disagree except at $\varphi(w)$. Let $\mathcal N=\varphi^{-1}(D(\varphi(w),r))$, so $w\in E\cup F$.

    We are done by connectedness.
\end{proof}
\[(\}-:\tag{Owen's Signature with moustache}\]
\begin{thm}[Open Mapping Thm for Riemann surfaces]
    If $f:R\to S$ is a non-constant analytic map, then $W$ open in $R$ $\implies$ $f(W)$ is open in $S$.
\end{thm}
\begin{proof}
    Take $w\in W$ and charts $(\varphi, U)$, $(\psi, V)$ around $w, f(w)$ resp. Let $N=\varphi(U\cap f^{-1}\cap W)$, which is open in $\CC$, and consider $\psi f\varphi^{-1}$. WLOG, assume $N$ is connected, so by identity thm on $\CC$, either const (contradiction) or $\psi f\varphi^{-1}(N)$ is open in $\psi(V)$, so $f\varphi^{-1}(N)$ is open in $V$. Let $M_w=\varphi^{-1}(N)$ which is a nbd of $w$ in $R$ with $M_w\subseteq W$ and $f(w)\in f(M_w)$ is open in $S$, so $f(w)$ is an interior point of $f(W)$.
\end{proof}
\begin{crly}
    Let $f:R\to S$ be analytic with $R$ compact. Then either $f$ is surjective (so $S$ is compact) or $f$ is constant. 
\end{crly}

\subsection{Local Representation of Analytic Maps}
Let $U\subseteq\CC$ be a domain and let $f:U\to\CC$ be a non-const holo'c function with $z_0\in U$. Locally, $f(z)=f(z_0)+\sum_{k\ge 1}a_k(z-z_0)^k$. Let $m\ge 1$ be the smallest s.t. $a_m\neq 0$. Then $f(z)=f(z_0)+(z-z_0)^mg(z)$ where $g(z_0)\neq 0$.
\begin{defn}
    The multiplicity of $f$ at $z_0$ is $m_f(z_0)=m$ as above.
\end{defn}
Note that the set of points with multiplicity $>1$ form a discrete set. Also, the multiplicity is multiplicative (w.r.t. composition).
\begin{thm}[Local mapping thm for domains]
    For non-const $f:U\to\CC$ holo'c, $z_0\in U$, $m=m_f(z_0)$. Then $\exists$ nbd $N$ of $z_0\in U$ and a  biholomorphic $\beta:N\to D(0,\delta)$ (for arbitrarily small $\delta>0$) s.t. $f(z)=f(z_0)+\beta(z)^m$ on $N$.
\end{thm}
\begin{proof}
    Write $f(z)=f(z_0)+(z-z_0)^mg(z)$ on some $D(z_0,r)\subseteq U$ w/ $g$ holo'c and non-zero at $z_0$. Write $g(z_0)=re^{i\alpha}\neq 0$. Have a holo'c branch $\log_\alpha$. Define $g^{-1}(\CC\setminus L_{\pi+\alpha}\cap D(z_0,r))$ is a nbd of $z_0$ on which $g$ is holo'c. Define $\beta(z)=(z-z_0)e^{1/m}\log_\alpha g(z)$. Can see that $\beta'(z_0)\neq 0$. By IFT, have nbd $N''\subseteq N'$ w/ $\beta:N''\to\beta(N'')$ biholo'c. Take $\beta^{-1}(D(0,\delta))$ and restrict.
\end{proof}
Let $f:R\to S$ be analytic and $z_0\in R$ with charts $(\varphi_0,U_0)$, $(\psi_0,V_0)$ around $z_0$ and $f(z_0) $ resp.
\begin{defn}
    The multiplicity $m_f(z_0)$ is $m_{\psi_0f\varphi^{-1}}(\varphi_0(z_0))$.
\end{defn}
\begin{lem}
    This is independent of charts.
\end{lem}
\begin{proof}
    For another pair of charts $(\varphi_1,U_1)$ and $(\psi_1,V_1)$. Write $\psi_1f\varphi_1^{-1}=(\psi_1\psi_0^{-1})(\psi_0f\varphi_0^{-1})(\varphi_0\varphi_1^{-1})$. Done (multiplicative).
\end{proof}

\subsection{Degree}
\begin{thm}[Valency Thm]
    Let $f:R\to S$ be a non-const and analytic. Suppose $R$ is compact. Then $\exists n\ge 1$ (degree/valency) s.t. $\forall w\in S$, $\# f^{-1}(w)=n$ (with multiplicity).
\end{thm}
\[D:\tag{Owen's Signature}\]
\begin{proof}
    For all $w_0\in S$, $f^{-1}(w_0)$ is compact and discrete so finite, say $\{z_1,...,z_q\}\neq\varnothing$. For $A\subseteq R$, and $w\in S$, let $n_A(w)=\sum_{z\in f^{-1}(w)\cap A}m_f(z)$. 

    $R$ Hausdorff, so have $q$ disjoint nbds $M_i$ of $z_i$. Choose charts $(\varphi_i,U_i)$ around $z_i$ and $(\psi,V)$ around $w_0$. Wlog, $\psi(w_0)=0$. Then $\psi f\varphi_i^{-1}$ is holo'c around $\varphi_i(z_i)$, so there exists a nbd $H_i$ in  $\CC$ around $\varphi_i(z_i)$ s.t. $\psi f\varphi_i^{-1}(\zeta)=\psi f(z_i)+\beta_i(\zeta)^{m_f(z_i)}$ for some $\beta_i:H_i\to D(0,r_i)$ is biholo'c. Take $s=\min r_i$ and define $N_i=\varphi_i^{-1}(H_i)$. $f$ is $m_i$-to-1 $N_i\to\psi^{-1}(D(0,s))$ and $N_i$ are disjoint nbd of $z_i$ in $R$.

    Consider $R\setminus\bigcup_{i=1}^q N_i$. It's compact, so $M=S\setminus f(R\setminus \bigcup N_i)$ is open, so a nbd of $w_0$. Take any $w\in\bigcap f(N_i)\cap M$ (nbd of $w_0$) with $R=\bigsqcup N_i\bigsqcup (R\setminus\bigcup N_i)$. Have $n_R(w)=n_{N_1}(w)+\cdots+n_{N_q}(w)+n_{R\setminus \bigcup U_i}(w)=m_f(z_1)+\cdots+m_f(z_q)+0$, so $n_R$ is loc.const.
\end{proof}
\begin{crly}
    $f:\CC_\infty\to\CC_\infty$ is biholo'c (bianalytic) iff $f$ is a M\"obius transformation.
\end{crly}

\subsection{Harmonic Functions}
\begin{defn}
    A func $u:U\to\RR$, $U$ domain, is harmonic if $u\in C^2(U)$ and $\nabla^2u=0$.
\end{defn}
\begin{thm}
    $u$ is harmonic on disk $D$ $\implies$ $\exists f:D\to\CC$ holo'c with $u=\Re f$.
\end{thm}
\begin{crly}
    If $u$ is harmonic on a domain $U$, then $u$ is smooth.
\end{crly}
\begin{crly}
    If $U\to V$ holo'c on a domain and $u:V\to\RR$ harmonic, then $u\circ g$ harmonic.
\end{crly}
\begin{defn}
    $R$ is a Riemann surface. $u:R\to\RR$ is harmonic if $u\varphi_\alpha^{-1}$ is harmonic for all charts $\varphi_\alpha$.
\end{defn}
\begin{thm}[Identity Thm for harmonic functions]
    Let $R$ be a Riemann surface with $u,v:R\to\RR$ both harmonic, then either $u\equiv v$ or $\{z\in R:u(z)=v(z)\}$ has empty interior.
\end{thm}
\begin{proof}
    ES1
\end{proof}
\begin{crly}[Open Mapping for harmonic functions]
    Let $R$ be a Riemann surface with a non-const harmonic function $R\to \RR$, then $u$ is an open map.
\end{crly}
\begin{proof}
    Pick $w\in W$ open and chart $(\varphi, U)$ at $w$. Wlog, assume $U\subseteq W$ and $\varphi(U)$ is a disk, so there exists a holo'c function $f:\varphi(U)\to\CC$ s.t. $\Re f=u\varphi^{-1}$. Note that $f$ is non-constant, so open mapping theorem for holo'c functions, $f(\varphi(U))$ is open. Project to the real part, deduce that $u(U)$ is open in $\RR$, so $u(w)$ is an interior point of $W$ for all $w\in W$.
\end{proof}
\begin{crly}
    If $u:R\to\RR$ is a harmonic function and $R$ is compact, then $u$ is const.
\end{crly}
\section{Covering Maps}
In this section, assume all topological spaces are Hausdorff, connected, and locally path-connected, unless stated otherwise.
\subsection{Local Homeomorphisms}
\begin{defn}
    A map $f:X\to Y$ is a local-homeo if $\forall x\in X$, $\exists N_x$ open nbd of $x$ s.t. $f(N_x)$ is open in $Y$ and $f|_{N_x}:N_x\to f(N_x)$ is a homeo.
\end{defn}
\begin{lem}
    Let $f:R\to S$ be non-const and analytic.
    \begin{enumerate}[(i)]
        \item If $m_f(z)\equiv 1$ then $f$ is a local homeo.
        \item If $Z=\{z\in R: m_f(z)>1\}$, have $R\setminus Z$ RS so $f|_{R\setminus Z}$ is a local homeo
    \end{enumerate}
\end{lem}
\begin{proof}
    (i): For $z\in R$, pick charts $(\varphi, U), (\psi,W)$. Do stuff over $\CC$ and shrink the domain.

    (ii): $Z$ is discrete. $\phi(z)$ has an accumulation point, then identity theorem on $\CC$ and on riemann surfaces imply that $f$ is const.
\end{proof}
\subsection{Paths and  Lifts}
\begin{defn}
    Lift of paths 
\end{defn}
\begin{thm}
    For $f:\tilde X\to X$ local homeo and $\tilde\gamma,\tilde{\tilde\gamma}$ two lifts of $\gamma$ s.t. $\tilde\gamma(0)=\tilde{\tilde\gamma}(0)$, then $\tilde\gamma\equiv\tilde{\tilde\gamma}$.
\end{thm}
\begin{proof}
    Define $E=\{t:\in[0,1]:\tilde\gamma(t)=\tilde{\tilde\gamma}(t)\}$. $E$ is closed since $\tilde X$ is Hausdorff and paths are cts. $E$ is also open. Let $\tau\in E$. Pick nbd $\tilde N$ around $\tilde\gamma(\tau)$ s.t. $f|_{\hat N}$ is inj. For $t\in(\tau-\delta,\tau+\delta)\cap I$, $\tilde\gamma(t),\tilde{\tilde\gamma}(t)\in\hat N$. $f$ is inj on $\hat N$, so $\tilde\gamma(t)=\tilde{\tilde\gamma}(t)$
\end{proof}
\begin{defn}
    Covering maps
\end{defn}
\begin{lem}
    The cardinality of the fiber is constant.
\end{lem}
\begin{proof}
    Equiv rel: $x\sim x'$ iff $|\pi^{-1}(x)|=|\pi^{-1}(x')|$. $|\pi^{-1}(x)|$ is locally constant. Done by connectedness.
\end{proof}
\begin{thm}
    For non-const $f:R\to S$ analytic and $R$ cpt, the map $f:R\setminus f^{-1}f(Z)\to S\setminus f(Z)$ is a covering map of Riemann surfaces.
\end{thm}
\begin{proof}
    $Z$ is discrete and closed in a cpt set, so $Z$ is finite, so $f(Z)$ is finite. For all $w\in S$, $f^{-1}(w)$ is finite (cpt discrete), so $f^{-1}f(Z)$ is finite.
\end{proof}
\[(\&\tag{Owen's Soognature}\]
\subsection{Branched Covering Maps}
\begin{defn}
    If $\tilde S, S$ are Riemann surfaces, a branched covering map $p:\tilde S\to S$ is where $\forall s\in S$ there exists a nbd  $N$ of $s$ in $S$ and a homeo $\Phi:N\to\Delta$ (unit disk) with $\Phi(s)=0$ s.t. $p^{-1}(N)=\bigsqcup_{i\in I} U_i$ a disjoint union of (connected) open subsets of $\tilde S$, each with a homeo $\Psi_i:U_i\to \Delta$ s.t. $\forall z\in \Delta$, $\Phi p\Psi_i^{-1}=z^{m_i}$ for some $m_i\in\NN$.

    A branch point $s$ of $p$ is a point $s=\Psi_i^{-1}(0)\in\tilde S$ where $m_i>1$. (The unique point $x\in U_i$ with $p(x)=s$) A critical value of $p$ is $p(x)\in S$ for any branch point $x$.
\end{defn}
e.g., any non-const analytic map $f:R\to S$ for $R$ compact Riemann surface. (follows from thm 3.7 and valency theorem thm 2.11)

\begin{thm}
    Any compact orientable connected topological surface (without boundary) $S$ is homeo to a genus $g$ surface.
\end{thm}
\begin{thm}[Riemann-Hurwitz formula]
    Let $f:R\to S$ non-const analytic map between compact Riemann surfaces. Suppose $f$ has deg $n$. Then \[\sum_{p\in R}(m_f(p)-1)=2(g_R-1)-2n(g_S-1)\]
    where $g_R,g_S$ are genera of $R,S$ resp.
\end{thm}
\begin{proof}[Proof Sketch [Non-examinable]]
    Finite number of branch points $\{r_1,...,r_k\}$ on $R$ (where $m_f(r_i)>1$). For any $w\in S$ not critical value, have $|f^{-1}(w)|=n$ by 2.11. Take a polygonal decomposition $\calD$ of $S$ (existence by Rado) including all critical values in the vertices. Then the preimage $f^{-1}(D)$ is a polygonal decomposition of $R$ with $nF$ faces, $nE$ edges and $nV-\sum_{j=1}^k(m_f(r_j)-1)$. Compute.
\end{proof}
In particular, $g_R\ge g_S$ (cf. ES2).

\subsection{The (Topological) Monodromy Theorem}
\begin{thm}
    Let $\pi:\tilde X\to X$ be a covering map, $\gamma$ path in $X$ and $p\in\tilde X$ any pt with $\pi(p)=\gamma(0)$. Then there exists a lift $\tilde\gamma$ of $\gamma$ with $\tilde\gamma(0)=p$ (unique by thm 3.4)
\end{thm}
\begin{proof}
    (cf. Htpy lifting)
\end{proof}
\begin{defn}
    Homotopy rel $\partial[0,1]$. (cf. algtop)
\end{defn}
\begin{defn}
    Simply connected (cf. algtop)
\end{defn}
\begin{thm}[Topological Monodromy thm]
    Let $f:Y\to X$ be a local homeo. Suppose $\alpha,\beta$ are paths in $X$ which are homotopic rel $\{0,1\}$. Take $y_0\in Y$ with $f(y)=\alpha(0)=\beta(0)=x_0$. Suppose that any path $\gamma$ in $X$ with $\gamma(0)=x_0$ has a lift $\tilde\gamma(0)=y_0$. Then $\tilde\alpha,\tilde\beta$ are homotopic in $Y$ rel $\{0,1\}$. In particular, $\tilde\alpha(1)=\tilde\beta(1)$.
\end{thm}
Proof omitted.
\begin{crly}
    If $\pi:\tilde X\to X$ is a covering map and if $X$ is simply connected, then $\pi$ is a homeo. 
\end{crly}
\begin{proof}
    ES2.
\end{proof}.
\end{document}