\documentclass{article}
\usepackage{graphicx} % Required for inserting images
\usepackage[utf8]{inputenc}
\usepackage{amsmath,amsfonts,amssymb,amsthm}
\usepackage{enumerate,bbm}
\usepackage{leftindex}
\usepackage{tikz,tikz-cd,graphicx,color,mathrsfs,color,hyperref,boldline}
\usepackage{caption,float}
\usepackage[a4paper,margin=1in,footskip=0.25in]{geometry}

\usepackage{listings}
\usepackage{xcolor}

\usepackage{tabularx,capt-of}

\usepackage{blindtext}
%Image-related packages
\usepackage{graphicx}
\usepackage{subcaption}
\usepackage[export]{adjustbox}
\usepackage{lipsum}

%hyperref setup
\hypersetup{
    colorlinks=true,
    linkcolor=blue,
    filecolor=magenta,      
    urlcolor=cyan,
    pdftitle={Overleaf Example},
    pdfpagemode=FullScreen,
    }

%New colors defined below
\definecolor{codegreen}{rgb}{0,0.6,0}
\definecolor{codegray}{rgb}{0.5,0.5,0.5}
\definecolor{codepurple}{rgb}{0.58,0,0.82}
\definecolor{backcolour}{rgb}{0.95,0.95,0.92}

%Code listing style named "mystyle"
\lstdefinestyle{mystyle}{
  backgroundcolor=\color{backcolour}, commentstyle=\color{codegreen},
  keywordstyle=\color{magenta},
  numberstyle=\tiny\color{codegray},
  stringstyle=\color{codepurple},
  basicstyle=\ttfamily\footnotesize,
  breakatwhitespace=false,         
  breaklines=true,                 
  captionpos=b,                    
  keepspaces=true,                 
  numbers=left,                    
  numbersep=5pt,                  
  showspaces=false,                
  showstringspaces=false,
  showtabs=false,                  
  tabsize=2
}

%"mystyle" code listing set
\lstset{style=mystyle}

\theoremstyle{definition}
\newtheorem{defn}{Definition}[section]
\newtheorem{example}[defn]{Example}
\theoremstyle{remark}
\newtheorem{rem}{Remark}
\newtheorem{remS}[section]{defn}
\theoremstyle{plain}
\newtheorem{lem}[defn]{Lemma}
\newtheorem{thm}[defn]{Theorem}
\newtheorem{prop}[defn]{Proposition}
\newtheorem{fact}[defn]{Fact}
\newtheorem{crly}[defn]{Corollary}
\newtheorem{conj}[defn]{Conjecture}

%\newtheorem*{programming*}{Programming Task}

%\newtheorem{innercustomgeneric}{\customgenericname}
%\providecommand{\customgenericname}{}
%\newcommand{\newcustomtheorem}[2]{%
%  \newenvironment{#1}[1]
%  {%
%   \renewcommand\customgenericname{#2}%
%   \renewcommand\theinnercustomgeneric{##1}%
%   \innercustomgeneric
%  }
%  {\endinnercustomgeneric}
%}

%\newcustomtheorem{question}{Question}
%\newcustomtheorem{programming}{Programming Task}

\newcommand{\NN}{\mathbb{N}}
\newcommand{\ZZ}{\mathbb{Z}}
\newcommand{\QQ}{\mathbb{Q}}
\newcommand{\RR}{\mathbb{R}}
\newcommand{\CC}{\mathbb{C}}
\newcommand{\PP}{\mathbb{P}}
\newcommand{\FF}{\mathbb{F}}
\newcommand{\Hom}{\operatorname{Hom}}
\newcommand{\im}{\operatorname{im}}
\newcommand{\id}{\operatorname{id}}
\newcommand{\Ind}{\operatorname{Ind}}
\newcommand{\Res}{\operatorname{Res}}
\newcommand{\btop}{\mathbf{Top}}
\newcommand{\bdel}{\mathbf{\Delta}}
\newcommand{\sing}{\operatorname{Sing}}
\newcommand{\bset}{\mathbf{Set}}
\newcommand{\Rmod}{\mathbf{Mod}_R}

\newcommand{\calD}{\mathcal{D}}

\newcommand{\sol}{\textit{Solution: }}

\title{Simplicial Homotopy Theory}
\author{Kevin}
\date{January 2025}

\begin{document}
\maketitle
\section{Introduction}
Recall basic notions from category theory
\begin{defn}
    A category $\mathscr{C}$ consists of a collection of objects, denoted $\operatorname{ob}(\mathscr C)$, and for each pair of objects $A,B$, a collection of morphisms $\Hom_{\mathscr C}(A,B)$ such that
    \begin{itemize}
        \item [(i)] There is an associative composition law of morphisms.
        \item[(ii)] For each $A\in\operatorname{ob}(\mathscr C)$, there is a distinguished element $\id_A\in\Hom_{\mathscr C}(A,A)$, which is left and right unital.
    \end{itemize}
\end{defn}
\begin{defn}
    A morphism is called an isomorphism if it has a two-sided inverse.
\end{defn}

Given a category $\mathscr{C}$, want to classify objects up to isomorphism.

For the purpose of this course, $\mathbf{Top}$ will denote the category  with objects those topological spaces that are homeomorphic to a CW complex. Morphisms are cts functions. Isomorphisms are homeomorphisms. We might attempt to classify objects in $\mathbf{Top}$ up to homeo.

A useful tool is a functor, for every integer $i$ and commutative irng $R$
\[H_i(-;R):\mathbf{Top}\to\Rmod,\ X\mapsto H_i(X;R)\]
A key ingredient in the defns of these functors is the study of cts maps from the n-simplex into $X$, as $n$ varies.

There is a factorization of functors.
\[H_i(-;R):\mathbf{Top}\to \mathbf{HoTop}\to\Rmod\]
where $\mathbf{HoTop}$ is the full subcategory of $\mathbf{Top}$ obtained by forcing homotopic cts maps to be equal (so an isomoprhism in this category is a htpy equivalence). Homotopy theory is the study of $\mathbf{HoTop}$.
\begin{rem}
    Isomorphism classes in $\mathbf{HoTop}$ can be understood via the moduli stack of formal groups and other objects in arithmetic geometry.
\end{rem}

\textbf{Question: How to think about objects in $\mathbf{HoTop}$ up to isomorphism?}

Roughly, a homotopy type consists of 
\begin{itemize}
    \item A collection of objects
    \item For every pair of objects a collection of isomorphism between them
    \item For every pair of parallel isomorphism $f,g:A\to B$ a collection of 2-isomorphisms between them.
\end{itemize}
\begin{example}
    The following shows two different homotopy types.
    \begin{figure}[H]
        \centering
        \includegraphics[width=0.5\linewidth]{Lent/pictures/htpy_type_example.PNG}
        \caption{Two different homotopy types}
        \label{fig:1}
    \end{figure}
    On the top, we have a homotopy type consisting of two objects $A, B$ and an isomorphism between them (represented by the edge). This is isomorphic to a single object $A$ as we can (formally) contract the isomorphism to identify equivalent objects. It is also equivalent to a htpy type consisting of $A,B,C,D$ where $B,C,D$ are all isomorphic to $A$ via a unique isomorphism. It is also euqivalent to the htpy type consisting of two objects $A,B$, two isomorphisms $A\to B$ and a 2-isomorphism, i.e., an equivalence of these two isomorphisms.

    On the bottom, we have another homotopy type with two objects and two non-equivalent isomorphisms. We may contract one isomoprhism and get a htpy type of consisting of a single object with an interesting automorphism.

    Sets are homotopy types with no non-trivial isomoprhisms or 2-isomorphisms.

    Groupoids, i.e., categories in which all morphisms are isomorphisms, are homotopy types with no non-trivial 2-isomorphisms and 3-isomorphisms, etc.
\end{example}
Given a topological space $X$, there is an associated homotopy type s.t.
    \begin{itemize}
        \item points of $X$ are objects. (cts maps from the $0$-simplex)
        \item paths in $X$ are isomorphisms. (cts maps from the $1$-simplex)
        \item A cts function $|\Delta^2|\to X$ is a 2-isomorphism. (Regard the map restricted to two edges of $|\Delta^2|$ as a concatenation of paths, then the map is essentially a homotopy rel the boundary of $|\Delta^1|$.)
    \end{itemize}

\begin{defn}
    The simplex category $\mathbf{\Delta}$ has 
    \begin{itemize}
        \item objects $[n]$ for every integer $n\ge 0$
        \item A morphism $[n]\to[m]$ is an order preserving function from $\{1\le2\le...\le n\}\to\{1\le 2\le...\le m\}$.
    \end{itemize}
\end{defn}
There is a functor $\mathbf{\Delta}\to\mathbf{Top}$ sending $[n]\mapsto|\Delta^n|$.
\begin{defn}
    A simplicial set is a functor $\mathbf{\Delta}^{op}\to\mathbf{Set}$.
\end{defn}
\begin{example}
    If $X$ is a topological space, there is an associated simplicial set $\sing(X):\mathbf{\Delta}^{op}\to \mathbf{Set}$ which sends $[n]$ to the set of continuous functions $|\Delta^n|\to X$.
\end{example}
\end{document}