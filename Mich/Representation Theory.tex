\documentclass{article}
\usepackage{graphicx} % Required for inserting images
\usepackage[utf8]{inputenc}
\usepackage{amsmath,amsfonts,amssymb,amsthm}
\usepackage{enumerate,bbm}
\usepackage{tikz,graphicx,color,mathrsfs,color,hyperref,boldline}
\usepackage{caption,float}
\usepackage[a4paper,margin=1in,footskip=0.25in]{geometry}

\usepackage{listings}
\usepackage{xcolor}

\usepackage{tabularx,capt-of}

\usepackage{blindtext}
%Image-related packages
\usepackage{graphicx}
\usepackage{subcaption}
\usepackage[export]{adjustbox}
\usepackage{lipsum}

%hyperref setup
\hypersetup{
    colorlinks=true,
    linkcolor=blue,
    filecolor=magenta,      
    urlcolor=cyan,
    pdftitle={Overleaf Example},
    pdfpagemode=FullScreen,
    }

%New colors defined below
\definecolor{codegreen}{rgb}{0,0.6,0}
\definecolor{codegray}{rgb}{0.5,0.5,0.5}
\definecolor{codepurple}{rgb}{0.58,0,0.82}
\definecolor{backcolour}{rgb}{0.95,0.95,0.92}

%Code listing style named "mystyle"
\lstdefinestyle{mystyle}{
  backgroundcolor=\color{backcolour}, commentstyle=\color{codegreen},
  keywordstyle=\color{magenta},
  numberstyle=\tiny\color{codegray},
  stringstyle=\color{codepurple},
  basicstyle=\ttfamily\footnotesize,
  breakatwhitespace=false,         
  breaklines=true,                 
  captionpos=b,                    
  keepspaces=true,                 
  numbers=left,                    
  numbersep=5pt,                  
  showspaces=false,                
  showstringspaces=false,
  showtabs=false,                  
  tabsize=2
}

%"mystyle" code listing set
\lstset{style=mystyle}

\theoremstyle{definition}
\newtheorem{defn}{Definition}[section]
\newtheorem{example}[defn]{Example}
\theoremstyle{remark}
\newtheorem{rem}{Remark}
\newtheorem{remS}[section]{defn}
\newtheorem{lem}[defn]{Lemma}
\theoremstyle{plain}
\newtheorem{thm}[defn]{Theorem}
\newtheorem{prop}[defn]{Proposition}
\newtheorem{fact}[defn]{Fact}
\newtheorem{crly}[defn]{Corollary}
\newtheorem{conj}[defn]{Conjecture}

%\newtheorem*{programming*}{Programming Task}

%\newtheorem{innercustomgeneric}{\customgenericname}
%\providecommand{\customgenericname}{}
%\newcommand{\newcustomtheorem}[2]{%
%  \newenvironment{#1}[1]
%  {%
%   \renewcommand\customgenericname{#2}%
%   \renewcommand\theinnercustomgeneric{##1}%
%   \innercustomgeneric
%  }
%  {\endinnercustomgeneric}
%}

%\newcustomtheorem{question}{Question}
%\newcustomtheorem{programming}{Programming Task}

\newcommand{\NN}{\mathbb{N}}
\newcommand{\ZZ}{\mathbb{Z}}
\newcommand{\QQ}{\mathbb{Q}}
\newcommand{\RR}{\mathbb{R}}
\newcommand{\CC}{\mathbb{C}}
\newcommand{\PP}{\mathbb{P}}
\newcommand{\FF}{\mathbb{F}}

\newcommand{\calD}{\mathcal{D}}

\newcommand{\sol}{\textit{Solution: }}
\newcommand{\Res}{\operatorname{Res}}

\title{Representation theory}
\author{Kevin}
\date{October 2024}

\begin{document}
\maketitle
\section{Revision of LA}
\section{Definitions}
\begin{defn}
    $G$ is a group, $V$ a vector space over a field $k$, a representation of $G$ on $V$ is a group homomorphism $\rho:G\to GL(V)$
\end{defn}
\begin{rem}
    Synonym 1: $G$ acts on $V$ by linear transformations
    Synonym 2: $V$ is a $kG$-module...
\end{rem}
\begin{defn}
    The degree of $\rho$ is $\dim V$.
\end{defn}
\begin{example}
\begin{enumerate}
    \item $G=\ZZ_2=\{\pm 1\}$, $V=\RR^2$, $1\mapsto I$, $-1\mapsto \begin{pmatrix}
        -1&\\
        &1
    \end{pmatrix}$
    \item (Trivial representation) $G$ any group, $V=k$, $\rho(g)=1\in k^\ast$.
    \item $G=\ZZ$, $V$ vector space. We have 
    \[\{\text{rep of }\ZZ\}\leftrightarrow\{\text{invertible linear transformation}\}\]
    given by $\rho\mapsto \rho(1)$. (This follows from definition)
    \item $G=\ZZ_N$, then similarly, we have
    \[\{\text{rep of }\ZZ_N\}\leftrightarrow\{\text{invertible linear maps }f\text{ s.t. }f^N=I\}\]
\end{enumerate}
\end{example}
\begin{rem}
    If $G$ has group presentation $\langle g_1,...,g_k\mid r_1,...,r_l\rangle$, then
    \begin{align*}
        \{\text{reps of }G\text{ on }V\}\leftrightarrow\{(f_1,...,f_k)\in GL(V)\mid \forall i,\rho(r_i)=I\}
    \end{align*}
\end{rem}
\begin{example}
    $G=S_3$, $V=\RR^2$. Given $g\in G$ make it act on an equilateral triangle by permuting vertices and extending linearly. This gives a representation by definition.
\end{example}

\section{Category of representations}
Problem: too many reps. We want to
\begin{itemize}
    \item know when two reps are the ``same'';
    \item break up reps into ``smaller'' reps.
\end{itemize}
\begin{defn}
    Let $\rho:G\to\operatorname{Aut}(V)$, $\rho':G\to V'$ are isomorphic if there is an isomorphism $\varphi:V\to V'$ as vector spaces s.t. $\varphi\circ\rho(g)=\rho'(g)\circ\varphi$ for all $g\in G$.
\end{defn}
Note that in the definition above, $\varphi$ is independent of $g$.

\textbf{Terminology:} $\varphi$ intertwines $\rho,\rho'$. This is the same as saying $\varphi^{-1}$ intertwines $\rho',\rho$.
\begin{example}
    $\rho:\ZZ\to\operatorname{Aut}(V)$ and $\rho':\ZZ\to\operatorname{Aut}(V')$ are isomorphic iff there exists a basis of $V$ and a basis of $V'$ s.t. 
    \[\begin{bmatrix}\text{matrix of }\rho(1)\text{ basis of }V\end{bmatrix}=\begin{bmatrix}
        \text{matrix of }\rho'(1)\text{ wrt to basis of }V'
    \end{bmatrix}\]
    As a corollary, isomorphism classes of reps of $\ZZ$ on $V$ is in bijection correspondence with rational canonical forms.
\end{example}
\begin{rem}
    If two representations $\rho,\rho'$ are isomorphic, then $\dim\rho=\dim\rho'$, but the converse is false, e.g., $\rho:\ZZ\to GL_1(\RR),1\mapsto 1$ and $\rho:\ZZ\to GL_1(\RR), 1\mapsto -1$. Another example is if $G=\ZZ_2$, then the matrices $\operatorname{diag}(-1,1),\operatorname{diag}(1,-1)$ and "off-diag"(1,1) define isomorphic reps which are different from $\operatorname{diag}(-1,-1)$ and $I$.
\end{rem}
    Let $\rho:G\to\operatorname{Aut}(V)$ be a rep. Suppose $W\le V$ is preserved by $G$, i.e., $gW\subseteq W$ for all $g\in G$ (implies $gW=W$ as $g$ is invertible). Then $G$ acts on $W$ and we get $\tilde{rho}: G\to\operatorname{Aut}(W)$ and we say that $\tilde{\rho}$ is a subrepresentation of $\rho$.
\begin{defn}
    A $G$-invariant subspace $W\le V$ is called a subrepresentation. If $W=0$ or $W=V$ then we say the subrep is trivial. Otherwise we say the subrep is non-trivial
\end{defn}
\begin{defn}
    If there are no non-trivial subreps of $V$ then we say that $V$ is irreducible (simple).
\end{defn}
 \begin{example}
     \begin{itemize}
         \item Any 1-dim rep is irred.
         \item $G=\ZZ_2$, $1\mapsto\operatorname{diag}(1,-1)$ is not irreducible because it has exactly two non-trivial subreps.
         \item $S^3$ acts on an equilateral triangle by permuting vertices, one can extend linearly to $\CC^2$. This rep is irreducible by noting the fact that an invariant subspace is a complex line which corresponds to common eigenvectors.
         \item $G=\ZZ$, $V$ vec. space over $\CC$, $\rho(1)=A$. Can check that the invariant subspaces are precisely defined by Jordan blocks of $A$ when written in JNF. Suppose wrt to some basis $e_1,...,e_n$ one has $A$ in JNF, then $\langle e_1\rangle,..,\langle e_1,...,e_n\langle$ are all invariant subspaces, so the only irreducible rep of $\ZZ$ on vector space over $\CC$ are the $1$-dim reps.
     \end{itemize}
 \end{example}
 \begin{prop}
     Let $\rho:G\to\operatorname{Aut}(V)$ be a rep and $W\le V$ a subspace. Then $W$ is a subrep if and only if for every basis $v_1,...,v_a$ of $W$ and its extension $v_1,...,v_d$ which is a basis of $V$, the matrix of $\rho(g)$ is of the form
     \begin{align*}
     \begin{pmatrix}
         A & B\\ 0&C
     \end{pmatrix}
     \end{align*} for all $g\in G$
 \end{prop}
 \begin{proof}
 Trivial.
 \end{proof}
 \begin{defn}
     A rep $V$ of $G$ is a direct sum of $W,W'$ if 
     \begin{itemize}
         \item they are invariant subspaces
         \item and $V=W\oplus W'$ as a vector space.
     \end{itemize}
 \end{defn}
 \begin{prop}
     $V$ is a direct sum of $W,W'$ if and only if for every basis $v_1,...,v_a$ of $W$ and basis $v_{a+1},...,v_{d}$ of W', the matrix of $\rho(g)$ is of the form
     \begin{align*}
         \begin{pmatrix}
             A&0\\0&B
         \end{pmatrix}
     \end{align*} for all $g\in G$
 \end{prop}
 \begin{proof}
     Trivial.
 \end{proof}
 One can define external direct sums $\rho\oplus\rho'$ as well by requiring the action to be componentwise.
 \begin{defn}
     Given any rep $V$ of $G$ (abuse of notation), write it as a direct sum $V=W_1\oplus...\oplus W_r$, where $W_i$ is $G$-invariant and does not itself break up into direct sum. Such a $W$ is called indecomposable
 \end{defn}
 An example is given by JNF.
 \begin{rem}
     Indecomposable $\not\implies$ Irreducible.
 \end{rem}
 \begin{defn}
     Let $\rho:G\to\operatorname{Aut}(V)$, $\rho':G\to\operatorname{Aut}(V')$ be reps of $G$. A morphism (or $G$-equivariant map) from $\rho$ to $\rho'$ is a linear map $\varphi:V\to V'$ s.t. $\varphi\rho(g)=\rho'(g)\varphi$.

     We write $\operatorname{Hom}_G(V,V')$ for the set of these $G$-equivariant maps. This is a vector space.
     \begin{itemize}
         \item If $\varphi$ is an iso, then $V,V'$ are isomorphic reps;
         \item If $\varphi$ is injective, then $V$ is a subrep of $V'$
         \item If $\varphi$ is surjective, then $V'$ is a quotient rep of $V$.
     \end{itemize}
 \end{defn}
 \begin{example}
     Suppose $G$ acts on a set $X$, then we can lienarize the action and get a representation on $k[X]$ which is the vector space with basis $\{e_x\}_{x\in X}$. Note that this rep is never irreducible.

     Exercise:
     \begin{itemize}
         \item Show that if $X=\{x_1,...,x_d\}$, then $W=\langle e_{x_1}+...+e_{x_d}\rangle$ and $W'=\{\sum a_ix_i\mid \sum_ia_i=0\}$ are both subreps of $k[X]$ and if $\operatorname{char}(k)=0$, $k[X]=W\oplus W'$.
         \item Can you find an example where $W'$ is not irred.
         \item Is $W\oplus W'=k[X]$ true if $\operatorname{char}(k)=p$ prime?
     \end{itemize}
 \end{example}

\textbf{Clarification:} If $\varphi\in\operatorname{Hom}_G(V,V')$ is surjective, then $V/\ker\varphi\cong V'$, so we say that $V'$ is a quotient rep of $V$.

If $G$ acts on a set $X$ and $X=X_1\sqcup X_2$ and $GX_i=X_i$, then $k[X]=k[X_1]\oplus k[X_2]$ (direct sum as rep). More generally, $X=\bigsqcup O_i$ where $O_i$ are orbits. If $y\in O_i$, then $G/\operatorname{Stab}(y)$ is in bijective correspondence with $O_i$. Then, $k[X]=\bigoplus_i k[O_i]$. These are called permutation reps

\begin{example}
    $S_3$ acts on $\{1,2,3\}$ then $k[X]=k^3$ and the action is by permuting standard basis vectors.
\end{example}
\begin{defn}
    $\rho:G\to GL(V)$ rep, then $\rho$ is faithful if $\ker\rho$ is trivial.
\end{defn}
\begin{crly}
    If $G$ is simple, then any non-trivial rep is faithful. Every finite group can be embedded as the subgroup of some $GL(V)$ so admits a faithful rep.
\end{crly}
\begin{proof}
    1st part is trivial. The 2nd part uses Cayley's theorem and linear extension to make it an action on vector spaces.
\end{proof}
\begin{rem}
    Can compose $\rho$ rep with a group hom to get new reps. When the group hom is an inclusion, we say that the resulting rep is the restriction to the subgroup.
\end{rem}

\section{Complete Reducibility}
\begin{thm}
Let $G$ be a be a finite group. $V$ a vector space over field $k$ s.t. $\operatorname{char}(k)=0$. Suppose $W\le V$ is a $G$-invariant subspace. Then $\exists$ a $G$-invariant complement $W'$ s.t. $V=W\oplus W'$ as reps of $G$.
\end{thm}
We will prove this theorem later. We now discuss some consequences of this theorem.
\begin{crly}
    In the situation of the preceding theorem, $V\cong W_1\oplus...\oplus W_r$ as reps s.t. $W_i$ is irreducible.
\end{crly}
\begin{proof}
    Induction on dimension.
\end{proof}
\begin{example}
    $G=\ZZ_N$, $k$ algebraically closed field with characteristic $0$. We can invoke JNF.
     If $V$ is indecomposable as a rep of $G$, then it's also indecomposable as a rep of $\ZZ$ by precomposing with the projection map, and the matrix of $\rho(1)=A$ is a Jordan block $J_\lambda$. We may compute $J_\lambda^i$ to conclude that $d=1$ and $\lambda^N=1$. So the only Jordan blocks that occur in a finite dim rep of $\ZZ_N$ are $1\times 1$ blocks with e-values $N$th roots of unity, i.e., diagonal. Exercise: write down all invariant subspaces of $V$ is $\rho(1)$ is as above.

     If $k=\FF_p$, $G=\ZZ_p$, then $G$ acts on $\FF_p^2$ by $1\mapsto \begin{pmatrix}
         1&a\\0&1
     \end{pmatrix}$ Show that there is no complement to the invariant line $\langle e_1\rangle$.
\end{example}
\textbf{Exercise:}  prove the theorem if $G$ is abelian and $V$ a v.s. over $\CC$. [Pick a $g\in G$ and find an e-value $\lambda$. Now any $h\in G$ preserves the $\lambda$-eigenspace by commutativity. Induction.]

Recall relevant knowledge on Hermitian inner product. [linear in the 1st argument and conjugate-linear in the 2nd argument.]

\begin{lem}
If $W\le V$ is a $G$-invariant subsapce and $\langle-,-\rangle$ a $G$-invariant inner product on $V$, then $W^\perp$ is $G$-invariant.
\end{lem}
\begin{proof}
    If $x\in W^\perp$ and $g\in G$, then $\langle gx,w\rangle=\langle g^{-1}gx,g^{-1}w\rangle=\langle x,g^{-1}w\rangle=0$.
\end{proof}
\begin{lem}
    $G$ finite group acting on a vector space $V$ over $\CC$. Then there exists a $G$-invariant Hermitian inner product on $V$.
\end{lem}
\begin{proof}
    Choose any Hermitian inner product and denote it $\langle-,-\rangle$. Define a new inner product 
    \[(v,w)=|G|^{-1}\sum_{g\in G}\langle gv,gw\rangle\]
    Most properties are trivial. We check $G$-invariance and non-degeneracy. Let $h\in G$, then
    \[(hv,hw)=|G|^{-1}\sum_{g\in G}\langle ghv,ghw\rangle=|G|^{-1}\sum_{g'\in G}\langle g'v,g'w\rangle=(v,w)\]
    where we use the change of index $g'=gh$. For non-degeneracy, note that $(v,v)=0\leftrightarrow\langle gv,gv\rangle=0$ for all $g\in G$ $\Leftrightarrow v=0$.
\end{proof}
\begin{defn}
    A rep $V$ of $G$ is completely reducible if it's isomorphic to a direct sum of irreducible reps.
\end{defn}
\begin{crly}
    Representations of finite group $G$ on vector space $V$ over $\CC$ are completely reducible.
\end{crly}
\begin{crly}
    If $G\le GL_n(\CC)$ is finite, then $G$ is conjugate to a subgroup of $U(n)$.
\end{crly}
\begin{proof}
    Recall that if $(-,-)$ is any Hermitian inner product, then can choose a basis $v_1,...,v_n$ of $\CC^n$ s.t. $(v_i,v_j)=\delta_{ij}$. Let $X$ be the change of basis matrix from $\{v_1,...,v_n\}$ to the standard basis, then $\langle x,y\rangle=(Xx,Xy)$. Apply this to a $G$-invariant inner product, then 
    \[\langle X^{-1}gXx,X^{-1}gXy\rangle = (gXx,gXy)=(Xx,Yy)=\langle x,y\rangle\]
    so $\{X^{-1}gX:g\in G\}\le U(n)$.
\end{proof}
\begin{crly}
    If $A\in GL_n(\CC)$ and $A^n=I$, then $A$ is diagonalizable.
\end{crly}
\begin{proof}
    $A$ defines a rep of $\ZZ_n$. We have three variants of the same proof.
    \begin{itemize}
        \item $A$ is conjugate to a unitary matrix, so diagonalizable by the preceding corollary.
        \item $v\in\CC^n$ e-vector for $A$. Complete reducibility implies that $\CC^n=\CC v\oplus W$ as reps. Done by induction on dimensions.
        \item The only irreducible reps are one dimensional (e-vectors exist in $\CC$), so $\CC^n$ splits as the direct sum of $A$-invariant lines, so $A$ is diagonal in some basis.
    \end{itemize}
\end{proof}
\begin{thm}[Complete reducibility]
    Let $G$ be a finite group, $V$ a vector space over field $k$ such that $\operatorname{char}(k)\nmid |G|$. If $W\le V$ is $G$-invariant, then there exists a $G$-invariant complement.
\end{thm}
\begin{proof}
    Pick any $W'$ which is a complement of $W$. Let $q:V\to W$ be the projection onto $W$ with $\ker q=W$. We then have $gq(w)=gw=qgw$ for all $w\in W$. Now define $\Bar{q}(v)=|G|^{-1}\sum_{g\in G}gqg^{-1}v$
    then $\Bar{q}$ has several properties.
    \begin{itemize}
        \item $\Bar{q}(V)=W$ as $q(V)=W$.
        \item $gq(v)\in W$ for all $v\in V$ as $W$ is $G$-stable
        \item If $w\in W$, then $\Bar{q}(w)=|G|^{-1}\sum_{g\in G}gqg^{-1}w=|G|^{-1}\sum_{g\in G}gg^{-1}w=w$, so $\Bar{q}$ projects onto $W$.
        \item $\ker\bar{q}$ is $G$-stable because for $h\in G$, $h\bar{q}(v)=|G|^{-1}\sum_{g'\in G}g'qg^{-1}hv=\bar{q}hv$ where $g'=gh$.
        \item Finally, can prove that $\ker\bar q\oplus\operatorname{im}\bar q=V$ using properties of projection (idempotency).
    \end{itemize}
    So we have found a $G$-invariant subspace.
\end{proof}
Illuminating exercise: Do this process for $G=\ZZ_2$.
What fails when $\operatorname{char}(k)\mid G$?

\section{Characters}
\begin{defn}
    $\rho:G\to\operatorname{Aut}(V)$, $V$ a vec. space over $k$. The character of $\rho$ is a function $\chi:G\to k$, $g\mapsto\operatorname{tr}(\rho(g))$.
\end{defn}
\begin{rem}
    It is clear that $\chi$ satisfies the following properties.
    \begin{itemize}
        \item $\chi(g)$ doesn't depend on the choice of basis of $V$.
        \item If $\rho,\rho'$ are isomorphic reps via a linear iso represented by matrix $X$, then $\operatorname{tr}(\rho'(g))=X\rho(g) X^{-1}$ for all $g\in G$.
    \end{itemize}
\end{rem}
\begin{lem}
    $\rho$ rep of $G$ on $V$ (over $\CC$).
    \begin{itemize}
        \item $\chi(1)=\dim V$
        \item $\chi(g)=\chi(hgh^{-1})$ for all $h,g\in G$
        \item $\chi(g^{-1})=\chi(g)^\ast$ if $G$ is finite.
        \item If $\chi'$ is the character of $\rho'$ then $\chi_{\rho\oplus\rho'}=\chi+\chi'$ defined as $g\mapsto\chi(g)+\chi'(g)$.
    \end{itemize}
\end{lem}
\begin{proof}
    Trivial.
\end{proof}
\begin{defn}
    The set of class functions is $\mathcal{C}_G=\{f:G\to\CC:\forall h,g\in G, f(ghg^{-1})=f(h)\}$.
\end{defn}
Note that $\mathcal{C}_G$ is a vector space over $\CC$ with basis the functions \[\delta_{\mathcal{O}}:G\to\CC,g\mapsto \begin{cases}1&g\in\mathcal{O}\\ 0&g\not\in\mathcal{O}\end{cases}\]
where $\mathcal{O}$ is a conjugacy class.

Write $G$ as a disjoint union of conj. classes, then $\chi\in\mathcal{C}_G$ if $V$ is a complex rep of $G$.

Define a Hermitian inner product on $\mathcal{C}_G$ by
\[\langle f,f'\rangle=|G|^{-1}\sum_{g\in G}f(g)\overline{f'(g)}=\sum_{1}^r\dfrac{1}{|Z_G(x_i)|}f(x_i)\overline{f'(x_i)}\]
where $x_i$ are in distinct orbits.
\begin{thm}
    If $V,V'$ are irreducible complex of a finite group $G$ with character $\chi,\chi'$, then
    \[\langle\chi,\chi'\rangle=\begin{cases}
        1 & V\cong V'\\
        0 & \text{o/w}
    \end{cases}\]
\end{thm}
Proof deferred.
\begin{crly}
    The number of distinct irred. reps of $G$ = the number of distinct irred. reps?????
\end{crly}
\begin{thm}
    There are as many irred. complex reps as conj. classes in $G$ (finite), i.e., $\{\chi_V: V\text{ irred.}\}$ is an orthonormal basis of $\mathcal{C}_G$.
\end{thm}

\begin{crly}
    If $\rho',\rho'$ are two complex reps of $G$ with the same character, then $\rho\cong\rho'$
\end{crly}
Proof deferred.


We introduce some notation. If $\rho$ is a rep of $G$, then write $n\rho$ for the $n$-fold direct sum of $\rho$, whose character is then denoted by $n\chi_\rho$. So if $\rho$ is isomorphic to a direct sum of reps, i.e., $\rho=n_1\rho_1\oplus...\oplus n_r\rho_r$, $\rho_i$ irred (complete reducibility) so that $\chi=n_1\chi+...+n_r\chi_r$. But $\langle\chi_i,\chi_j\rangle=\delta_{ij}$, so $\langle\chi_i,\chi_j\rangle=n_i$, so have an expansion in terms of the orthonormal basis $\{\chi_i\}$.

In particular, in any decomposition of a rep $V$ into direct summand of irred. reps, the multiplicity with which the irred. reps occur is $\langle\chi,\chi_i\rangle$, which is independent of the decomposition.



\end{document}