\documentclass{article}
\usepackage{graphicx} % Required for inserting images
\usepackage[utf8]{inputenc}
\usepackage{amsmath,amsfonts,amssymb,amsthm}
\usepackage{enumerate,bbm}
\usepackage{leftindex}
\usepackage{tikz,tikz-cd,graphicx,color,mathrsfs,color,hyperref,boldline}
\usepackage{caption,float}
\usepackage[a4paper,margin=1in,footskip=0.25in]{geometry}

\usepackage{listings}
\usepackage{xcolor}

\usepackage{tabularx,capt-of}

\usepackage{blindtext}
%Image-related packages
\usepackage{graphicx}
\usepackage{subcaption}
\usepackage[export]{adjustbox}
\usepackage{lipsum}

%hyperref setup
\hypersetup{
    colorlinks=true,
    linkcolor=blue,
    filecolor=magenta,      
    urlcolor=cyan,
    pdftitle={Overleaf Example},
    pdfpagemode=FullScreen,
    }

%New colors defined below
\definecolor{codegreen}{rgb}{0,0.6,0}
\definecolor{codegray}{rgb}{0.5,0.5,0.5}
\definecolor{codepurple}{rgb}{0.58,0,0.82}
\definecolor{backcolour}{rgb}{0.95,0.95,0.92}

%Code listing style named "mystyle"
\lstdefinestyle{mystyle}{
  backgroundcolor=\color{backcolour}, commentstyle=\color{codegreen},
  keywordstyle=\color{magenta},
  numberstyle=\tiny\color{codegray},
  stringstyle=\color{codepurple},
  basicstyle=\ttfamily\footnotesize,
  breakatwhitespace=false,         
  breaklines=true,                 
  captionpos=b,                    
  keepspaces=true,                 
  numbers=left,                    
  numbersep=5pt,                  
  showspaces=false,                
  showstringspaces=false,
  showtabs=false,                  
  tabsize=2
}

%"mystyle" code listing set
\lstset{style=mystyle}

\theoremstyle{definition}
\newtheorem{defn}{Definition}[section]
\newtheorem{example}[defn]{Example}
\theoremstyle{remark}
\newtheorem{rem}{Remark}
\newtheorem{remS}[section]{defn}
\theoremstyle{plain}
\newtheorem{lem}[defn]{Lemma}
\newtheorem{thm}[defn]{Theorem}
\newtheorem{prop}[defn]{Proposition}
\newtheorem{fact}[defn]{Fact}
\newtheorem{crly}[defn]{Corollary}
\newtheorem{conj}[defn]{Conjecture}

%\newtheorem*{programming*}{Programming Task}

%\newtheorem{innercustomgeneric}{\customgenericname}
%\providecommand{\customgenericname}{}
%\newcommand{\newcustomtheorem}[2]{%
%  \newenvironment{#1}[1]
%  {%
%   \renewcommand\customgenericname{#2}%
%   \renewcommand\theinnercustomgeneric{##1}%
%   \innercustomgeneric
%  }
%  {\endinnercustomgeneric}
%}

%\newcustomtheorem{question}{Question}
%\newcustomtheorem{programming}{Programming Task}

\newcommand{\NN}{\mathbb{N}}
\newcommand{\ZZ}{\mathbb{Z}}
\newcommand{\QQ}{\mathbb{Q}}
\newcommand{\RR}{\mathbb{R}}
\newcommand{\CC}{\mathbb{C}}
\newcommand{\PP}{\mathbb{P}}
\newcommand{\FF}{\mathbb{F}}
\newcommand{\bA}{\mathbb{A}}
\newcommand{\Hom}{\operatorname{Hom}}
\newcommand{\im}{\operatorname{im}}
\newcommand{\id}{\operatorname{id}}
\newcommand{\Ind}{\operatorname{Ind}}
\newcommand{\Res}{\operatorname{Res}}


\newcommand{\calD}{\mathcal{D}}

\newcommand{\sol}{\textit{Solution: }}

\title{Algebraic Geometry}
\author{Kevin}
\date{January 2025}

\begin{document}
\maketitle
\section{Motivating Examples and Introduction}
\[\backslash:\tag{Owen's signature}\]
Consider $f(x,y)=x^2+y^2-1$ over $\RR$. Roots of $f$: unit circle.
\begin{itemize}
    \item $1$-dim manifold
    \item smooth
    \item irreducible
\end{itemize}
Algebraically, consider the quotient ring $R[x,y]/I$, where $I=(x^2+y^2-1)$.
\begin{itemize}
    \item transcendence degree $1$ over $\RR$.
    \item localizations are independent of choices
    \item $I$ is a prime ideal
\end{itemize}
\begin{example}
    $x^n+y^n=z^n$ over $\ZZ$. (no non-trivial solutions when $n\ge 3$) Assume $n=2$. Can identify (not a bijective correspondence) the solution sets with $\{(x,y)\in\QQ^2:x^2+y^2=1\}$. Consider the line $L_t:y=tx+1$. $L_t$ meets the circle at $(-2t/(1+t^2),(1-t^2)/(1+t^2))$. In $\QQ^2$, if $t\in \QQ$, then 
    \[t\leftrightarrow\left(\dfrac{-2t}{1+t^2},\dfrac{1-t^2}{1+t^2}\right)\]
    is a fully algebraic identification of the solution set with the base field.
\end{example}
\begin{rem}
    Care about intersections.
\end{rem}
\begin{example}
    FTA. The set of zeros of a polynomial over $\CC$ is $\{(z,y)=(z,p(z)):z\in\CC\}\cap\{(z,y):y=0\}\subseteq\CC^2$.
\end{example}
\begin{defn}
    Let $L/k$ be a field extension. We say $x\in L$ is algebraic over $k$ if there exists a non-zero $p_x\in k[z]$ s.. $p_x(x)=0$. Otherwise, $x$ is transcendental over $k$. Say $L/k$ is algebraic if all elemnts of $L$ are algebraic over $k$
\end{defn}
Recall that every field $k$ has a (unique up to iso) maximal algebraic extension $\bar k$, its algebraic closure.

\textbf{In this course, we work over an algebraically closed field of characteristic $0$.}

\subsection{The Projective Plane}
FTA predicts that every two lines in a plane intersect at a point, unless they are parallel.
\begin{defn}
    The projective plane: $\PP^2_k=\PP^2=\{(x,y,z)\in k^3\setminus\{(0,0,0)\}\}/\{(x,y,z)\sim\lambda(x,y,z),\lambda\neq 0\}$.
\end{defn}
Denote $[(x,y,z)]=(x:y:z)$. We have an inclusion $k^2\hookrightarrow\PP_k^2, (x,y)\mapsto (x:y:1)$. The points at infinity are $\{(x:y:0)\}\subseteq\PP_k^2$.

A line $ax+by-c=0$ in $k^2$ dosen't have well-defined solution set in $\PP^2$, but its homogenization $ax+by+cz=0$ does.

\begin{defn}
    A projective plane curve is
    \[\{(x:y:z)\in\PP^2\colon F(x,y,z)=0\}\]
    for some non-zero homogeneous poly $F$.
\end{defn}
If $f\in k[x,y]$, then $z^{\deg f}f(x/z,y/z)=F(x,y,z)$ is homogeneous of the same degree.
\begin{defn}
    If $C=\{f(x,y)=0\}\subseteq k^2$ and $F$ is the homogenization of $f$, we say that $\{(x:y:z)\in\PP^2:F(x,y,z)=0\}$ is the projective closure of $C$ in $\PP^2$.
\end{defn}

\section{Affine Varieties}
\begin{defn}
    Affine $n$-space over $k$ is the set $\bA_k^n=k^n$. A polynomial $f(x_1,...,x_n)\in k[x_1,...,x_n]$ is also a function $\bA^n\to k$. The zero set (vanishing locus) of a subset $S\subseteq k[x_1,...,x_n]$ is the set $Z(S)=\{P\in\bA^n:\forall f\in S,\ f(P)=0\}$. An affine algebraic set is any subset of some $\bA^n$ of the form $Z(S)$ for some $S\subseteq k[x_1,...,x_n]$.
\end{defn}
\begin{defn}
    If $f\in k[x_1,...,x_n]$ is a non-constant polynoimal, then $Z(f)$ is a hypersurface. In particular, if $f$ is linear, $Z(f)$ is a hyperplane.
\end{defn}
\begin{example}
    The twisted cubic is $\{(t,t^2,t^3)\in\bA^3:t\in k\}=Z(t-x^2,z-x^3)$.It is non-planar (not contained in any hyperplane).
\end{example}
\begin{prop}
    Let $S\subseteq k[x_1,...,x_n]$ be a set of polys. Then
    \begin{enumerate}
        \item[(i)] $Z(S)=Z((S))$, where $(S)\trianglelefteq k[x_1,...,x_n]$ is the ideal generated by $S$.
        \item[(ii)] There exists $f_1,...,f_r\in S$ s.t. $Z(S)=Z(f_1,...,f_r)$.
    \end{enumerate}
\end{prop}
\begin{proof}
    (i) is trivial. (ii) follows from $k[x_1,...,x_n]$ being Noetherian.
\end{proof}
\begin{prop}
    Affine algebraic sets satisfy
    \begin{enumerate}
        \item[(i)]$S\subseteq T\subseteq k[x_1,...,x_n]\implies Z(T)\subseteq Z(S)$.
        \item[(ii)] $\bA^n,\varnothing$ are affine algebriac sets.
        \item[(iii)] Given a collection $\{S_i\}_{i\in I}$ of subsets of $k[x_1,...,x_n]$, $\bigcap_{i\in I} Z(S_i)=Z(\bigcup_{i\in I}S)$.
        \item[(iv)] If $S,T\subseteq k[x_1,...,x_n]$ are finite, then $Z(S)\cup Z(T)=Z(ST)$.
    \end{enumerate}
\end{prop}
\begin{proof}
    (i)-(iii) clear. (iv) by direct calculation.
\end{proof}
\begin{defn}
    The Zariski topology on $\bA$ is the topology whose closed sets are affine algebraic subsets. This is indeed a topology by preceding proposition.
\end{defn}
\begin{defn}
    A distinguished open set in $\bA^n$ is any set $\bA^n\setminus Z(f)$ for a single $f$.
\end{defn}
Note that Zariski topology is very coarse. The intersection of two non-empty open sets is non-empty and dense. Will prove in ES1 that distinguished open sets form a basis of Zariski topology. (Also on ES1) The Zariski topology on a product is the the product of Zariski topology. If $X\in \bA^n$ is affine algebraic, then the subspace topology agrees with the Zariski topology on $X$.

\begin{defn}
    A topological space $X$ is irreducible if $X$ is cannot be written as $X=X_1\cup X_2$ with $X_1, X_2$ closed and proper. Otherwise, $X$ is reducible.
\end{defn}
e.g. $Z(xy)$ is reducible.
\begin{defn}
    An affine variety is an irreducible (w.r.t. Zariski topology) affine algebraic set.
\end{defn}
If $f\in k[x_1,...,x_n]$ is irreducible then $Z(f)$ is irreducible.

\section{Ideals and the Nullstellensatz}
\begin{defn}
    $X\subseteq \bA^n$. The ideal of $X$ is $I(X)=\{f\in k[x_1,...,x_n]:\forall P\in X,\ f(P)=0\}$
\end{defn}
\begin{prop}[Properties of $I(X)$ for $X$ algebraic]
    Let $X, Y$ be affine algebraic sets in $\bA^n$.
    \begin{enumerate}[(1)]
        \item If $S\subseteq k[x_1,...,x_n]$, then $S\subseteq I(Z(S))$;
        \item $X=Z(I(X))$
        \item $X=Y$ iff $I(X)=I(Y)$.
        \item $X\subseteq Y$ iff $I(Y)\subseteq I(X)$
    \end{enumerate}
\end{prop}
\begin{proof}
    (1) clear from defn. 

    (2) Clearly $X\subseteq Z(I(X))$. Conversely, write $X=Z(S)$, so $S\subseteq I(X)$, so $Z(I(X))\subseteq Z(S)=X$.

    (3) follows from (2).

    (4). If $X\subseteq Y$, then $I(Y)\subseteq I(X)$ by defn. Conversely, if $P\in X\setminus Y$, then (2) implies that $P\not\in Z(I(Y))$, so there exists $f\in I(Y)$ with $f(P)\neq 0$.
\end{proof}
\begin{prop}
    Any affine algebraic set is a finite union (unique up to ordering, ES1) of irred. affine algebraic sets (varieties).
\end{prop}
\begin{proof}
    Let $X$ be affine algebraic. Suppose $X$ is reducible (otherwise done), i.e., $X=X_1\cup X_1'$. If $X$ is not a finite union of varieties, then wlog $X_1$ is not a finite union of varieties. We can write $X_1=X_2\cup X_2'$ s.t. $X_2$ fails to be a finite union of varieties. Continue, get a descending chain of affine alg sets. By the preceding prop, get an ACC in $k[x_1,...,x_n]$, which eventually stabilizes, i.e., eventually $X_n$ is a finite union of varieties.
\end{proof}
Get maps $Z(\cdot ),\ I(\cdot )$
\[\left\{\text{affine alg subsets of }\bA^n\right\}\overset{X\leftrightarrow I(X)}{\longleftrightarrow}\left\{I\trianglelefteq k[x_1,...,x_n]\right\}\]
$I(\cdot)$ does not have full image, e.g., $(x^2)$.
\begin{prop}
    $X\subseteq\bA^n$ affine alg set. Then $X$ is irreducible iff $I(X)$ is a prime ideal.
\end{prop}
\begin{proof}
    Suppose $X$ is reducible, write $X=X_1\cup X_2$ proper closed. Then $I(X)=I(X_1)\cap I(X_2)$. By (3) of the previous proposition, there exists $f\in I(X_1)\setminus I(X_2)$ and $g\in I(X_2)\setminus I(X_1)$. Then $fg\in I(X)$ but $f,g\not\in I(X)$, so $I(X)$ is not prime.

    Conversely, if $I(X)$ is not prime, then can find $f,g\not\in I(X)$ but $fg\in I(X)$. Define $X_1=X\cap Z(f)$ and $X_2=X\cap Z(g)$. These are proper closed subsets and $X_1\cup X_2=X$, so $X$ is reducible. 
\end{proof}

\begin{thm}[Weak Nullstellensatz]
    The maximal ideals of $k[x_1,...,x_n]$ are those of the form $(x_1-a_1,x_2-a_2,...,x_n-a_n)$ for some $(a_1,...,a_n)\in k^n$.
\end{thm}
Proof postponed.
\begin{crly}[Weak Nullstellensatz]
    If $I\subsetneq k[x_1,...,x_n]$ is a proper ideal, then $Z(I)\neq\varnothing$.
\end{crly}
\begin{proof}
    Any proper ideal is contained in a maximal ideal which has the form $(x_1-a_1,...,x_n-a_n)=\mathfrak m$, so $(a_1,...,a_n)\in Z(\mathfrak m)$.
\end{proof}
\begin{defn}
    Let $I\trianglelefteq k[x_1,...,x_n]$. The radical ideal of $I$ is $\sqrt I=\{f\in k[x_1,...,x_n]:\exists m>0,\ f^m\in I\}$.
\end{defn}
Note that $I\subseteq\sqrt I$ and $Z(I)=Z(\sqrt I)$.
\begin{thm}[Hilbert's Nullstellensatz]
    Let $J\trianglelefteq k[x_1,...,x_n]$. Then $\sqrt J=I(Z(J))$.
\end{thm}
\begin{proof}
    By defn, $\sqrt J\subseteq I(Z(J))$.

    Write $J=(f_1,...,f_r)$ and let $g\in I(Z(J))$. Define another ideal $\tilde J=(f_1,...,f_r,x_{n+1}g(x_1,...,x_n)-1)\trianglelefteq k[x_1,...,x_{n+1}]$. If $\tilde P\in Z(\tilde J)$, then the projection $P$ to the first $n$ coords is in $Z(f_i)$ for all $1\le i\le r$, so $g(P)=0$. Contradicting $x_{n+1}g-1=0$, so $Z(\tilde J)=\varnothing$, so $1\in \tilde J$ by weak Nullstellensatz, so $\exists h_1,...,h_{n+1}\in k[x_1,...,x_{r+1}]$ with $\sum_{i=1}^r h_if_i+h_{r+1}(x_{n+1}g-1)=1$. On the set where $x_{n+1}g=1$, we have $\sum_{i=1}^r h(x_1,...,x_n,1/g(x_1,...,x_n))g(x_1,...,x_n)=1$. Clear denominators by a sufficiently high power of $g$. Get
    \[\sum_{i=1}^rh_i'(x_1,...,x_n)f_i(x_1,...,x_n)=g(x_1,...,x_n)^N\]
    so $g\in\sqrt J$.
\end{proof}


\end{document}