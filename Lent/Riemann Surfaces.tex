\documentclass{article}
\usepackage{graphicx} % Required for inserting images
\usepackage[utf8]{inputenc}
\usepackage{amsmath,amsfonts,amssymb,amsthm}
\usepackage{enumerate,bbm}
\usepackage{leftindex}
\usepackage{tikz,tikz-cd,graphicx,color,mathrsfs,color,hyperref,boldline}
\usepackage{caption,float}
\usepackage[a4paper,margin=1in,footskip=0.25in]{geometry}

\usepackage{listings}
\usepackage{xcolor}

\usepackage{tabularx,capt-of}

\usepackage{blindtext}
%Image-related packages
\usepackage{graphicx}
\usepackage{subcaption}
\usepackage[export]{adjustbox}
\usepackage{lipsum}

%hyperref setup
\hypersetup{
    colorlinks=true,
    linkcolor=blue,
    filecolor=magenta,      
    urlcolor=cyan,
    pdftitle={Overleaf Example},
    pdfpagemode=FullScreen,
    }

%New colors defined below
\definecolor{codegreen}{rgb}{0,0.6,0}
\definecolor{codegray}{rgb}{0.5,0.5,0.5}
\definecolor{codepurple}{rgb}{0.58,0,0.82}
\definecolor{backcolour}{rgb}{0.95,0.95,0.92}

%Code listing style named "mystyle"
\lstdefinestyle{mystyle}{
  backgroundcolor=\color{backcolour}, commentstyle=\color{codegreen},
  keywordstyle=\color{magenta},
  numberstyle=\tiny\color{codegray},
  stringstyle=\color{codepurple},
  basicstyle=\ttfamily\footnotesize,
  breakatwhitespace=false,         
  breaklines=true,                 
  captionpos=b,                    
  keepspaces=true,                 
  numbers=left,                    
  numbersep=5pt,                  
  showspaces=false,                
  showstringspaces=false,
  showtabs=false,                  
  tabsize=2
}

%"mystyle" code listing set
\lstset{style=mystyle}

\theoremstyle{definition}
\newtheorem{defn}{Definition}[section]
\newtheorem{example}[defn]{Example}
\theoremstyle{remark}
\newtheorem{rem}{Remark}
\newtheorem{remS}[section]{defn}
\theoremstyle{plain}
\newtheorem{lem}[defn]{Lemma}
\newtheorem{thm}[defn]{Theorem}
\newtheorem{prop}[defn]{Proposition}
\newtheorem{fact}[defn]{Fact}
\newtheorem{crly}[defn]{Corollary}
\newtheorem{conj}[defn]{Conjecture}

%\newtheorem*{programming*}{Programming Task}

%\newtheorem{innercustomgeneric}{\customgenericname}
%\providecommand{\customgenericname}{}
%\newcommand{\newcustomtheorem}[2]{%
%  \newenvironment{#1}[1]
%  {%
%   \renewcommand\customgenericname{#2}%
%   \renewcommand\theinnercustomgeneric{##1}%
%   \innercustomgeneric
%  }
%  {\endinnercustomgeneric}
%}

%\newcustomtheorem{question}{Question}
%\newcustomtheorem{programming}{Programming Task}

\newcommand{\NN}{\mathbb{N}}
\newcommand{\ZZ}{\mathbb{Z}}
\newcommand{\QQ}{\mathbb{Q}}
\newcommand{\RR}{\mathbb{R}}
\newcommand{\CC}{\mathbb{C}}
\newcommand{\PP}{\mathbb{P}}
\newcommand{\FF}{\mathbb{F}}
\newcommand{\Hom}{\operatorname{Hom}}
\newcommand{\im}{\operatorname{im}}
\newcommand{\id}{\operatorname{id}}
\newcommand{\Ind}{\operatorname{Ind}}
\newcommand{\Res}{\operatorname{Res}}

\newcommand{\calD}{\mathcal{D}}

\newcommand{\sol}{\textit{Solution: }}

\title{Riemann Surfaces}
\author{Kevin}
\date{January 2025}

\begin{document}
\maketitle
\section{The Complex Plane}
): (Owen's signature)
\subsection{Review of Complex Analysis}
\begin{itemize}
    \item A domain $U\subseteq \CC$ is an open (non-empty) and connected (hence path-connected) subset. 
    \item Disk centered at $z_0$ with radius $r$: $D(z_0,r)$.
    \item Punctured disk $D'(z_0,r)$.
\end{itemize}
If $f:U\to\CC$ is complex-diff, then $\det J=|f'|^2$, where $J$ is the Jacobian of $f$ regarded as a function on $\RR^2$.
\begin{defn}
    $f:U\to\CC$, $U$ domain, is holo'c on $U$ if $f$ is $\CC$-diff for all $z_0\in U$.
\end{defn}
\begin{thm}[Taylor]
    If $f$ is holo'c on $U$ and $D(z_0,r)\subseteq U$, then $f(z)=\sum_{n=0}^\infty a_n(z-z_0)^n$ converges on $D(z_0,r)$ with $a_n=f^{(n)}(z_0)/n!$. Then $f$ holo'c on $U$ implies that $f^{(n)}$ holo'c on $U$.
\end{thm}
\begin{thm}[Identity theorem]
    $f,g$ be holo'c on a domain $U$, then either $f\equiv g$ or for all $z_0\in U$ there exists $r>0$ s.t. $f(z)\neq g(z)$ for all $z\in D'(z_0,r)\subseteq U$.
\end{thm}

Recall that if $f:D'(z_0,r)\to\CC$ is holo'c then $z_0$ is an isolated singularity.
In this situation we have the (unique) Laurent expansion at $z_0$, $f(z)=\sum_{n=-\infty}^\infty a_n(z-z_0)^n$ on $D'(z_0,r)$.
\begin{thm}
    Exactly one occurs
    \begin{itemize}
        \item Removable
        \item Pole
        \item Isolated essential.
    \end{itemize}
\end{thm}
\begin{defn}
    $f:U\to\CC$ ($U$ domain) meromorphic if it's holo'c away from poles.
\end{defn}
\begin{thm}[Open mapping theorem]
    $f$ holo'c on a domain $U$. Then $f$ is either constant or an open map.
\end{thm}
\begin{crly}[Inverse function theorem]
    If $f$ is holo'c on a domain, then
    \begin{enumerate}
        \item $f$ injective $\implies$ $f^{-1}$ is holo'c with $(f^{-1})'(z)=1/f'(f^{-1}(z))$.
        \item if $f'(z_0)\neq 0$ for some $z_0\in U$, then there exists an open nbd (nbd will always be open in this course) $N$ of $z_0$ s.t. $f:N\to f(N)$ is biholomorphic (i.e., $f$ and $f^{-1}$ are both holo'c).
    \end{enumerate}
\end{crly}

\subsection{Continuation of Power Series}
\textit{Notation: $\Delta:=D(0,1)$, and $T=\{z\mid |z|=1\}$.}

Suppose we have power series with radius of convergence $R$ (not $0$ or $\infty$) so $f:D(z_0,R)\to\CC$ holo'c. WLOG, assume $R=1$ and power series as domain $\Delta$.
\begin{defn}
    A point $z\in T$ is regular if $\exists D(z,r)$ and $g:D(z,r)\to\CC$ holo'c s.t. $f\equiv g$ on $\Delta\cap D(z,r)$. Otherwise, we say that $z$ is singular.
\end{defn}
So $z$ being regular means that $f$ can be extended holomorphically to $\bar f$ on $\Delta\cup D(z,r)$.
\begin{example}
    \begin{enumerate}
        \item[(i)] $f(z)=\sum_nz^n$ diverges on $T$ but the $f(z)=1/(1-z)$ means that $T\setminus\{1\}$ are regular points
        \item[(ii)] Take $f(z)=\sum_{n\ge 2}\frac{z^n}{n(n-1)}$ is absolutely conv. on $T$. Note that $1$ is singular. If it's regular, then $f''$ extends but $f''$ is the geometric series.
    \end{enumerate}
\end{example}
\begin{rem}
    \begin{enumerate}
        \item[(i)] The set of regular points is open in $T$.
        \item[(ii)] For $w\in\Delta$, let $\rho(w)$ be radius of convergence of the Taylor series at $w$. For $z\in T$, have $z$ regular if and only if $\rho(z/2)>1/2$.
    \end{enumerate}
\end{rem}
\begin{thm}
    If $f(z)=\sum_{n\ge 0} a_nz^n$ has r.o.c. $1$, then there is a singular point on $T$.
\end{thm}
\begin{proof}
    If not, then for all $w\in T$, $\exists r_w>0$ s.t. $f$ can be extended holomorphically to $f_w$ on $\Delta\cup D(w,r_w)$. Note that $D(w,r_w)\cap D(w',r_{w'})$ is convex (if non-empty). Let $F:\Delta\cup\bigcup_{w\in T}D(w,r_w)\to\CC$ be the extension of $f$. $F$ is well-defined. If $z\in D(w,r_w)\cap D(w',r_{w'})$ then $f_w,f_{w'}$ both holo'c on $D(w,r_w)\cap D(w',r_{w'})\cup\Delta$. By identity theorem, they agree. Now, we note that the domain of $F$ includes $D(o,R)$ for some $R>1$. (If not, take $z_n$ with $1|z_n|<1+1/n$) and $z_n\not\in\operatorname{dom}(F)$, then by sequential compactness, get a contradiction.) Contradiction with r.o.c being $1$.
\end{proof}
\begin{crly}
    If $a_n\ge0$ for all $n$, then $1$ is singular.
\end{crly}
\begin{proof}
    For $z\in T$ and each $k$, we have $f^{(k)}(z/2)=\sum_{n=k}^\infty n(n-1)...(n-k+1)a_n(z/2)^{n-k}$. So $|f^{(k)}(z/2)|\le |f^{(k)}(1/2)|$, so $\rho(z/2)\ge \rho(1/2)$. If $\rho(1/2)>1/2$, then all $z\in T$ is regular. Contradiction.
\end{proof}

\subsection{Complex Logarithm}
Notation: $L_\theta=\{re^{i\theta}:r\ge 0\}$.
\begin{defn}
    For any $\alpha\in\RR$, define $\log_\alpha:\CC\setminus L_{\alpha+\pi}\to\CC$ as $\log_\alpha(re^{i\theta})=\ln r+i\theta$, where $\theta\in(\alpha-\pi,\alpha+\pi)$.
\end{defn}

\subsection{Analytic Continuation (Plane)}
Let $D$ be a fixed domain of $\CC$.
\begin{defn}
    A function element of $D$ is a pair $(f,U)$, where $U$ is a subdomain $(\subseteq D)$ and $f$ is a holo'c function defined on $U$.
\end{defn}
\begin{defn}
    Let $(f,U)$ and $(g,V)$ be function elements of $D$. Say $(g,V)$ is a direct analytic continuation of $(f,U)$ if $U\cap V\neq\varnothing$ and $f\equiv g$ on $U\cap V$.
\end{defn}
\begin{rem}
    The relation of being direct analytic continuation is relfexive and symmetric but not transitive. For instance, $f=\log_{\pi/2}$ on its domain, $g=\log_0$ on the right half plane, $h=\log_{-\pi/2}$ on its domain.
\end{rem}
\begin{defn}
    $(g,V)$ is an analytic continuation of $(f,U)$ along some path $\gamma$ if there exists function elements $(f, U)=(f_1,U_1), (f_2,U_2),...,(f_n,U_n)=(g,V)$ and a dissection $\{t_0=0\le t_1\le...\le t_n=1\}$ s.t. $(f_{i+1},U_{i+1})$ is a direct analytic continuation of $(f_i, U_i)$ and $\gamma([t_{i-1},t_i])\subseteq U_i$. Write $(f,U)\approx (g,V)$.

    An equivalence class $\mathfrak F$ is called a complete analytic function of $D$.
\end{defn}
Note that if $(g,V)$ and $(h,V)$ are both analytic continuations of $(f,U)$ along the \textbf{same} path $\gamma$, then $g\equiv h$ on $V$. (See ES1 or later).

\begin{rem}
    Can do the same for meromorphic continuation.
\end{rem}
\end{document}