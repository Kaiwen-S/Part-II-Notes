\documentclass{article}
\usepackage{graphicx} % Required for inserting images
\usepackage[utf8]{inputenc}
\usepackage{amsmath,amsfonts,amssymb,amsthm}
\usepackage{enumerate,bbm}
\usepackage{leftindex}
\usepackage{tikz,tikz-cd,graphicx,color,mathrsfs,color,hyperref,boldline}
\usepackage{caption,float}
\usepackage[a4paper,margin=1in,footskip=0.25in]{geometry}

\usepackage{listings}
\usepackage{xcolor}

\usepackage{tabularx,capt-of}

\usepackage{blindtext}
%Image-related packages
\usepackage{graphicx}
\usepackage{subcaption}
\usepackage[export]{adjustbox}
\usepackage{lipsum}

%hyperref setup
\hypersetup{
    colorlinks=true,
    linkcolor=blue,
    filecolor=magenta,      
    urlcolor=cyan,
    pdftitle={Overleaf Example},
    pdfpagemode=FullScreen,
    }

%New colors defined below
\definecolor{codegreen}{rgb}{0,0.6,0}
\definecolor{codegray}{rgb}{0.5,0.5,0.5}
\definecolor{codepurple}{rgb}{0.58,0,0.82}
\definecolor{backcolour}{rgb}{0.95,0.95,0.92}

%Code listing style named "mystyle"
\lstdefinestyle{mystyle}{
  backgroundcolor=\color{backcolour}, commentstyle=\color{codegreen},
  keywordstyle=\color{magenta},
  numberstyle=\tiny\color{codegray},
  stringstyle=\color{codepurple},
  basicstyle=\ttfamily\footnotesize,
  breakatwhitespace=false,         
  breaklines=true,                 
  captionpos=b,                    
  keepspaces=true,                 
  numbers=left,                    
  numbersep=5pt,                  
  showspaces=false,                
  showstringspaces=false,
  showtabs=false,                  
  tabsize=2
}

%"mystyle" code listing set
\lstset{style=mystyle}

\theoremstyle{definition}
\newtheorem{defn}{Definition}[section]
\newtheorem{example}[defn]{Example}
\theoremstyle{remark}
\newtheorem{rem}{Remark}
\newtheorem{remS}[section]{defn}
\theoremstyle{plain}
\newtheorem{lem}[defn]{Lemma}
\newtheorem{thm}[defn]{Theorem}
\newtheorem{prop}[defn]{Proposition}
\newtheorem{fact}[defn]{Fact}
\newtheorem{crly}[defn]{Corollary}
\newtheorem{conj}[defn]{Conjecture}

%\newtheorem*{programming*}{Programming Task}

%\newtheorem{innercustomgeneric}{\customgenericname}
%\providecommand{\customgenericname}{}
%\newcommand{\newcustomtheorem}[2]{%
%  \newenvironment{#1}[1]
%  {%
%   \renewcommand\customgenericname{#2}%
%   \renewcommand\theinnercustomgeneric{##1}%
%   \innercustomgeneric
%  }
%  {\endinnercustomgeneric}
%}

%\newcustomtheorem{question}{Question}
%\newcustomtheorem{programming}{Programming Task}

\newcommand{\NN}{\mathbb{N}}
\newcommand{\ZZ}{\mathbb{Z}}
\newcommand{\QQ}{\mathbb{Q}}
\newcommand{\RR}{\mathbb{R}}
\newcommand{\CC}{\mathbb{C}}
\newcommand{\PP}{\mathbb{P}}
\newcommand{\FF}{\mathbb{F}}
\newcommand{\Hom}{\operatorname{Hom}}
\newcommand{\im}{\operatorname{im}}
\newcommand{\id}{\operatorname{id}}
\newcommand{\Ind}{\operatorname{Ind}}
\newcommand{\Res}{\operatorname{Res}}

\newcommand{\calD}{\mathcal{D}}

\newcommand{\sol}{\textit{Solution: }}

\title{Logic and Set Theory}
\author{Kevin}
\date{January 2025}

\begin{document}
\maketitle
\section{Propositional Logic}
\((\; :\langle\) (by Owen)
\begin{defn}
    The language of propositional logic consists of a set $P$ of primitive propositions and the set $L=L(P)$ of propositions (or compound propositions) built as follows
    \begin{enumerate}
        \item[(i)] $P\subseteq L$
        \item[(ii)] $\bot\in L$ ($\bot$ is called `false' or `bottom')
        \item[(iii)] If $p,q\in L$, then $(p\Rightarrow q)\in L$.
    \end{enumerate}
\end{defn}
\begin{example}
    Often $P=\{p_1,p_2,...\}$ (we need a countable infinity supply). Then $(p_1\Rightarrow p_2),\ (\bot\Rightarrow(p_1\Rightarrow p_2)),\ ((p_1\Rightarrow p_2)\Rightarrow(p_1\Rightarrow p_2))\in L$. Also, if $p\in L$, then $((p\Rightarrow \bot)\Rightarrow\bot)\in L$.
\end{example}
\begin{rem}\
    \begin{itemize}
        \item The set $L$ is built from $P$ and $\bot$ inductively, which means that $L=\bigcup_{n\in\NN}L_n$, where $L_1=P\cup\{\bot\}$ and for $n\in \NN$, $L_{n+1}=L_n\cup\{(p\Rightarrow q)\colon p,q\in L_n\}$
        \item Every proposition is a finite string of symbols in the alphabet $P\cup\{\bot,\Rightarrow,(,)\}$. It's easy to check that $L$ is the smallest subset of the set $\Sigma$ of all such strings satsifying (i)-(iii). Note that $L\neq\Sigma$, e.g. $)\Rightarrow p$ is not a proposition.
        \item Every proposition $p$ is built uniquely using (i)-(iii), i.e., either $p\in P$ or $p=\bot$ or $p=(q\Rightarrow r)$ for some unique $q,r\in L$.
        \item We will introduce $\wedge,\ \vee$ later.
    \end{itemize}
\end{rem}
\subsection{Semantic Entailment}
\begin{defn}
    A valuation on $L$ is a function $v:L\to\{0,1\}$ s.t. 
    \begin{enumerate}
        \item[(i)] $v(\bot)=0$
        \item[(ii)] $v(p\Rightarrow q)=\begin{cases}
            0 & v(p)=1 \text{ and } v(q)=0\\
            1 & \text{o/w}
        \end{cases}$
    \end{enumerate}
\end{defn}
\begin{prop}\
    \begin{enumerate}
        \item[(i)] If $v,v'$ are valuations and $v|_P=v'|_P$, then $v=v'$.
        \item[(ii)] For any function $w:P\to\{0,1\}$, there exists a valuation on $L$ s.t. $v|_P=w$.
    \end{enumerate}
\end{prop}
\begin{proof}\
    \begin{enumerate}
        \item[(i)] We have $v(\bot)=v'(\bot)=0$ by definition, so $v|_{L_1}=v'|_{L_1}$. Continue inductively. If $v|_{L_n}=v'|_{L_{n}}$, then for all $p\in L_{n+1}\setminus L_n$, there exists unique $q,r\in L_n$ s.t. $p=(q\Rightarrow r)$. Then $v(q\Rightarrow r)=v'(q\Rightarrow r)$.
        \item[(ii)] Define $v(p)=w(p)$ for all $p\in P$. Extend $v$ to $L$ inductively.
    \end{enumerate}
\end{proof}
\begin{defn}
    Let $t\in L$. We say that $t$ is a tautology if $v(t)=1$ for all valuations $v$ on $L$.
\end{defn}
\textbf{Notations:}
\begin{itemize}
    \item (true/top) $\top=(\bot\Rightarrow\bot)$
    \item (not) $\neg p=(p\Rightarrow\bot)$
    \item (or) $p\vee q=(\neg p\Rightarrow q)=((p\Rightarrow\bot)\Rightarrow q)$
    \item (and) $p\wedge q=\neg (p\Rightarrow\neg q)$
\end{itemize}


\end{document}