\documentclass{article}
\usepackage{graphicx} % Required for inserting images
\usepackage[utf8]{inputenc}
\usepackage{amsmath,amsfonts,amssymb,amsthm}
\usepackage{enumerate,bbm}
\usepackage{tikz,graphicx,color,mathrsfs,color,hyperref,boldline}
\usepackage{caption,float}
\usepackage[a4paper,margin=1in,footskip=0.25in]{geometry}

\usepackage{listings}
\usepackage{xcolor}

\usepackage{tabularx,capt-of}

\usepackage{blindtext}
%Image-related packages
\usepackage{graphicx}
\usepackage{subcaption}
\usepackage[export]{adjustbox}
\usepackage{lipsum}

%hyperref setup
\hypersetup{
    colorlinks=true,
    linkcolor=blue,
    filecolor=magenta,      
    urlcolor=cyan,
    pdftitle={Overleaf Example},
    pdfpagemode=FullScreen,
    }

%New colors defined below
\definecolor{codegreen}{rgb}{0,0.6,0}
\definecolor{codegray}{rgb}{0.5,0.5,0.5}
\definecolor{codepurple}{rgb}{0.58,0,0.82}
\definecolor{backcolour}{rgb}{0.95,0.95,0.92}

%Code listing style named "mystyle"
\lstdefinestyle{mystyle}{
  backgroundcolor=\color{backcolour}, commentstyle=\color{codegreen},
  keywordstyle=\color{magenta},
  numberstyle=\tiny\color{codegray},
  stringstyle=\color{codepurple},
  basicstyle=\ttfamily\footnotesize,
  breakatwhitespace=false,         
  breaklines=true,                 
  captionpos=b,                    
  keepspaces=true,                 
  numbers=left,                    
  numbersep=5pt,                  
  showspaces=false,                
  showstringspaces=false,
  showtabs=false,                  
  tabsize=2
}

%"mystyle" code listing set
\lstset{style=mystyle}

\theoremstyle{definition}
\newtheorem{defn}{Definition}[section]
\theoremstyle{remark}
\newtheorem{rem}{Remark}
\newtheorem{remS}[section]{defn}
%\newtheorem{lem}[defn]{Lemma}
\theoremstyle{plain}
\newtheorem{thm}[defn]{Theorem}
\newtheorem{prop}[defn]{Proposition}
\newtheorem{fact}[defn]{Fact}
\newtheorem{crly}[defn]{Corollary}
\newtheorem{conj}[defn]{Conjecture}
\newtheorem{example}[defn]{Example}
\newtheorem{lem}[defn]{Lemma}
%\newtheorem*{programming*}{Programming Task}

%\newtheorem{innercustomgeneric}{\customgenericname}
%\providecommand{\customgenericname}{}
%\newcommand{\newcustomtheorem}[2]{%
%  \newenvironment{#1}[1]
%  {%
%   \renewcommand\customgenericname{#2}%
%   \renewcommand\theinnercustomgeneric{##1}%
%   \innercustomgeneric
%  }
%  {\endinnercustomgeneric}
%}

%\newcustomtheorem{question}{Question}
%\newcustomtheorem{programming}{Programming Task}

\newcommand{\NN}{\mathbb{N}}
\newcommand{\ZZ}{\mathbb{Z}}
\newcommand{\QQ}{\mathbb{Q}}
\newcommand{\RR}{\mathbb{R}}
\newcommand{\CC}{\mathbb{C}}
\newcommand{\PP}{\mathbb{P}}
\newcommand{\FF}{\mathbb{F}}

\newcommand{\calD}{\mathcal{D}}

\newcommand{\sol}{\textit{Solution: }}
\newcommand{\Res}{\operatorname{Res}}

\title{Introduction to Stable Homotopy Theory}
\author{}
\date{}

\begin{document}

\maketitle
\section{Simplicial Homotopy Theory}
\begin{defn}
    $\Delta$ is the category of finite non-empty totally ordered sets with morphisms non-decreasing maps.
\end{defn}
The typical object of $\Delta$ is $[n]=\{0<1<...<n\}$. We have three important examples of morphisms.
\begin{itemize}
    \item $s:[n]\to[0]$ which collapses everything to $0$.
    \item ($i$th face map) $\partial_i:[n]\to[n+1]$ which omits the value $i$, i.e.,
    \[\partial_i(j)=\begin{cases}
        j & j< i\\
        j+1 & j\ge i
    \end{cases}\]
    \item ($i$th degeneracy map) $s_i:[n]\to[n-1]$ which identifies two adjacent values, i.e.,
    \[s_i(j)=\begin{cases}
        j & j<i\\
        j-1 & j\ge i
    \end{cases}\]
\end{itemize}
\begin{rem}
    By $\mathbf{Top}$, we mean the category of compactly generated topological spaces.
\end{rem}
There is a functor $\Delta\to\mathbf{Top}$ sending $[n]$ to $|\Delta^n|=\{(t_0,...,t_{n+1})\in\RR^{n+1}\colon \sum_it_i=1, t_i\ge 0\}$, the geometric $n$-simplex. To see functoriality, simply note that if $f:[n]\to[m]$ is order-preserving, then
$f$ induces a map $f:|\Delta^n|\to|\Delta^m|$ by extending the map of basis linearly, i.e.,
\[f(t_0,...,t_n)=\left(\sum_{i\in f^{-1}(0)}t_i,...,\sum_{i\in f^{-1}(m)}t_i\right)\]
So, the $i$th face map induces the inclusion of a face of a simplex; the $i$th degeneracy map projects everythign down to a face.
\begin{defn}
    A \textbf{simplicial set} is a functor $X:\Delta^{\text{op}}\to\mathbf{Set}$.
\end{defn}
\begin{example}
    Let $X$ be a topological space. $\operatorname{Sing}(X)$ is the simplicial set $[n]\mapsto \operatorname{Hom}_{\mathbf{Top}}(|\Delta^n|,X)$
\end{example}
\begin{rem}
    Slogan: $\operatorname{Sing}(X)$ knows all about the homotopy type of $X$.
\end{rem}
Inspired by the example above, we often refer to $X[0]$ as the set of points of $X$ and $X[1]$ as the set of paths in $X$ (more generally $X[n]$ the set of $n$-simplices).
\begin{example}
    Can define $C_\ast(X)_n=$ free abelian group generated by $X[n]$, with boundary maps given by $\partial=\sum_i(-1)^i\partial_i$,
\end{example}

We now look at more examples of simplicial sets.
\begin{example}
    Suppose $\mathbf{C}$ is a category, then we define its \textbf{nerve} as the simplicial set $N\mathbf{C}:[n]\mapsto \operatorname{Fun}([n],\mathbf{C})$ ($[n]$ seen as a poset). Then, $N\mathbf{C}_0$ is the collection of objects in $\mathbf{C}$; $N\mathbf{C}_1$ the collection of arrows in $\mathbf{C}$; and $N\mathbf{C}_2$ the collection of composable pairs of arrows in $\mathbf{C}$.
\end{example}
\begin{example}
    Can also define $\Delta^n$ as the simplicial set $N([n])$, where $[n]$ is seen as a poset, meaning that $\Delta^n:[m]\mapsto\operatorname{Hom}([m],[n])$. This is the simplicial analog of $|\Delta^n|$. This comes with some important simplicial subsets. For instance, $\partial\Delta^n=\text{``the union of }\partial_i\Delta^n\text{ ''}:[m]\mapsto\{f\in\operatorname{Hom}([m],[n]):f\text{ not onto}\}$. For $0\le i\le n$, one can also take the \textbf{horn}, defined as $\Lambda_i^n:[m]\mapsto\{f\in\operatorname{Hom}([m],[n]):\operatorname{im}(f)\not\supseteq\{0,...,\hat{i},...,n\}\}$ (every face except the $i$th one).
\end{example}
\begin{prop}
    The functor $\operatorname{Sing}:\mathbf{Top}\to\mathbf{sSet}$ has a left adjoint $|-|:\mathbf{sSet}\to\mathbf{Top}$ called geometric realization, and $\operatorname{Hom}_{\mathbf{SSet}}(X,\operatorname{Sing}(Y))\cong\operatorname{Hom}_{\mathbf{Top}}(|X|,Y)$. An explicit formula of $|-|$ is given by
    \[|X|=\left(\bigsqcup_{n\ge 0}X[n]\times|\Delta^n|\right)\Bigg/(\forall f:[n]\to[m],\forall x\in X[n]),\forall t\in|\Delta^n|,(f^\ast(x),t)\sim(x,f(t))\]    
\end{prop}
\begin{example}
    $|\Delta^n|=|\Delta^n|$ trivially.

    $|-|$ preserves colimits: $|\partial \Delta^n|=$union of the proper faces in $|\Delta^n|$.
\end{example}
\begin{proof}
    A map $X\to\operatorname{Sing}(Y)$ is a natural transformation $x\in X[n]\mapsto \sigma_x:|\Delta^n|\to Y$. For all $f:[n]\to[m]$ and for all $x\in X[m]$, we have $\sigma_{f^\ast(x)}=f\circ\sigma_x$, where $f:|\Delta^n|\to|\Delta^m|$, if and only if the map $\bigsqcup_{n\ge0}X[n]\times|\Delta^n|\to Y$ induced from this preserves equiv relation, if and only if we have a map $|X|\to Y$.
\end{proof}
Let's examine $|\Delta^1\times\Delta^1|$. The simplices of $\Delta^1\times\Delta^1$ are paris of maps $(f_i:[n]\to[1],g_{i'}:[n]\to[1])$, where $f_i(j)=0$ if $j<i$ and $f_i(j)=1$ if $j\ge i$. We claim that $\Delta^1\times\Delta^1=\Delta^2\cup_{\Delta^1}\Delta^2$. It's two dimensional as every higher simplex factors through one/two dim simplices. Can interpret this as $|\Delta^1\times\Delta^1|\to|\Delta^1|\times|\Delta^1|$ being a homeo.
\begin{prop}
    Let $X,Y$ be simplicial sets. The map $|X\times Y|\to|X|\times|Y|$ is a homeomorphism
\end{prop}
\begin{proof}
    (Sketch)
    Reduction to $X=\Delta^n$, $Y=\Delta^m$ and do a combinatorial argument.

    First assume $Y=\Delta^n$. Consider the poset of simplicial subsets of $X$, i.e., the collection of $A\subseteq X$, s.t. $|A\times\Delta^n|\to|A|\times|\Delta^n|$ is a homeo. By Zorn's lemma, this has a maximal element $A$ [\textit{we're using compact generation to show that $-\times|\Delta^n|$ commutes with colimits}]. Let $A$ be such a maximal element. If $A=X$ then we are done, if $A\neq X$, then we can find a simplex $\sigma\in X[m]$ not in $A$ of minimal dim. Note that $\sigma$ has to be non-degenerate due to minimality of its dimension. We take $A'=A\cup_{\partial\Delta^m}\Delta^m$. Using the fact that $|-\times\Delta^n|$ and $|-|\times|\Delta^n|$ commutes with pushout, we get a contradiction.

    Then to prove it for general $Y$, fix $X$ and consider the poset $B\subseteq Y$ s.t. $|X\times B|\cong|X|\times |B|$. 

    For details of the combinatorial argument, see Gabriel-Zisman Thm 3.1 of \textit{Calculus of Fractions and Homotopy Theory}.
\end{proof}
\begin{rem}
    We are using the following fact. If $Z$ is a compactly generated topological space. The functor $-\times Z$ has a left adjoint $\operatorname{Maps}(Z,-)$, which implies that the product functor commutes with colimits. Specializing to pushouts, we have $(A\cup_B C)\times Z\cong (A\times Z)\cup_{B\times Z}(C\times Z)$. The general statement is that $|X\times Y|\to(|X|\times|Y|)^{cg}$ is a homeo, where the product $|X|\times |Y|$ is not compactly generated in general, so we take its k-ification.
\end{rem}

\subsection{Homotopy groups}
The next goal is to define $\pi_n(X,x)$. To achieve this we need something more from $X$. To illustrate that simplicial set is not sufficient, consider $\pi_0(X)$. Ideally we want to take $X[0]$ and quotient by the relation $x\sim y$ if there is a path from $x$ to $y$, i.e., $\exists \gamma\in X[1]$ s.t. $\partial_1\gamma=x,\partial_0\gamma=y$. However, $\sim$ is not an equiv relation in general. For instance, let $X=\Delta^1$. We have $0\sim 1$ but $1\not\sim 0$. Transitivity is not guaranteed. if $\gamma:x\to y$ and $\delta:y\to z$, then we want to fill the horn $\Delta^1\cup_{\Delta^0}\Delta^1=\Lambda_1^2$ to obtain a path $x\to z$, but this is not possible in general.
\begin{defn}
    $X\in\mathbf{sSet}$ is a \textbf{Kan complex} if $\forall n\ge0,\forall 0\le i\le n,\forall f:\Lambda_i^n\to X$, there exists an extension $\tilde f:\Delta^n\to X$.
\end{defn}
\begin{example}
    Let $Y$ be a topological space, then $\operatorname{Sing}(Y)$ is a Kan complex.
    \begin{proof}
        a map $\Lambda_i^n\to\operatorname{Sing}(Y)$ is the same thing as a map $|\Lambda_i^n|\to Y$. Our goal is to extend it to $|\Delta^n|\overset{|\tilde f|}{\to} Y$. The inclusion $|\Lambda_i^n|\subseteq|\Delta^n|$ has a retraction (sending the barycenter of the $i$th face to the $i$th vertex and extend linearly). Then we just define $|\tilde{f}|=|f|\circ r$, where $r$ is the retraction.
    \end{proof}
    We note that this is how we define composition of path.
\end{example}
\begin{rem}
    If $X$ is a Kan complex, the relation $\sim$ on $X[0]$ defined above is an equivalence relation.
\end{rem}
\begin{example}
    Let $X, Y$ be topological spaces [not necessarily compactly generated in this situation]. Take $\operatorname{Maps}(X,Y)$ as the simplicial set $[n]\mapsto \operatorname{Hom}_{\mathbf{Top}}(X\times |\Delta^n|, Y)$. Then $\operatorname{Maps}(X,Y)$ is a Kan complex. Notice that points are continuous maps and paths are homotopies. If $X,Y$ are compactly generated, then this is seen as the mapping space. Have $\pi_0(\operatorname{Maps}(X,Y))=[X,Y]$.
\end{example}
\end{document}