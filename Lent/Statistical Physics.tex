\documentclass{article}
\usepackage{graphicx} % Required for inserting images
\usepackage[utf8]{inputenc}
\usepackage{amsmath,amsfonts,amssymb,amsthm}
\usepackage{enumerate,bbm}
\usepackage{tikz,graphicx,color,mathrsfs,color,hyperref,boldline}
\usepackage{caption,float}
\usepackage[a4paper,margin=1in,footskip=0.25in]{geometry}

\usepackage{listings}
\usepackage{xcolor}

\usepackage{tabularx,capt-of}

\usepackage{blindtext}
%Image-related packages
\usepackage{graphicx}
\usepackage{subcaption}
\usepackage[export]{adjustbox}
\usepackage{lipsum}

%hyperref setup
\hypersetup{
    colorlinks=true,
    linkcolor=blue,
    filecolor=magenta,      
    urlcolor=cyan,
    pdftitle={Overleaf Example},
    pdfpagemode=FullScreen,
    }

%New colors defined below
\definecolor{codegreen}{rgb}{0,0.6,0}
\definecolor{codegray}{rgb}{0.5,0.5,0.5}
\definecolor{codepurple}{rgb}{0.58,0,0.82}
\definecolor{backcolour}{rgb}{0.95,0.95,0.92}

%Code listing style named "mystyle"
\lstdefinestyle{mystyle}{
  backgroundcolor=\color{backcolour}, commentstyle=\color{codegreen},
  keywordstyle=\color{magenta},
  numberstyle=\tiny\color{codegray},
  stringstyle=\color{codepurple},
  basicstyle=\ttfamily\footnotesize,
  breakatwhitespace=false,         
  breaklines=true,                 
  captionpos=b,                    
  keepspaces=true,                 
  numbers=left,                    
  numbersep=5pt,                  
  showspaces=false,                
  showstringspaces=false,
  showtabs=false,                  
  tabsize=2
}

%"mystyle" code listing set
\lstset{style=mystyle}

\theoremstyle{definition}
\newtheorem{defn}{Definition}[section]
\newtheorem{example}[defn]{Example}
\theoremstyle{remark}
\newtheorem{rem}{Remark}
\newtheorem{remS}[section]{defn}
\newtheorem{lem}[defn]{Lemma}
\theoremstyle{plain}
\newtheorem{thm}[defn]{Theorem}
\newtheorem{prop}[defn]{Proposition}
\newtheorem{fact}[defn]{Fact}
\newtheorem{crly}[defn]{Corollary}
\newtheorem{conj}[defn]{Conjecture}

%\newtheorem*{programming*}{Programming Task}

%\newtheorem{innercustomgeneric}{\customgenericname}
%\providecommand{\customgenericname}{}
%\newcommand{\newcustomtheorem}[2]{%
%  \newenvironment{#1}[1]
%  {%
%   \renewcommand\customgenericname{#2}%
%   \renewcommand\theinnercustomgeneric{##1}%
%   \innercustomgeneric
%  }
%  {\endinnercustomgeneric}
%}

%\newcustomtheorem{question}{Question}
%\newcustomtheorem{programming}{Programming Task}

\newcommand{\NN}{\mathbb{N}}
\newcommand{\ZZ}{\mathbb{Z}}
\newcommand{\QQ}{\mathbb{Q}}
\newcommand{\RR}{\mathbb{R}}
\newcommand{\CC}{\mathbb{C}}
\newcommand{\PP}{\mathbb{P}}
\newcommand{\FF}{\mathbb{F}}

\newcommand{\calD}{\mathcal{D}}
\newcommand{\id}{\operatorname{id}}

\newcommand{\sol}{\textit{Solution: }}
\newcommand{\Res}{\operatorname{Res}}

\title{Statistical Physics}
\author{Kevin}
\date{January 2025}

\begin{document}
\maketitle
\section{}
\begin{defn}
    Isolated system: no exchange of energy/particles with the outside world.
\end{defn}
Assume QM, but applied to classical physics.
Recall Schr\"odinger's eqn $\hat H\psi=E\psi$.
\begin{defn}
    A microstate: the actual (quantum) state of the system (max info allowed by QM). ``pure state''

    A macrostate: specify a few macroscopically observable quantities, e.g., energy, volume, temperature, the number of particles, etc.

    A mixed state: probability $p(n)$ for the state $|n\rangle$ (in some basis).
\end{defn}
One macrostate $\leftrightarrow$ many microstates. 

The expectation value is 
\[\langle\hat{\mathcal{O}}\rangle=\sum_np(n)\langle n|\hat{\mathcal O}|n\rangle\]
Density matrix (combines QM and classical probability)
\[\rho=\sum_n p(n)|n\rangle\langle n|\]
Equilibrium: the probability distribution $p(n)$ (or $\rho$) is time-independent.

\textbf{Fundamental Assumption:} For an isolated system in equilibrium, all accessible microstates are equally likely. For now, ``accessible'' $=$ ``same energy $E$''. More precisely, $E<E_n<E+\Delta E$ (there is a small range).
\begin{defn}
    Microcanonical ensemble: $\Omega(E)=\#$ of states with energy $E$. So
    \[p(n)=\begin{cases}
        1/\Omega(E) & \text{if }E<E_n<E+\Delta E\\
        0 & \text{otherwise}
     \end{cases}\]
\end{defn}
We can assume that the energy levels form a continuum since they are freely spread and $\Omega(E)$ is large.

\begin{defn}
    Boltzmann entropy of system (macrostate) is defined as 
    $S(E)=k_B\ln\Omega(E)$, where $k_B\approx 1.381\times 10^{-23}J/K$ is Boltzmann constant.
\end{defn}
e.g., gas of $10^{23}$ molecules, then $S/k_B\sim 10^{23}$, then $\Omega=e^{S/k_B}\sim 10^{10^{23}}$, so the choice of $\Delta E$ doesn't matter. If $\Delta(E)\mapsto 10^6\Delta E$, then $S\mapsto S+k_B\log(10^6)$. The difference is negligible.

$S$ is additive: consider two (independent and isolated) systems, then $\Omega(E_1,E_2)=\Omega(E_1)\Omega(E_2)$ so $S(E_1,E_2)=S(E_1)+S(E_2)$.

Now assume that these two systems exchange energy, $E^{\text{old}}_1\mapsto E_1$, $E_2^{\text{old}}\mapsto E_2$, $E_{\text{tot}}=E_1^{\text{old}}+E_2^{\text{old}}=E_1+E_2$. Then,
\[\Omega(E_{\text{tot}})=\sum_{E_i}\Omega_1(E_i)\Omega_2(E_{\text{tot}}-E_i)=\sum_{E_i}\exp\left[\dfrac{S_1(E)}{k}+\dfrac{S_1(E_{\text{tot}}-E_i)}{k}\right]\]
The system is dominated by $\max S$, so $\Delta S\ge 0$. (2nd Law)

\end{document}